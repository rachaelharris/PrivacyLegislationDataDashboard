As Introduced
134th General Assembly
Regular Session H. B. No. 367
2021-2022
Representatives Crawley, Jarrells
Cosponsors: Representatives Miranda, Boyd, Howse, West, Leland, Russo,
Galonski, Brent, Brown, Denson, Boggs, Weinstein, Miller, A., Smith, M., Smith,
K., Upchurch, Blackshear, Sheehy, Crossman, Robinson, Skindell, Liston,
Sobecki, Lightbody, Lepore-Hagan, Hicks-Hudson, Sweeney, Kelly
A B I L L
To enact sections 149.437, 2935.40, 2935.41,
2935.42, and 2935.43 of the Revised Code
regarding the use of body-worn cameras and
dashboard cameras by peace officers, the use of
excessive force by peace officers, and the
public release of body-worn camera or dashboard
camera recordings when there is an allegation of
peace officer misconduct.
BE IT ENACTED BY THE GENERAL ASSEMBLY OF THE STATE OF OHIO:
Section 1. That sections 149.437, 2935.40, 2935.41,
2935.42, and 2935.43 of the Revised Code be enacted to read as
follows:
 Sec. 149.437. (A) As used in this section:
(1) "Body-worn camera," "dashboard camera," and
"restricted portions of a body-worn camera or dashboard camera
recording" have the same meanings as in section 149.43 of the
Revised Code.
1
2
3
4
5
6
7
8
9
10
11
12
13
14
15
16
H. B. No. 367 Page 2
As Introduced
(2) "Enforcement action" and "law enforcement agency" have
the same meanings as in section 2935.40 of the Revised Code.
(B) Every law enforcement agency in the state shall
establish and follow a retention schedule for body-worn camera
recordings and dashboard camera recordings, if applicable, in
compliance with section 149.33, 149.39, or 149.42 of the Revised
Code, as applicable.
(C) Subject to divisions (D), (E), and (F) of this section
and except as otherwise provided in division (G) of this
section, if a law enforcement agency receives a complaint
alleging misconduct by a peace officer while conducting an
enforcement action, regarding a peace officer employed by the
agency, from another peace officer, a nonprofit organization, or
a member of the public, that employing agency shall release to
the public all unedited video and audio recordings of the
enforcement action, including those from a body-worn camera,
dashboard camera, or otherwise collected during an
investigation, within twenty-one days after the law enforcement
agency receives notice of the complaint.
(D)(1) A law enforcement agency shall provide, upon
request, any audio or video recording under division (C) of this
section that includes the death of a person to any of the
following:
(a) The person's spouse;
(b) The person's parent;
(c) The person's legal guardian;
(d) The person's child;
(e) The person's sibling;
17
18
19
20
21
22
23
24
25
26
27
28
29
30
31
32
33
34
35
36
37
38
39
40
41
42
43
44
H. B. No. 367 Page 3
As Introduced
(f) The person's grandparent;
(g) The person's grandchild;
(h) The person's significant other;
(i) The person's legal representative.
(2) The law enforcement agency shall notify the person
described in division (D)(1) of this section who requests the
audio or video recording of the person's right to receive and
review the audio or video recording at least seventy-two hours
prior to the release of the audio or video recording to the
public under division (C) of this section.
(E)(1) If any audio or video recording that is to be
released to the public under division (C) of this section
includes a restricted portion of a body-worn camera or dashboard
camera recording, the law enforcement agency shall redact or
obscure that recording before releasing the recording to the
public. No law enforcement agency shall release any unredacted
or unobscured recording without obtaining the written
authorization of the person in the recording, or, if the person
is deceased, incapacitated, or a child, the written
authorization of the person's next of kin.
(2)(a) A person who is the subject of an audio or video
recording described in division (E)(1) of this section may
waive, in writing, the person's privacy interest that is
implicated in the audio or video recording. The law enforcement
agency shall notify that person, if the person's contact
information is available, within twenty days after a complaint
of peace officer misconduct, that the person has a right to
waive the person's privacy interest that is implicated in the
audio or video recording.
45
46
47
48
49
50
51
52
53
54
55
56
57
58
59
60
61
62
63
64
65
66
67
68
69
70
71
72
73
H. B. No. 367 Page 4
As Introduced
(b) If a law enforcement agency receives a written waiver
from that person, the law enforcement agency shall not redact,
obscure, or withhold the release of the audio or video recording
to the public.
(F) If redacting or obscuring an audio or video recording
pursuant to division (E) of this section is insufficient to
protect the identity or privacy interests of the person in the
audio or video recording, the law enforcement agency shall, upon
request, release the audio or video recording to the person or,
if the person is deceased, incapacitated, or a child, to the
person's spouse, parent, legal guardian, child, sibling,
grandparent, grandchild, significant other, or legal
representative within twenty days after the law enforcement
agency receives the complaint of peace officer misconduct.
(G) If an audio or video recording would substantially
interfere with or jeopardize an active or ongoing investigation,
the law enforcement agency may withhold the audio or video
recording from being released to the public for not more than
forty-five days from the date of the allegation of peace officer
misconduct. The prosecuting attorney, village solicitor, city
director of law, or similar chief legal officer, as applicable,
shall prepare a written explanation of reasons why release of
the audio or video recording to the public would substantially
interfere with or jeopardize the active or ongoing
investigation. Upon release of the audio or video recording to
the public, the prosecuting attorney, village solicitor, city
director of law, or similar chief legal officer shall release
that written explanation to the public.
(H) If criminal charges have been filed against any person
involved in the enforcement action, that person has the right to
74
75
76
77
78
79
80
81
82
83
84
85
86
87
88
89
90
91
92
93
94
95
96
97
98
99
100
101
102
103
H. B. No. 367 Page 5
As Introduced
object to the public release of any body-worn camera or
dashboard camera recording. The person shall file the objection
with the court within twenty-one days of the appointment of
counsel, the filing of an entry of appearance by counsel, or the
person's election to proceed pro se, whichever is later. If the
person elects to proceed pro se, the court shall advise the
person of the deadline to file an objection to the public
release of any body-worn camera or dashboard camera recording.
The court shall hold a hearing on any objection to the public
release of any body-worn camera or dashboard camera recording
not later than seven days after the filing of the objection with
the court and the court shall issue a ruling not later than
three days after the hearing.
 Sec. 2935.40. (A) As used in this section:
(1) "Body-worn camera" and "dashboard camera" have the
same meanings as in section 149.43 of the Revised Code.
(2) "Enforcement action" means any of the following:
(a) A call for service or a self-initiated service
activity;
(b) An investigatory stop;
(c) A traffic or pedestrian stop;
(d) A pursuit by foot, vehicle, bicycle, or any other
available means of transportation;
(e) A use of force;
(f) An arrest;
(g) A forced entry into a structure, vehicle, or other
premises.
104
105
106
107
108
109
110
111
112
113
114
115
116
117
118
119
120
121
122
123
124
125
126
127
128
129
130
H. B. No. 367 Page 6
As Introduced
(3) "Law enforcement agency" has the same meaning as in
section 2925.61 of the Revised Code.
(B) On or before July 1, 2023, every law enforcement
agency in the state shall provide body-worn cameras for each
peace officer of the law enforcement agency who interacts with
the public or who is a correctional officer in a jail.
(C) Except as otherwise provided in division (D), (E), or
(F) of this section, a peace officer shall wear and activate a
body-worn camera or, if the peace officer's vehicle is equipped
with a dashboard camera, activate a dashboard camera when
engaging in an enforcement action or, if the peace officer is a
correctional officer in a jail, when performing a task that
requires an anticipated use of force, including removing an
inmate from the inmate's cell or placing an inmate in a
restraint chair.
(D) A peace officer may deactivate a body-worn camera or
dashboard camera under the following circumstances:
(1) To avoid recording personal information that is not
related to a specific case;
(2) When working on an assignment that is not related to a
specific case;
(3) When there is an extended period of inactivity in the
enforcement activity or there is contact between the peace
officer and an individual that is not related to the enforcement
activity;
(4) When the peace officer is involved in administrative,
tactical, or management discussions.
(E) Division (C) of this section does not apply to any of
131
132
133
134
135
136
137
138
139
140
141
142
143
144
145
146
147
148
149
150
151
152
153
154
155
156
157
158
H. B. No. 367 Page 7
As Introduced
the following:
(1) A peace officer who is undercover;
(2) A peace officer who is a correctional officer in a
jail if the jail is equipped with video cameras;
(3) A peace officer who is assigned to a courtroom.
(F) A law enforcement agency may apply to the attorney
general for funds to purchase body-worn cameras for the agency.
(G) Notwithstanding any provisions of section 4117.10 of
the Revised Code to the contrary, this section and section
2935.41 of the Revised Code prevail over any conflicting
provision of a collective bargaining agreement entered into
under Chapter 4117. of the Revised Code on or after the
effective date of this section.
 Sec. 2935.41. (A) If a peace officer fails to activate a
body-worn camera or dashboard camera as required under section
2935.40 of the Revised Code, or tampers with the operation of or
any portion of a body-worn camera or dashboard camera recording
when required to activate the body-worn camera or dashboard
camera, there is a permissive inference in any internal
investigation or administrative or civil proceeding that the
missing recording would have demonstrated misconduct by the
peace officer.
(B) If a peace officer fails to activate or reactivate the
peace officer's body-worn camera or dashboard camera, as
required under section 2935.40 of the Revised Code, or tampers
with the operation of or any portion of a body-worn camera or
dashboard camera recording when required to activate the bodyworn camera or dashboard camera, there is a rebuttable
presumption of inadmissibility in either of the following
159
160
161
162
163
164
165
166
167
168
169
170
171
172
173
174
175
176
177
178
179
180
181
182
183
184
185
186
187
H. B. No. 367 Page 8
As Introduced
circumstances:
(1) A statement by the peace officer is sought to be
introduced that was not recorded due to the peace officer's
failure to activate or reactivate the body-worn camera or
dashboard camera.
(2) A statement by the peace officer is sought to be
introduced that was not recorded by other means.
(C) Division (B) of this section does not apply if the
peace officer did not activate the body-worn camera or dashboard
camera because of a malfunction of the body-worn camera or
dashboard camera and the peace officer was not aware of that
malfunction or was unable to rectify it prior to the incident,
provided that the law enforcement agency that employs the peace
officer has documentation that demonstrates that the peace
officer checked the functionality of the body-worn camera or
dashboard camera at the beginning of the peace officer's shift.
(D) In addition to any criminal penalty, if a court or
internal investigation finds that a peace officer intentionally
failed to activate a body-worn camera or dashboard camera or
tampered with any body-worn camera or dashboard camera, the law
enforcement agency that employs the peace officer shall
discipline the peace officer to the extent permitted by any
applicable existing collective bargaining agreement.
 Sec. 2935.42. (A) A peace officer may only use force if
the force is reasonably necessary to achieve a lawful objective,
including to effect a lawful arrest, prevent the escape of an
offender, defend the peace officer from physical harm, or defend
another person from physical harm.
(B) A peace officer may only use deadly force if the peace
188
189
190
191
192
193
194
195
196
197
198
199
200
201
202
203
204
205
206
207
208
209
210
211
212
213
214
215
216
H. B. No. 367 Page 9
As Introduced
officer has an objectively reasonable belief that deadly force
is necessary to defend the peace officer from serious physical
harm or death or defend another person from serious physical
harm or death.
 Sec. 2935.43. (A) As used in this section:
(1) "Excessive force" means force used by a peace officer
that exceeds the minimum amount of force necessary to diffuse an
incident or protect the peace officer or others from serious
physical harm. The use of excessive force is presumed when a
peace officer continues to use force in excess of the force
permitted pursuant to section 2935.42 of the Revised Code to a
person who has been rendered incapable to resist arrest.
(2) "Unconstitutional conduct" means, under color of law,
statute, ordinance, regulation, or custom, willfully subjecting
a person to the deprivation of any rights, privileges, or
immunities secured or protected by the United States
Constitution or the Ohio Constitution.
(B) No peace officer shall recklessly fail to intervene to
prevent or stop another peace officer from using excessive force
while placing a person under arrest or in detention, taking a
person into custody, booking a person, or while controlling or
managing a crowd.
 (C) A peace officer who intervenes in another peace
officer's use of excessive force shall report that intervention
to the peace officer's immediate supervisor. The peace officer
shall, at a minimum, include all of the following in that
report:
 (1) The date, time, and location of the occurrence;
 (2)The identity, if known, and a description of the
217
218
219
220
221
222
223
224
225
226
227
228
229
230
231
232
233
234
235
236
237
238
239
240
241
242
243
244
245
H. B. No. 367 Page 10
As Introduced
participants involved in the occurrence;
 (3) A description of the intervention actions taken by the
peace officer.
(D) Whoever violates division (B) of this section is
guilty of failure to intervene in excessive use of force, a
misdemeanor of the first degree.
(E) If an internal investigation by a law enforcement
agency finds that a peace officer employed by that law
enforcement agency violated division (B) of this section, the
chief law enforcement officer of the law enforcement agency
shall inform the prosecuting attorney of this finding to allow
the prosecuting attorney to determine whether the prosecuting
attorney should file charges against the peace officer. Nothing
in this division prohibits a prosecuting attorney from filing
charges against a peace officer for a violation of division (B)
of this section before the conclusion of any internal
investigation.
(F) In addition to any criminal penalty that is imposed
for a violation of division (B) of this section, if an internal
investigation by the law enforcement agency finds that the
incident in which the peace officer that is employed by the law
enforcement agency failed to intervene resulted in death or
serious physical harm to any person, the law enforcement agency
shall discipline the peace officer to the extent permitted by
any applicable existing collective bargaining agreement.
(G)(1) If the prosecuting attorney charges a peace officer
with any offense related to and based on the use of excessive
force, including a violation of division (B) of this section,
but does not file charges against any other peace officer who
246
247
248
249
250
251
252
253
254
255
256
257
258
259
260
261
262
263
264
265
266
267
268
269
270
271
272
273
274
H. B. No. 367 Page 11
As Introduced
was present during the other peace officer's use of excessive
force, the prosecuting attorney shall prepare a written report
explaining the prosecuting attorney's basis for the decision to
not charge any other peace officer who was present during the
use of excessive force with any offense related to and based on
the use of excessive force.
(2) The prosecuting attorney shall release the written
report to the public, and shall post the report on the web site
of the prosecuting attorney's office, if applicable, within
twenty-one days of the filing of charges against the peace
officer. If public disclosure of the report would substantially
interfere with or jeopardize an ongoing criminal investigation,
the prosecuting attorney may delay the public disclosure of the
report for not more than forty-five days from the date charges
are filed against the peace officer for any offense related to
or based on the use of excessive force.
(H) No employer who employs a peace officer shall
discharge without just cause or otherwise discriminate against a
peace officer with respect to tenure, terms, conditions, or
privileges of employment, or any matter directly or indirectly
related to employment, if the peace officer, while performing
the officer's duties, intervenes in another peace officer's use
of excessive force, reports unconstitutional conduct to the
employer, or fails to comply with an order that the peace
officer reasonably believes is unconstitutional.
(I) No employer shall discriminate in any manner against a
peace officer or any other person because that peace officer or
other person has reported another peace officer's use of
excessive force, or testified, assisted, or participated in any
manner in any investigation, proceeding, or hearing relating to
275
276
277
278
279
280
281
282
283
284
285
286
287
288
289
290
291
292
293
294
295
296
297
298
299
300
301
302
303
304
H. B. No. 367 Page 12
As Introduced
that use of excessive force. 305