A bill for an act
relating to consumer data privacy; giving various rights to consumers regarding
personal data; placing obligations on certain businesses regarding consumer data;
providing for enforcement by the attorney general; proposing coding for new law
as Minnesota Statutes, chapter 325O.

BE IT ENACTED BY THE LEGISLATURE OF THE STATE OF MINNESOTA:

Section 1. [325O.01] CITATION.
This chapter may be cited as the "Minnesota Consumer Data Privacy Act."
Sec. 2. [325O.02] DEFINITIONS.
(a) For purposes of this chapter, the following terms have the meanings given.
(b) "Affiliate" means a legal entity that controls, is controlled by, or is under common
control with, that other legal entity. For these purposes, "control" or "controlled" means:
ownership of, or the power to vote, more than 50 percent of the outstanding shares of any
class of voting security of a company; control in any manner over the election of a majority
of the directors or of individuals exercising similar functions; or the power to exercise a
controlling influence over the management of a company.
(c) "Authenticate" means to use reasonable means to determine that a request to exercise
any of the rights in section 325O.05, subdivision 1, paragraphs (b) to (e), is being made by
the consumer who is entitled to exercise such rights with respect to the personal data at
issue.
(d) "Child" has the meaning given in United States Code, title 15, section 6501.
(e) "Consent" means any freely given, specific, informed, and unambiguous indication
of the consumer's wishes by which the consumer signifies agreement to the processing of
personal data relating to the consumer for a narrowly defined particular purpose. Acceptance
of a general or broad terms of use or similar document that contains descriptions of personal
data processing along with other, unrelated information does not constitute consent. Hovering
over, muting, pausing, or closing a given piece of content does not constitute consent.
Likewise, consent cannot be obtained through a user interface designed or manipulated with
the substantial effect of subverting or impairing user autonomy, decision making, or choice.
(f) "Consumer" means a natural person who is a Minnesota resident acting only in an
individual or household context. It does not include a natural person acting in a commercial
or employment context.
(g) "Controller" means the natural or legal person which, alone or jointly with others,
determines the purposes and means of the processing of personal data.
(h) "Decisions that produce legal effects concerning a consumer or similarly significant
effects concerning a consumer" means decisions that result in the provision or denial of
financial and lending services, housing, insurance, education enrollment, criminal justice,
employment opportunities, health care services, or access to basic necessities, such as food
and water.
(i) "Deidentified data" means data that cannot reasonably be used to infer information
about, or otherwise be linked to, an identified or identifiable natural person, or a device
linked to such person, provided that the controller that possesses the data:
(1) takes reasonable measures to ensure that the data cannot be associated with a natural
person;
(2) publicly commits to maintain and use the data only in a deidentified fashion and not
attempt to reidentify the data; and
(3) contractually obligates any recipients of the information to comply with all provisions
of this paragraph.
(j) "Delete" means to remove or destroy information such that it is not maintained in
human- or machine-readable form and cannot be retrieved or utilized in the course of
business.
(k) "Identified or identifiable natural person" means a person who can be readily
identified, directly or indirectly.
(l) "Known child" means a child under circumstances where a controller has actual
knowledge of, or willfully disregards, the child's age.
(m) "Personal data" means any information that is linked or reasonably linkable to an
identified or identifiable natural person. Personal data does not include deidentified data or
publicly available information. For purposes of this paragraph, "publicly available
information" means information that is lawfully made available from federal, state, or local
government records.
(n) "Process" or "processing" means any operation or set of operations that are performed
on personal data or on sets of personal data, whether or not by automated means, such as
the collection, use, storage, disclosure, analysis, deletion, or modification of personal data.
(o) "Processor" means a natural or legal person who processes personal data on behalf
of a controller.
(p) "Profiling" means any form of automated processing of personal data to evaluate,
analyze, or predict personal aspects concerning an identified or identifiable natural person's
economic situation, health, personal preferences, interests, reliability, behavior, location,
or movements.
(q) "Pseudonymous data" means personal data that cannot be attributed to a specific
natural person without the use of additional information, provided that such additional
information is kept separately and is subject to appropriate technical and organizational
measures to ensure that the personal data are not attributed to an identified or identifiable
natural person.
(r) "Sale," "sell," or "sold" means the exchange of personal data for monetary or other
valuable consideration by the controller to a third party. Sale does not include the following:
(1) the disclosure of personal data to a processor who processes the personal data on
behalf of the controller;
(2) the disclosure of personal data to a third party with whom the consumer has a direct
relationship for purposes of providing a product or service requested by the consumer;
(3) the disclosure or transfer of personal data to an affiliate of the controller;
(4) the disclosure of information that the consumer intentionally made available to the
general public via a channel of mass media, and did not restrict to a specific audience; or
(5) the disclosure or transfer of personal data to a third party as an asset that is part of a
merger, acquisition, bankruptcy, or other transaction in which the third party assumes control
of all or part of the controller's assets.
(s) Sensitive data is a form of personal data. "Sensitive data" means:
(1) personal data revealing racial or ethnic origin, religious beliefs, mental or physical
health condition or diagnosis, sexual orientation, or citizenship or immigration status;
(2) the processing of genetic or biometric data for the purpose of uniquely identifying
a natural person;
(3) the personal data of a known child; or
(4) specific geolocation data.
(t) "Specific geolocation data" means information derived from technology including
but not limited to global positioning system level latitude and longitude coordinates or other
mechanisms that can be used to identify a natural person's specific location. Specific
geolocation data excludes the content of communications.
(u) "Targeted advertising" means displaying advertisements to a consumer where the
advertisement is selected based on personal data obtained from a consumer's activities over
time and across nonaffiliated websites or online applications to predict such consumer's
preferences or interests. It does not include advertising:
(1) based on activities within a controller's own websites or online applications;
(2) based on the context of a consumer's current search query or visit to a website or
online application; or
(3) to a consumer in response to the consumer's request for information or feedback.
(v) "Third party" means a natural or legal person, public authority, agency, or body other
than the consumer, controller, processor, or an affiliate of the processor or the controller.
Sec. 3. [325O.03] SCOPE; EXCLUSIONS.
Subdivision 1. Scope. This chapter applies to legal entities that conduct business in
Minnesota or produce products or services that are targeted to residents of Minnesota, and
that satisfy one or more of the following thresholds:
(1) during a calendar year, controls or processes personal data of 100,000 consumers or
more; or
(2) derives over 25 percent of gross revenue from the sale of personal data and processes
or controls personal data of 25,000 consumers or more.
Subd. 2. Exclusions. (a) This chapter does not apply to the following entities or types
of information:
(1) a government entity, as defined by section 13.02, subdivision 7a;
(2) a federally recognized Indian tribe;
(3) information that meets the definition of:
(i) protected health information as defined by and for purposes of the Health Insurance
Portability and Accountability Act of 1996, Public Law 104-191, and related regulations;
(ii) health records, as defined in section 144.291, subdivision 2;
(iii) patient identifying information for purposes of Code of Federal Regulations, title
42, part 2, established pursuant to United States Code, title 42, section 290dd-2;
(iv) identifiable private information for purposes of the federal policy for the protection
of human subjects, Code of Federal Regulations, title 45, part 46; identifiable private
information that is otherwise information collected as part of human subjects research
pursuant to the good clinical practice guidelines issued by the International Council for
Harmonisation; the protection of human subjects under Code of Federal Regulations, title
21, parts 50 and 56; or personal data used or shared in research conducted in accordance
with one or more of the requirements set forth in this paragraph;
(v) information and documents created for purposes of the federal Health Care Quality
Improvement Act of 1986, Public Law 99-660, and related regulations; or
(vi) patient safety work product for purposes of Code of Federal Regulations, title 42,
part 3, established pursuant to United States Code, title 42, sections 299b-21 to 299b-26;
(4) information that is derived from any of the health care-related information listed in
clause (3), but that has been deidentified in accordance with the requirements for
deidentification set forth in Code of Federal Regulations, title 45, part 164;
(5) information originating from, and intermingled to be indistinguishable with, any of
the health care-related information listed in clause (3) that is maintained by:
(i) a covered entity or business associate as defined by the Health Insurance Portability
and Accountability Act of 1996, Public Law 104-191, and related regulations;
(ii) a health care provider, as defined in section 144.291, subdivision 2; or
(iii) a program or a qualified service organization as defined by Code of Federal
Regulations, title 42, part 2, established pursuant to United States Code, title 42, section
290dd-2;
(6) information used only for public health activities and purposes as described in Code
of Federal Regulations, title 45, section 164.512;
(7) an activity involving the collection, maintenance, disclosure, sale, communication,
or use of any personal data bearing on a consumer's credit worthiness, credit standing, credit
capacity, character, general reputation, personal characteristics, or mode of living by a
consumer reporting agency, as defined in United States Code, title 15, section 1681a(f), by
a furnisher of information, as set forth in United States Code, title 15, section 1681s-2, who
provides information for use in a consumer report, as defined in United States Code, title
15, section 1681a(d), and by a user of a consumer report, as set forth in United States Code,
title 15, section 1681b, except that information is only excluded under this paragraph to the
extent that such activity involving the collection, maintenance, disclosure, sale,
communication, or use of such information by that agency, furnisher, or user is subject to
regulation under the federal Fair Credit Reporting Act, United States Code, title 15, sections
1681 to 1681x, and the information is not collected, maintained, used, communicated,
disclosed, or sold except as authorized by the Fair Credit Reporting Act;
(8) personal data collected, processed, sold, or disclosed pursuant to the federal
Gramm-Leach-Bliley Act, Public Law 106-102, and implementing regulations, if the
collection, processing, sale, or disclosure is in compliance with that law;
(9) personal data collected, processed, sold, or disclosed pursuant to the federal Driver's
Privacy Protection Act of 1994, United States Code, title 18, sections 2721 to 2725, if the
collection, processing, sale, or disclosure is in compliance with that law;
(10) personal data regulated by the federal Family Educations Rights and Privacy Act,
United States Code, title 20, section 1232g, and its implementing regulations;
(11) personal data collected, processed, sold, or disclosed pursuant to the federal Farm
Credit Act of 1971, as amended, United States Code, title 12, sections 2001 to 2279cc, and
its implementing regulations, Code of Federal Regulations, title 12, part 600, if the collection,
processing, sale, or disclosure is in compliance with that law;
(12) data collected or maintained:
(i) in the course of an individual acting as a job applicant to or an employee, owner,
director, officer, medical staff member, or contractor of that business if it is collected and
used solely within the context of that role;
(ii) as the emergency contact information of an individual under item (i) if used solely
for emergency contact purposes; or
(iii) that is necessary for the business to retain to administer benefits for another individual
relating to the individual under item (i) if used solely for the purposes of administering those
benefits;
(13) personal data collected, processed, sold, or disclosed pursuant to the Minnesota
Insurance Fair Information Reporting Act in sections 72A.49 to 72A.505; or
(14) data collected, processed, sold, or disclosed as part of a payment-only credit, check,
or cash transaction where no data about consumers, as defined in section 325O.02, are
retained.
(b) Controllers that are in compliance with the Children's Online Privacy Protection Act,
United States Code, title 15, sections 6501 to 6506, and its implementing regulations, shall
be deemed compliant with any obligation to obtain parental consent under this chapter.
Sec. 4. [325O.04] RESPONSIBILITY ACCORDING TO ROLE.
(a) Controllers and processors are responsible for meeting their respective obligations
established under this chapter.
(b) Processors are responsible under this chapter for adhering to the instructions of the
controller and assisting the controller to meet its obligations under this chapter. Such
assistance shall include the following:
(1) taking into account the nature of the processing, the processor shall assist the controller
by appropriate technical and organizational measures, insofar as this is possible, for the
fulfillment of the controller's obligation to respond to consumer requests to exercise their
rights pursuant to section 325O.05; and
(2) taking into account the nature of processing and the information available to the
processor, the processor shall assist the controller in meeting the controller's obligations in
relation to the security of processing the personal data and in relation to the notification of
a breach of the security of the system pursuant to section 325E.61, and shall provide
information to the controller necessary to enable the controller to conduct and document
any data protection assessments required by section 325O.08.
(c) Notwithstanding the instructions of the controller, a processor shall:
(1) ensure that each person processing the personal data is subject to a duty of
confidentiality with respect to the data; and
(2) engage a subcontractor only (i) after providing the controller with an opportunity to
object, and (ii) pursuant to a written contract in accordance with paragraph (e) that requires
the subcontractor to meet the obligations of the processor with respect to the personal data.
(d) Taking into account the context of processing, the controller and the processor shall
implement appropriate technical and organizational measures to ensure a level of security
appropriate to the risk and establish a clear allocation of the responsibilities between the
controller and the processor to implement such measures.
(e) Processing by a processor shall be governed by a contract between the controller and
the processor that is binding on both parties and that sets out the processing instructions to
which the processor is bound, including the nature and purpose of the processing, the type
of personal data subject to the processing, the duration of the processing, and the obligations
and rights of both parties. In addition, the contract shall include the requirements imposed
by this paragraph, paragraphs (c) and (d), as well as the following requirements:
(1) at the choice of the controller, the processor shall delete or return all personal data
to the controller as requested at the end of the provision of services, unless retention of the
personal data is required by law;
(2) the processor shall make available to the controller all information necessary to
demonstrate compliance with the obligations in this chapter; and
(3) the processor shall allow for, and contribute to, reasonable audits and inspections by
the controller or the controller's designated auditor. Alternatively, the processor may, with
the controller's consent, arrange for a qualified and independent auditor to conduct, at least
annually and at the processor's expense, an audit of the processor's policies and technical
and organizational measures in support of the obligations under this chapter. The auditor
must use an appropriate and accepted control standard or framework and audit procedure
for such audits as applicable, and shall provide a report of such audit to the controller upon
request.
(f) In no event shall any contract relieve a controller or a processor from the liabilities
imposed on them by virtue of their roles in the processing relationship under this chapter.
(g) Determining whether a person is acting as a controller or processor with respect to
a specific processing of data is a fact-based determination that depends upon the context in
which personal data are to be processed. A person that is not limited in the person's processing
of personal data pursuant to a controller's instructions, or that fails to adhere to such
instructions, is a controller and not a processor with respect to a specific processing of data.
A processor that continues to adhere to a controller's instructions with respect to a specific
processing of personal data remains a processor. If a processor begins, alone or jointly with
others, determining the purposes and means of the processing of personal data, it is a
controller with respect to such processing.
Sec. 5. [325O.05] CONSUMER PERSONAL DATA RIGHTS.
Subdivision 1. Consumer rights provided. (a) Except as provided in this chapter, a
controller must comply with a request to exercise the consumer rights provided in this
subdivision.
(b) A consumer has the right to confirm whether or not a controller is processing personal
data concerning the consumer and access the categories of personal data the controller is
processing.
(c) A consumer has the right to correct inaccurate personal data concerning the consumer,
taking into account the nature of the personal data and the purposes of the processing of the
personal data.
(d) A consumer has the right to delete personal data concerning the consumer.
(e) A consumer has the right to obtain personal data concerning the consumer, which
the consumer previously provided to the controller, in a portable and, to the extent technically
feasible, readily usable format that allows the consumer to transmit the data to another
controller without hindrance, where the processing is carried out by automated means.
(f) A consumer has the right to opt out of the processing of personal data concerning
the consumer for purposes of targeted advertising, the sale of personal data, or profiling in
furtherance of decisions that produce legal effects concerning a consumer or similarly
significant effects concerning a consumer.
Subd. 2. Exercising consumer rights. (a) A consumer may exercise the rights set forth
in this section by submitting a request, at any time, to a controller specifying which rights
the consumer wishes to exercise.
(b) In the case of processing personal data concerning a known child, the parent or legal
guardian of the known child may exercise the rights of this chapter on the child's behalf.
(c) In the case of processing personal data concerning a consumer legally subject to
guardianship or conservatorship under sections 524.5-101 to 524.5-502, the guardian or the
conservator of the consumer may exercise the rights of this chapter on the consumer's behalf.
Subd. 3. Controller response to consumer requests. (a) Except as provided in this
chapter, a controller must comply with a request to exercise the rights pursuant to subdivision
1.
(b) A controller must provide one or more secure and reliable means for consumers to
submit a request to exercise their rights under this section. These means must take into
account the ways in which consumers interact with the controller and the need for secure
and reliable communication of the requests.
(c) A controller may not require a consumer to create a new account in order to exercise
a right, but a controller may require a consumer to use an existing account to exercise the
consumer's rights under this section.
(d) A controller must comply with a request to exercise the right in subdivision 1,
paragraph (e), as soon as feasibly possible, but no later than 15 days of receipt of the request.
(e) A controller must inform a consumer of any action taken on a request under
subdivision 1, paragraphs (b) to (d), without undue delay and in any event within 45 days
of receipt of the request. That period may be extended once by 45 additional days where
reasonably necessary, taking into account the complexity and number of the requests. The
controller must inform the consumer of any such extension within 45 days of receipt of the
request, together with the reasons for the delay.
(f) If a controller does not take action on a consumer's request, the controller must inform
the consumer without undue delay and at the latest within 45 days of receipt of the request
of the reasons for not taking action and instructions for how to appeal the decision with the
controller as described in subdivision 3.
(g) Information provided under this section must be provided by the controller free of
charge, up to twice annually to the consumer. Where requests from a consumer are manifestly
unfounded or excessive, in particular because of their repetitive character, the controller
may either charge a reasonable fee to cover the administrative costs of complying with the
request, or refuse to act on the request. The controller bears the burden of demonstrating
the manifestly unfounded or excessive character of the request.
(h) A controller is not required to comply with a request to exercise any of the rights
under subdivision 1, paragraphs (b) to (e), if the controller is unable to authenticate the
request using commercially reasonable efforts. In such cases, the controller may request
the provision of additional information reasonably necessary to authenticate the request.
Subd. 4. Appeal process required. (a) A controller must establish an internal process
whereby a consumer may appeal a refusal to take action on a request to exercise any of the
rights under subdivision 1 within a reasonable period of time after the consumer's receipt
of the notice sent by the controller under subdivision 3, paragraph (f).
(b) The appeal process must be conspicuously available. The process must include the
ease of use provisions in subdivision 3 applicable to submitting requests.
(c) Within 30 days of receipt of an appeal, a controller must inform the consumer of any
action taken or not taken in response to the appeal, along with a written explanation of the
reasons in support thereof. That period may be extended by 60 additional days where
reasonably necessary, taking into account the complexity and number of the requests serving
as the basis for the appeal. The controller must inform the consumer of any such extension
within 30 days of receipt of the appeal, together with the reasons for the delay. The controller
must also provide the consumer with an e-mail address or other online mechanism through
which the consumer may submit the appeal, along with any action taken or not taken by the
controller in response to the appeal and the controller's written explanation of the reasons
in support thereof, to the attorney general.
(d) When informing a consumer of any action taken or not taken in response to an appeal
pursuant to paragraph (c), the controller must clearly and prominently provide the consumer
with information about how to file a complaint with the Office of the Attorney General.
The controller must maintain records of all such appeals and the controller's responses for
at least 24 months and shall, upon request by a consumer or by the attorney general, compile
and provide a copy of the records to the attorney general.
Sec. 6. [325O.06] PROCESSING DEIDENTIFIED DATA OR PSEUDONYMOUS
DATA.
(a) This chapter does not require a controller or processor to do any of the following
solely for purposes of complying with this chapter:
(1) reidentify deidentified data;
(2) maintain data in identifiable form, or collect, obtain, retain, or access any data or
technology, in order to be capable of associating an authenticated consumer request with
personal data; or
(3) comply with an authenticated consumer request to access, correct, delete, or port
personal data pursuant to section 325O.05, subdivision 1, paragraphs (b) to (e), if all of the
following are true:
(i) the controller is not reasonably capable of associating the request with the personal
data, or it would be unreasonably burdensome for the controller to associate the request
with the personal data;
(ii) the controller does not use the personal data to recognize or respond to the specific
consumer who is the subject of the personal data, or associate the personal data with other
personal data about the same specific consumer; and
(iii) the controller does not sell the personal data to any third party or otherwise
voluntarily disclose the personal data to any third party other than a processor, except as
otherwise permitted in this section.
(b) The rights contained in section 325O.05, subdivision 1, paragraphs (b) to (e), do not
apply to pseudonymous data in cases where the controller is able to demonstrate any
information necessary to identify the consumer is kept separately and is subject to effective
technical and organizational controls that prevent the controller from accessing such
information.
(c) A controller that uses pseudonymous data or deidentified data must exercise reasonable
oversight to monitor compliance with any contractual commitments to which the
pseudonymous data or deidentified data are subject, and must take appropriate steps to
address any breaches of contractual commitments.
Sec. 7. [325O.07] RESPONSIBILITIES OF CONTROLLERS.
Subdivision 1. Transparency obligations. (a) Controllers shall provide consumers with
a reasonably accessible, clear, and meaningful privacy notice that includes:
(1) the categories of personal data processed by the controller;
(2) the purposes for which the categories of personal data are processed;
(3) how and where consumers may exercise the rights contained in section 325O.05,
including how a consumer may appeal a controller's action with regard to the consumer's
request;
(4) the categories of personal data that the controller shares with third parties, if any;
and
(5) the categories of third parties, if any, with whom the controller shares personal data.
(b) If a controller sells personal data to third parties or processes personal data for targeted
advertising, it must clearly and conspicuously disclose such processing, as well as the manner
in which a consumer may exercise the right to opt out of such processing, in a clear and
conspicuous manner.
Subd. 2. Use of data. (a) A controller's collection of personal data must be limited to
what is reasonably necessary in relation to the purposes for which such data are processed.
(b) A controller's collection of personal data must be adequate, relevant, and limited to
what is reasonably necessary in relation to the purposes for which such data are processed,
as disclosed to the consumer.
(c) Except as provided in this chapter, a controller may not process personal data for
purposes that are not reasonably necessary to, or compatible with, the purposes for which
such personal data are processed, as disclosed to the consumer, unless the controller obtains
the consumer's consent.
(d) A controller shall establish, implement, and maintain reasonable administrative,
technical, and physical data security practices to protect the confidentiality, integrity, and
accessibility of personal data. Such data security practices shall be appropriate to the volume
and nature of the personal data at issue.
(e) Except as otherwise provided in this act, a controller may not process sensitive data
concerning a consumer without obtaining the consumer's consent, or, in the case of the
processing of personal data concerning a known child, without obtaining consent from the
child's parent or lawful guardian, in accordance with the requirement of the Children's
Online Privacy Protection Act, United States Code, title 15, sections 6501 to 6506, and its
implementing regulations.
Subd. 3. Nondiscrimination. (a) A controller shall not process personal data on the
basis of a consumer's or a class of consumers' actual or perceived race, color, ethnicity,
religion, national origin, sex, gender, gender identity, sexual orientation, familial status,
lawful source of income, or disability in a manner that unlawfully discriminates against the
consumer or class of consumers with respect to the offering or provision of: housing,
employment, credit, or education; or the goods, services, facilities, privileges, advantages,
or accommodations of any place of public accommodation.
(b) A controller may not discriminate against a consumer for exercising any of the rights
contained in this chapter, including denying goods or services to the consumer, charging
different prices or rates for goods or services, and providing a different level of quality of
goods and services to the consumer. This subdivision does not prohibit a controller from
offering a different price, rate, level, quality, or selection of goods or services to a consumer,
including offering goods or services for no fee, if the offering is in connection with a
consumer's voluntary participation in a bona fide loyalty, rewards, premium features,
discounts, or club card program.
(c) A controller may not sell personal data to a third-party controller as part of a bona
fide loyalty, rewards, premium features, discounts, or club card program under paragraph
(b) unless:
(1) the sale is reasonably necessary to enable the third party to provide a benefit to which
the consumer is entitled;
(2) the sale of personal data to third parties is clearly disclosed in the terms of the
program; and
(3) the third party uses the personal data only for purposes of facilitating such a benefit
to which the consumer is entitled and does not retain or otherwise use or disclose the personal
data for any other purpose.
Subd. 4. Waiver of rights unenforceable. Any provision of a contract or agreement of
any kind that purports to waive or limit in any way a consumer's rights under this chapter
shall be deemed contrary to public policy and shall be void and unenforceable.
Sec. 8. [325O.08] DATA PROTECTION ASSESSMENTS.
(a) A controller must conduct and document a data protection assessment of each of the
following processing activities involving personal data:
(1) the processing of personal data for purposes of targeted advertising;
(2) the sale of personal data;
(3) the processing of sensitive data;
(4) any processing activities involving personal data that present a heightened risk of
harm to consumers; and
(5) the processing of personal data for purposes of profiling, where such profiling presents
a reasonably foreseeable risk of:
(i) unfair or deceptive treatment of, or disparate impact on, consumers;
(ii) financial, physical, or reputational injury to consumers;
(iii) a physical or other intrusion upon the solitude or seclusion, or the private affairs or
concerns, of consumers, where such intrusion would be offensive to a reasonable person;
or
(iv) other substantial injury to consumers.
(b) A data protection assessment must take into account the type of personal data to be
processed by the controller, including the extent to which the personal data are sensitive
data, and the context in which the personal data are to be processed.
(c) A data protection assessment must identify and weigh the benefits that may flow
directly and indirectly from the processing to the controller, consumer, other stakeholders,
and the public against the potential risks to the rights of the consumer associated with such
processing, as mitigated by safeguards that can be employed by the controller to reduce
such risks. The use of deidentified data and the reasonable expectations of consumers, as
well as the context of the processing and the relationship between the controller and the
consumer whose personal data will be processed, must be factored into this assessment by
the controller.
(d) The attorney general may request, in writing, that a controller disclose any data
protection assessment that is relevant to an investigation conducted by the attorney general.
The controller must make a data protection assessment available to the attorney general
upon such a request. The attorney general may evaluate the data protection assessments for
compliance with the responsibilities contained in section 325O.07 and with other laws. Data
protection assessments are classified as nonpublic data, as defined by section 13.02,
subdivision 9. The disclosure of a data protection assessment pursuant to a request from the
attorney general under this paragraph does not constitute a waiver of the attorney-client
privilege or work product protection with respect to the assessment and any information
contained in the assessment.
(e) Data protection assessments conducted by a controller for the purpose of compliance
with other laws or regulations may qualify under this section if they have a similar scope
and effect.
Sec. 9. [325O.09] LIMITATIONS AND APPLICABILITY.
(a) The obligations imposed on controllers or processors under this chapter do not restrict
a controller's or a processor's ability to:
(1) comply with federal, state, or local laws, rules, or regulations;
(2) comply with a civil, criminal, or regulatory inquiry, investigation, subpoena, or
summons by federal, state, local, or other governmental authorities;
(3) cooperate with law enforcement agencies concerning conduct or activity that the
controller or processor reasonably and in good faith believes may violate federal, state, or
local laws, rules, or regulations;
(4) investigate, establish, exercise, prepare for, or defend legal claims;
(5) provide a product or service specifically requested by a consumer, perform a contract
to which the consumer is a party, or take steps at the request of the consumer prior to entering
into a contract;
(6) take immediate steps to protect an interest that is essential for the life of the consumer
or of another natural person, and where the processing cannot be manifestly based on another
legal basis;
(7) prevent, detect, protect against, or respond to security incidents, identity theft, fraud,
harassment, malicious or deceptive activities, or any illegal activity; preserve the integrity
or security of systems; or investigate, report, or prosecute those responsible for any such
action;
(8) assist another controller, processor, or third party with any of the obligations under
this paragraph; or
(9) engage in public or peer-reviewed scientific, historical, or statistical research in the
public interest that adheres to all other applicable ethics and privacy laws and is approved,
monitored, and governed by an institutional review board, human subjects research ethics
review board, or a similar independent oversight entity which has determined that:
(i) the research is likely to provide substantial benefits that do not exclusively accrue to
the controller;
(ii) the expected benefits of the research outweigh the privacy risks; and
(iii) the controller has implemented reasonable safeguards to mitigate privacy risks
associated with research, including any risks associated with reidentification.
(b) The obligations imposed on controllers or processors under this chapter do not restrict
a controller's or processor's ability to collect, use, or retain data to:
(1) identify and repair technical errors that impair existing or intended functionality; or
(2) perform solely internal operations that are reasonably aligned with the expectations
of the consumer based on the consumer's existing relationship with the controller, or are
otherwise compatible with processing in furtherance of the provision of a product or service
specifically requested by a consumer or the performance of a contract to which the consumer
is a party when those internal operations are performed during, and not following, the
consumer's relationship with the controller.
(c) The obligations imposed on controllers or processors under this chapter do not apply
where compliance by the controller or processor with this chapter would violate an
evidentiary privilege under Minnesota law and do not prevent a controller or processor from
providing personal data concerning a consumer to a person covered by an evidentiary
privilege under Minnesota law as part of a privileged communication.
(d) A controller or processor that discloses personal data to a third-party controller or
processor in compliance with the requirements of this chapter is not in violation of this
chapter if the recipient processes such personal data in violation of this chapter, provided
that, at the time of disclosing the personal data, the disclosing controller or processor did
not have actual knowledge that the recipient intended to commit a violation. A third-party
controller or processor receiving personal data from a controller or processor in compliance
with the requirements of this chapter is likewise not in violation of this chapter for the
obligations of the controller or processor from which it receives such personal data.
(e) Obligations imposed on controllers and processors under this chapter shall not:
(1) adversely affect the rights or freedoms of any persons, such as exercising the right
of free speech pursuant to the First Amendment of the United States Constitution; or
(2) apply to the processing of personal data by a natural person in the course of a purely
personal or household activity.
(f) Personal data that are processed by a controller pursuant to this section must not be
processed for any purpose other than those expressly listed in this section. Personal data
that are processed by a controller pursuant to this section may be processed solely to the
extent that such processing is:
(1) necessary, reasonable, and proportionate to the purposes listed in this section;
(2) adequate, relevant, and limited to what is necessary in relation to the specific purpose
or purposes listed in this section; and
(3) insofar as possible, taking into account the nature and purpose of processing the
personal data, subjected to reasonable administrative, technical, and physical measures to
protect the confidentiality, integrity, and accessibility of the personal data, and to reduce
reasonably foreseeable risks of harm to consumers.
(g) If a controller processes personal data pursuant to an exemption in this section, the
controller bears the burden of demonstrating that such processing qualifies for the exemption
and complies with the requirements in paragraph (f).
(h) Processing personal data solely for the purposes expressly identified in paragraph
(a), clauses (1) to (7), does not, by itself, make an entity a controller with respect to such
processing.
Sec. 10. [325O.10] ATTORNEY GENERAL ENFORCEMENT.
(a) In the event that a controller or processor violates this chapter, the attorney general,
prior to filing an enforcement action under paragraph (b), must provide the controller or
processor with a warning letter identifying the specific provisions of this chapter the attorney
general alleges have been or are being violated. If, after 30 days of issuance of the warning
letter, the attorney general believes the controller or processor has failed to cure any alleged
violation, the attorney general may bring an enforcement action under paragraph (b).
(b) The attorney general may bring a civil action against a controller or processor to
enforce a provision of this chapter in accordance with section 8.31. If the state prevails in
an action to enforce this chapter, the state may, in addition to penalties provided by paragraph
(c) or other remedies provided by law, be allowed an amount determined by the court to be
the reasonable value of all or part of the state's litigation expenses incurred.
(c) Any controller or processor that violates this chapter is subject to an injunction and
liable for a civil penalty of not more than $7,500 for each violation.
Sec. 11. [325O.11] PREEMPTION OF LOCAL LAW; SEVERABILITY.
(a) This chapter supersedes and preempts laws, ordinances, regulations, or the equivalent
adopted by any local government regarding the processing of personal data by controllers
or processors.
(b) If any provision of this act or its application to any person or circumstance is held
invalid, the remainder of the act or the application of the provision to other persons or
circumstances is not affected.
Sec. 12. EFFECTIVE DATE.
This act is effective July 31, 2022, except that postsecondary institutions regulated by
the Office of Higher Education, air carriers as defined in United States Code, title 49, section
40102, and nonprofit corporations governed by Minnesota Statutes, chapter 317A, are not
required to comply with this act until July 31, 2026.