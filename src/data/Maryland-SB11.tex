EXPLANATION: CAPITALS INDICATE MATTER ADDED TO EXISTING LAW.
 [Brackets] indicate matter deleted from existing law.
 *sb0011*
SENATE BILL 11
I3 2lr0763
(PRE–FILED) CF 2lr0764
By: Senator Lee
Requested: October 15, 2021
Introduced and read first time: January 12, 2022
Assigned to: Finance
A BILL ENTITLED
1 AN ACT concerning
2 Maryland Online Consumer Protection and Child Safety Act
3 FOR the purpose of regulating the collection and use of consumers’ personal information
4 by businesses; establishing the right of a consumer to receive information regarding
5 collection practices, have personal information deleted by a business, and prohibit
6 the disclosure of personal information by a business; requiring businesses to provide
7 certain notices to consumers and include certain information in online privacy
8 policies; authorizing the Office of the Attorney General to adopt regulations to carry
9 out this Act; and generally relating to privacy of consumer personal information.
10 BY repealing and reenacting, with amendments,
11 Article – Commercial Law
12 Section 13–301(14)(xxxiv) and (xxxv)
13 Annotated Code of Maryland
14 (2013 Replacement Volume and 2021 Supplement)
15 BY adding to
16 Article – Commercial Law
17 Section 13–301(14)(xxxvi); and 14–4401 through 14–4415 to be under the new
18 subtitle “Subtitle 44. Consumer Personal Information Privacy”
19 Annotated Code of Maryland
20 (2013 Replacement Volume and 2021 Supplement)
21 SECTION 1. BE IT ENACTED BY THE GENERAL ASSEMBLY OF MARYLAND,
22 That the Laws of Maryland read as follows:
23 Article – Commercial Law
24 13–301.
2 SENATE BILL 11
1 Unfair, abusive, or deceptive trade practices include any:
2 (14) Violation of a provision of:
3 (xxxiv) The federal Servicemembers Civil Relief Act; [or]
4 (xxxv) [§] SECTION 11–210 of the Education Article; or
5 (XXXVI) TITLE 14, SUBTITLE 44 OF THIS ARTICLE; OR
6 SUBTITLE 44. CONSUMER PERSONAL INFORMATION PRIVACY.
7 14–4401.
8 (A) IN THIS SUBTITLE THE FOLLOWING WORDS HAVE THE MEANINGS
9 INDICATED.
10 (B) (1) “AGGREGATE CONSUMER INFORMATION” MEANS INFORMATION
11 THAT RELATES TO A GROUP OR CATEGORY OF CONSUMERS, FROM WHICH
12 INDIVIDUAL CONSUMER IDENTITIES HAVE BEEN REMOVED, THAT IS NOT LINKED OR
13 REASONABLY LINKABLE TO ANY CONSUMER, INCLUDING THROUGH A DEVICE.
14 (2) “AGGREGATE CONSUMER INFORMATION” DOES NOT INCLUDE AN
15 INDIVIDUAL CONSUMER RECORD THAT HAS BEEN DE–IDENTIFIED.
16 (C) “BUSINESS” MEANS:
17 (1) A SOLE PROPRIETORSHIP, A PARTNERSHIP, A LIMITED LIABILITY
18 COMPANY, A CORPORATION, AN ASSOCIATION, OR ANY OTHER LEGAL ENTITY THAT:
19 (I) IS ORGANIZED OR OPERATED FOR THE PROFIT OR
20 FINANCIAL BENEFIT OF ITS OWNERS;
21 (II) COLLECTS THE PERSONAL INFORMATION OF AN
22 INDIVIDUAL OR A CONSUMER; AND
23 (III) SATISFIES ONE OR MORE OF THE FOLLOWING THRESHOLDS:
24 1. HAS ANNUAL GROSS REVENUES IN EXCESS OF
25 $25,000,000;
26 2. ANNUALLY BUYS, RECEIVES FOR THE BUSINESS’S
27 COMMERCIAL PURPOSES, SELLS, OR SHARES FOR COMMERCIAL PURPOSES, ALONE
28 OR IN COMBINATION, THE PERSONAL INFORMATION OF 100,000 OR MORE 
SENATE BILL 11 3
1 CONSUMERS, HOUSEHOLDS, OR DEVICES; OR
2 3. DERIVES AT LEAST 50% OF ITS ANNUAL REVENUES
3 FROM SELLING CONSUMERS’ PERSONAL INFORMATION; OR
4 (2) ANY ENTITY THAT:
5 (I) CONTROLS OR IS CONTROLLED BY A BUSINESS UNDER ITEM
6 (1) OF THIS SUBSECTION; AND
7 (II) SHARES A NAME, SERVICE MARK, OR TRADEMARK WITH THE
8 BUSINESS.
9 (D) “BUSINESS PURPOSE” MEANS THE USE OF PERSONAL INFORMATION BY
10 A BUSINESS OR A SERVICE PROVIDER IN A MANNER REASONABLY NECESSARY TO
11 ACHIEVE THE OPERATIONAL PURPOSE FOR WHICH THE INFORMATION WAS
12 COLLECTED.
13 (E) (1) “COLLECT” MEANS TO BUY, RENT, GATHER, OBTAIN, RECEIVE, OR
14 ACCESS ANY PERSONAL INFORMATION RELATING TO A CONSUMER BY ANY MEANS.
15 (2) “COLLECT” INCLUDES TO RECEIVE INFORMATION FROM THE
16 CONSUMER OR BY OBSERVING THE CONSUMER’S BEHAVIOR.
17 (F) “CONSUMER” MEANS AN INDIVIDUAL WHO RESIDES IN THE STATE.
18 (G) “DE–IDENTIFIED” MEANS, WITH RESPECT TO INFORMATION,
19 PROCESSED SO THAT THE INFORMATION CANNOT REASONABLY IDENTIFY, RELATE
20 TO, DESCRIBE, BE CAPABLE OF BEING ASSOCIATED WITH, OR BE LINKED TO A
21 PARTICULAR CONSUMER, IF A BUSINESS THAT USES DE–IDENTIFIED INFORMATION:
22 (1) HAS IMPLEMENTED TECHNICAL SAFEGUARDS THAT PROHIBIT
23 RE–IDENTIFICATION OF THE CONSUMER TO WHOM THE INFORMATION MAY RELATE;
24 (2) HAS IMPLEMENTED BUSINESS PROCESSES THAT SPECIFICALLY
25 PROHIBIT RE–IDENTIFICATION OF THE INFORMATION;
26 (3) HAS IMPLEMENTED BUSINESS PROCESSES TO PREVENT
27 INADVERTENT RELEASE OF DE–IDENTIFIED INFORMATION; AND
28 (4) MAKES NO ATTEMPT TO RE–IDENTIFY THE INFORMATION.
29 (H) (1) “DESIGNATED METHOD FOR SUBMITTING VERIFIABLE 
4 SENATE BILL 11
1 CONSUMER REQUESTS” MEANS A MAILING ADDRESS, AN E–MAIL ADDRESS, AN
2 INTERNET WEBSITE, AN INTERNET PORTAL, A TELEPHONE NUMBER, OR ANY OTHER
3 APPLICABLE CONTACT INFORMATION THROUGH WHICH A CONSUMER MAY SUBMIT
4 A REQUEST OR DIRECTION UNDER THIS SUBTITLE.
5 (2) “DESIGNATED METHOD FOR SUBMITTING VERIFIABLE
6 CONSUMER REQUESTS” INCLUDES A CONSUMER–FRIENDLY MEANS OF CONTACTING
7 A BUSINESS APPROVED BY THE ATTORNEY GENERAL UNDER § 14–4412(4) OF THIS
8 SUBTITLE.
9 (I) “DEVICE” MEANS A PHYSICAL OBJECT THAT IS CAPABLE OF
10 CONNECTING TO THE INTERNET OR TO ANOTHER DEVICE.
11 (J) “HOMEPAGE” MEANS:
12 (1) THE INTRODUCTORY PAGE OF AN INTERNET WEBSITE AND ANY
13 INTERNET WEBPAGE WHERE PERSONAL INFORMATION IS COLLECTED; OR
14 (2) IN THE CASE OF AN ONLINE SERVICE OR APPLICATION:
15 (I) THE SERVICE OR APPLICATION PLATFORM PAGE OR
16 DOWNLOAD PAGE;
17 (II) A LINK WITHIN THE SERVICE OR APPLICATION, SUCH AS
18 FROM THE SERVICE OR APPLICATION CONFIGURATION, “ABOUT”, “INFORMATION”,
19 OR SETTINGS PAGE; OR
20 (III) ANY OTHER LOCATION THAT ALLOWS A CONSUMER TO
21 REVIEW THE NOTICE REQUIRED BY § 14–4403(A) OF THIS SUBTITLE, WHETHER
22 BEFORE OR AFTER DOWNLOADING THE APPLICATION OR SERVICE.
23 (K) (1) “PERSONAL INFORMATION” MEANS INFORMATION THAT
24 IDENTIFIES, RELATES TO, DESCRIBES, IS REASONABLY CAPABLE OF BEING
25 ASSOCIATED WITH, OR COULD REASONABLY BE LINKED, DIRECTLY OR INDIRECTLY,
26 WITH A PARTICULAR CONSUMER OR THE CONSUMER’S DEVICE.
27 (2) “PERSONAL INFORMATION” DOES NOT INCLUDE:
28 (I) PUBLICLY AVAILABLE INFORMATION THAT IS LAWFULLY
29 MADE AVAILABLE FROM FEDERAL, STATE, OR LOCAL GOVERNMENT RECORDS;
30 (II) DE–IDENTIFIED CONSUMER INFORMATION; OR
SENATE BILL 11 5
1 (III) AGGREGATE CONSUMER INFORMATION.
2 (L) “PROBABILISTIC IDENTIFIER” MEANS THE IDENTIFICATION OF A
3 CONSUMER OR A DEVICE TO A DEGREE OF CERTAINTY OF MORE PROBABLE THAN
4 NOT BASED ON CATEGORIES OF PERSONAL INFORMATION INCLUDED IN, OR SIMILAR
5 TO, THE CATEGORIES LISTED UNDER SUBSECTION (K) OF THIS SECTION.
6 (M) “PROCESSING” MEANS AN OPERATION OR A SET OF OPERATIONS THAT
7 IS PERFORMED ON PERSONAL INFORMATION OR ON SETS OF PERSONAL
8 INFORMATION, WHETHER OR NOT BY AUTOMATED MEANS.
9 (N) “PSEUDONYMIZE” MEANS THE PROCESSING OF PERSONAL
10 INFORMATION IN A MANNER THAT RENDERS THE PERSONAL INFORMATION NO
11 LONGER ATTRIBUTABLE TO A SPECIFIC CONSUMER WITHOUT THE USE OF
12 ADDITIONAL INFORMATION, IF THE ADDITIONAL INFORMATION IS KEPT
13 SEPARATELY AND IS SUBJECT TO TECHNICAL AND ADMINISTRATIVE SAFEGUARDS
14 TO ENSURE THAT THE PERSONAL INFORMATION IS NOT ATTRIBUTED TO AN
15 IDENTIFIED OR IDENTIFIABLE CONSUMER.
16 (O) “RESEARCH” MEANS SCIENTIFIC, SYSTEMATIC STUDY AND
17 OBSERVATION, INCLUDING BASIC RESEARCH OR APPLIED RESEARCH THAT IS IN THE
18 PUBLIC INTEREST AND THAT ADHERES TO APPLICABLE ETHICS AND PRIVACY LAWS
19 OR STUDIES CONDUCTED IN THE PUBLIC INTEREST IN THE AREA OF PUBLIC HEALTH.
20 (P) “SERVICE” MEANS WORK, LABOR, AND SERVICES, INCLUDING SERVICES
21 FURNISHED IN CONNECTION WITH THE SALE OR REPAIR OF GOODS.
22 (Q) “SERVICE PROVIDER” MEANS A PERSON THAT PROCESSES
23 INFORMATION ON BEHALF OF A BUSINESS AND TO WHICH THE BUSINESS DISCLOSES
24 A CONSUMER’S PERSONAL INFORMATION FOR A BUSINESS PURPOSE IN
25 ACCORDANCE WITH A WRITTEN CONTRACT IF THE CONTRACT PROHIBITS THE
26 ENTITY RECEIVING THE INFORMATION FROM RETAINING, USING, OR DISCLOSING
27 THE PERSONAL INFORMATION FOR ANY PURPOSE OTHER THAN FOR THE SPECIFIC
28 PURPOSE OF PERFORMING THE SERVICES SPECIFIED IN THE CONTRACT FOR THE
29 BUSINESS, OR AS OTHERWISE ALLOWED BY THIS SUBTITLE.
30 (R) “THIRD PARTY” MEANS A PERSON THAT IS NOT THE BUSINESS THAT
31 COLLECTS PERSONAL INFORMATION FROM CONSUMERS UNDER THIS SUBTITLE OR
32 A SERVICE PROVIDER OF THAT BUSINESS.
33 (S) (1) “THIRD–PARTY DISCLOSURE” MEANS A TRANSFER OF A
34 CONSUMER’S PERSONAL INFORMATION BY THE BUSINESS TO A THIRD PARTY,
35 INCLUDING SELLING, RENTING, RELEASING, DISSEMINATING, MAKING AVAILABLE,
6 SENATE BILL 11
1 TRANSFERRING, OR OTHERWISE COMMUNICATING ORALLY, IN WRITING, OR BY
2 ELECTRONIC OR OTHER MEANS.
3 (2) “THIRD–PARTY DISCLOSURE” DOES NOT INCLUDE:
4 (I) A DISCLOSURE BY A BUSINESS OF PERSONAL INFORMATION
5 OF A CONSUMER TO A SERVICE PROVIDER THAT IS NECESSARY TO THE
6 PERFORMANCE OF A BUSINESS PURPOSE INCLUDED IN A NOTICE UNDER § 14–4403
7 OF THIS SUBTITLE;
8 (II) IDENTIFICATION BY A BUSINESS OF A CONSUMER WHO HAS
9 OPTED OUT OF THE SALE OF THE CONSUMER’S PERSONAL INFORMATION FOR THE
10 PURPOSE OF ALERTING THIRD PARTIES THAT THE CONSUMER HAS OPTED OUT OF
11 THE SALE OF THE CONSUMER’S PERSONAL INFORMATION; OR
12 (III) THE TRANSFER BY A BUSINESS TO A THIRD PARTY OF THE
13 PERSONAL INFORMATION OF A CONSUMER AS AN ASSET THAT IS PART OF A MERGER,
14 AN ACQUISITION, A BANKRUPTCY, OR ANY OTHER TRANSACTION IN WHICH THE
15 THIRD PARTY ASSUMES CONTROL OF ALL OR PART OF THE BUSINESS IF THAT
16 INFORMATION IS USED OR SHARED CONSISTENTLY WITH THE NOTICE RECEIVED BY
17 CONSUMERS UNDER § 14–4403 OF THIS SUBTITLE.
18 (T) “UNIQUE IDENTIFIER” MEANS A PERSISTENT IDENTIFIER THAT CAN BE
19 USED TO RECOGNIZE A CONSUMER OR A DEVICE THAT IS LINKED TO A CONSUMER
20 OR HOUSEHOLD, OVER TIME AND ACROSS DIFFERENT TECHNOLOGIES, INCLUDING:
21 (1) A DEVICE IDENTIFIER;
22 (2) AN INTERNET PROTOCOL ADDRESS;
23 (3) A COOKIE, BEACON, PIXEL TAG, MOBILE AD IDENTIFIER, OR
24 SIMILAR TECHNOLOGY;
25 (4) A CONSUMER NUMBER, UNIQUE PSEUDONYM, OR USER ALIAS; OR
26 (5) A TELEPHONE NUMBER OR ANY OTHER FORM OF PERSISTENT OR
27 PROBABILISTIC IDENTIFIER THAT CAN BE USED TO IDENTIFY A PARTICULAR
28 CONSUMER OR DEVICE.
29 14–4402.
30 THIS SUBTITLE DOES NOT APPLY:
SENATE BILL 11 7
1 (1) TO A BUSINESS COLLECTING OR DISCLOSING PERSONAL
2 INFORMATION OF THE BUSINESS’S EMPLOYEES TO THE EXTENT THAT THE BUSINESS
3 IS COLLECTING OR DISCLOSING THE INFORMATION WITHIN THE SCOPE OF ITS ROLE
4 AS AN EMPLOYER;
5 (2) WITH RESPECT TO MEDICAL OR HEALTH INFORMATION THAT IS
6 COLLECTED BY A COVERED ENTITY OR BUSINESS ASSOCIATE GOVERNED BY THE
7 PRIVACY, SECURITY, AND BREACH NOTIFICATION RULES ISSUED BY THE U.S.
8 DEPARTMENT OF HEALTH AND HUMAN SERVICES IN 45 C.F.R. PARTS 160 AND 164,
9 ESTABLISHED IN ACCORDANCE WITH THE FEDERAL HEALTH INSURANCE
10 PORTABILITY AND ACCOUNTABILITY ACT OF 1996 AND THE FEDERAL HEALTH
11 INFORMATION TECHNOLOGY FOR ECONOMIC AND CLINICAL HEALTH ACT;
12 (3) TO A HEALTH CARE PROVIDER OR COVERED ENTITY GOVERNED
13 BY THE PRIVACY, SECURITY, AND BREACH NOTIFICATION RULES ISSUED BY THE U.S.
14 DEPARTMENT OF HEALTH AND HUMAN SERVICES IN 45 C.F.R. PARTS 160 AND 164,
15 ESTABLISHED IN ACCORDANCE WITH THE FEDERAL HEALTH INSURANCE
16 PORTABILITY AND ACCOUNTABILITY ACT OF 1996, TO THE EXTENT THE PROVIDER
17 OR COVERED ENTITY MAINTAINS PATIENT INFORMATION IN THE SAME MANNER AS
18 MEDICAL INFORMATION OR PROTECTED HEALTH INFORMATION AS DESCRIBED IN
19 ITEM (2) OF THIS SECTION;
20 (4) WITH RESPECT TO INFORMATION COLLECTED AS PART OF A
21 CLINICAL TRIAL SUBJECT TO THE FEDERAL POLICY FOR THE PROTECTION OF
22 HUMAN SUBJECTS, ALSO KNOWN AS THE COMMON RULE, IN ACCORDANCE WITH
23 GOOD CLINICAL PRACTICE GUIDELINES ISSUED BY THE INTERNATIONAL COUNCIL
24 FOR HARMONISATION OR IN ACCORDANCE WITH HUMAN SUBJECT PROTECTION
25 REQUIREMENTS OF THE U.S. FOOD AND DRUG ADMINISTRATION;
26 (5) WITH RESPECT TO THE SALE OF PERSONAL INFORMATION TO OR
27 FROM A CONSUMER REPORTING AGENCY IF THAT INFORMATION IS TO BE REPORTED
28 IN, OR USED TO GENERATE, A “CONSUMER REPORT” AS DEFINED BY 15 U.S.C. §
29 1681(A) AND USE OF THAT INFORMATION IS LIMITED BY THE FEDERAL FAIR CREDIT
30 REPORTING ACT;
31 (6) WITH RESPECT TO PERSONAL INFORMATION COLLECTED,
32 PROCESSED, SOLD, OR DISCLOSED UNDER THE FEDERAL GRAMM–LEACH–BLILEY
33 ACT AND IMPLEMENTING REGULATIONS;
34 (7) WITH RESPECT TO PERSONAL INFORMATION COLLECTED,
35 PROCESSED, SOLD, OR DISCLOSED UNDER THE FEDERAL DRIVER’S PRIVACY
36 PROTECTION ACT OF 1994; OR
8 SENATE BILL 11
1 (8) WITH RESPECT TO EDUCATION INFORMATION COVERED BY THE
2 FEDERAL FAMILY EDUCATIONAL RIGHTS AND PRIVACY ACT, 20 U.S.C. § 1232G
3 AND 34 C.F.R. PART 99.
4 14–4403.
5 (A) A BUSINESS THAT COLLECTS A CONSUMER’S PERSONAL INFORMATION
6 SHALL, AT OR BEFORE THE POINT OF COLLECTION, CLEARLY AND CONSPICUOUSLY
7 NOTIFY A CONSUMER OF:
8 (1) THE CATEGORIES OF PERSONAL INFORMATION THE BUSINESS
9 WILL COLLECT ABOUT THAT CONSUMER;
10 (2) THE BUSINESS PURPOSES FOR WHICH THE CATEGORIES OF
11 PERSONAL INFORMATION MAY BE USED;
12 (3) THE CATEGORIES OF THIRD PARTIES TO WHICH THE BUSINESS
13 DISCLOSES PERSONAL INFORMATION;
14 (4) THE BUSINESS PURPOSES FOR THIRD–PARTY DISCLOSURE; AND
15 (5) THE CONSUMER’S RIGHT TO REQUEST:
16 (I) A COPY OF THE CONSUMER’S PERSONAL INFORMATION
17 UNDER § 14–4404 OF THIS SUBTITLE;
18 (II) DELETION OF THE CONSUMER’S PERSONAL INFORMATION
19 UNDER § 14–4406 OF THIS SUBTITLE; AND
20 (III) TO OPT OUT OF THIRD–PARTY DISCLOSURE UNDER §
21 14–4407 OF THIS SUBTITLE.
22 (B) A BUSINESS MAY NOT COLLECT ADDITIONAL CATEGORIES OF PERSONAL
23 INFORMATION OR USE PERSONAL INFORMATION COLLECTED FOR ADDITIONAL
24 PURPOSES WITHOUT FIRST PROVIDING THE CONSUMER WITH NOTICE CONSISTENT
25 WITH THIS SECTION.
26 14–4404.
27 (A) A CONSUMER MAY REQUEST THAT A BUSINESS THAT COLLECTS THE
28 CONSUMER’S PERSONAL INFORMATION DISCLOSE TO THE CONSUMER:
29 (1) THE SPECIFIC PIECES OF PERSONAL INFORMATION THE 
SENATE BILL 11 9
1 BUSINESS HAS COLLECTED ABOUT THE CONSUMER;
2 (2) THE SOURCES FROM WHICH THE CONSUMER’S PERSONAL
3 INFORMATION WAS COLLECTED;
4 (3) THE NAMES OF THIRD PARTIES TO WHICH THE BUSINESS
5 DISCLOSED THE CONSUMER’S PERSONAL INFORMATION; AND
6 (4) THE BUSINESS PURPOSES FOR THIRD–PARTY DISCLOSURE.
7 (B) A BUSINESS SHALL PROVIDE THE INFORMATION SPECIFIED IN
8 SUBSECTION (A) OF THIS SECTION TO A CONSUMER ONLY ON RECEIPT OF A
9 VERIFIABLE CONSUMER REQUEST.
10 (C) (1) SUBJECT TO PARAGRAPH (2) OF THIS SUBSECTION, AFTER
11 RECEIVING A VERIFIABLE CONSUMER REQUEST, A BUSINESS SHALL PROMPTLY
12 TAKE STEPS TO PROVIDE, FREE OF CHARGE TO THE CONSUMER, THE PERSONAL
13 INFORMATION REQUIRED BY THIS SECTION.
14 (2) THE INFORMATION MAY BE PROVIDED BY:
15 (I) UNITED STATES MAIL; OR
16 (II) ELECTRONIC DELIVERY THAT IS PORTABLE AND, TO THE
17 EXTENT TECHNICALLY FEASIBLE, IN A READILY USABLE FORMAT THAT ALLOWS THE
18 CONSUMER TO TRANSMIT THIS INFORMATION TO ANOTHER ENTITY WITHOUT
19 HINDRANCE.
20 (D) NOTWITHSTANDING § 14–4405 OF THIS SUBTITLE, A BUSINESS MAY
21 PROVIDE PERSONAL INFORMATION TO A CONSUMER AT ANY TIME, BUT IS NOT
22 REQUIRED TO PROVIDE PERSONAL INFORMATION TO THE SAME CONSUMER MORE
23 THAN ONCE IN A 6–MONTH PERIOD.
24 (E) IF VERIFIED REQUESTS FROM A CONSUMER ARE EXCESSIVE, BECAUSE
25 OF THEIR REPETITIVE CHARACTER, A BUSINESS MAY:
26 (1) CHARGE A REASONABLE FEE, TAKING INTO ACCOUNT THE
27 ADMINISTRATIVE COSTS OF PROVIDING THE INFORMATION OR COMMUNICATION OR
28 TAKING THE ACTION REQUESTED; OR
29 (2) REFUSE TO ACT ON THE REQUEST AND NOTIFY THE CONSUMER OF
30 THE REASON FOR REFUSING THE REQUEST.
10 SENATE BILL 11
1 (F) A BUSINESS MAY NOT REQUIRE A CONSUMER TO CREATE AN ACCOUNT
2 WITH THE BUSINESS IN ORDER TO MAKE A VERIFIABLE CONSUMER REQUEST.
3 (G) A BUSINESS MAY NOT:
4 (1) RETAIN PERSONAL INFORMATION ABOUT A CONSUMER
5 COLLECTED FROM A SINGLE ONE–TIME TRANSACTION, UNLESS THE BUSINESS
6 REGULARLY RETAINS PERSONAL INFORMATION OF THAT TYPE IN THE ORDINARY
7 COURSE OF BUSINESS;
8 (2) RE–IDENTIFY OR LINK ANY DATA THAT IN THE ORDINARY COURSE
9 OF BUSINESS IS NOT MAINTAINED IN A MANNER THAT WOULD BE CONSIDERED
10 PERSONAL INFORMATION; OR
11 (3) DISCLOSE PERSONAL INFORMATION IF THE DISCLOSURE WOULD
12 ADVERSELY AFFECT THE LEGAL RIGHTS OF OTHER CONSUMERS.
13 14–4405.
14 (A) (1) SUBJECT TO PARAGRAPH (2) OF THIS SUBSECTION, A BUSINESS
15 SHALL, IN A FORM THAT IS REASONABLY ACCESSIBLE TO CONSUMERS, MAKE
16 AVAILABLE TO CONSUMERS TWO OR MORE DESIGNATED METHODS FOR SUBMITTING
17 VERIFIABLE CONSUMER REQUESTS.
18 (2) (I) IF A BUSINESS MAINTAINS AN INTERNET WEBSITE IN
19 CONNECTION WITH THE BUSINESS, THE BUSINESS SHALL MAINTAIN A WEBSITE PAGE
20 THAT MEETS THE REQUIREMENT UNDER PARAGRAPH (1) OF THIS SUBSECTION.
21 (II) A BUSINESS SHALL PROVIDE A TOLL–FREE TELEPHONE
22 NUMBER FOR THE PURPOSE OF ACCEPTING VERIFIABLE CONSUMER REQUESTS
23 UNDER THIS SUBSECTION, UNLESS THE BUSINESS MAINTAINS A DIRECT
24 RELATIONSHIP WITH THE CONSUMER.
25 (B) (1) WITHIN 45 DAYS AFTER RECEIVING A VERIFIABLE CONSUMER
26 REQUEST FROM THE CONSUMER, A BUSINESS SHALL DELIVER TO THE CONSUMER
27 FREE OF CHARGE THE INFORMATION REQUIRED UNDER § 14–4404 OF THIS
28 SUBTITLE IN A READILY USABLE FORMAT THAT ALLOWS THE CONSUMER TO
29 TRANSMIT THE INFORMATION FROM ONE ENTITY TO ANOTHER ENTITY WITHOUT
30 HINDRANCE.
31 (2) THE TIME PERIOD TO PROVIDE THE REQUIRED INFORMATION
32 MAY BE EXTENDED ONCE BY UP TO AN ADDITIONAL 45 DAYS WHEN REASONABLY
33 NECESSARY, IF THE CONSUMER IS PROVIDED NOTICE OF THE EXTENSION WITHIN 
SENATE BILL 11 11
1 THE FIRST 45–DAY PERIOD.
2 (C) A BUSINESS IS NOT REQUIRED TO PROVIDE THE INFORMATION
3 REQUIRED BY § 14–4404 OF THIS SUBTITLE TO THE SAME CONSUMER MORE THAN
4 TWICE IN A 12–MONTH PERIOD.
5 (D) (1) IF A BUSINESS HAS AN ONLINE PRIVACY POLICY, THE BUSINESS
6 SHALL INCLUDE IN THE POLICY:
7 (I) THE CATEGORIES OF PERSONAL INFORMATION THE
8 BUSINESS COLLECTS ABOUT CONSUMERS;
9 (II) THE BUSINESS PURPOSES FOR WHICH THE CATEGORIES OF
10 PERSONAL INFORMATION ARE USED;
11 (III) THE CATEGORIES OF THIRD PARTIES TO WHICH THE
12 BUSINESS DISCLOSES PERSONAL INFORMATION;
13 (IV) THE BUSINESS PURPOSE FOR THIRD–PARTY DISCLOSURE;
14 AND
15 (V) THE CONSUMER’S RIGHT TO REQUEST:
16 1. A COPY OF THE CONSUMER’S PERSONAL
17 INFORMATION IN ACCORDANCE WITH § 14–4404 OF THIS SUBTITLE;
18 2. THE DELETION OF THE CONSUMER’S PERSONAL
19 INFORMATION IN ACCORDANCE WITH § 14–4406 OF THIS SUBTITLE; AND
20 3. TO OPT OUT OF THIRD–PARTY DISCLOSURE IN
21 ACCORDANCE WITH § 14–4407 OF THIS SUBTITLE.
22 (2) IF A BUSINESS DOES NOT HAVE AN ONLINE PRIVACY POLICY BUT
23 DOES HAVE A BUSINESS WEBSITE, THE BUSINESS SHALL:
24 (I) INCLUDE THE INFORMATION REQUIRED UNDER
25 PARAGRAPH (1) OF THIS SUBSECTION ON THE WEBSITE; AND
26 (II) UPDATE THE INFORMATION AT LEAST ONCE EVERY 12
27 MONTHS.
28 (E) A BUSINESS SHALL ENSURE THAT AN INDIVIDUAL RESPONSIBLE FOR
29 HANDLING CONSUMER INQUIRIES ABOUT THE BUSINESS’S PRIVACY PRACTICES OR 
12 SENATE BILL 11
1 THE BUSINESS’S COMPLIANCE WITH THIS SUBTITLE IS INFORMED OF THE
2 REQUIREMENTS IN THIS SUBTITLE AND HOW TO DIRECT A CONSUMER TO EXERCISE
3 THE CONSUMER’S RIGHTS UNDER THIS SUBTITLE.
4 (F) A BUSINESS MAY USE PERSONAL INFORMATION COLLECTED FROM A
5 CONSUMER IN CONNECTION WITH THE BUSINESS’S VERIFICATION OF THE
6 CONSUMER’S REQUEST ONLY FOR THE PURPOSES OF VERIFICATION.
7 14–4406.
8 (A) A CONSUMER MAY REQUEST THAT A BUSINESS DELETE ALL PERSONAL
9 INFORMATION ABOUT THE CONSUMER THAT THE BUSINESS HAS COLLECTED FROM
10 THE CONSUMER.
11 (B) A BUSINESS THAT COLLECTS PERSONAL INFORMATION ABOUT A
12 CONSUMER SHALL DISCLOSE, IN ACCORDANCE WITH § 14–4403 OF THIS SUBTITLE,
13 THE CONSUMER’S RIGHT TO REQUEST THE DELETION OF THE CONSUMER’S
14 PERSONAL INFORMATION.
15 (C) A BUSINESS THAT RECEIVES A VERIFIABLE CONSUMER REQUEST FROM
16 A CONSUMER TO DELETE THE CONSUMER’S PERSONAL INFORMATION UNDER
17 SUBSECTION (A) OF THIS SECTION SHALL DELETE THE PERSONAL INFORMATION
18 FROM ITS RECORDS AND DIRECT SERVICE PROVIDERS TO DELETE THE PERSONAL
19 INFORMATION FROM THE SERVICE PROVIDERS’ RECORDS.
20 (D) A BUSINESS OR A SERVICE PROVIDER IS NOT REQUIRED TO COMPLY
21 WITH A CONSUMER’S REQUEST TO DELETE THE CONSUMER’S PERSONAL
22 INFORMATION IF IT IS NECESSARY FOR THE BUSINESS OR SERVICE PROVIDER TO
23 MAINTAIN THE PERSONAL INFORMATION IN ORDER TO:
24 (1) COMPLETE THE TRANSACTION FOR WHICH THE PERSONAL
25 INFORMATION WAS COLLECTED, PROVIDE A GOOD OR SERVICE REQUESTED BY THE
26 CONSUMER OR REASONABLY ANTICIPATED WITHIN THE CONTEXT OF A BUSINESS’S
27 ONGOING BUSINESS RELATIONSHIP WITH THE CONSUMER, OR OTHERWISE
28 PERFORM A CONTRACT BETWEEN THE BUSINESS AND THE CONSUMER;
29 (2) DETECT SECURITY INCIDENTS, PROTECT AGAINST MALICIOUS,
30 DECEPTIVE, FRAUDULENT, OR ILLEGAL ACTIVITY, OR PROSECUTE THOSE
31 RESPONSIBLE FOR THAT ACTIVITY;
32 (3) IDENTIFY OR REPAIR ERRORS THAT IMPAIR EXISTING INTENDED
33 FUNCTIONALITY;
SENATE BILL 11 13
1 (4) EXERCISE FREE SPEECH, ENSURE THE RIGHT OF ANOTHER
2 CONSUMER TO EXERCISE THE RIGHT OF FREE SPEECH, OR EXERCISE ANOTHER
3 RIGHT PROVIDED FOR BY LAW;
4 (5) ENGAGE IN PUBLIC OR PEER–REVIEWED SCIENTIFIC,
5 HISTORICAL, OR STATISTICAL RESEARCH IN THE PUBLIC INTEREST THAT ADHERES
6 TO OTHER APPLICABLE ETHICS AND PRIVACY LAWS, WHEN THE BUSINESS’S
7 DELETION OF THE INFORMATION IS LIKELY TO RENDER IMPOSSIBLE OR TO
8 SERIOUSLY IMPAIR THE ACHIEVEMENT OF THE RESEARCH, IF THE CONSUMER HAS
9 PROVIDED INFORMED CONSENT; OR
10 (6) COMPLY WITH A LEGAL OBLIGATION.
11 14–4407.
12 (A) (1) A CONSUMER MAY, AT ANY TIME, DEMAND THAT A BUSINESS NOT
13 DISCLOSE THE CONSUMER’S PERSONAL INFORMATION TO THIRD PARTIES.
14 (2) THIS RIGHT MAY BE REFERRED TO AS THE “RIGHT TO OPT OUT OF
15 THIRD–PARTY DISCLOSURE”.
16 (B) NOTWITHSTANDING SUBSECTION (A) OF THIS SECTION, A BUSINESS MAY
17 NOT DISCLOSE THE PERSONAL INFORMATION OF A CONSUMER TO A THIRD PARTY IF
18 THE BUSINESS HAS ACTUAL KNOWLEDGE OR WILLFULLY DISREGARDS THE FACT
19 THAT THE CONSUMER IS UNDER THE AGE OF 16 YEARS.
20 (C) A BUSINESS THAT HAS RECEIVED DIRECTION FROM A CONSUMER NOT
21 TO DISCLOSE THE CONSUMER’S PERSONAL INFORMATION TO THIRD PARTIES MAY
22 NOT:
23 (1) DISCLOSE THE PERSONAL INFORMATION TO THIRD PARTIES
24 UNLESS THE CONSUMER LATER PROVIDES EXPRESS AUTHORIZATION FOR THAT
25 DISCLOSURE; OR
26 (2) REQUEST AUTHORIZATION TO DISCLOSE THE PERSONAL
27 INFORMATION TO THIRD PARTIES FOR AT LEAST 12 MONTHS AFTER THE DATE ON
28 WHICH THE BUSINESS RECEIVED THE DIRECTION FROM THE CONSUMER.
29 (D) A BUSINESS SHALL PROVIDE A CLEAR AND CONSPICUOUS LINK ON THE
30 INTERNET HOMEPAGE OF THE BUSINESS TO AN INTERNET WEBPAGE THAT ENABLES
31 A CONSUMER OR A PERSON AUTHORIZED BY THE CONSUMER TO OPT OUT OF THE
32 THIRD–PARTY DISCLOSURE OF THE CONSUMER’S PERSONAL INFORMATION.
14 SENATE BILL 11
1 (E) (1) A CONSUMER MAY EXERCISE THE RIGHT TO OPT OUT OF THE SALE
2 OR DISCLOSURE OF THE CONSUMER’S PERSONAL INFORMATION THROUGH A
3 TECHNOLOGY INDICATING THE CONSUMER’S INTENT TO OPT OUT, INCLUDING A
4 PREFERENCE OR BROWSER SETTING, BROWSER EXTENSION, OR GLOBAL DEVICE
5 SETTING.
6 (2) A BUSINESS SHALL COMPLY WITH PARAGRAPH (1) OF THIS
7 SUBSECTION IN ACCORDANCE WITH REGULATIONS ADOPTED BY THE ATTORNEY
8 GENERAL.
9 (F) (1) A CONSUMER MAY AUTHORIZE ANOTHER PERSON TO OPT OUT OF
10 THE SALE OR DISCLOSURE OF THE CONSUMER’S PERSONAL INFORMATION ON THE
11 CONSUMER’S BEHALF.
12 (2) A BUSINESS SHALL COMPLY WITH AN OPT–OUT REQUEST
13 RECEIVED FROM A PERSON AUTHORIZED BY THE CONSUMER TO ACT ON THE
14 CONSUMER’S BEHALF, IN ACCORDANCE WITH REGULATIONS ADOPTED BY THE
15 ATTORNEY GENERAL.
16 (G) A BUSINESS MAY REQUIRE AUTHENTICATION OF A CONSUMER REQUEST
17 RECEIVED UNDER THIS SECTION IN A MANNER THAT IS REASONABLE IN LIGHT OF
18 THE NATURE OF THE PERSONAL INFORMATION REQUESTED.
19 (H) A BUSINESS MAY NOT REQUIRE A CONSUMER TO CREATE AN ACCOUNT
20 IN ORDER TO EXERCISE THE RIGHT TO OPT OUT OF THIRD–PARTY DISCLOSURE.
21 14–4408.
22 (A) A BUSINESS MAY NOT DISCRIMINATE AGAINST A CONSUMER BASED ON
23 THE CONSUMER’S DECISION TO EXERCISE RIGHTS UNDER THIS SUBTITLE.
24 (B) FOR PURPOSES OF THIS SECTION, DISCRIMINATION INCLUDES:
25 (1) DENYING GOODS OR SERVICES TO THE CONSUMER;
26 (2) CHARGING DIFFERENT PRICES OR RATES FOR GOODS OR
27 SERVICES, INCLUDING THROUGH THE USE OF DISCOUNTS OR OTHER BENEFITS OR
28 PENALTIES;
29 (3) PROVIDING A DIFFERENT LEVEL OR QUALITY OF GOODS OR
30 SERVICES TO THE CONSUMER; OR
31 (4) SUGGESTING THAT THE CONSUMER WILL RECEIVE A DIFFERENT 
SENATE BILL 11 15
1 PRICE OR RATE FOR GOODS OR SERVICES OR A DIFFERENT LEVEL OR QUALITY OF
2 GOODS OR SERVICES.
3 14–4409.
4 THE OBLIGATIONS IMPOSED BY THIS SUBTITLE MAY NOT RESTRICT THE
5 ABILITY OF A BUSINESS OR THIRD PARTY TO:
6 (1) COMPLY WITH FEDERAL, STATE, OR LOCAL LAWS;
7 (2) COMPLY WITH A CIVIL, CRIMINAL, OR REGULATORY INQUIRY,
8 INVESTIGATION, SUBPOENA, OR SUMMONS BY A FEDERAL, STATE, OR LOCAL
9 AUTHORITY;
10 (3) COOPERATE WITH A LAW ENFORCEMENT AGENCY CONCERNING
11 CONDUCT OR ACTIVITY THAT THE BUSINESS, SERVICE PROVIDER, OR THIRD PARTY
12 REASONABLY AND IN GOOD FAITH BELIEVES MAY VIOLATE FEDERAL, STATE, OR
13 LOCAL LAW;
14 (4) EXERCISE LEGAL RIGHTS OR PRIVILEGES; OR
15 (5) ENGAGE IN NEWS–GATHERING ACTIVITIES PROTECTED BY THE
16 FIRST AMENDMENT OF THE U.S. CONSTITUTION.
17 14–4410.
18 RESEARCH WITH PERSONAL INFORMATION THAT MAY HAVE BEEN
19 COLLECTED FROM A CONSUMER IN THE COURSE OF THE CONSUMER’S
20 INTERACTIONS WITH A BUSINESS’S SERVICE OR DEVICE FOR OTHER PURPOSES
21 SHALL BE:
22 (1) USED SOLELY FOR RESEARCH PURPOSES THAT ARE COMPATIBLE
23 WITH THE CONTEXT IN WHICH THE PERSONAL INFORMATION WAS COLLECTED;
24 (2) RESTRICTED FROM USE FOR ANY COMMERCIAL PURPOSE;
25 (3) SUBSEQUENTLY PSEUDONYMIZED AND DE–IDENTIFIED, OR
26 DE–IDENTIFIED AND IN THE AGGREGATE, SO THAT THE INFORMATION CANNOT
27 REASONABLY IDENTIFY, RELATE TO, DESCRIBE, BE CAPABLE OF BEING ASSOCIATED
28 WITH, OR BE LINKED, DIRECTLY OR INDIRECTLY, TO A PARTICULAR CONSUMER;
29 (4) SUBJECT TO TECHNICAL SAFEGUARDS THAT PROHIBIT
30 RE–IDENTIFICATION OF THE CONSUMER TO WHOM THE INFORMATION MAY 
16 SENATE BILL 11
1 PERTAIN;
2 (5) SUBJECT TO BUSINESS PROCESSES THAT SPECIFICALLY
3 PROHIBIT RE–IDENTIFICATION OF THE INFORMATION;
4 (6) SUBJECT TO BUSINESS PROCESSES TO PREVENT INADVERTENT
5 RELEASE OF DE–IDENTIFIED INFORMATION;
6 (7) PROTECTED FROM ANY RE–IDENTIFICATION ATTEMPTS; AND
7 (8) SUBJECT TO THE ADDITIONAL SECURITY CONTROLS OF THE
8 BUSINESS THAT LIMIT ACCESS TO THE RESEARCH DATA TO ONLY THOSE
9 INDIVIDUALS IN A BUSINESS AS ARE NECESSARY TO CARRY OUT THE RESEARCH
10 PURPOSE.
11 14–4411.
12 (A) A VIOLATION OF THIS SUBTITLE IS:
13 (1) AN UNFAIR, ABUSIVE, OR DECEPTIVE TRADE PRACTICE WITHIN
14 THE MEANING OF TITLE 13 OF THIS ARTICLE; AND
15 (2) SUBJECT TO THE ENFORCEMENT AND PENALTY PROVISIONS
16 CONTAINED IN TITLE 13 OF THIS ARTICLE.
17 (B) (1) A BUSINESS THAT DISCLOSES PERSONAL INFORMATION TO A
18 SERVICE PROVIDER MAY NOT BE LIABLE UNDER THIS SUBTITLE IF:
19 (I) THE SERVICE PROVIDER RECEIVING THE PERSONAL
20 INFORMATION USES THE PERSONAL INFORMATION IN VIOLATION OF THE
21 RESTRICTIONS SET FORTH IN THIS SUBTITLE; AND
22 (II) AT THE TIME OF THE DISCLOSURE, THE BUSINESS DOES NOT
23 HAVE ACTUAL KNOWLEDGE OR REASON TO BELIEVE THAT THE SERVICE PROVIDER
24 INTENDS TO COMMIT A VIOLATION.
25 (2) A SERVICE PROVIDER MAY NOT BE LIABLE UNDER THIS SUBTITLE
26 FOR THE OBLIGATIONS OF A BUSINESS FOR WHICH IT PROVIDES SERVICES AS SET
27 FORTH IN THIS SUBTITLE.
28 14–4412.
29 THE OFFICE OF THE ATTORNEY GENERAL MAY ADOPT REGULATIONS 
SENATE BILL 11 17
1 NECESSARY TO CARRY OUT THIS SUBTITLE, INCLUDING REGULATIONS TO:
2 (1) IDENTIFY CATEGORIES OF PERSONAL INFORMATION IN ADDITION
3 TO THOSE DESCRIBED UNDER § 14–4402 OF THIS SUBTITLE IN ORDER TO ADDRESS
4 CHANGES IN TECHNOLOGY, DATA COLLECTION PRACTICES, OBSTACLES TO
5 IMPLEMENTATION, AND PRIVACY CONCERNS;
6 (2) UPDATE AS NEEDED THE DEFINITION OF “UNIQUE IDENTIFIER”
7 TO ADDRESS CHANGES IN TECHNOLOGY, DATA COLLECTION, OBSTACLES TO
8 IMPLEMENTATION, AND PRIVACY CONCERNS;
9 (3) ESTABLISH ANY EXCEPTIONS NECESSARY TO COMPLY WITH
10 STATE OR FEDERAL LAW, INCLUDING EXCEPTIONS RELATING TO TRADE SECRETS
11 AND INTELLECTUAL PROPERTY RIGHTS;
12 (4) ADOPT STANDARDS AND PROCEDURES:
13 (I) TO FACILITATE AND GOVERN THE SUBMISSION OF
14 VERIFIABLE CONSUMER REQUESTS UNDER §§ 14–4404 THROUGH 14–4407 OF THIS
15 SUBTITLE;
16 (II) TO GOVERN RESPONSES BY BUSINESSES AND SERVICE
17 PROVIDERS TO VERIFIABLE CONSUMER REQUESTS UNDER §§ 14–4404 THROUGH
18 14–4407 OF THIS SUBTITLE; AND
19 (III) FOR THE DEVELOPMENT AND USE OF A RECOGNIZABLE AND
20 UNIFORM OPT–OUT LOGO OR BUTTON BY ALL BUSINESSES TO PROMOTE CONSUMER
21 AWARENESS OF THE OPPORTUNITY TO OPT OUT OF THIRD–PARTY DISCLOSURE OF
22 CONSUMER PERSONAL INFORMATION;
23 (5) ADJUST THE MONETARY THRESHOLD IN § 14–4401(D)(1)(III)1 OF
24 THIS SUBTITLE TO REFLECT ANY INCREASE IN THE CONSUMER PRICE INDEX AS
25 PUBLISHED BY THE UNITED STATES BUREAU OF LABOR STATISTICS;
26 (6) ENSURE THAT THE NOTICES AND INFORMATION THAT
27 BUSINESSES ARE REQUIRED TO PROVIDE UNDER THIS SUBTITLE ARE PROVIDED IN
28 A MANNER THAT MAY BE EASILY UNDERSTOOD BY THE AVERAGE CONSUMER, ARE
29 ACCESSIBLE TO CONSUMERS WITH DISABILITIES, AND ARE AVAILABLE IN THE
30 LANGUAGE PRIMARILY USED TO INTERACT WITH THE CONSUMER, INCLUDING
31 ADOPTING REGULATIONS, PROCEDURES, AND GUIDELINES REGARDING FINANCIAL
32 INCENTIVE OFFERINGS; AND
33 (7) FURTHER THE PURPOSES OF §§ 14–4404 THROUGH 14–4407 OF 
18 SENATE BILL 11
1 THIS SUBTITLE, WITH THE GOAL OF MINIMIZING THE ADMINISTRATIVE BURDEN ON
2 CONSUMERS, TAKING INTO ACCOUNT AVAILABLE TECHNOLOGY, SECURITY
3 CONCERNS, AND THE BURDEN ON THE BUSINESS, TO GOVERN A DETERMINATION BY
4 A BUSINESS THAT A REQUEST FOR INFORMATION RECEIVED BY A CONSUMER IS A
5 VERIFIABLE CONSUMER REQUEST, INCLUDING TREATING A REQUEST SUBMITTED
6 THROUGH A PASSWORD–PROTECTED ACCOUNT MAINTAINED BY THE CONSUMER
7 WITH THE BUSINESS WHILE THE CONSUMER IS LOGGED INTO THE ACCOUNT AS A
8 VERIFIABLE CONSUMER REQUEST AND PROVIDING A MECHANISM FOR A CONSUMER
9 WHO DOES NOT MAINTAIN AN ACCOUNT WITH THE BUSINESS TO REQUEST
10 INFORMATION THROUGH THE BUSINESS’S AUTHENTICATION OF THE CONSUMER’S
11 IDENTITY.
12 14–4413.
13 (A) WHEREVER POSSIBLE, LAW RELATING TO CONSUMERS’ PERSONAL
14 INFORMATION SHOULD BE CONSTRUED TO HARMONIZE WITH THE PROVISIONS OF
15 THIS SUBTITLE.
16 (B) IN THE EVENT OF A CONFLICT BETWEEN OTHER LAWS AND THIS
17 SUBTITLE, THE PROVISIONS OF LAW THAT AFFORD THE GREATEST PROTECTION FOR
18 THE RIGHT OF PRIVACY FOR CONSUMERS SHALL CONTROL.
19 14–4414.
20 IF A SERIES OF STEPS OR TRANSACTIONS IS ENGAGED WHERE COMPONENT
21 PARTS OF A SINGLE TRANSACTION ARE TAKEN WITH THE INTENT OF AVOIDING THE
22 REQUIREMENTS OF THIS SUBTITLE, A COURT SHALL DISREGARD THE
23 INTERMEDIATE STEPS OR TRANSACTIONS FOR PURPOSES OF CARRYING OUT THIS
24 SUBTITLE.
25 14–4415.
26 A PROVISION OF A CONTRACT OR AN AGREEMENT OF ANY KIND THAT
27 PURPORTS TO WAIVE OR LIMIT IN ANY WAY THE RIGHTS OF A CONSUMER UNDER
28 THIS SUBTITLE, INCLUDING A RIGHT TO A REMEDY OR MEANS OF ENFORCEMENT,
29 SHALL BE CONSIDERED CONTRARY TO PUBLIC POLICY AND SHALL BE VOID AND
30 UNENFORCEABLE.
31 SECTION 2. AND BE IT FURTHER ENACTED, That this Act shall take effect
32 January 1, 2023.