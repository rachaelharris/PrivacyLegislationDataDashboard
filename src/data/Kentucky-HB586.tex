UNOFFICIAL COPY 21 RS BR 1691
Page 1 of 68
XXXX Jacketed
1 AN ACT relating to elections.
2 Be it enacted by the General Assembly of the Commonwealth of Kentucky:
3 Section 1. KRS 116.013 is amended to read as follows:
4 As used in this chapter unless context otherwise requires:
5 (1) [Unless the context otherwise requires,] The word "voter" means any name
6 contained in any registration list;
7 (2) The word "election" means any primary, runoff primary, regular election, or
8 special election.
9 Section 2. KRS 116.055 is amended to read as follows:
10 (1) Before a person shall be qualified to vote in a primary, he or she:
11 (a) Shall possess all the qualifications required of voters in a regular election;
12 (b) Shall have been a registered member of the party in whose primary he or she
13 seeks to vote on December 31 immediately preceding the primary; and
14 (c) Shall have remained continuously registered as a member of that party in
15 whose primary he or she seeks to vote between December 31 immediately
16 preceding the primary and the date set for the primary.
17 (2) In the case of a new registration made after December 31 immediately preceding the
18 primary, a voter shall have registered and remained continuously registered as a
19 member of the party in whose primary he or she seeks to vote from the date of
20 registration until the date set for the primary.
21 (3) Any voter who withdraws his or her registration after December 31 immediately
22 preceding the primary, and reregisters as a voter with a different party affiliation,
23 during those periods that the registration books are open immediately preceding the
24 primary, shall not be eligible to vote in the upcoming primary.
25 (4) No person shall be allowed to vote for any party candidates or slates of candidates
26 other than that of the party of which he or she is a registered member.
27 (5) The qualifications shall be determined as of the date of the primary, without regard 
UNOFFICIAL COPY 21 RS BR 1691
Page 2 of 68
XXXX Jacketed
1 to the qualifications or disqualifications as they may exist at the succeeding regular
2 election, except that minors seventeen (17) years of age who will become eighteen
3 (18) years of age on or before the day of the regular election shall be entitled to vote
4 in the primary if otherwise qualified. However, any registered voter, whether
5 registered as a member of a party, political organization, political group, or as an
6 independent, shall be qualified to vote in a primary for candidates listed in all
7 nonpartisan races. Any voter eligible to vote in a primary shall also be eligible to
8 vote in a subsequent runoff primary if one (1) shall be necessary.
9 SECTION 3. A NEW SECTION OF KRS CHAPTER 117 IS CREATED TO
10 READ AS FOLLOWS:
11 As used in this chapter, unless the context otherwise requires, the word "election"
12 means any primary, runoff primary, regular election, or special election.
13 Section 4. KRS 117.085 is amended to read as follows:
14 (1) All requests for an application for a mail-in absentee ballot may be transmitted by
15 telephone, facsimile machine, by mail, by electronic mail, or in person. The county
16 clerk shall transmit all applications for a mail-in absentee ballot to the voter by
17 mail, electronic mail, or in person at the option of the voter, except as provided in
18 paragraph (b) of this subsection. Except as otherwise provided in KRS 117.077, the
19 mail-in absentee ballot application may be requested by the voter or the spouse,
20 parents, or children of the voter, but shall be restricted to the use of the voter.
21 (a) Except as otherwise provided in KRS 117.077, a qualified voter may apply to
22 cast his or her vote by mail-in absentee ballot if the completed application is
23 received not later than the close of business hours seven (7) days before the
24 election, and if the voter is:
25 1. A resident of Kentucky who is a covered voter as defined in KRS
26 117A.010;
27 2. A student who temporarily resides outside the county of his or her 
UNOFFICIAL COPY 21 RS BR 1691
Page 3 of 68
XXXX Jacketed
1 residence;
2 3. Incarcerated in jail and charged with a crime, but has not been convicted
3 of the crime;
4 4. Changing or has changed his or her place of residence to a different state
5 while the registration books are closed in the new state of residence
6 before an election of electors for President and Vice President of the
7 United States, in which case the voter shall be permitted to cast a mail-in
8 absentee ballot for electors for President and Vice President of the
9 United States only;
10 5. Temporarily residing outside the state but still eligible to vote in this
11 state;
12 6. Prevented from voting in person at the polls on election day and from
13 casting an in-person absentee ballot[ in the county clerk's office] on all
14 days in-person absentee voting is conducted because his or her
15 employment location requires him or her to be absent from the county of
16 his or her residence all hours and all days in-person absentee voting is
17 conducted[ in the county clerk's office];
18 7. A participant in the Secretary of State's crime victim address
19 confidentiality protection program as authorized by KRS 14.312; or
20 8. Not able to appear at the polls on election day or during the dates and
21 times in-person absentee voting is being conducted due to[on the
22 account of] age, disability, or illness, and who has not been declared
23 mentally disabled by a court of competent jurisdiction.
24 (b) Residents of Kentucky who are covered voters as defined in KRS 117A.010
25 may apply for a mail-in absentee ballot by means of the federal post-card
26 application, which may be transmitted to the county clerk's office by mail, by
27 facsimile machine, or by means of the electronic transmission system 
UNOFFICIAL COPY 21 RS BR 1691
Page 4 of 68
XXXX Jacketed
1 established under KRS 117A.030(4). The federal post-card application may be
2 used to register, reregister, and to apply for a mail-in absentee ballot. If the
3 federal post-card application is received at any time not less than seven (7)
4 days before the election, the county clerk shall affix his or her seal to the
5 application form upon receipt.
6 (c) In-person absentee voting shall be conducted in the county clerk's office or
7 other place designated by the county board of elections and approved by the
8 State Board of Elections[ during normal business hours] for at least the six
9 (6)[twelve (12)] working days and two (2) Saturdays before the election. A
10 county board of elections may permit in-person absentee voting to be
11 conducted on a voting machine for a period longer than the six (6)[twelve
12 (12)] working days and two Saturdays before the election. In-person
13 absentee voting shall begin at 8 a.m., and continue for be no less than seven
14 (7) hours, but no more than nine (9) hours each day in-person absentee
15 voting is being conducted.
16 (d) A qualified voter may choose, at any time[ during normal business hours on
17 those days] in-person absentee voting is conducted,[ in the county clerk's
18 office, make application in person to the county clerk] to vote on a voting
19 machine in the county clerk's office or other place designated by the county
20 board of elections and approved by the State Board of Elections, if the voter
21 provides proof of identification as defined in KRS 117.375 or meets the
22 requirements of KRS 117.228 and 117.229[, and the voter:
23 1. Is a resident of Kentucky who is a covered voter as defined in KRS
24 117A.010, who will be absent from the county of his or her residence on
25 any election day;
26 2. Is a student who temporarily resides outside the county of his or her
27 residence;
UNOFFICIAL COPY 21 RS BR 1691
Page 5 of 68
XXXX Jacketed
1 3. Has surgery, or whose spouse has surgery, scheduled that will require
2 hospitalization on election day;
3 4. Temporarily resides outside the state, but is still eligible to vote in this
4 state and will be absent from the county of his or her residence on any
5 election day;
6 5. Is a resident of Kentucky who is a uniformed-service voter as defined in
7 KRS 117A.010 confined to a military base on election day, learns of that
8 confinement within seven (7) days or less of an election, and is not
9 eligible for a mail-in absentee ballot under this subsection;
10 6. Is in her last trimester of pregnancy at the time she wishes to vote under
11 this paragraph. The application form for a voter under this subparagraph
12 shall be prescribed by the State Board of Elections, which shall contain
13 the woman's sworn statement that she is in fact in her last trimester of
14 pregnancy at the time she wishes to vote;
15 7. Has not been declared mentally disabled by a court of competent
16 jurisdiction and, on account of age, disability, or illness, is not able to
17 appear at the polls on election day; or
18 8. Is not permitted to vote by a mail-in absentee ballot under paragraph (a)
19 of this subsection, but who will be absent from the county of his or her
20 residence on election day].
21 (e) Voters who change their place of residence to a different state while the
22 registration books are closed in the new state of residence before a presidential
23 election shall be permitted to cast an in-person absentee ballot for President
24 and Vice President only[, by making application in person to the county clerk
25 to vote on a voting machine in the county clerk's office or other place
26 designated by the county board of elections and approved by the State Board
27 of Elections,] up to the close of normal business hours on the day before the 
UNOFFICIAL COPY 21 RS BR 1691
Page 6 of 68
XXXX Jacketed
1 primary or election.
2 (f) [Any member of the county board of elections, any precinct election officer
3 appointed to serve in a precinct other than that in which he or she is registered,
4 any alternate precinct election officer, any deputy county clerk, any staff for
5 the State Board of Elections, and any staff for the county board of elections
6 may vote on a voting machine in the county clerk's office or other place
7 designated by the county board of elections, and approved by the State Board
8 of Elections, up to the close of normal business hours on the day before the
9 election. The application form for those persons shall be prescribed by the
10 State Board of Elections and, in the case of application by precinct election
11 officers, shall contain a verification of appointment signed by a member of the
12 county board of elections. If an alternate precinct election officer or a precinct
13 election officer appointed to serve in a precinct other than that in which he or
14 she is registered receives his or her appointment while in-person absentee
15 voting is being conducted in the county, the officer may vote on a voting
16 machine in the county clerk's office or other place designated by the county
17 board of elections, and approved by the State Board of Elections, up to the
18 close of normal business hours on the day before the election. Precinct
19 election officers' verification of appointment shall also contain the date of
20 appointment. The applications shall be restricted to the use of the voter only.
21 (g) ]The members of the county board of elections or their designees who provide
22 equal representation of both political parties may serve as precinct election
23 officers, without compensation, for all in-person absentee voting[ performed
24 on a voting machine in the county clerk's office or other place designated by
25 the county board of elections and approved by the State Board of Elections]. If
26 the members of the county board of elections or their designees serve as
27 precinct election officers for the in-person absentee voting, they shall perform 
UNOFFICIAL COPY 21 RS BR 1691
Page 7 of 68
XXXX Jacketed
1 the same duties and exercise the same authority as precinct election officers
2 who serve on the day of an election. If the members of the county board of
3 elections or their designees do not serve as precinct election officers for in4 person absentee voting, the county clerk or deputy county clerks shall
5 supervise the in-person absentee voting.
6 (g)[(h)] Any individual qualified to appoint challengers for the day of an election
7 may also appoint challengers to observe all in-person absentee voting[
8 performed at the county clerk's office or other place designated by the county
9 board of elections, and approved by the State Board of Elections], and those
10 challengers may exercise the same privileges as challengers appointed for
11 observing voting on the day of a primary or an election[ at a regular polling
12 place].
13 (2) The county clerk shall type the name of the voter permitted to vote by mail-in
14 absentee ballot on the mail-in absentee ballot application form for that person's use
15 and no other. The mail-in absentee ballot application form shall be in the form
16 prescribed by the State Board of Elections, which shall include the voter affirmation
17 form as prescribed in KRS 117.228(1)(c), shall bear the seal of the county clerk, and
18 shall contain the following information: name, residential address, precinct, party
19 affiliation, statement of the reason the person cannot vote in person on election day
20 or during the dates and time in-person absentee voting is being conducted,
21 statement of where the voter shall be on election day or during the dates and times
22 in-person absentee voting is being conducted, statement of compliance with
23 residency requirements for voting in the precinct, an instructional statement
24 prescribing the requirements for providing a copy of the voter's proof of
25 identification or voter affirmation when applicable, and the voter's mailing address
26 for a mail-in absentee ballot. The mail-in absentee ballot application form shall be
27 verified and signed by the voter, and the voter shall provide a copy of his or her 
UNOFFICIAL COPY 21 RS BR 1691
Page 8 of 68
XXXX Jacketed
1 proof of identification, as defined in KRS 117.375, or the executed voter
2 affirmation as described in KRS 117.228(1)(c). A notice of the actual penalty
3 provisions in KRS 117.995(2) and (5) shall be printed on the mail-in absentee ballot
4 application form.
5 (3) (a) If the county clerk finds that the voter is properly registered as stated in his or
6 her mail-in absentee ballot application form and qualifies to receive a mail-in
7 absentee ballot[ by mail], he or she shall mail to the voter a mail-in absentee
8 ballot, two (2) official envelopes for returning the mail-in absentee ballot, and
9 instructions for voting.
10 (b) The county clerk shall complete a postal form for a certificate of mailing for
11 mail-in absentee ballots mailed within the fifty (50) states, and it shall be
12 stamped by the postal service when the mail-in absentee ballots are mailed. A
13 mail-in absentee ballot may be transmitted by facsimile machine or by the
14 electronic transmission system established under KRS 117A.030(4) to a
15 covered voter as defined in KRS 117A.010. The covered voter shall be
16 notified of the options for transmittal of the mail-in absentee ballot, and the
17 mail-in absentee ballot shall be transmitted by the method chosen for receipt
18 by the resident of Kentucky who is a covered voter.
19 (4) Mail-in absentee ballots which are requested prior to the printing of the mail-in
20 absentee ballots shall be mailed or otherwise transmitted as provided in subsection
21 (3) of this section by the county clerk to the voter within three (3) days of the receipt
22 of the printed ballots. Mail-in absentee ballots requested after the receipt of the
23 ballots by the county clerk shall be mailed or otherwise transmitted as provided in
24 subsection (3) of this section to the voter within three (3) days of the receipt of the
25 request.
26 (5) The county clerk shall cause mail-in absentee ballots to be printed:
27 (a) Fifty (50) days prior to each primary or regular election;[, and ]
UNOFFICIAL COPY 21 RS BR 1691
Page 9 of 68
XXXX Jacketed
1 (b) Forty-five (45) days prior to a special election; and
2 (c) Fifteen (15) days prior to a runoff primary.
3 (6) The mail-in absentee ballot outer envelope shall bear the words "Absentee Ballot"
4 and the address and official title of the county clerk and shall provide space for the
5 voter's signature, voting address, precinct number, and signatures of two (2)
6 witnesses if the voter signs the form with the use of a mark instead of the voter's
7 signature. A detachable flap on the secrecy envelope shall provide space for the
8 voter's signature, voting address, precinct number, signatures of two (2) witnesses if
9 the voter signs the form with the use of a mark instead of the voter's signature and
10 notice of penalty provided in KRS 117.995(5). The county clerk shall type the
11 voter's address and precinct number in the upper left hand corner of the outer
12 envelope and of the detachable flap on the secrecy envelope immediately below the
13 blank space for the voter's signature. The secrecy envelope shall be blank. The
14 county clerk shall retain the voter's mail-in ballot application form, which shall
15 include the photographed copy of the voter's proof of identification or the voter
16 affirmation as prescribed by KRS 117.228(1)(c), and the postal form required by
17 subsection (3) of this section for twenty-two (22) months after the primary or
18 election.
19 (7) Except as provided in subsection (9) of this section, any person who has received a
20 mail-in absentee ballot[ by mail but who knows at least seven (7) days before the
21 date of the election that he or she will be in his or her county of residence on
22 election day and] who has not voted pursuant to the mail-in absentee ballot
23 provisions of KRS 117.086, and who elects to vote in person on election day or
24 during the times in-person absentee voting is being conducted, shall cancel his or
25 her mail-in absentee ballot and vote in person. The voter shall return the mail-in
26 absentee ballot to the county clerk's office on or before the day the voter votes in27 person, but no later than seven (7) days prior to the date of the election. Upon the 
UNOFFICIAL COPY 21 RS BR 1691
Page 10 of 68
XXXX Jacketed
1 return of the mail-in absentee ballot, the county clerk shall mark on the outer
2 envelope of the sealed ballot or the unmarked ballot the words "Canceled because
3 voter appeared to vote in person." Sealed envelopes so marked shall not be opened.
4 The county clerk shall remove the voter's name from the list of persons who were
5 sent mail-in absentee ballots, and the voter may vote in-person absentee in the
6 precinct in which he or she is properly registered.
7 (8) Any voter qualified for a mail-in absentee ballot who does not receive a requested
8 mail-in absentee ballot within a reasonable amount of time shall contact the county
9 clerk, who shall reissue a second mail-in absentee ballot. The county clerk shall
10 keep a record of the mail-in absentee ballots issued and returned by mail, the in11 person absentee voting and federal in-person provisional absentee voting that is
12 performed[ on the voting machine in the county clerk's office or other place
13 designated by the county board of elections and approved by the State Board of
14 Elections,] to verify that only the first voted ballot to be returned by the voter is
15 counted. Upon the return of any mail-in absentee ballot after the first mail-in
16 absentee ballot is returned, the county clerk shall mark on the outer envelope of the
17 sealed ballot the words "Canceled because ballot reissued."
18 (9) Any covered voter as defined in KRS 117A.010 who has received a mail-in
19 absentee ballot but who knows that he or she will be in the county on election day
20 or during the dates and times in-person absentee voting is being conducted, and
21 who has not voted pursuant to the provisions of KRS 117.086, shall cancel his or
22 her mail-in absentee ballot and vote in person. The voter shall return the mail-in
23 absentee ballot to the county clerk's office on or before the day the voter votes in
24 person[election day]. Upon the return of the mail-in absentee ballot, the county
25 clerk shall mark on the outer envelope of the sealed mail-in absentee ballot or the
26 unmarked mail-in absentee ballot the words "Canceled because voter appeared to
27 vote in person." Sealed envelopes so marked shall not be opened. If the covered 
UNOFFICIAL COPY 21 RS BR 1691
Page 11 of 68
XXXX Jacketed
1 voter is unable to return the mail-in absentee ballot to the county clerk's office on or
2 before the day the voter[election day, at the time he or she] votes in person, he or
3 she shall sign a written oath as to his or her qualifications on the form prescribed by
4 the State Board of Elections pursuant to KRS 117.245. The county clerk shall
5 remove the voter's name from the list of persons who were sent mail-in absentee
6 ballots and[,] provide the voter with written authorization to vote[ at the precinct,
7 and the voter may vote in the precinct in which he or she is properly registered].
8 (10) Notwithstanding the provisions of the Kentucky Open Records Act, KRS 61.870 to
9 61.884, the information contained in an application for a mail-in absentee ballot
10 shall not be made public until after the close of business hours on the election day
11 for which the application applies. This subsection shall not prohibit at any time the
12 disclosure, upon request, of the total number of applications for mail-in absentee
13 ballots that have been filed, or the disclosure to the Secretary of State or the State
14 Board of Elections, if requested or if otherwise required by law, of any information
15 in an application for a mail-in absentee ballot.
16 Section 5. KRS 117.066 is amended to read as follows:
17 (1) In the case of a precinct comprised of a small number of registered voters, the
18 county board of elections may, pursuant to KRS 117.055, utilize the facilities of
19 another precinct as a voting location. Additionally, the county board of elections
20 may petition the State Board of Elections to allow the precinct election officers of
21 the larger precinct to serve as precinct election officers for the precinct that is the
22 subject of the petition. The petition shall designate both the smaller precinct and the
23 larger precinct with which it is to be included, the type of voting machine or
24 machines to be used, and whether supplemental paper ballots are to be used. The
25 petition shall contain a full explanation of the reasons why inclusion is desirable.
26 (2) If the petition submitted pursuant to subsection (1) of this section is approved by the
27 State Board of Elections, the election shall be conducted according to the following 
UNOFFICIAL COPY 21 RS BR 1691
Page 12 of 68
XXXX Jacketed
1 provisions:
2 (a) One voting machine may be utilized for both precincts if the State Board of
3 Elections certifies that separate ballots may be placed upon the voting
4 machine to be used without endangering the integrity of the ballots or without
5 violating any other election law. Otherwise, separate voting machines shall be
6 used for each precinct. In the instance of a precinct which has a small number
7 of voters such that the use of a separate voting machine would be cost8 prohibitive, the county clerk may make application to the State Board of
9 Elections to use supplemental paper ballots under KRS 118.215 to conduct the
10 voting for the small precinct on election day. If the use of supplemental paper
11 ballots is approved by the State Board of Elections, at the close of voting on
12 election day, the locked supplemental paper ballot box shall be transported to
13 the county board of elections along with the federal provisional ballot
14 receptacle, and ballots shall be counted by the county board of elections as
15 provided by subsections 11 to 15 of Section 10 of this Act[KRS 117.275(10)
16 to (14)];
17 (b) Separate precinct voter rosters shall be maintained for each precinct, and steps
18 shall be taken to insure that voters cast their ballot in their duly authorized
19 precinct; and
20 (c) A separate set of elections forms and reports required by this chapter and the
21 State Board of Elections shall be maintained for each precinct.
22 Section 6. KRS 117.086 is amended to read as follows:
23 (1) (a) The voter returning his or her absentee ballot by mail shall mark his or her
24 ballot, seal it in the secrecy envelope, and then seal the outer envelope, and
25 mail it to the county clerk as provided in this chapter.
26 (b) The voter shall sign the detachable flap and the outer envelope in order to
27 validate the ballot. A person having power of attorney for the voter and who 
UNOFFICIAL COPY 21 RS BR 1691
Page 13 of 68
XXXX Jacketed
1 signs the detachable flap and outer envelope for the voter shall complete the
2 voter assistance form as required by KRS 117.255. The signatures of two (2)
3 witnesses are required if the voter signs the form with the use of a mark
4 instead of the voter's signature. A resident of Kentucky who is a covered voter
5 as defined in KRS 117A.010 who has received an absentee ballot transmitted
6 by facsimile machine or by means of the electronic transmission system
7 established under KRS 117A.030(4) shall transmit the voted ballot to the
8 county clerk by mail only, conforming with ballot security requirements that
9 may be promulgated by the State Board of Elections by administrative
10 regulation under KRS Chapter 13A. In order to be counted, the ballots shall be
11 received by the county clerk by at least the time established by the election
12 laws generally for the closing of the polls, which time shall not include the
13 extra hour during which those voters may vote who were waiting in line to
14 vote at the scheduled poll closing time.
15 (2) [Any voter who shall be absent from the county on election day, but who does not
16 qualify to receive a mail-in absentee ballot under the provisions of KRS 117.085,
17 and ]All voters qualified to vote in-person absentee prior to the election under the
18 provisions of KRS 117.085, shall vote at the main office of the county clerk or other
19 place designated by the county board of elections, and approved by the State Board
20 of Elections, prior to the day of election. The county clerk may provide for such
21 voting by the voting equipment in general use in the county[ either at the precinct,
22 the equipment as may be used to tabulate absentee ballots,] or any other voting
23 equipment approved by the State Board of Elections for use in Kentucky, except as
24 follows:
25 (a) Any voter qualifying to vote[ in the county clerk's office or other place
26 designated by the county board of elections, and approved by the State Board
27 of Elections,] who receives assistance to vote shall complete the voter 
UNOFFICIAL COPY 21 RS BR 1691
Page 14 of 68
XXXX Jacketed
1 assistance form required by KRS 117.255;
2 (b) Any voter qualifying to vote[ in the county clerk's office or other place
3 designated by the county board of elections, and approved by the State Board
4 of Elections,] whose qualifications are challenged on grounds other than
5 inability to provide proof of identification by any clerk or deputy shall
6 complete an "Oath of Voter" affidavit; and
7 (c) Any voter qualifying to vote[ in the county clerk's office or other place
8 designated by the county board of elections and approved by the State Board
9 of Elections,] who is unable to provide proof of identification as defined in
10 KRS 117.375, may cast an in-person absentee ballot or federal provisional in11 person absentee ballot in accordance with KRS 117.228 or 117.229.
12 (3) When the county clerk uses general voting equipment as provided for in subsection
13 (2) of this section, each voter casting his or her vote[ at the county clerk's office or
14 other place designated by the county board of elections, and approved by the State
15 Board of Elections,] shall sign an "In-Person Absentee Ballot Signature Roster."
16 (4) The county clerk shall designate a location within his or her office where the in17 person absentee ballots shall be cast secretly. The county clerk, with the approval
18 of the State Board of Elections, may establish locations other than his or her main
19 office in which the voters may execute their in-person absentee ballots. Public
20 notice of the locations shall be given pursuant to KRS Chapter 424, and similar
21 notice by mail shall be given to the county chairs of the two (2) political parties
22 whose candidates polled the largest number of votes in the county at the last regular
23 election.
24 (5) The State Board of Elections shall promulgate administrative regulations under
25 KRS Chapter 13A to provide for casting ballots in accordance with subsection (2)
26 of this section.
27 (6) The county clerk shall deposit all of the mail-in absentee ballots in a locked ballot 
UNOFFICIAL COPY 21 RS BR 1691
Page 15 of 68
XXXX Jacketed
1 box immediately upon receipt without opening the outer envelope. The ballot box
2 shall be locked with three (3) locks. The keys to the box shall be retained by the
3 three (3) members of the central absentee ballot counting board, if one is appointed,
4 or by the members of the board of elections, and the box shall remain locked until
5 the ballots are counted. All voting equipment on which ballots are cast as permitted
6 in subsection (2) of this section shall also remain locked and the keys shall be
7 retained by the three (3) members of the central absentee ballot counting board, if
8 one is appointed, or by the members of the board of elections, and the equipment
9 shall remain locked until the ballots are counted.
10 (7) The county clerk shall keep separate lists for each election of all persons who:
11 (a) Return their absentee ballots by mail;
12 (b) Cast their absentee ballots in-person[in the county clerk's office or other place
13 designated by the county board of elections and approved by the State Board
14 of Elections]; and
15 (c) Cast their federal provisional in-person absentee ballots under subsection
16 (2)(c) of this section.
17 The county clerk shall send a copy of each list to the State Board of Elections after
18 any primary or election day. Notwithstanding the provisions of the Kentucky Open
19 Records Act, KRS 61.870 to 61.884, each list of all persons who return their
20 absentee ballots by mail or who cast their in-person absentee ballots [in the clerk's
21 office or other designated and approved place] shall not be made public until after
22 the close of business hours on the primary or election day for which the list applies.
23 The county clerk and the Secretary of State shall keep a record of the number of
24 votes cast by each method listed in paragraphs (a) to (c) of this subsection, which
25 are cast in any primary or election as a part of the official returns of the primary or
26 election.
27 (8) The county board of elections shall report to the State Board of Elections within ten 
UNOFFICIAL COPY 21 RS BR 1691
Page 16 of 68
XXXX Jacketed
1 (10) days after any primary or regular election as to the number of rejected absentee
2 ballots, including rejected mail-in absentee ballots and ballots cast under subsection
3 (2) of this section, and the reasons for rejecting the ballots on a form prescribed and
4 furnished by the State Board of Elections in administrative regulations promulgated
5 under KRS Chapter 13A.
6 Section 7. KRS 117.087 is amended to read as follows:
7 (1) The challenge of an absentee ballot returned by mail shall be in writing and in the
8 hands of the county clerk before 8 a.m. on election day.
9 (2) The county board of elections shall count the absentee ballots returned by mail and
10 the in-person absentee votes cast[ on the voting machine in the county clerk's office
11 or other place designated by the county board of elections and approved by the State
12 Board of Elections]. Federal provisional in-person absentee ballots shall be
13 processed in accordance with KRS 117.229. The board may appoint a central ballot
14 counting board of not less than three (3) members, who shall be qualified voters and
15 no more than two-thirds (2/3) of whom shall be members of the same political
16 party, to count the ballots at the direction of the county board of elections.
17 (3) Beginning at 8 a.m. on election day, the board shall meet at the county clerk's office
18 to count the absentee ballots returned by mail and the in-person absentee ballots
19 cast[ on the voting machine in the county clerk's office or other place designated by
20 the county board of elections and approved by the State Board of Elections].
21 Candidates or their representatives shall be permitted to be present. The county
22 board of elections shall authorize representatives of the news media to observe the
23 counting of the ballots. The board shall open the boxes containing absentee ballots
24 returned by mail and remove the envelopes one (1) at a time. As each envelope is
25 removed, it shall be examined to ascertain whether the outer envelope and the
26 detachable flap are in proper order and have been signed by the voter. A person
27 having power of attorney for the voter and who signs the detachable flap and outer 
UNOFFICIAL COPY 21 RS BR 1691
Page 17 of 68
XXXX Jacketed
1 envelope for the voter shall complete the voter assistance form required by KRS
2 117.255. The signatures of two (2) witnesses are required if the voter signs the form
3 with the use of a mark instead of the voter's signature. All unsigned mail-in
4 absentee ballots shall be rejected automatically. The chair of the county board of
5 elections shall compare the signatures on the outer envelope, the detachable flap
6 with the signature of the voter that appears on the registration card. If the outer
7 envelope and the detachable flap are found to be in order, the chair shall read aloud
8 the name of the voter. If the vote of the voter is not rejected on a challenge then
9 made as provided in subsection (4) of this section, the chair shall remove the
10 detachable flap and place the secrecy envelope unopened in a ballot box which has
11 been provided for the purpose.
12 (4) When the name of a voter who cast a mail-in absentee ballot is read aloud by the
13 chair, the vote of the voter may be challenged by any board member or by the
14 written challenge provided in subsection (1) of this section and the challenge may
15 be determined and the vote accepted or rejected by the board as if the voter was
16 present and voting in person; but if the outer envelope and the detachable flap are
17 regular, and each substantially comply with the provisions of this chapter, they shall
18 be considered as showing that the voter is prima facie entitled to vote. If the vote of
19 a voter is rejected pursuant to the challenge, the secrecy envelope shall not be
20 opened, but returned to the outer envelope upon which the chair shall write on the
21 envelope the word "rejected."
22 (5) After the challenges have been made and all the blank secrecy envelopes have been
23 placed in a ballot box, the box shall be thoroughly shaken to redistribute the
24 absentee ballots in the box. The board shall open the ballot box, remove the
25 absentee ballots from the secrecy envelopes, and count the ballots.
26 (6) The board shall unlock any voting equipment used to cast in-person absentee
27 ballots[ in the county clerk's office or other place designated by the county board of 
UNOFFICIAL COPY 21 RS BR 1691
Page 18 of 68
XXXX Jacketed
1 elections, and approved by the State Board of Elections,] as provided for in KRS
2 117.086, and a total of all in-person absentee ballots shall be made and recorded on
3 the form provided by the State Board of Elections.
4 (7) The county board of elections, the county clerk, and all individuals permitted to be
5 present for the counting of absentee ballots pursuant to subsection (2) of this section
6 shall not make public the absentee ballot results determined as provided in this
7 section until after 6 p.m. prevailing time.
8 Section 8. KRS 117.088 is amended to read as follows:
9 (1) For purposes of this section, "blind or visually impaired individual" means an
10 individual who:
11 (a) Has a visual acuity of 20/200 or less in the better eye with correcting lenses or
12 has a limited field of vision so that the widest diameter of the visual field
13 subtends an angle no greater than twenty (20) degrees;
14 (b) Has a medically indicated expectation of visual deterioration;
15 (c) Has a medically diagnosed limitation in visual functioning that restricts the
16 individual's ability to read and write standard print at levels expected of
17 individuals of comparable ability;
18 (d) Has been certified as requiring permanent assistance to vote under KRS
19 117.255(5) for reason of blindness; or
20 (e) Qualifies to receive assistance to vote under KRS 117.255(2) for reason of
21 blindness.
22 (2) For purposes of this section, "pilot program" means a program in a county
23 containing a consolidated local government or containing a city of the first class for
24 unassisted voting by blind or visually impaired individuals.
25 (3) A county board of elections in a county containing a consolidated local government
26 or containing a city of the first class may establish a pilot program. As part of this
27 pilot program, the State Board of Elections shall approve the use of voting 
UNOFFICIAL COPY 21 RS BR 1691
Page 19 of 68
XXXX Jacketed
1 equipment under KRS 117.379 that is designed to permit blind and visually
2 impaired individuals to vote without assistance, for use beginning in the 2002
3 regular[general] election. No county board of elections in a county containing a
4 consolidated local government or containing a city of the first class shall be required
5 to operate a pilot program.
6 (4) The State Board of Elections, if it approves the voting equipment under KRS
7 117.379, may approve the use of voting equipment designed to permit blind and
8 visually impaired individuals to vote without assistance in as many locations within
9 a county containing a consolidated local government or containing a city of the first
10 class as are designated by the county board of elections.
11 (5) A county board of elections in a county containing a consolidated local government
12 or containing a city of the first class shall provide a report to the State Board of
13 Elections after every primary or regular[general] election regarding the number of
14 blind or visually impaired individuals that have utilized the voting equipment
15 during the pilot program.
16 (6) Notwithstanding the provisions of KRS 116.025, or any other statute to the
17 contrary, a blind or visually impaired voter residing in a county containing a
18 consolidated local government or containing a city of the first class that is operating
19 a pilot program shall be permitted to vote at a location outside the precinct of his or
20 her registration by voting at a location within the county of his or her registration on
21 a voting machine designed to permit blind or visually impaired individuals to vote
22 without assistance, which may include voting at the county clerk's office, or other
23 place designated by the county board of elections, and approved by the State Board
24 of Elections.
25 (7) Notwithstanding the provisions of KRS 117.085, 117.086, or 117.0863 or any other
26 statute to the contrary, a blind or visually impaired individual residing in a county
27 containing a consolidated local government or containing a city of the first class that 
UNOFFICIAL COPY 21 RS BR 1691
Page 20 of 68
XXXX Jacketed
1 is operating a pilot program shall be permitted to vote in the location within the
2 county of his or her registration as provided under subsection (6) of this section, on
3 a voting machine designed to permit blind or visually impaired individuals to vote
4 without assistance, at any time during which absentee voting is conducted[ in the
5 clerk's office or other place designated by the county board of elections during
6 normal business hours] on at least any of the six (6)[twelve (12)] working days and
7 two (2) Saturdays before the election, and the county board of elections may permit
8 the voting to be conducted on a voting machine for a period longer than the six
9 (6)[twelve (12)] working days and two (2) Saturdays, before the election prescribed
10 above. An application for those blind or visually impaired individuals wishing to
11 vote on a voting machine approved for use by blind or visually impaired individuals
12 shall be prescribed by the State Board of Elections and shall include the individual's
13 sworn statement that the individual is blind or visually impaired.
14 (8) Notwithstanding the requirements of KRS 117.381, or any other statute to the
15 contrary, the State Board of Elections may certify, as a part of the pilot project of a
16 county containing a consolidated local government or containing a city of the first
17 class, voting equipment which utilizes audio recordings, voice-activated technology,
18 or vocal recognition technology to record a vote, and may require such
19 accommodations as would permit a blind or visually impaired voter to cast a vote in
20 secret.
21 (9) Notwithstanding the provisions of KRS 117.255, a blind or visually impaired voter
22 residing in a county containing a consolidated local government or containing a city
23 of the first class that is operating a pilot project may cast his or her vote alone and
24 without assistance on a voting machine approved for use by blind or visually
25 impaired individuals. However, the blind or visually impaired voter shall be
26 instructed by the officers of election, with the aid of the instruction cards and the
27 model, in the use of the machine, if the voter so requests.
UNOFFICIAL COPY 21 RS BR 1691
Page 21 of 68
XXXX Jacketed
1 (10) Nothing in this section shall impair the right of any qualified voter under KRS
2 117.255 to receive assistance and vote according to the procedures specified in that
3 section.
4 Section 9. KRS 117.145 is amended to read as follows:
5 (1) At least forty-five[fifteen (15)] days before any special election,[ and] at least fifty
6 (50) days before the day of any primary or regular election, and fifteen (15) days
7 before any runoff primary, the county clerk of each county shall cause to be printed
8 and ready for use ballot labels for each candidate who, and each question which, is
9 entitled to be voted upon in such election. The ballot labels shall be printed on clear
10 white paper or other material which shall be furnished by the printer. They shall be
11 printed in black ink, in plain, clear type clearly legible to a person with normal
12 vision, and shall be of a size to fit the ballot frames. The labels shall include the
13 necessary party designations.
14 (2) Each county clerk shall have printed a sufficient number of mail-in[paper] absentee
15 ballots, voter affirmations, and election official affirmations.[ The voter affirmation,
16 if applicable, and the absentee ballot shall be used for voting by absent voters; by
17 precinct officers who have been assigned to a precinct other than their own; by
18 members of a county board of elections; by voters so disabled by age, infirmity, or
19 illness as to be unable to appear at the polls; and for voting in an emergency
20 situation.] The ballot stubs shall be consecutively numbered and the county board
21 shall keep a record, by number, of all mail-in absentee ballots issued[used for any
22 of the purposes listed in this subsection].
23 (3) Each county clerk shall have printed a sufficient number of federal provisional
24 ballots, which, except for the candidates listed, shall have the same form as the
25 absentee ballots. A federal provisional ballot shall indicate that the ballot is a
26 federal provisional ballot. The federal provisional ballot stubs shall be
27 consecutively numbered, and the county board of elections shall keep a record, by 
UNOFFICIAL COPY 21 RS BR 1691
Page 22 of 68
XXXX Jacketed
1 number, of all federal provisional ballots used for votes cast by provisional voters in
2 federal elections.
3 (4) No later than the Friday preceding a special or regular election, the county clerk
4 shall equip the voting machines with the necessary supplies for the purpose of
5 write-in votes. The county clerk shall also attach a pencil or pen to the voting
6 machine for write-in purposes. The county clerk shall equip the in-person
7 absentee voting machine with the necessary supplies, including attaching a pencil
8 or pen to the voting machine for the purpose of write-in votes, at least five (5)
9 days before the in-person absentee voting begins.
10 (5) If supplemental paper ballots have been approved as provided in KRS 118.215, the
11 county clerk shall cause to be printed a sufficient number of paper ballots for the
12 registered voters of each precinct. The paper ballots shall have stubs which are
13 numbered consecutively. The quality of paper on which the supplemental paper
14 ballots are printed shall be determined by administrative regulations promulgated
15 under KRS Chapter 13A by the secretary of the Finance and Administration
16 Cabinet.
17 Section 10. KRS 117.275 is amended to read as follows:
18 (1) At the count of the votes in any precinct, any candidate or slate of candidates and
19 any representatives to witness and check the count of the votes therein, who are
20 authorized to be appointed as is provided in subsection (9) of this section, shall be
21 admitted and be permitted to be present and witness the count.
22 (2) As soon as the polls are closed, and the last voter has voted, the judges shall
23 immediately lock and seal the voting equipment so that the voting and counting
24 mechanism will be prevented from operation, and they shall sign a certificate
25 stating:
26 (a) That the voting equipment has been locked against voting and sealed;
27 (b) The number of voters, as shown on the public counters;
UNOFFICIAL COPY 21 RS BR 1691
Page 23 of 68
XXXX Jacketed
1 (c) The number registered on the protective or accumulative counter or device, if
2 any; and
3 (d) The number or other designation of the voting equipment, which certificate
4 shall be returned by the judges of election to the officials authorized by law to
5 receive it. The judges shall compare the number of voters, as shown by the
6 counter of the voting equipment, with the number of those who have voted as
7 shown by the protective or accumulative counter or device, if any.
8 (3) Where voting equipment is used which does not print the candidates' names along
9 with the total votes received on a general return sheet or record for that equipment,
10 the procedure to be followed shall be as follows:
11 (a) The judges, in the presence of the representatives mentioned in subsection (1)
12 of this section, if any, and of all other persons who may be lawfully within the
13 polling place, shall give full view of all the counter numbers;
14 (b) The judges shall enter, in ink, the total votes cast for each candidate, and slate
15 of candidates, and for and against each question on the return sheets; and
16 (c) Each precinct election officer shall sign the return sheets, and a copy of the
17 return sheets shall be posted on the precinct door.
18 (4) Where voting equipment is used that prints the candidates' names along with the
19 total votes received on a return sheet or record for that equipment, the precinct
20 election officers shall sign the return sheets or record for the voting equipment,
21 which shall be posted on the door of the precinct.
22 (5) If any officer shall decline to sign the return sheets, he or she shall state the reason
23 in writing, and a copy thereof, signed by the officer, shall be enclosed with the
24 return sheets.
25 (6) Each of the return sheets, if applicable, and the record of the voting equipment shall
26 be enclosed in an envelope. One (1) copy of the return sheets, if applicable, one (1)
27 copy of the record of the voting equipment, and the write-in roll, if any write-in 
UNOFFICIAL COPY 21 RS BR 1691
Page 24 of 68
XXXX Jacketed
1 votes were cast in the precinct, shall be directed to the county board of elections of
2 the county in which the election is being held. One (1) copy of the return sheets or
3 record of the voting equipment shall be given to the county clerk of the county in
4 which the election is being held and to each of the local governing bodies of the two
5 (2) dominant political parties, but a local governing body of a dominant political
6 party may decline a copy of the precinct election return by filing a written
7 declination with the county board of elections prior to the election, and upon this
8 declination, a printed copy shall not be issued to the political party so declining. The
9 declination on file shall be effective for that election and any subsequent elections
10 until revoked by the local governing body of a dominant political party by filing a
11 written revocation with the county board of elections. The envelope shall have
12 endorsed thereon a certificate of the election officers, stating the number of the
13 machine, the precinct where it has been used, the number on the seal, and the
14 number on the protective or accumulative counter or device at the close of the polls.
15 (7) Following the tabulation of all votes cast in the election, including absentee votes
16 and write-in votes, the county board shall mail a copy of the precinct-by-precinct
17 summary of the tabulation sheets showing the results from each precinct to the State
18 Board of Elections and the county clerk shall mail or deliver the precinct signature
19 rosters from each precinct to the State Board of Elections during the period
20 established by KRS 117.355(3).
21 (8) In a primary where each party's slates of candidates seeking the nomination of
22 their parties for Governor and Lieutenant Governor are voted on, the Secretary of
23 State, upon receiving the certified results of voting from each county's precincts,
24 shall determine whether a runoff primary shall be necessary for either or both
25 parties pursuant to KRS 118.425. The Secretary of State shall, within twenty four
26 (24) hours of making his or her determination, inform the affected slates of
27 candidates, the county clerks, the county boards of elections, the State Board of 
UNOFFICIAL COPY 21 RS BR 1691
Page 25 of 68
XXXX Jacketed
1 Elections, the Registry of Election Finance, and the news media of his or her
2 determination, and the date of the runoff primary, which shall be subject to
3 change pursuant to Section 13 of this Act if an election contest or vote recount
4 shall be requested.
5 (9) As soon as possible after the completion of the count, the two (2) judges shall return
6 to the county board of elections the keys to the voting machine received and
7 receipted for by them, and the county clerk in which the precinct is located shall
8 have the voting machine properly boxed or securely covered and removed to a
9 proper and secure place of storage.
10 (10)[(9)] In primaries, each candidate or group of candidates may designate to the
11 county board of elections a representative to witness and check the vote count. In
12 regular elections, the governing authority of each political party, each candidate for
13 member of board of education, nonpartisan candidate, independent candidate, or
14 independent ticket may designate a representative to the county board of elections to
15 witness and check the vote count. The county board of elections shall authorize
16 representatives of the news media to witness the vote count.
17 (11)[(10)] For all federal provisional ballots, if applicable, and supplemental paper
18 ballots if approved as provided in KRS 118.215, after the polls are closed, the two
19 (2) judges shall return to the county clerk's office the locked federal provisional
20 ballot receptacle and the supplemental paper ballot box, all ballot stubs, spoiled
21 ballots, and unvoted ballots at the same time as the tabulation of votes from the
22 voting machine is delivered. The county clerk shall issue a receipt for the number of
23 ballot stubs, unvoted ballots, spoiled ballots, and the ballot boxes or ballot
24 receptacle.
25 (12)[(11)] The county board of elections, or its designee, shall count and tally the
26 supplemental paper ballots manually or with the use of tabulating equipment which
27 does not involve an additional voting system. The results of the vote tally shall be 
UNOFFICIAL COPY 21 RS BR 1691
Page 26 of 68
XXXX Jacketed
1 certified by the county board of elections to the county clerk and to the Secretary of
2 State.
3 (13)[(12)] The county board of elections shall tabulate the valid federal provisional
4 ballots. The results of the vote tally shall be certified by the county board of
5 elections to the county clerk and to the Secretary of State. The county board shall
6 mail a copy of the precinct-by-precinct summary of the valid federal provisional
7 ballot tabulation sheets showing the results from each precinct to the State Board of
8 Elections.
9 (14)[(13)] The county board of elections shall authorize the candidates, slates of
10 candidates, or their representatives, and representatives of the news media to be
11 present during the counting of the supplemental and federal provisional paper
12 ballots.
13 (15)[(14)] Except as otherwise required in this chapter that certain records and papers
14 relating to specified elections be retained for twenty-two (22) months, the county
15 clerk shall retain the voted federal provisional ballots, voter affirmations, election
16 official affirmations, and the supplemental paper ballots for twenty-two (22) months
17 and the unvoted federal provisional ballots, the voter affirmations, election official
18 affirmations, and the supplemental paper ballots for sixty (60) days after each
19 election day, after which time they shall be destroyed in a manner to render them
20 unreadable by the county board of elections if no contest or recount action has been
21 filed.
22 Section 11. KRS 117.295 is amended to read as follows:
23 (1) For a period of ten (10) days following any primary election, and for a period of
24 thirty (30) days following any general or special election, the voting machine shall
25 remain locked against voting and the ballot boxes containing all paper ballots shall
26 remain locked, except that the voting machines and the ballot boxes may be opened
27 and all the data and figures therein examined, upon the order of any court of 
UNOFFICIAL COPY 21 RS BR 1691
Page 27 of 68
XXXX Jacketed
1 competent jurisdiction, or judge thereof, or by direction of any legislative
2 committee authorized and empowered to investigate and report upon contested
3 elections, and all the data and figures shall be examined by the court, judge, or
4 committee in the presence of the officer having the custody of the machine and
5 ballot boxes. In the event of a contest of election, the court in which the contest is
6 pending or the committee before which the contest is being heard may, upon motion
7 of any party to the contest, issue an order requiring that the voting machines and
8 ballot boxes shall remain continuously locked for further time as may be reasonable
9 or necessary, with due regard for the preparation of the machines for a succeeding
10 primary, regular, or special election, but in no event shall the order compel that the
11 machines remain locked to a time within thirty (30) days next preceding any
12 approaching primary, runoff primary, regular, or special election.
13 (2) During the period when the machine and the ballot boxes are required to be kept
14 locked, the keys thereto shall remain in the possession of the county board of
15 elections. After that period, it shall be the duty of the county board of elections to
16 return the keys to the custody of the county clerk.
17 SECTION 12. A NEW SECTION OF ARTICLE 025 OF KRS CHAPTER 118
18 IS CREATED TO READ AS FOLLOWS:
19 As used in this chapter, unless the context otherwise requires, the word "election"
20 means any primary, runoff primary, regular election, or special election.
21 Section 13. KRS 118.025 is amended to read as follows:
22 (1) Except as otherwise provided by law, voting in all primaries and elections shall be
23 by secret ballot on voting machines.
24 (2) The general laws applying to primaries, runoff primary, regular, and special
25 elections shall apply to primaries, regular, and special elections conducted with the
26 use of voting machines, and all provisions of the general laws applying to the
27 custody of ballot boxes shall apply, as far as applicable, to the custody of the voting 
UNOFFICIAL COPY 21 RS BR 1691
Page 28 of 68
XXXX Jacketed
1 machine.
2 (3) Primaries for the nomination of candidates to be voted for at the next regular
3 election shall be held on the last[first] Tuesday[ after the third Monday] in
4 June[May] of each year.
5 (4) The election of all officers of all governmental units shall be held on the first
6 Tuesday after the first Monday in November.
7 (5) If the law authorizes the calling of a special election on a day other than the day of
8 the regular election in November, the election shall be held on a Tuesday.
9 (6) If the law requires that a special election be held within a period of time during
10 which the voting machines must be locked as required by KRS 117.295, the special
11 election shall be held on the fourth Tuesday following the expiration of the period
12 during which the voting machines are locked.
13 (7) A runoff primary shall be held thirty-five (35) days after the date of the June
14 primary, if it shall be necessary, pursuant to Section 19 of this Act. If the date to
15 hold the runoff primary falls on a holiday, the runoff primary shall be held on
16 the succeeding Tuesday. If either a primary is contested or a recount of the votes
17 cast in a primary is requested, a runoff primary shall be held on the first Tuesday
18 following the thirty-fifth day at the conclusion of any contest proceeding or
19 recount, unless that date falls on a holiday; in that case, a runoff primary shall
20 be held on the succeeding Tuesday.
21 Section 14. KRS 118.035 is amended to read as follows:
22 (1) The polls shall be opened on the day of a primary, runoff primary, special election,
23 or regular election at 6 a.m., prevailing time, and shall remain open until each voter
24 who is waiting in line at the polls at 7[6] p.m., prevailing time, has voted. At 7[6]
25 p.m., prevailing time, if voters are waiting at the polls to vote, the precinct election
26 sheriff shall announce that a voter wishing to vote must immediately get in line.
27 When all voters waiting at the polls at that time are in line, the precinct election 
UNOFFICIAL COPY 21 RS BR 1691
Page 29 of 68
XXXX Jacketed
1 sheriff shall then determine which voter is the last in line, and that voter shall be the
2 last voter permitted to vote. The precinct election sheriff shall wait in line with the
3 last voter who shall be permitted to vote until that voter has voted and shall inform
4 a voter who subsequently arrives at the polls that no one shall be permitted to vote
5 after the last voter in line at 7[6] p.m., prevailing time. After the last voter waiting
6 in line at 7[6] p.m., prevailing time, has voted, the polls shall then be closed.
7 (2) As provided in Section 148 of the Constitution of Kentucky, any person entitled to a
8 vote at any election in this state shall, if he or she has made application for leave
9 prior to the day he or she appears before the county clerk to request an application
10 for or to execute an absentee ballot, be entitled to absent himself or herself from
11 any services or employment in which he or she is then engaged or employed for a
12 reasonable time, but not less than four (4) hours on the day he or she appears before
13 the county clerk to request an application for a mail-in absentee ballot or to
14 execute an in-person absentee ballot during the in-person absentee voting period,[
15 during normal business hours of the office of the clerk] or to cast his ballot or on
16 the day of the election between the time of opening and closing the polls. The
17 employer may specify the hours during which an employee may absent himself or
18 herself.
19 (3) No person shall be penalized for taking a reasonable time off to vote, unless, under
20 circumstances which did not prohibit him or her from voting, he or she fails to
21 vote. Any qualified voter who exercises his or her right to voting leave under this
22 section but fails to cast his or her vote, under circumstances which did not prohibit
23 him or her from voting, may be subject to disciplinary action.
24 (4) Any person selected to serve as an election officer shall be entitled to absent himself
25 or herself from any services or employment in which he or she is then engaged or
26 employed for a period of an entire day to attend training or to serve as an election
27 officer. The person shall not, because of so absenting himself or herself, be liable to 
UNOFFICIAL COPY 21 RS BR 1691
Page 30 of 68
XXXX Jacketed
1 any penalty. The employer may specify the hours during which the employee may
2 absent himself herself. No person shall refuse an employee the privilege hereby
3 conferred, or discharge or threaten to discharge an employee or subject an employee
4 to a penalty, because of the exercise of the privilege.
5 Section 15. KRS 118.215 is amended to read as follows:
6 (1) After the order of the names has been determined as provided in KRS 118.225, the
7 Secretary of State shall certify, to the county clerks of the respective counties
8 entitled to participate in the nomination or election of the respective candidates, the
9 name, place of residence, and party of each candidate or slate of candidates for each
10 office, as specified in the nomination papers or certificates and petitions of
11 nomination filed with him or her, and shall designate the device with which the
12 candidate groups, slates of candidates, or lists of candidates of each party shall be
13 printed, in the order in which they are to appear on the ballot, with precedence to be
14 given to the party that polled the highest number of votes at the preceding election
15 for presidential electors, followed by the political party which received the second
16 highest number of votes, with the order of any other political parties and
17 independents to be determined by lot. Candidates for county offices and local state
18 offices shall be listed in the following order: Commonwealth's attorney, circuit
19 clerk, property valuation administrator, county judge/executive, county attorney,
20 county clerk, sheriff, jailer, county commissioner, coroner, justice of the peace, and
21 constable. The names of candidates for President and Vice President shall be
22 certified in lieu of certifying the names of the candidates for presidential electors.
23 The names shall be certified as follows:
24 (a) Not later than the second Monday after the filing deadline for the primary as
25 established in KRS 83A.045, 118.165, and 118A.060;
26 (b) Not less than twenty-five (25) days before a runoff primary;
27 (c) Not later than the second Monday following the filing deadline for the regular 
UNOFFICIAL COPY 21 RS BR 1691
Page 31 of 68
XXXX Jacketed
1 election, except as provided in paragraph (d)[(c)] of this subsection; and
2 (d)[(c)] Not later than the Monday after the Friday following the first Tuesday in
3 September preceding a regular election, for those years in which there is an
4 election for President and Vice President of the United States.
5 (2) Except as otherwise provided in subsection (3) of this section, all independent
6 candidates or slates of candidates whose nominating petitions are filed with the
7 county clerk or the Secretary of State shall be listed under the title and device
8 designated by them as provided in KRS 118.315, or if none is designated, under the
9 word "independent," and shall be placed on the ballot in a separate column or
10 columns or in a separate line or lines according to the office which they seek. The
11 order in which independent candidates or slates of candidates shall appear on the
12 ballot shall be determined by lot by the county clerk. If the same device is selected
13 by two (2) groups of petitioners, it shall be given to the first selecting it and the
14 county clerk shall permit the other group to select a suitable device. This section
15 shall not apply to candidates for municipal offices which come under subsection (3)
16 of this section.
17 (3) The ballots used at any election in which city officers are to be elected as provided
18 in subsection (2) of this section shall contain the names of candidates for the city
19 offices grouped according to the offices they seek, and the candidates shall be
20 immediately arranged with and designated by the title of office they seek. The order
21 in which the names of the candidates for each office are to be printed on the ballot
22 shall be determined by lot. Each group of candidates for each separate office for
23 which the candidates are to be elected shall be clearly separated from other groups
24 on the ballot and spaced to avoid confusion on the part of the voter.
25 (4) The Secretary of State shall not knowingly certify to the county clerk of any county
26 the name of any candidate or slate of candidates who has not filed the required
27 nomination papers, nor knowingly fail to certify the name of any candidate or slate 
UNOFFICIAL COPY 21 RS BR 1691
Page 32 of 68
XXXX Jacketed
1 of candidates who has filed the required nomination papers.
2 (5) If the county clerk determines that the number of certified candidates or slates of
3 candidates cannot be placed on a ballot which can be accommodated by the voting
4 machines currently in use by the county, he or she shall so notify the State Board of
5 Elections not later than the last Tuesday in February preceding the primary or the
6 last Tuesday in August preceding the regular election. The State Board of Elections
7 shall meet within five (5) days of the notice, review the ballot conditions, and
8 determine whether supplemental paper ballots are necessary for the election. Upon
9 approval of the State Board of Elections, supplemental paper ballots may be used
10 for nonpartisan candidates or slates of candidates for an office or offices and public
11 questions submitted for a yes or no vote. All candidates or slates of candidates for
12 any particular office shall be placed either on the machine ballot or on the paper
13 ballot. Supplemental paper ballots may also be used to conduct the voting, in the
14 instance of a small precinct as provided in KRS 117.066.
15 (6) The ballot position of a candidate or slate of candidates shall not be changed after
16 the ballot position has been designated by the county clerk.
17 Section 16. KRS 118.225 is amended to read as follows:
18 (1) For the purpose of determining the order in which the names of candidates or slates
19 of candidates to be voted for by the electors of the entire state shall be certified and
20 printed on the ballots with the designation of the respective offices, the Secretary of
21 State shall prepare lists of the counties of each congressional district of the state.
22 The Secretary of State shall arrange the surnames of all candidates or slates of
23 candidates for each office in alphabetical order for the First Congressional District,
24 and the names shall be certified in this order to the county clerks of all the counties
25 comprising that district. For each succeeding congressional district, taken in
26 numerical order, the name appearing first for each office in the last preceding
27 district shall be placed last, and the name appearing second in the last preceding 
UNOFFICIAL COPY 21 RS BR 1691
Page 33 of 68
XXXX Jacketed
1 district shall be placed first, and each other name shall be moved up one (1) place.
2 The lists shall be certified accordingly.
3 (2) For all other offices for which nomination papers and petitions are filed with the
4 Secretary of State, the order of names of candidates for each office shall be
5 determined by lot at a public drawing to be held in the office of the Secretary of
6 State at 2 p.m., standard time, on the Thursday following the filing deadline for the
7 primary as established in KRS 83A.045, 118.165, and 118A.060, twenty-six (26)
8 days before a runoff primary, or the Thursday following the first Tuesday after the
9 first Monday in June preceding the regular election.
10 (3) For all offices for which nomination papers and petitions are filed in the office of
11 the county clerk, the order in which the names of candidates for each office are to
12 be printed on the ballot shall be determined by lot at a public drawing in the office
13 of the county clerk at 2 p.m., standard time, on the Thursday following the filing
14 deadline for the primary as established in KRS 83A.045, 118.165, and 118A.060 or
15 the Thursday following the first Tuesday after the first Monday in June preceding
16 the regular election.
17 (4) For all offices for which the deadline for filing nomination papers and petitions is
18 governed by KRS 83A.165(4)(c) or 118.375(2), the order in which the names of
19 candidates for each office are to be printed shall be determined by lot at a public
20 drawing in the office at the place of filing at 2 p.m., standard time, on the Thursday
21 following the second Tuesday in August preceding the regular election.
22 (5) If the number of certified candidates or slates of candidates cannot be placed on a
23 ballot which can be accommodated on voting machines currently in use in the
24 county, the county clerk shall notify the State Board of Elections, as provided in
25 KRS 118.215.
26 Section 17. KRS 118.315 is amended to read as follows:
27 (1) A candidate for any office to be voted for at any regular election may be nominated 
UNOFFICIAL COPY 21 RS BR 1691
Page 34 of 68
XXXX Jacketed
1 by a petition of electors qualified to vote for him or her, complying with the
2 provisions of subsection (2) of this section. No person whose registration status is
3 as a registered member of a political party shall be eligible to election as an
4 independent, or political organization, or political group candidate, nor shall any
5 person be eligible to election as an independent, or political organization, or
6 political group candidate whose registration status was as a registered member of a
7 political party on January 1 immediately preceding the regular election for which
8 the person seeks to be a candidate. This restriction shall not apply to candidates to
9 those offices specified in KRS 118.105(7), for supervisor of a soil and water
10 conservation district, for candidates for mayor or legislative body in cities of the
11 home rule class, or to candidates participating in nonpartisan elections.
12 (2) The form of the petition shall be prescribed by the State Board of Elections. It shall
13 be signed by the candidate and by registered voters from the district or jurisdiction
14 from which the candidate seeks nomination. The petition shall include a declaration,
15 sworn to by the candidate, that he or she possesses all the constitutional and
16 statutory requirements of the office for which the candidate has filed. Signatures for
17 a petition of nomination for a candidate seeking any office[, excluding President of
18 the United States in accordance with KRS 118.591(1),] shall not be affixed on the
19 document to be filed prior to the first Wednesday after the first Monday in
20 November of the year preceding the year in which the office will appear on the
21 ballot. Signatures for nomination papers shall not be affixed on the document to be
22 filed prior to the first Wednesday after the first Monday in November of the year
23 preceding the year in which the office will appear on the ballot. A petition of
24 nomination for a state officer, or any officer for whom all the electors of the state
25 are entitled to vote, shall contain five thousand (5,000) petitioners; for a
26 representative in Congress from any congressional district, or for any officer from
27 any other district except as herein provided, four hundred (400) petitioners; for a 
UNOFFICIAL COPY 21 RS BR 1691
Page 35 of 68
XXXX Jacketed
1 county officer, member of the General Assembly, or Commonwealth's attorney, one
2 hundred (100) petitioners; for a soil and water conservation district supervisor,
3 twenty-five (25) petitioners; for a city officer or board of education member, two (2)
4 petitioners; and for an officer of a division less than a county, except as herein
5 provided, twenty (20) petitioners. It shall not be necessary that the signatures of the
6 petition be appended to one (1) paper. Each petitioner shall include the date he or
7 she affixes the signature, address of residence, and date of birth. Failure of a voter
8 to include the signature affixation date, date of birth, and address of residence shall
9 result in the signature not being counted. If any person joins in nominating, by
10 petition, more than one (1) nominee for any office to be filled, he or she shall be
11 counted as a petitioner for the candidate whose petition is filed first, except a
12 petitioner for the nomination of candidates for soil and water conservation district
13 supervisors may be counted for every petition to which his or her signature is
14 affixed.
15 (3) Titles, ranks, or spurious phrases shall not be accepted on the filing papers and shall
16 not be printed on the ballots as part of the candidate's name; however, nicknames,
17 initials, and contractions of given names may be accepted as the candidate's name.
18 (4) The Secretary of State and county clerks shall examine the petitions of all
19 candidates who file with them to determine whether each petition is regular on its
20 face. If there is an error, the Secretary of State or the county clerk shall notify the
21 candidate by certified mail within twenty-four (24) hours of filing.
22 (5) A judge who elected to retire as a Senior Status Special Judge in accordance with
23 KRS 21.580 shall not become a candidate or a nominee for any elected office
24 during the five (5) year term prescribed in KRS 21.580(1)(a)1., regardless of the
25 number of days served by the judge acting as a Senior Status Special Judge.
26 Section 18. KRS 118.555 is amended to read as follows:
27 [(1) ]The state executive committee of each political party shall, pursuant to its party's 
UNOFFICIAL COPY 21 RS BR 1691
Page 36 of 68
XXXX Jacketed
1 rules,[ determine whether to] distribute its party's authorized delegate votes for
2 presidential candidates at its party's national convention based on the results of a party
3 caucus[, a presidential preference primary, or a combination of the two (2) methods. Each
4 state executive committee shall notify the State Board of Elections of its decision not later
5 than the December 31 preceding the day for conducting a presidential preference primary
6 as set by KRS 118.561.
7 (2) If a state executive committee determines that its party's authorized delegate votes
8 for presidential candidates at its party's national convention shall be distributed
9 based on the results of a party caucus, a presidential preference primary shall not be
10 conducted for that political party, and the provisions of KRS 118.561 to 118.651
11 shall not apply]. The distribution of delegates among the presidential candidates
12 shall be determined by party rule.
13 [(3) If a state executive committee determines that its party's authorized delegate votes
14 for presidential candidates at its party's national convention shall be distributed
15 based on the results of both a party caucus and a presidential preference primary, the
16 formula for distribution of authorized delegate votes based on the results of a party
17 caucus shall be determined by party rule. The distribution of delegates based on the
18 results of a presidential preference primary shall be in accordance with the
19 provisions of KRS 118.641(1). Regardless of the method by which the authorized
20 delegate votes are distributed,] The casting of votes on the first ballot at each party's
21 national convention shall be in accordance with the provisions of KRS 118.641(2).
22 Section 19. KRS 118.245 is amended to read as follows:
23 (1) The candidate for office, other than the offices of Governor and Lieutenant
24 Governor, receiving the highest number of votes in a primary for the office for
25 which he or she is a candidate shall be the nominee of his or her party for that office
26 and shall receive the certificate of nomination.
27 (2) A slate of candidates for Governor and Lieutenant Governor that receives more 
UNOFFICIAL COPY 21 RS BR 1691
Page 37 of 68
XXXX Jacketed
1 than fifty percent (50%) of its party's votes cast shall be the nominee of its party
2 for those offices and that slate of candidates shall receive the certificate of
3 nomination, except that if two (2) slates of candidates receive more than fifty
4 percent (50%) of his or her party's votes, the slate receiving the higher number of
5 votes shall be its party's nominee, and no runoff primary shall be conducted.
6 (3) A slate of candidates for Governor and Lieutenant Governor that receives the
7 highest number of its party's votes but which number is less than fifty percent
8 (50%) of the votes cast for all slates of candidates of that party, shall be required
9 to participate in a runoff primary with the slate of candidates of the same party
10 receiving the second highest number of votes.
11 (4) The slate of candidates for Governor and Lieutenant Governor receiving the
12 highest number of votes in a runoff primary shall be the nominees of that party
13 for Governor and Lieutenant Governor, and that slate of candidates shall receive
14 the certificate of nomination.
15 (5)[(2)] Subject to the foregoing provisions relating to a runoff primary, if two (2) or
16 more candidates in a runoff primary or primary are found to have received the
17 highest and an equal number of votes for nomination to the same office, the election
18 shall be determined by lot in the manner the board directs, in the presence of not
19 less than three (3) other persons. This section does not apply to presidential
20 primaries.
21 Section 20. KRS 118A.060 is amended to read as follows:
22 (1) Except as provided in KRS 118A.100, no person's name shall appear on a ballot
23 label or absentee ballot for an office of the Court of Justice without first having
24 been nominated as provided in this section.
25 (2) Each candidate for nomination shall file a petition for nomination with the Secretary
26 of State not earlier than the first Wednesday after the first Monday in November of
27 the year preceding the year in which the office will appear on the ballot and not later 
UNOFFICIAL COPY 21 RS BR 1691
Page 38 of 68
XXXX Jacketed
1 than the first Friday following the first Monday in January preceding the day fixed
2 by law for holding the primary for the office. The petition shall be sworn to before
3 an officer authorized to administer an oath by the candidate and by not less than two
4 (2) registered voters from the district or circuit from which he or she seeks
5 nomination. Signatures for nomination papers shall not be affixed on the document
6 to be filed prior to the first Wednesday after the first Monday in November of the
7 year preceding the year in which the office will appear on the ballot. The petition
8 shall be filed no later than 4 p.m. local time at the place of filing when filed on the
9 last date on which the papers are permitted to be filed.
10 (3) The petition for nomination shall be in the form prescribed by the State Board of
11 Elections. The petition shall include a declaration sworn to by the candidate, that he
12 or she possesses all the constitutional and statutory requirements of the office for
13 which the candidate has filed. Titles, ranks, or spurious phrases shall not be
14 accepted on the petition and shall not be printed on the ballots as part of the
15 candidate's name; however, nicknames, initials, and contractions of given names
16 may be acceptable as the candidate's name.
17 (4) The Secretary of State shall examine the petition of each candidate to determine
18 whether it is regular on its face. If there is an error, the Secretary of State shall
19 notify the candidate by certified mail within twenty-four (24) hours of filing. The
20 order of names on the ballot for each district or circuit, and numbered division if
21 divisions exist, shall be determined by lot at a public drawing to be held in the
22 office of the Secretary of State at 2 p.m., standard time, on the Thursday following
23 the filing deadline for the primary as established in this section and in KRS
24 83A.045 and 118.165.
25 (5) Not later than the date set forth in KRS 118.215[(1)(a)] preceding the primary, and
26 after the order of names on the ballot has been determined as required in subsection
27 (4) of this section, the Secretary of State shall:
UNOFFICIAL COPY 21 RS BR 1691
Page 39 of 68
XXXX Jacketed
1 (a) Certify to the county clerks of the respective counties entitled to participate in
2 the election of the various candidates, the name and place of residence of each
3 candidate for each office, by district or circuit, and numbered division if
4 divisions exist, as specified in the petitions for nomination filed with him or
5 her; and
6 (b) Designate for the county clerks the office of the Court of Justice with which
7 the names of candidates shall be printed and the order in which they are to
8 appear on the ballot.
9 (6) The ballot position of a candidate shall not be changed after the ballot position has
10 been designated by the Secretary of State.
11 (7) The county clerks of each county shall cause to be printed on the ballot labels for
12 the voting machines and on the special ballots for the primary the names of the
13 candidates for offices in the Court of Justice.
14 (8) The names of the candidates shall be placed on the voting machine in a separate
15 column or columns or in a separate line or lines and identified by the words
16 "Judicial Ballot." The words "Vote for one," or "Vote for one in each division,"
17 shall be printed on the ballot in an appropriate location. The office, numbered
18 division if divisions exist, and the candidates shall be clearly labeled. No party
19 designation or emblem of any kind, nor any sign indicating any candidate's political
20 belief or party affiliation, shall be used on voting machines or special ballots.
21 (9) The two (2) candidates receiving the highest number of votes for nomination for
22 justice or judge of a district or circuit, or numbered division if divisions exist, shall
23 be nominated. Certificates of nomination shall be issued as provided in KRS
24 118A.190.
25 (10) If it appears after expiration of the time for filing petitions for nomination that there
26 are not more than two (2) candidates who have filed the necessary petitions for a
27 place on the ballot in the regular election, no drawing for ballot position shall be 
UNOFFICIAL COPY 21 RS BR 1691
Page 40 of 68
XXXX Jacketed
1 held and the Secretary of State shall immediately issue and file in the Secretary's
2 office certificates of nomination, and send copies to the candidates.
3 Section 21. KRS 118A.090 is amended to read as follows:
4 (1) For the regular election, the order of names on the ballot for each district or circuit,
5 and numbered division if divisions exist, shall be determined by lot at a public
6 drawing to be held in the office of the Secretary of State at 2 p.m., standard time, on
7 the Thursday following the first Tuesday after the first Monday in June preceding
8 the regular election, except as provided in KRS 118A.100(6).
9 (2) Not later than the date set forth in KRS 118.215[(1)(b)] after the filing deadline for
10 the regular election in a year in which there is no election for President and Vice
11 President of the United States, or not later than the date set forth in KRS
12 118.215[(1)(c)] preceding a regular election in a year in which there is an election
13 for President and Vice President of the United States, and after the order of names
14 on the ballot has been determined as required in subsection (1) of this section, the
15 Secretary of State shall:
16 (a) Certify to the county clerks of the respective counties entitled to participate in
17 the election of the various candidates, the name and place of residence of each
18 candidate for each office, by district or circuit, and numbered division if
19 divisions exist, as certified under KRS 118A.060; and
20 (b) Designate for the county clerks the office of the Court of Justice with which
21 the names of candidates shall be printed and the order in which they are to
22 appear on the ballot.
23 (3) The ballot position of a candidate shall not be changed after the ballot position has
24 been designated by the Secretary of State. The county clerks of each county shall
25 cause to be printed on the ballot labels for the voting machines and on the special
26 ballots for the regular elections the names of the candidates for offices of the Court
27 of Justice.
UNOFFICIAL COPY 21 RS BR 1691
Page 41 of 68
XXXX Jacketed
1 (4) The names of the candidates shall be placed on the voting machine in a separate
2 column or columns or in a separate line or lines and identified by the words
3 "Judicial Ballot," and in such a manner that the casting of a vote for all of the
4 candidates of a political party will not operate to cast a vote for judicial candidates.
5 The words "Vote for one" or "Vote for one in each division," shall be printed on the
6 ballot in an appropriate location. The office, numbered division thereof if divisions
7 exist, and the candidates therefor shall be clearly labeled. No party designation or
8 emblem of any kind, nor any sign indicating any candidate's political belief or party
9 affiliation, shall be used on voting machines or special ballots.
10 (5) The candidate receiving the highest number of votes cast at the regular election for a
11 district or circuit, or for a numbered division thereof if divisions exist, shall be
12 elected.
13 Section 22. KRS 121.015 is amended to read as follows:
14 As used in this chapter:
15 (1) "Registry" means the Kentucky Registry of Election Finance;
16 (2) "Election" means any primary, runoff primary, regular, or special election to fill
17 vacancies regardless of whether a candidate or slate of candidates is opposed or
18 unopposed in an election. Each primary, runoff primary, regular, or special election
19 shall be considered a separate election;
20 (3) "Committee" includes the following:
21 (a) "Campaign committee," which means one (1) or more persons who receive
22 contributions and make expenditures to support or oppose one (1) or more
23 specific candidates or slates of candidates for nomination or election to any
24 state, county, city, or district office, but does not include an entity established
25 solely by a candidate which is managed solely by a candidate and a campaign
26 treasurer and whose name is generic in nature, such as "Friends of (the
27 candidate)," and does not reflect that other persons have structured themselves 
UNOFFICIAL COPY 21 RS BR 1691
Page 42 of 68
XXXX Jacketed
1 as a committee, designated officers of the committee, and assigned
2 responsibilities and duties to each officer with the purpose of managing a
3 campaign to support or oppose a candidate in an election;
4 (b) "Caucus campaign committee," which means members of one (1) of the
5 following caucus groups who receive contributions and make expenditures to
6 support or oppose one (1) or more specific candidates or slates of candidates
7 for nomination or election, or a committee:
8 1. House Democratic caucus campaign committee;
9 2. House Republican caucus campaign committee;
10 3. Senate Democratic caucus campaign committee;
11 4. Senate Republican caucus campaign committee; or
12 5. Subdivisions of the state executive committee of a minor political party,
13 which serve the same function as the above-named committees, as
14 determined by regulations promulgated by the registry;
15 (c) "Political issues committee," which means three (3) or more persons joining
16 together to advocate or oppose a constitutional amendment or public question
17 which appears on the ballot if that committee receives or expends money in
18 excess of one thousand dollars ($1,000);
19 (d) "Permanent committee," which means a group of individuals, including an
20 association, committee, or organization, other than a campaign committee,
21 political issues committee, inaugural committee, caucus campaign committee,
22 or party executive committee, which is established as, or intended to be, a
23 permanent organization having as a primary purpose expressly advocating the
24 election or defeat of one (1) or more clearly identified candidates, slates of
25 candidates, or political parties, which functions on a regular basis throughout
26 the year;
27 (e) An executive committee of a political party; and
UNOFFICIAL COPY 21 RS BR 1691
Page 43 of 68
XXXX Jacketed
1 (f) "Inaugural committee," which means one (1) or more persons who receive
2 contributions and make expenditures in support of inauguration activities for
3 any candidate or slate of candidates elected to any state, county, city, or
4 district office;
5 (4) "Contributing organization" means a group which merely contributes to candidates,
6 slates of candidates, campaign committees, caucus campaign committees, or
7 executive committees from time to time from funds derived solely from within the
8 group, and which does not solicit or receive funds from sources outside the group
9 itself. However, any contributions made by the groups in excess of one hundred
10 dollars ($100) shall be reported to the registry;
11 (5) "Testimonial affair" means an affair held in honor of a person who holds or who is
12 or was a candidate for nomination or election to a political office in this state
13 designed to raise funds for any purpose not charitable, religious, or educational;
14 (6) "Contribution" means any:
15 (a) Payment, distribution, loan, deposit, or gift of money or other thing of value,
16 to a candidate, his or her agent, a slate of candidates, its authorized agent, a
17 committee, or contributing organization. As used in this subsection, "loan"
18 shall include a guarantee, endorsement, or other form of security where the
19 risk of nonpayment rests with the surety, guarantor, or endorser, as well as
20 with a committee, contributing organization, candidate, slate of candidates, or
21 other primary obligor. No person shall become liable as surety, endorser, or
22 guarantor for any sum in any one (1) election which, when combined with all
23 other contributions the individual makes to a candidate, his or her agent, a
24 slate of candidates, its agent, a committee, or a contributing organization,
25 exceeds the contribution limits provided in KRS 121.150;
26 (b) Payment by any person other than the candidate, his or her authorized
27 treasurer, a slate of candidates, its authorized treasurer, a committee, or a 
UNOFFICIAL COPY 21 RS BR 1691
Page 44 of 68
XXXX Jacketed
1 contributing organization, of compensation for the personal services of
2 another person which are rendered to a candidate, slate of candidates,
3 committee, or contributing organization, or for inauguration activities;
4 (c) Goods, advertising, or services with a value of more than one hundred dollars
5 ($100) in the aggregate in any one (1) election which are furnished to a
6 candidate, slate of candidates, committee, or contributing organization or for
7 inauguration activities without charge, or at a rate which is less than the rate
8 normally charged for the goods or services; or
9 (d) Payment by any person other than a candidate, his or her authorized treasurer,
10 a slate of candidates, its authorized treasurer, a committee, or contributing
11 organization for any goods or services with a value of more than one hundred
12 dollars ($100) in the aggregate in any one (1) election which are utilized by a
13 candidate, slate of candidates, committee, or contributing organization, or for
14 inauguration activities;
15 (7) Notwithstanding the foregoing meanings of "contribution," the word shall not be
16 construed to include:
17 (a) Services provided without compensation by individuals volunteering a portion
18 or all of their time on behalf of a candidate, a slate of candidates, committee,
19 or contributing organization;
20 (b) A loan of money by any financial institution doing business in Kentucky made
21 in accordance with applicable banking laws and regulations and in the
22 ordinary course of business; or
23 (c) An independent expenditure by any individual or permanent committee;
24 (8) "Candidate" means any person who has received contributions or made
25 expenditures, has appointed a campaign treasurer, or has given his or her consent
26 for any other person to receive contributions or make expenditures with a view to
27 bringing about his or her nomination or election to public office, except federal 
UNOFFICIAL COPY 21 RS BR 1691
Page 45 of 68
XXXX Jacketed
1 office;
2 (9) "Slate of candidates" means:
3 (a) Between the time a certificate or petition of nomination has been filed for a
4 candidate for the office of Governor under KRS 118.365 and the time the
5 candidate designates a running mate for the office of Lieutenant Governor
6 under KRS 118.126, a slate of candidates consists of the candidate for the
7 office of Governor; and
8 (b) After that candidate has designated a running mate under KRS 118.126, that
9 same slate of candidates consists of that same candidate for the office of
10 Governor and the candidate's running mate for the office of Lieutenant
11 Governor. Unless the context requires otherwise, any provision of law that
12 applies to a candidate shall also apply to a slate of candidates;
13 (10) "Knowingly" means, with respect to conduct or to a circumstance described by a
14 statute defining an offense, that a person is aware or should have been aware that
15 his or her conduct is of that nature or that the circumstance exists;
16 (11) "Fundraiser" means an individual who directly solicits and secures contributions on
17 behalf of a candidate or slate of candidates for a statewide-elected state office or an
18 office in a jurisdiction with a population in excess of two hundred thousand
19 (200,000) residents;
20 (12) "Independent expenditure" means the expenditure of money or other things of value
21 for a communication which expressly advocates the election or defeat of a clearly
22 identified candidate or slate of candidates, and which is made without any
23 coordination, consultation, or cooperation with any candidate, slate of candidates,
24 campaign committee, or any authorized person acting on behalf of any of them, and
25 which is not made in concert with, or at the request or suggestion of any candidate,
26 slate of candidates, campaign committee, or any authorized person acting on behalf
27 of any of them;
UNOFFICIAL COPY 21 RS BR 1691
Page 46 of 68
XXXX Jacketed
1 (13) "Electronic reporting" means the use of technology, having electrical, digital,
2 magnetic, wireless, optical, electromagnetic, or similar capabilities, by which an
3 individual or other entity submits, compiles, or transmits campaign finance reports
4 to the registry, or by which the registry receives, stores, analyzes, or discloses the
5 reports;
6 (14) "Security procedure" means a procedure employed for the purpose of verifying that
7 an electronic signature, record, or performance is that of a specific person or for
8 detecting changes or errors in the information in an electronic record. The term
9 includes a procedure that requires the use of algorithms or other codes, identifying
10 words or numbers, encryption, or callback or other acknowledgment procedures;
11 (15) "Electronic signature" means an electronic sound, symbol, or process attached to or
12 logically associated with a record and executed or adopted by a person with the
13 intent to sign the record;
14 (16) "Filer" means any candidate, a slate of candidates, committee, or other individual or
15 entity required to submit financial disclosure reports to the registry; and
16 (17) "Filer-side software" means software provided to or used by the filer that enables
17 transmittal of financial reports to the registry.
18 Section 23. KRS 121.150 is amended to read as follows:
19 (1) No contribution shall be made or received, directly or indirectly, other than an
20 independent expenditure, to support inauguration activities or to support or defeat a
21 candidate, slate of candidates, constitutional amendment, or public question which
22 will appear on the ballot in an election, except through the duly appointed campaign
23 manager, or campaign treasurer of the candidate, slate of candidates, or registered
24 committee. Any person making an independent expenditure, shall report these
25 expenditures when the expenditures by that person exceed five hundred dollars
26 ($500) in the aggregate in any one (1) election, on a form provided or using a format
27 approved by the registry and shall sign a statement on the form, under penalty of 
UNOFFICIAL COPY 21 RS BR 1691
Page 47 of 68
XXXX Jacketed
1 perjury, that the expenditure was an actual independent expenditure and that there
2 was no prior communication with the campaign on whose behalf it was made.
3 (2) Except as provided in KRS 121.180(10), the solicitation from and contributions by
4 campaign committees, caucus campaign committees, political issues committees,
5 permanent committees, and party executive committees to any religious, charitable,
6 civic, eleemosynary, or other causes or organizations established primarily for the
7 public good is expressly prohibited; except that it shall not be construed as a
8 violation of this section for a candidate or a slate of candidates to contribute to
9 religious, civic, or charitable groups.
10 (3) No candidate, slate of candidates, committee, or contributing organization, nor
11 anyone acting on their behalf, shall accept any anonymous contribution in excess of
12 one hundred dollars ($100), and all anonymous contributions in excess of one
13 hundred dollars ($100) shall be returned to the donor, if the donor can be
14 determined. If no donor is found, the contribution shall escheat to the state. No
15 candidate, slate of candidates, committee, or contributing organization, nor anyone
16 acting on their behalf shall accept anonymous contributions in excess of two
17 thousand dollars ($2,000) in the aggregate in any one (1) election. Anonymous
18 contributions in excess of two thousand dollars ($2,000) in the aggregate which are
19 received in any one (1) election shall escheat to the state.
20 (4) No candidate, slate of candidates, committee, or contributing organization, nor
21 anyone on their behalf, shall accept a cash contribution in excess of one hundred
22 dollars ($100) in the aggregate from each contributor in any one (1) election. No
23 candidate, slate of candidates, committee, or contributing organization, nor anyone
24 on their behalf, shall accept a cashier's check or money order in excess of the
25 maximum cash contribution limit unless the instrument clearly identifies both the
26 payor and the payee. A contribution made by cashier's check or money order which
27 identifies both the payor and payee shall be treated as a contribution made by check 
UNOFFICIAL COPY 21 RS BR 1691
Page 48 of 68
XXXX Jacketed
1 for purposes of the contribution limits contained in this section. No person shall
2 make a cash contribution in excess of one hundred dollars ($100) in the aggregate in
3 any one (1) election to a candidate, slate of candidates, committee, or contributing
4 organization, nor anyone on their behalf.
5 (5) No candidate, slate of candidates, committee, contributing organization, nor anyone
6 on their behalf, shall accept any contribution in excess of one hundred dollars
7 ($100) from any person who shall not become eighteen (18) years of age on or
8 before the day of the next general election.
9 (6) Except as provided in subsection (22) of this section, no candidate, slate of
10 candidates, campaign committee, nor anyone acting on their behalf, shall accept a
11 contribution of more than two thousand dollars ($2,000)as indexed for inflation
12 every odd-numbered year using the preceding year's percent increase in the non13 seasonally adjusted annual average Consumer Price Index for all Urban Consumers
14 (CPI-U), U.S. City Average, All Items, for that year as published by the United
15 States Bureau of Labor Statistics and rounded to the nearest hundred dollars, from
16 any person, permanent committee, or contributing organization in any one (1)
17 election. No person, permanent committee, or contributing organization shall
18 contribute more than two thousand dollars ($2,000) as indexed for inflation every
19 odd-numbered year using the preceding year's percent increase in the non-seasonally
20 adjusted annual average Consumer Price Index for all Urban Consumers (CPI-U),
21 U.S. City Average, All Items, as published by the United States Bureau of Labor
22 Statistics and rounded to the nearest hundred dollars, to any one (1) candidate,
23 campaign committee, nor anyone acting on their behalf, in any one (1) election.
24 (7) Permanent committees or contributing organizations affiliated by bylaw structure or
25 by registration, as determined by the Registry of Election Finance, shall be
26 considered as one (1) committee for purposes of applying the contribution limits of
27 subsection (6) of this section.
UNOFFICIAL COPY 21 RS BR 1691
Page 49 of 68
XXXX Jacketed
1 (8) No permanent committee shall contribute funds to another permanent committee for
2 the purpose of circumventing contribution limits of subsection (6) of this section.
3 (9) No person shall contribute funds to a permanent committee, political issues
4 committee, or contributing organization for the purpose of circumventing the
5 contribution limits of subsection (6) of this section.
6 (10) No person shall contribute more than two thousand dollars ($2,000)as indexed for
7 inflation every odd-numbered year using the preceding year's percent increase in the
8 non-seasonally adjusted annual average Consumer Price Index for all Urban
9 Consumers (CPI-U), U.S. City Average, All Items, for that year as published by the
10 United States Bureau of Labor Statistics and rounded to the nearest hundred dollars,
11 to a permanent committee or contributing organization in any one (1) year.
12 (11) (a) No person shall contribute more than five thousand dollars ($5,000) to the
13 state executive committee of a political party in any one (1) year. The
14 contribution limit in this paragraph shall not apply to a contribution
15 designated exclusively for a state executive committee's building fund account
16 established under KRS 121.172.
17 (b) No person shall contribute more than five thousand dollars ($5,000) to a
18 subdivision or affiliate of a state political party in any one (1) year.
19 (c) No person shall contribute more than five thousand dollars ($5,000) to a
20 caucus campaign committee in any one (1) year.
21 (12) No person shall make a payment, distribution, loan, advance, deposit, or gift of
22 money to another person to contribute to a candidate, a slate of candidates,
23 committee, contributing organization, or anyone on their behalf. No candidate, slate
24 of candidates, committee, contributing organization, nor anyone on their behalf
25 shall accept a contribution made by one (1) person who has received a payment,
26 distribution, loan, advance, deposit, or gift of money from another person to
27 contribute to a candidate, a slate of candidates, committee, contributing 
UNOFFICIAL COPY 21 RS BR 1691
Page 50 of 68
XXXX Jacketed
1 organization, or anyone on their behalf.
2 (13) Subject to the provisions of subsection (18)[(17)] of this section, no candidate or
3 slate of candidates for nomination to any state, county, city, or district office, nor
4 their campaign committees, nor anyone on their behalf, shall solicit or accept
5 contributions for primary election expenses after the date of the primary. No person
6 other than the candidate or slate of candidates shall contribute for primary election
7 expenses after the date of the primary.
8 (14) Subject to the provisions of subsection (18) of this section, no slate of candidates
9 for nomination for election to the offices of Governor and Lieutenant Governor,
10 nor its campaign committees, nor anyone on their behalf, shall solicit or accept
11 contributions for runoff primary election expenses after the date of the runoff
12 primary. No person other than the slated candidates shall contribute for runoff
13 primary election expenses after the date of the runoff primary.
14 (15)[(14)] Subject to the provisions of subsection (18)[(17)] of this section, no candidate
15 or slate of candidates for any state, county, city, or district office at a regular
16 election, nor their campaign committees, nor anyone on their behalf, shall solicit or
17 accept contributions for regular election expenses after the date of the regular
18 election. No person other than the candidate or slate of candidates shall contribute
19 for regular election expenses after the date of the regular election.
20 (16)[(15)] Subject to the provisions of subsection (18)[(17)] of this section, no candidate
21 or slate of candidates for nomination or election to any state, county, city, or district
22 office, nor their campaign committees, nor anyone on their behalf, shall solicit or
23 accept contributions for special election expenses after the date of the special
24 election. No person other than the candidate or slate of candidates shall contribute
25 for special election expenses after the date of the special election.
26 (17)[(16)] The provisions of subsections (13) and (15)[(13) and (14)] of this section
27 shall apply only to those candidates in a primary or regular election which shall be 
UNOFFICIAL COPY 21 RS BR 1691
Page 51 of 68
XXXX Jacketed
1 conducted subsequent to January 1, 1989. The provisions of subsections (16) and
2 (17)[subsection (15)] of this section shall apply only to those candidates or slates of
3 candidates in a runoff primary or special election which shall be conducted
4 subsequent to January 1, 1993.
5 (18)[(17)] A candidate, slate of candidates, or a campaign committee may solicit and
6 accept contributions after the date of a primary[ election], runoff primary, regular
7 election, or special election to defray necessary expenses that arise after the date of
8 the election associated with election contests, recounts, and recanvasses of a
9 specific election, complaints regarding alleged campaign finance violations that are
10 filed with the registry pertaining to a specific election, or other legal actions
11 pertaining to a specific election to which a candidate, slate of candidates, or
12 campaign committee is a party, and for repayment of debts and obligations owed by
13 the campaign. Reports of contributions received and expenditures made after the
14 date of the specific election shall be made in accordance with KRS 121.180.
15 (19)[(18)] No candidate, slate of candidates, committee, except a political issues
16 committee, or contributing organization, nor anyone on their behalf, shall
17 knowingly accept a contribution from a corporation, directly or indirectly, except to
18 the extent that the contribution is designated to a state executive committee's
19 building fund account established under KRS 121.172.
20 (20)[(19)] Nothing in this section shall be construed to restrict the ability of a
21 corporation to administer its permanent committee insofar as its actions can be
22 deemed not to influence an election as prohibited by KRS 121.025.
23 (21)[(20)] No candidate, slate of candidates, or committee, nor anyone on their behalf,
24 shall solicit a contribution of money or services from a state employee, whether or
25 not the employee is covered by the classified service provisions of KRS Chapter
26 18A. However, it shall not be a violation of this subsection for a state employee to
27 receive a solicitation directed to him as a registered voter in an identified precinct as 
UNOFFICIAL COPY 21 RS BR 1691
Page 52 of 68
XXXX Jacketed
1 part of an overall plan to contact voters not identified as state employees.
2 (22)[(21)] No candidate or slate of candidates for any office in this state shall accept a
3 contribution, including an in-kind contribution, which is made from funds in a
4 federal campaign account. No person shall make a contribution, including an in5 kind contribution, from funds in a federal campaign account to any candidate or
6 slate of candidates for any office in this state.
7 (23)[(22)] It shall be permissible for a married couple to make a contribution with one
8 (1) check that reflects the combined individual contribution limits of each
9 individual spouse per election, as set forth in subsection (6) of this section, for all
10 elections in a calendar year and the following shall be required to be written on the
11 check:
12 (a) The signatures of both spouses on the signature line of the check; and
13 (b) The designation of each contribution amount and the election or elections to
14 which they apply shall be memorialized on the memo line of the check.
15 Section 24. KRS 121.180 is amended to read as follows:
16 (1) (a) Any candidate, slate of candidates, or political issues committee shall be
17 exempt from filing any campaign finance reports required by subsections (3)
18 and (4) of this section if the candidate, slate of candidates, or political issues
19 committee chair files a form prescribed and furnished by the registry stating
20 that currently no contributions have been received and that contributions will
21 not be accepted or expended in excess of three thousand dollars ($3,000) in
22 any one (1) election. A separate form shall be required for each primary,
23 runoff primary, regular, or special election in which the candidate or slate of
24 candidates participates or in which the public question appears on the ballot,
25 unless the candidate, slate of candidates, or political issues committee chair
26 indicates on a request for exemption that the request will be applicable to
27 more than one (1) election. The form shall be filed with the same office with 
UNOFFICIAL COPY 21 RS BR 1691
Page 53 of 68
XXXX Jacketed
1 which a candidate or slate of candidates files nomination papers or, in the case
2 of a political issues committee, with the registry.
3 (b) For a primary, a candidate or slate of candidates shall file a request for
4 exemption not later than the deadline for filing nomination papers and, except
5 as provided in subparagraph 2. of paragraph (c) of this subsection, shall be
6 bound by its terms unless it is rescinded in writing not later than thirty (30)
7 days preceding the primary. For a runoff primary, a slate of candidates shall
8 file its request for exemption not later than five (5) days after the date of the
9 preceding primary and shall be bound by its terms unless rescinded in
10 writing not later than ten (10) days after the date of the preceding primary.
11 For a regular election, a candidate or slate of candidates shall file or rescind in
12 writing a request for exemption not later than sixty (60) days preceding the
13 regular election, except as provided in subparagraph 2. of paragraph (c) of this
14 subsection. For a special election, a candidate or slate of candidates shall file a
15 request for exemption not later than ten (10) days after the candidate or slate
16 of candidates is nominated for a special election and shall be bound by its
17 terms unless it is rescinded in writing not later than thirty (30) days preceding
18 the special election. A political issues committee chair shall file a request for
19 exemption when the committee registers with the registry and shall be bound
20 by its terms unless it is rescinded in writing not later than thirty (30) days
21 preceding the date the issue appears on the ballot.
22 (c) 1. A candidate or slate of candidates that revokes a request for exemption
23 in a timely manner shall file all reports required of a candidate intending
24 to raise or spend in excess of three thousand dollars ($3,000) in an
25 election. To revoke the request for an exemption, the candidate or slate
26 of candidates shall file the appropriate form with the registry not later
27 than the deadline for filing a revocation.
UNOFFICIAL COPY 21 RS BR 1691
Page 54 of 68
XXXX Jacketed
1 2. A candidate or slate of candidates that is exempted from campaign
2 finance reporting requirements pursuant to paragraph (a) of this
3 subsection but who accepts contributions or makes expenditures in
4 excess of the exempted amount in an election, shall file all applicable
5 reports required for the remainder of that election, based upon the
6 amount of contributions or expenditures the candidate or slate of
7 candidates accepts or receives in that election. The filing of applicable
8 required reports by a candidate or slate of candidates after the exempted
9 amount is exceeded shall serve as notice to the registry that the initial
10 exemption has been rescinded. No further notice to the registry shall be
11 required and no penalty for exceeding the initial exempted amount shall
12 be imposed against the candidate or slate of candidates, except for
13 failure to file applicable reports required after the exempted amount is
14 exceeded.
15 (d) Any candidate or slate of candidates that is subject to a June or August filing
16 deadline and that intends to execute a request for exemption shall file the
17 appropriate request for exemption not later than the filing deadline and, except
18 as provided in subparagraph 2. of paragraph (c) of this subsection, shall be
19 bound by its terms unless it is rescinded in writing not later than sixty (60)
20 days preceding the regular election. A candidate or slate of candidates that is
21 covered by this paragraph shall have the same reversion rights as those
22 provided in subparagraph 1. of paragraph (c) of this subsection.
23 (e) Any candidate or slate of candidates that will appear on the ballot in a regular
24 election that has signed a request for exemption for that election may exercise
25 the reversion rights provided in subparagraph 1. of paragraph (c) of this
26 subsection if a candidate or slate of candidates that is subject to a June or
27 August filing deadline subsequently files in opposition to the candidate or 
UNOFFICIAL COPY 21 RS BR 1691
Page 55 of 68
XXXX Jacketed
1 slate of candidates. Except as provided in subparagraph 2. of paragraph (c) of
2 this subsection, a candidate or slate of candidates covered by this paragraph
3 shall comply with the deadline for rescission provided in subparagraph 1. of
4 paragraph (c) of this subsection.
5 (f) Except as provided in subparagraph 2. of paragraph (c) of this subsection, any
6 candidate or slate of candidates that has filed a request for exemption for a
7 regular election that later is opposed by a person who has filed a declaration of
8 intent to receive write-in votes may rescind the request for exemption and
9 exercise the reversion rights provided in subparagraph 1. of paragraph (c) of
10 this subsection.
11 (g) Any candidate or slate of candidates that has filed a request for exemption
12 may petition the registry to determine whether another person is campaigning
13 as a write-in candidate prior to having filed a declaration of intent to receive
14 write-in votes, and, if the registry determines upon a preponderance of the
15 evidence that a person who may later be a write-in candidate is conducting a
16 campaign, the candidate or slate of candidates, except as provided in
17 subparagraph 2. of paragraph (c) of this subsection, may petition the registry
18 to permit the candidate or slate of candidates to exercise the reversion rights
19 provided in subparagraph 1. of paragraph (c) of this subsection.
20 (h) If the opponent of a candidate or slate of candidates is replaced due to his or
21 her withdrawal because of death, disability, or disqualification, the candidate
22 or slate of candidates, except as provided in subparagraph 2. of paragraph (c)
23 of this subsection, may exercise the reversion rights provided in subparagraph
24 1. of paragraph (c) of this subsection not later than fifteen (15) days after the
25 party executive committee nominates a replacement for the withdrawn
26 candidate or slate of candidates.
27 (i) A person intending to be a write-in candidate for any office in a regular or 
UNOFFICIAL COPY 21 RS BR 1691
Page 56 of 68
XXXX Jacketed
1 special election may execute a request for exemption under paragraph (a) of
2 this subsection and shall be bound by its terms unless it is rescinded in writing
3 not later than fifteen (15) days preceding the regular or special election. A
4 person intending to be a write-in candidate who revokes a request for
5 exemption in a timely manner shall file all reports required of a candidate
6 intending to raise or spend in excess of three thousand dollars ($3,000) in an
7 election. Except as provided in subparagraph 2. of paragraph (c) of this
8 subsection, a person intending to be a write-in candidate who revokes a
9 request for exemption shall file the appropriate form with the registry.
10 (j) Except as provided in subparagraph 2. of paragraph (c) of this subsection, the
11 campaign committee of any candidate or slate of candidates that has filed a
12 request for exemption or a political issues committee whose chair has filed a
13 request for exemption shall be bound by its terms unless it is rescinded in a
14 timely manner.
15 (k) 1. Except as provided in subparagraph 2. of paragraph (c) of this
16 subsection, any candidate, slate of candidates, or political issues
17 committee that is exempt from filing campaign finance reports pursuant
18 to paragraph (a), (d), or (i) of this subsection that accepts contributions
19 or makes expenditures, or whose campaign treasurer accepts
20 contributions or makes expenditures, in excess of the applicable limit in
21 any one (1) election without rescinding the request for exemption in a
22 timely manner shall comply with all applicable reporting requirements
23 and, in lieu of other penalties prescribed by law, pay a fine of not less
24 than five hundred dollars ($500).
25 2. Except as provided in subparagraph 2. of paragraph (c) of this
26 subsection, a candidate, slate of candidates, campaign committee, or
27 political issues committee that is exempt from filing campaign finance 
UNOFFICIAL COPY 21 RS BR 1691
Page 57 of 68
XXXX Jacketed
1 reports pursuant to paragraph (a), (d), or (i) of this subsection that
2 knowingly accepts contributions or makes expenditures in excess of the
3 applicable spending limit in any one (1) election without rescinding the
4 request for exemption in a timely manner shall comply with all
5 applicable reporting requirements and shall be guilty of a Class D
6 felony.
7 (2) (a) State and county executive committees, and caucus campaign committees
8 shall make a full report, upon a prescribed form, to the registry, of all money,
9 loans, or other things of value, received from any source, and expenditures
10 authorized, incurred, or made, since the date of the last report, including:
11 1. For each contribution of any amount made by a permanent committee,
12 the name and business address of the permanent committee, the date of
13 the contribution, the amount contributed, and a description of the major
14 business, social, or political interest represented by the permanent
15 committee;
16 2. For other contributions in excess of one hundred dollars ($100), the full
17 name, address, age if less than the legal voting age, the date of the
18 contribution, the amount of the contribution, and the employer and
19 occupation of each contributor. If the contributor is self-employed, the
20 name under which he or she is doing business shall be listed;
21 3. The total amount of cash contributions received during the reporting
22 period; and
23 4. A complete statement of expenditures authorized, incurred, or made.
24 The complete statement of expenditures shall include the name and
25 address of each person to whom an expenditure is made in excess of
26 twenty-five dollars ($25), and the amount, date, and purpose of each
27 expenditure.
UNOFFICIAL COPY 21 RS BR 1691
Page 58 of 68
XXXX Jacketed
1 (b) In addition to the reporting requirements in paragraph (a) of this subsection,
2 the state executive committee of a political party that has established a
3 building fund account under KRS 121.172 shall make a full report, upon a
4 prescribed form, to the registry, of all contributions received from any source,
5 and expenditures authorized, incurred, or made, since the date of the last
6 report for the separate building fund account, including:
7 1. For each contribution of any amount made by a corporation, the name
8 and business address of the corporation, the date of the contribution, the
9 amount contributed, and a description of the major business conducted
10 by the corporation;
11 2. For other contributions in excess of one hundred dollars ($100), the full
12 name and address of the contributor, the date of the contribution, the
13 amount of the contribution, and the employer and occupation of each
14 contributor. If the contributor is self-employed, the name under which he
15 or she is doing business shall be listed;
16 3. The total amount of cash contributions received during the reporting
17 period; and
18 4. A complete statement of expenditures authorized, incurred, or made.
19 The complete statement of expenditures shall include the name and
20 address of each person to whom an expenditure is made in excess of
21 twenty-five dollars ($25), and the amount, date, and purpose of each
22 expenditure.
23 (c) The report required by paragraph (a) of this subsection shall be made on a
24 semiannual basis and shall be received by the registry by January 31 and by
25 July 31. The January report shall cover the period from July 1 to December
26 31. The July report shall cover the period from January 1 to June 30. If an
27 individual gives a reportable contribution to a caucus campaign committee or 
UNOFFICIAL COPY 21 RS BR 1691
Page 59 of 68
XXXX Jacketed
1 to a state or county executive committee with the intention that the
2 contribution or a portion of the contribution go to a candidate or slate of
3 candidates, the name of the contributor and the sum shall be indicated on the
4 committee report. The report required by paragraph (b) of this subsection
5 relating to a state executive committee's building fund account shall be
6 received by the registry within two (2) business days after the close of each
7 calendar quarter. The receipts and expenditures of funds remitted to each
8 political party under KRS 141.071 to 141.073 shall be separately accounted
9 for and reported to the registry in the manner required by KRS 121.230. The
10 separate report may be made a separate section within the report required by
11 this subsection to be received by the registry by January 31.
12 (3) (a) Except for candidates or slates of candidates, campaign committees, or
13 political issues committees exempted from reporting requirements pursuant to
14 subsection (1) of this section, each campaign treasurer of a candidate, slate of
15 candidates, campaign committee, or political issues committee who accepts
16 contributions or expends, expects to accept contributions or expend, or
17 contracts to expend more than three thousand dollars ($3,000) in any one (1)
18 election, and each fundraiser who secures contributions in excess of three
19 thousand dollars ($3,000) in any one (1) election, shall make a full report to
20 the registry, on a form provided or using a format approved by the registry, of
21 all money, loans, or other things of value, received from any source, and
22 expenditures authorized, incurred, and made, since the date of the last report,
23 including:
24 1. For each contribution of any amount made by a permanent committee,
25 the name and business address of the permanent committee, the date of
26 the contribution, the amount contributed, and a description of the major
27 business, social, or political interest represented by the permanent 
UNOFFICIAL COPY 21 RS BR 1691
Page 60 of 68
XXXX Jacketed
1 committee;
2 2. For each contribution in excess of one hundred dollars ($100) made to a
3 candidate or slate of candidates for a statewide-elected state office, or to
4 a campaign committee for a candidate or slate of candidates for a
5 statewide-elected state office, the date, name, address, occupation, and
6 employer of each contributor and the spouse of the contributor or, if the
7 contributor or spouse of the contributor is self-employed, the name
8 under which he or she is doing business, and the amount contributed by
9 each contributor;
10 3. For each contribution in excess of one hundred dollars ($100) made to
11 any candidate or campaign committee other than those specified in
12 subparagraph 2. of this paragraph or a political issues committee, the full
13 name, address, age if less than the legal voting age, the date of the
14 contribution, the amount of the contribution, and the employer and
15 occupation of each other contributor. If the contributor is self-employed,
16 the name under which he or she is doing business shall be listed;
17 4. The total amount of cash contributions received during the reporting
18 period; and
19 5. A complete statement of all expenditures authorized, incurred, or made.
20 The complete statement of expenditures shall include the name, address,
21 and occupation of each person to whom an expenditure is made in
22 excess of twenty-five dollars ($25), and the amount, date, and purpose of
23 each expenditure.
24 (b) Reports of all candidates, slates of candidates, campaign committees, political
25 issues committees, and registered fundraisers shall be made as follows:
26 1. Candidates as defined in KRS 121.015(8), slates of candidates,
27 candidate-authorized and unauthorized campaign committees, political 
UNOFFICIAL COPY 21 RS BR 1691
Page 61 of 68
XXXX Jacketed
1 issues committees, and fundraisers which register in the year before the
2 year an election in which the candidate, a slate of candidates, or public
3 question shall appear on the ballot, shall file financial reports with the
4 registry at the end of the first calendar quarter after persons become
5 candidates or slates of candidates, or following registration of the
6 committee or fundraiser, and each calendar quarter thereafter, ending
7 with the last calendar quarter of that year. Candidates, slates of
8 candidates, committees, and registered fundraisers shall make all reports
9 required by this section during the year in which the election takes place;
10 2. All candidates, slates of candidates, candidate-authorized and
11 unauthorized campaign committees, political issues committees, and
12 registered fundraisers shall make reports on the sixtieth day preceding a
13 regular election, including all previous contributions and expenditures;
14 3. All candidates, slates of candidates, candidate-authorized and
15 unauthorized campaign committees, political issues committees, and
16 registered fundraisers shall make reports on the thirtieth day preceding
17 an election, including all previous contributions and expenditures;
18 4. All candidates, slates of candidates, candidate-authorized and
19 unauthorized campaign committees, political issues committees, and
20 registered fundraisers shall make reports on the fifteenth day preceding
21 the date of the election; and
22 5. All reports to the registry shall cover campaign activity during the entire
23 reporting period and must be received by the registry within two (2)
24 business days after the date the reporting period ends to be deemed
25 timely filed.
26 (4) Except for candidates, slates of candidates, and political issues committees,
27 exempted pursuant to subsection (1)(a) of this section, all candidates, regardless of 
UNOFFICIAL COPY 21 RS BR 1691
Page 62 of 68
XXXX Jacketed
1 funds received or expended, candidate-authorized and unauthorized campaign
2 committees, political issues committees, and registered fundraisers shall make post3 election reports within thirty (30) days after the election. All post-election reports to
4 the registry shall cover campaign activity during the entire reporting period and
5 must be received by the registry within two (2) business days after the date the
6 reporting period ends to be deemed timely filed.
7 (5) In making the preceding reports, the total gross receipts from each of the following
8 categories shall be listed: proceeds from the sale of tickets for events such as
9 testimonial affairs, dinners, luncheons, rallies, and similar fundraising events, mass
10 collections made at the events, and sales of items such as campaign pins, buttons,
11 hats, ties, literature, and similar materials. When any individual purchase or the
12 aggregate purchases of any item enumerated above from a candidate or slate of
13 candidates for a statewide-elected state office or a campaign committee for a
14 candidate or slate of candidates for a statewide-elected state office exceeds one
15 hundred dollars ($100), the purchaser shall be identified by name, address, age, if
16 less than the legal voting age, occupation, and employer and the employer of the
17 spouse of the purchaser or, if the purchaser or the spouse of the purchaser is self18 employed, the name under which he or she is doing business, and the amount of the
19 purchase. When any individual purchase or the aggregate purchases of any item
20 enumerated above from any candidate or campaign committee other than a
21 candidate or slate of candidates for a statewide-elected state office or campaign
22 committee for a candidate or slate of candidates for a statewide-elected state office
23 exceeds one hundred dollars ($100), the purchaser shall be identified by name,
24 address, age if less than the legal voting age, occupation, and employer, or if the
25 purchaser is self-employed, the name under which he or she is doing business, and
26 the amount of the purchase. The lists shall be maintained by the campaign treasurer,
27 political issues committee treasurer, registered fundraiser, or other sponsor for 
UNOFFICIAL COPY 21 RS BR 1691
Page 63 of 68
XXXX Jacketed
1 inspection by the registry for six (6) years following the date of the election.
2 (6) Each permanent committee, except a federally registered permanent committee,
3 inaugural committee, or contributing organization shall make a full report to the
4 registry, on a form provided or using a format approved by the registry, of all
5 money, loans, or other things of value, received by it from any source, and all
6 expenditures authorized, incurred, or made, since the date of the last report,
7 including:
8 (a) For each contribution of any amount made by a permanent committee, the
9 name and business address of the permanent committee, the date of the
10 contribution, the amount contributed, and a description of the major business,
11 social, or political interest represented by the permanent committee;
12 (b) For other contributions in excess of one hundred dollars ($100), the full name,
13 address, age if under the legal voting age, the date of the contribution, the
14 amount of the contribution, and the employer and occupation of each
15 contributor. If the contributor is self-employed, the name under which he or
16 she is doing business shall be listed;
17 (c) An aggregate amount of cash contributions, the amount contributed by each
18 contributor, and the date of each contribution; and
19 (d) A complete statement of all expenditures authorized, incurred, or made,
20 including independent expenditures. This report shall be made by a permanent
21 committee, inaugural committee, or contributing organization to the registry
22 on the last day of the first calendar quarter following the registration of the
23 committee with the registry and on the last day of each succeeding calendar
24 quarter until such time as the committee terminates. A contributing
25 organization shall file a report of contributions received and expenditures on a
26 form provided or using a format approved by the registry not later than the last
27 day of each calendar quarter in which contributions are received or 
UNOFFICIAL COPY 21 RS BR 1691
Page 64 of 68
XXXX Jacketed
1 expenditures are made. All reports to the registry shall be received on or
2 before each filing deadline, and any report received by the registry within two
3 (2) business days after each filing deadline shall be deemed timely filed.
4 (7) If the final statement of a candidate, campaign committee, or political issues
5 committee shows an unexpended balance of contributions, continuing debts and
6 obligations, or an expenditure deficit, the campaign treasurer shall file with the
7 registry a supplemental statement of contributions and expenditures not more than
8 thirty (30) days after the deadline for filing the final statement. Subsequent
9 supplemental statements shall be filed annually, to be received by the registry by
10 December 1 of each year, until the account shows no unexpended balance,
11 continuing debts and obligations, expenditures, or deficit, or until the year before
12 the candidate or a slate of candidates seeks to appear on the ballot for the same
13 office for which the funds in the campaign account were originally contributed, in
14 which case the candidate or a slate of candidates shall file the supplemental annual
15 report by December 1 of that year or at the end of the first calendar quarter of that
16 year after the candidate or slate of candidates files nomination papers for the next
17 year's primary or regular election. All post-election reports to the registry shall
18 cover campaign activity during the entire reporting period and must be received by
19 the registry within two (2) business days after the date the reporting period ends to
20 be deemed timely filed. All contributions shall be subject to KRS 121.150 as of the
21 date of the election in which the candidate appeared on the ballot.
22 (8) All reports filed under the provisions of this chapter shall be a matter of public
23 record open to inspection by any member of the public immediately upon receipt of
24 the report by the registry.
25 (9) A candidate or slate of candidates is relieved of the duty personally to file reports
26 and keep records of receipts and expenditures if the candidate or slate states in
27 writing or on forms provided by the registry that:
UNOFFICIAL COPY 21 RS BR 1691
Page 65 of 68
XXXX Jacketed
1 (a) Within five (5) business days after personally receiving any contributions, the
2 candidate or slate of candidates shall surrender possession of the contributions
3 to the treasurer of their principal campaign committee without expending any
4 of the proceeds thereof. No contributions shall be commingled with the
5 candidate's or slated candidates' personal funds or accounts. Contributions
6 received by check, money order, or other written instrument shall be endorsed
7 directly to the campaign committee and shall not be cashed or redeemed by
8 the candidate;
9 (b) The candidate or slate of candidates shall not make any unreimbursed
10 expenditure for the campaign, except that this paragraph does not preclude a
11 candidate or slate from making an expenditure from personal funds to the
12 designated principal campaign committee, which shall be reported by the
13 committee as a contribution received; and
14 (c) The waiver shall continue in effect as long as the candidate or slate of
15 candidates complies with the conditions under which it was granted.
16 (10) No candidate, slate of candidates, campaign committee, political issues committee,
17 or contributing organization shall use or permit the use of contributions or funds
18 solicited or received for the person or in support of or opposition to a public issue
19 which will appear on the ballot to further the candidacy of the person for a different
20 public office, to support or oppose a different public issue, or to further the
21 candidacy of any other person for public office; except that nothing in this
22 subsection shall be deemed to prohibit a candidate or slate of candidates from using
23 funds in the campaign account to purchase admission tickets for any fundraising
24 event or testimonial affair for another candidate or slate of candidates if the amount
25 of the purchase does not exceed two hundred dollars ($200) per event or affair. Any
26 funds or contributions solicited or received by or on behalf of a candidate, slate of
27 candidates, or any committee, which has been organized in whole or in part to 
UNOFFICIAL COPY 21 RS BR 1691
Page 66 of 68
XXXX Jacketed
1 further any candidacy for the same person or to support or oppose the same public
2 issue, shall be deemed to have been solicited or received for the current candidacy
3 or for the election on the public issue if the funds or contributions are solicited or
4 received at any time prior to the regular election for which the candidate, slate of
5 candidates, or public issue is on the ballot. Any unexpended balance of funds not
6 otherwise obligated for the payment of expenses incurred to further a political issue
7 or the candidacy of a person shall, in whole or in part, at the election of the
8 candidate or committee, escheat to the State Treasury, be returned pro rata to all
9 contributors, or, in the case of a partisan candidate, be transferred to a caucus
10 campaign committee, or to the state or county executive committee of the political
11 party of which the candidate is a member except that a candidate, committee, or an
12 official may retain the funds to further the same public issue or to seek election to
13 the same office or may donate the funds to any charitable, nonprofit, or educational
14 institution recognized under Section 501(c)(3) of the United States Internal Revenue
15 Code of 1986, as amended, and any successor thereto.
16 (11) If adequate and appropriate agency funds are available to implement this subsection,
17 electronic reporting shall be made available by the registry to all candidates, slates
18 of candidates, committees, contributing organizations, registered fundraisers, and
19 persons making independent expenditures. The electronic report submitted to the
20 registry shall be the official campaign finance report for audit and other legal
21 purposes, whether mandated or filed by choice.
22 (12) Filers not required to file reports electronically, as set forth in this section, are
23 strongly encouraged to do so voluntarily.
24 (13) The date that an electronic or on-line report shall be deemed to have been filed with
25 the registry shall be the date on which it is received by the registry.
26 (14) All electronic or online filers shall affirm, under penalty of perjury, that the report
27 filed with the registry is complete and accurate.
UNOFFICIAL COPY 21 RS BR 1691
Page 67 of 68
XXXX Jacketed
1 (15) Filers who submit electronic campaign finance reports which are not readable, or
2 cannot be copied, or are not accompanied by any requisite paper copy shall be
3 deemed to not be in compliance with the requirements set forth in this section.
4 (16) Beginning with the primary scheduled in calendar year 2020, and for each
5 subsequent election scheduled thereafter, reports required to be submitted to the
6 registry involving candidates, slates of candidates, committees, contributing
7 organizations, and independent expenditures shall be reported electronically.
8 (17) (a) On each paper and electronic form that the registry supplies for the reports
9 required under subsections (2), (3), and (6) of this section, the registry shall
10 include an entry reading, "No change since last report."
11 (b) If a person or entity that is required to report under subsection (2), (3), or (6)
12 of this section has received no money, loans, or other things of value from any
13 source since the date of its last report and has not authorized, incurred, or
14 made any expenditures since that date, the person or entity may check or
15 otherwise designate the entry that reads, "No change since last report." A
16 person or entity designating this entry in a report shall state the balance carried
17 forward from the last report but need not specify receipts or expenditures in
18 further detail.
19 Section 25. The following KRS sections are repealed:
20 118.551 Definition of political party.
21 118.561 Presidential preference primary election.
22 118.571 Voter qualification.
23 118.581 Nomination of candidates by State Board of Elections.
24 118.591 Nomination of presidential preference primary candidate by petition --
25 Qualification of candidate through filing of notice of candidacy.
26 118.601 Notification of nominees by Secretary of State -- Order of names on ballot --
27 Certification of candidates. 
UNOFFICIAL COPY 21 RS BR 1691
Page 68 of 68
XXXX Jacketed
1 118.611 Candidates required to make deposit with Secretary of State -- Refund --
2 Escheat to Commonwealth.
3 118.621 Secretary of State to place candidates' names on ballot -- Provisions for casting
4 uncommitted vote.
5 118.631 Certification of results of preference primary.
6 118.641 Distribution of authorized delegate vote among party candidates.
7 118.651 Notice to political party's national committee.