Printed on recycled paper
130th MAINE LEGISLATURE
SECOND REGULAR SESSION-2022
Legislative Document No. 1982
S.P. 713 In Senate, February 16, 2022
An Act To Protect Consumers' Privacy by Giving Them Greater
Control of Their Data and To Establish Consumer Protections
Regarding Small Dollar Loans
Approved for introduction by a majority of the Legislative Council pursuant to Joint Rule
203.
Reference to the Committee on Innovation, Development, Economic Advancement and
Business suggested and ordered printed.
DAREK M. GRANT
Secretary of the Senate
Presented by Senator RAFFERTY of York.
Cosponsored by Representative TALBOT ROSS of Portland and
Senators: CURRY of Waldo, DAUGHTRY of Cumberland, President JACKSON of
Aroostook, Representative: ROBERTS of South Berwick.
Page 1 - 130LR2336(01)
1 Be it enacted by the People of the State of Maine as follows:
2 PART A
3 Sec. A-1. 10 MRSA c. 1057 is enacted to read:
4 CHAPTER 1057
5 MAINE CONSUMER PRIVACY ACT
6 §9601. Short title
7 This chapter may be known and cited as "the Maine Consumer Privacy Act."
8 §9602. Definitions
9 As used in this chapter, unless the context otherwise indicates, the following terms
10 have the following meanings.
11 1. Aggregate consumer information. "Aggregate consumer information" means
12 information that relates to a group or category of consumers, from which individual
13 consumer identities have been removed, that is not linked or reasonably linkable to any
14 consumer or household, including via a device. "Aggregate consumer information" does
15 not mean one or more individual consumer records that have been deidentified.
16 2. Biometric information. "Biometric information" means an individual's
17 physiological, biological or behavioral characteristics, including an individual's
18 deoxyribonucleic acid, or DNA, that can be used, singly or in combination with each other
19 or with other identifying data, to establish the individual's identity. "Biometric
20 information" includes, but is not limited to, imagery of the iris, retina, fingerprint, face,
21 hand, palm, vein patterns and voice recordings, from which an identifier template, such as
22 a faceprint, a minutiae template or a voiceprint, can be extracted and keystroke patterns or
23 rhythms, gait patterns or rhythms and sleep, health or exercise data that contain identifying
24 information.
25 3. Business. "Business" means:
26 A. A sole proprietorship, partnership, limited liability company, corporation,
27 association or other legal entity that is organized or operated for the profit or financial
28 benefit of its shareholders or other owners that collects consumers' personal
29 information or on the behalf of which that information is collected and that alone or
30 jointly with others determines the purposes and means of the processing of consumers'
31 personal information, that does business in the State and that satisfies one or more of
32 the following thresholds:
33 (1) Has annual gross revenues in excess of $25,000,000, as adjusted pursuant to
34 section 9619, subsection 1, paragraph E;
35 (2) Alone or in combination annually buys, receives for the business's commercial
36 purposes, sells or shares for commercial purposes alone or in combination the
37 personal information of 50,000 or more consumers, households or devices; or
38 (3) Derives 50% or more of its annual revenues from selling consumers' personal
39 information; or
Page 2 - 130LR2336(01)
1 B. An entity that controls or is controlled by a business as defined in paragraph A and
2 that shares common branding with the business. For the purposes of this paragraph,
3 "control" or "controlled" means ownership of, or the power to vote, more than 50% of
4 the outstanding shares of any class of voting security of a business; control in any
5 manner over the election of a majority of the directors or of individuals exercising
6 similar functions; or the power to exercise a controlling influence over the management
7 of a company, and "common branding" means a shared name, service mark or
8 trademark.
9 4. Business purpose. "Business purpose" means the use of personal information for
10 the business's or a service provider's operational purposes or other purposes of which
11 consumers are given notice, if the use of personal information is reasonably necessary and
12 proportionate to achieve the operational purpose for which the personal information was
13 collected or processed or for another operational purpose that is compatible with the context
14 in which the personal information was collected. "Business purpose" includes:
15 A. Auditing related to a current interaction with the consumer and concurrent
16 transactions, including, but not limited to, counting so-called ad impressions to unique
17 visitors, verifying positioning and quality of ad impressions and auditing compliance
18 with this specification and other standards;
19 B. Detecting security incidents, protecting against malicious, deceptive, fraudulent or
20 illegal activity and prosecuting those responsible for that activity;
21 C. Debugging to identify and repair errors that impair existing intended functionality;
22 D. Short-term, transient use if the personal information is not disclosed to another 3rd
23 party and is not used to build a profile about a consumer or otherwise alter an individual
24 consumer's experience outside the current interaction, including, but not limited to, the
25 contextual customization of ads shown as part of the same interaction;
26 E. Performing services on behalf of the business or service provider, including
27 maintaining or servicing accounts, providing customer service, processing or fulfilling
28 orders and transactions, verifying consumer information, processing payments,
29 providing financing, providing advertising or marketing services, providing analytic
30 services or providing similar services on behalf of the business or service provider;
31 F. Undertaking internal research for technological development and demonstration;
32 and
33 G. Undertaking activities to verify or maintain the quality or safety of a service or
34 device that is owned, manufactured, manufactured for or controlled by the business
35 and to improve, upgrade or enhance the service or device that is owned, manufactured,
36 manufactured for or controlled by the business.
37 5. Collect. "Collect" means to buy, rent, gather, obtain, receive or access any personal
38 information pertaining to a consumer by any means and includes receiving information
39 from the consumer, either actively or passively, or by observing the consumer's behavior.
40 6. Commercial purposes. "Commercial purposes" means to advance a person's
41 commercial or economic interests, such as by inducing another person to buy, rent, lease,
42 join, subscribe to, provide or exchange products, goods, property, information or services
43 or enabling or effecting, directly or indirectly, a commercial transaction. "Commercial 
Page 3 - 130LR2336(01)
44 purposes" does not include engaging in speech that state or federal courts have recognized
45 as noncommercial speech, including political speech and journalism.
3 7. Consumer. "Consumer" means a natural person who is a resident of the State.
4 8. Deidentified. "Deidentified" means cannot reasonably identify, relate to, describe,
5 be capable of being associated with or be linked, directly or indirectly, to a particular
6 consumer, if a business that uses deidentified information:
7 A. Has implemented technical safeguards that prohibit reidentification of the consumer
8 to whom the information may pertain;
9 B. Has implemented business processes that specifically prohibit reidentification of
10 the information;
11 C. Has implemented business processes to prevent inadvertent release of deidentified
12 information; and
13 D. Makes no attempt to reidentify the information.
14 9. Designated method for submitting requests. "Designated method for submitting
15 requests" means a mailing address, e-mail address, Internet web page, Internet web portal,
16 toll-free telephone number or other applicable contact information by which a consumer
17 may submit a request or direction under this chapter and any consumer-friendly means of
18 contacting a business as approved by the Attorney General pursuant to section 9619.
19 10. Device. "Device" means any physical object that is capable of connecting to the
20 Internet, directly or indirectly, or to another device that is capable of connecting to the
21 Internet, directly or indirectly.
22 11. Homepage. "Homepage" means the introductory page of an Internet website and
23 any Internet web page where personal information is collected. In the case of an online
24 service, such as a mobile application, "homepage" means the application's platform page
25 or download page, a link within the application, such as from the application's
26 configuration, "about," "information" or settings page, and any other location that allows
27 consumers to review the form required by section 9610, subsection 1, including, but not
28 limited to, before downloading the application.
29 12. Person. "Person" means an individual, proprietorship, firm, partnership, joint
30 venture, syndicate, business trust, company, corporation, limited liability company,
31 association, committee and any other organization or group of persons acting in concert.
32 13. Personal information. "Personal information":
33 A. Means information that identifies, relates to, describes, is reasonably capable of
34 being associated with or could reasonably be linked, directly or indirectly, with a
35 particular consumer or household and includes, but is not limited to:
36 (1) Identifiers such as a real name, alias, mailing address, signature, telephone
37 number, unique identifier, online identifier, Internet protocol address, e-mail
38 address, account name, social security number, driver's license number, nondriver
39 identification card number, passport number or other similar identifiers;
40 (2) Physical characteristics or description, insurance policy number, bank account
41 number, credit card number, debit card number or any other financial information,
42 medical information or health information;
1
2
Page 4 - 130LR2336(01)
1 (3) Characteristics of protected classifications under state or federal law;
2 (4) Commercial information, including records of personal property, products or
3 services purchased, obtained or considered, or other purchasing or consuming
4 histories or tendencies;
5 (5) Biometric information;
6 (6) Internet or other electronic network activity information, including, but not
7 limited to, browsing history, search history and information regarding a consumer's
8 interaction with an Internet website, application or advertisement;
9 (7) Geolocation data;
10 (8) Audio, electronic, visual, thermal, olfactory or similar information;
11 (9) Professional or employment-related information;
12 (10) Education information, defined as information that is not publicly available
13 personally identifiable information as defined in the Family Educational Rights
14 and Privacy Act of 1974, 20 United States Code, Section 1232g; and
15 (11) Inferences drawn from any of the information identified in this paragraph to
16 create a profile about a consumer reflecting the consumer's preferences,
17 characteristics, psychological trends, predispositions, behavior, attitudes,
18 intelligence, abilities and aptitudes;
19 B. Does not include publicly available information. For purposes of this paragraph,
20 "publicly available" means information that is lawfully made available from federal,
21 state or local government records. "Publicly available" does not mean biometric
22 information collected by a business about a consumer without the consumer's
23 knowledge; and
24 C. Does not include consumer information that is deidentified or aggregate consumer
25 information.
26 14. Probabilistic identifier. "Probabilistic identifier" means the identification of a
27 consumer or a device to a degree of certainty of more probable than not based on any
28 categories of personal information included in, or similar to, the categories enumerated in
29 the definition of "personal information" in subsection 13.
30 15. Processing. "Processing" means any operation or set of operations that is
31 performed on personal information or on sets of personal information, whether or not by
32 automated means.
33 16. Reidentify. "Reidentify" means the process of reversal of deidentification
34 techniques, including, but not limited to, the addition of specific pieces of information or
35 data elements that can, individually or in combination, be used to uniquely identify an
36 individual or usage of any statistical method, contrivance, computer software or other
37 means that have the effect of associating deidentified information with a specific
38 identifiable individual.
39 17. Research. "Research" means scientific, systematic study and observation,
40 including basic research or applied research that is in the public interest and that adheres to
41 all other applicable ethics and privacy laws or studies conducted in the public interest in
42 the area of public health. Research with personal information that may have been collected 
Page 5 - 130LR2336(01)
43 from a consumer in the course of the consumer's interactions with a business's service or
44 device for other purposes must be:
3 A. Compatible with the business purpose for which the personal information was
4 collected;
5 B. Subsequently deidentified, or deidentified and in the aggregate, such that the
6 information cannot reasonably identify, relate to, describe, be capable of being
7 associated with or be linked, directly or indirectly, to a particular consumer;
8 C. Made subject to technical safeguards that prohibit reidentification of the consumer
9 to whom the information may pertain;
10 D. Subject to business processes that specifically prohibit reidentification of the
11 information;
12 E. Made subject to business processes to prevent inadvertent release of deidentified
13 information;
14 F. Protected from any reidentification attempts;
15 G. Used solely for research purposes that are compatible with the context in which the
16 personal information was collected;
17 H. Not be used for any commercial purpose; and
18 I. Subjected by the business conducting the research to additional security controls that
19 limit access to the research data to only those individuals in a business as are necessary
20 to carry out the research purpose.
21 18. Sell. "Sell" means sell, rent, release, disclose, disseminate, make available, transfer
22 or otherwise communicate orally, in writing or by electronic or other means, a consumer's
23 personal information by a business to another business or a 3rd party for monetary or other
24 valuable consideration.
25 For purposes of this chapter, a business does not sell personal information when:
26 A. A consumer uses or directs the business to intentionally disclose personal
27 information or uses the business to intentionally interact with a 3rd party, if the 3rd
28 party does not also sell the personal information, unless that disclosure would be
29 consistent with the provisions of this chapter. An intentional interaction occurs when
30 the consumer intends to interact with the 3rd party via one or more deliberate
31 interactions. Hovering over, muting, pausing or closing a given piece of content does
32 not constitute a consumer's intent to interact with a 3rd party;
33 B. The business uses or shares an identifier for a consumer who has opted out of the
34 sale of the consumer's personal information for the purposes of alerting 3rd parties that
35 the consumer has opted out of the sale of the consumer's personal information;
36 C. The business uses or shares with a service provider personal information of a
37 consumer that is necessary to perform a business purpose if both of the following
38 conditions are met:
39 (1) The business has provided notice of that information being used or shared in
40 its terms and conditions consistent with section 9610; and
1
2
Page 6 - 130LR2336(01)
1 (2) The service provider does not further collect, sell or use the personal
2 information of the consumer except as necessary to perform the business purpose;
3 or
4 D. The business transfers to a 3rd party the personal information of a consumer as an
5 asset that is part of a merger, acquisition, bankruptcy or other transaction in which the
6 3rd party assumes control of all or part of the business, if that information is used or
7 shared consistent with the provisions of sections 9605 and 9606. If a 3rd party
8 materially alters how it uses or shares the personal information of a consumer in a
9 manner that is materially inconsistent with the promises made at the time of collection,
10 it shall provide prior notice of the new or changed practice to the consumer. The notice
11 must be sufficiently prominent and robust to ensure that existing consumers can easily
12 exercise their choices consistent with section 9607. This paragraph does not authorize
13 a business to make material, retroactive privacy policy changes or make other changes
14 in its privacy policy in a manner that would violate the Maine Unfair Trade Practices
15 Act or chapter 206.
16 19. Service. "Service" means work or labor, including work or labor furnished in
17 connection with the sale or repair of goods.
18 20. Service provider. "Service provider" means a sole proprietorship, partnership,
19 limited liability company, corporation, association or other legal entity that is organized or
20 operated for the profit or financial benefit of its shareholders or other owners, that processes
21 information on behalf of a business and to which the business discloses a consumer's
22 personal information for a business purpose pursuant to a written contract, if the contract
23 prohibits the entity receiving the information from retaining, using or disclosing the
24 personal information for any purpose other than for the specific purpose of performing the
25 services specified in the contract for the business, or as otherwise permitted by this chapter,
26 including retaining, using or disclosing the personal information for a commercial purpose
27 other than providing the services specified in the contract with the business.
28 21. Third party. "Third party" means a person that is not any of the following:
29 A. The business that collects personal information from consumers under this chapter;
30 or
31 B. A person to whom the business discloses a consumer's personal information for a
32 business purpose pursuant to a written contract, if the contract:
33 (1) Prohibits the person receiving the personal information from:
34 (a) Selling the personal information;
35 (b) Retaining, using or disclosing the personal information for any purpose
36 other than for the specific purpose of performing the services specified in the
37 contract, including retaining, using or disclosing the personal information for
38 a commercial purpose other than providing the services specified in the
39 contract; or
40 (c) Retaining, using or disclosing the information outside of the direct business
41 relationship between the person and the business; and
Page 7 - 130LR2336(01)
1 (2) Includes a certification made by the person receiving the personal information
2 that the person understands the restrictions in subparagraph (1) and will comply
3 with them.
4 22. Unique identifier. "Unique identifier" means a persistent identifier that can be
5 used to recognize a consumer, a household or a device that is linked to a consumer or
6 household, over time and across different services, including, but not limited to: a device
7 identifier; an Internet protocol address; cookies, beacons, pixel tags, mobile ad identifiers
8 or similar technology; customer number, unique pseudonym or user alias; telephone
9 numbers; or other forms of persistent or probabilistic identifiers that can be used to identify
10 a particular consumer or device. For purposes of this subsection, "household" means a
11 custodial parent or guardian and any minor children of whom the parent or guardian has
12 custody.
13 23. Verifiable consumer request. "Verifiable consumer request" means a request
14 that is made by a consumer or by a consumer on behalf of the consumer's minor child, or
15 by a natural person or a person registered with the Secretary of State authorized by the
16 consumer to act on the consumer's behalf, and that the business can reasonably verify,
17 pursuant to rules adopted by the Attorney General pursuant to section 9619, subsection 1,
18 paragraph G to be the consumer about whom the business has collected personal
19 information.
20 §9603. Consumer's right to request personal information
21 1. Request. A consumer has the right to request that a business that collects a
22 consumer's personal information disclose to that consumer the categories of personal
23 information the business has collected regarding the consumer.
24 2. Categories to be collected; purposes. A business that collects a consumer's
25 personal information shall, at or before the point of collection, inform consumers as to the
26 categories of personal information to be collected and the purposes for which the categories
27 of personal information may be used. A business may not collect additional categories of
28 personal information or use personal information collected for additional purposes without
29 providing the consumer with notice consistent with this section.
30 3. Verifiable consumer request required. A business shall provide the information
31 specified in subsection 1 to a consumer only upon receipt of a verifiable consumer request.
32 A business is not obligated to provide information to the consumer pursuant to this section
33 and sections 9604, 9605 and 9606 if the business cannot verify, pursuant to this subsection
34 and rules adopted by the Attorney General pursuant to section 9619, subsection 1,
35 paragraph G, that the consumer making the request is the consumer about whom the
36 business has collected information or is a person authorized by the consumer to act on such
37 consumer's behalf.
38 4. Disclose and deliver personal information. A business that receives a verifiable
39 consumer request from a consumer to access personal information shall promptly take steps
40 to disclose and deliver, free of charge to the consumer, the personal information required
41 by this section. The information may be delivered by mail or electronically, and, if
42 provided electronically, the information must be in a portable and, to the extent technically
43 feasible, readily useable format that allows the consumer to transmit this information to
44 another entity without hindrance. A business may provide personal information to a 
Page 8 - 130LR2336(01)
45 consumer at any time, but is not required to provide personal information to a consumer
46 more than twice in a 12-month period.
3 5. Single, one-time transaction. This section does not require a business to retain
4 any personal information collected for a single, one-time transaction, if such information
5 is not sold or retained by the business or to reidentify or otherwise link information that is
6 not maintained in a manner that would be considered personal information.
7 §9604. Consumer's right to request deletion of personal information
8 1. Request to delete. A consumer has the right to request that a business delete any
9 personal information about the consumer that the business has collected from the consumer.
10 2. Disclose right to request deletion. A business that collects personal information
11 about a consumer shall disclose, pursuant to section 9609, the consumer's right to request
12 the deletion of the consumer's personal information.
13 3. Verifiable consumer request. A business that receives a verifiable consumer
14 request from a consumer to delete the consumer's personal information pursuant to
15 subsection 1 shall delete the consumer's personal information from its records and direct
16 any service providers to delete the consumer's personal information from their records. A
17 business is not obligated to provide information to the consumer pursuant to this section
18 and sections 9603, 9605 and 9606 if the business cannot verify, pursuant to this subsection
19 and rules adopted by the Attorney General pursuant to section 9619, subsection 1,
20 paragraph G, that the consumer making the request is the consumer about whom the
21 business has collected information or is a person authorized by the consumer to act on such
22 consumer's behalf.
23 4. Deletion not required. A business or a service provider is not required to comply
24 with a consumer's request to delete the consumer's personal information if it is necessary
25 for the business or service provider to maintain the consumer's personal information in
26 order to:
27 A. Complete the transaction for which the personal information was collected, fulfill
28 the terms of a written warranty or product recall conducted in accordance with federal
29 law, provide a good or service requested by the consumer or reasonably anticipated
30 within the context of a business's ongoing business relationship with the consumer or
31 otherwise perform a contract between the business and the consumer;
32 B. Detect security incidents, protect against malicious, deceptive, fraudulent or illegal
33 activity or prosecute those responsible for that activity;
34 C. Debug to identify and repair errors that impair existing intended functionality;
35 D. Exercise free speech, ensure the right of another consumer to exercise that
36 consumer's right of free speech or exercise another right provided for by law;
37 E. Comply with a warrant under Title 16, chapter 3, subchapter 9-A, 10 or 11;
38 F. Engage in public or peer-reviewed scientific, historical or statistical research in the
39 public interest that adheres to all other applicable ethics and privacy laws, when the
40 business's deletion of the information is likely to render impossible or seriously impair
41 the achievement of such research, if the consumer has provided informed consent;
1
2
Page 9 - 130LR2336(01)
1 G. To enable solely internal uses that are reasonably aligned with the expectations of
2 the consumer based on the consumer's relationship with the business;
3 H. Comply with a legal obligation; or
4 I. Otherwise use the consumer's personal information, internally, in a lawful manner
5 that is compatible with the context in which the consumer provided the information.
6 §9605. Consumer's right to disclosure of information
7 1. Request. A consumer has the right to request that a business that collects personal
8 information about the consumer disclose to the consumer the following:
9 A. The categories of sources from which the personal information is collected;
10 B. The business purpose or commercial purpose for collecting or selling personal
11 information;
12 C. The categories of 3rd parties with whom the business shares personal information;
13 and
14 D. The specific pieces of personal information it has collected about that consumer.
15 2. Verifiable consumer request. A business that collects personal information about
16 a consumer shall disclose to the consumer, pursuant to section 9609, subsection 1,
17 paragraph C, the information specified in subsection 1 upon receipt of a verifiable
18 consumer request from the consumer. A business is not obligated to provide information
19 to the consumer pursuant to this section and sections 9603, 9604 and 9606 if the business
20 cannot verify, pursuant to this subsection and rules adopted by the Attorney General
21 pursuant to section 9619, subsection 1, paragraph G, that the consumer making the request
22 is the consumer about whom the business has collected information or is a person
23 authorized by the consumer to act on such consumer's behalf.
24 3. Information disclosed. A business that collects personal information about
25 consumers shall disclose, pursuant to section 9609, subsection 1, paragraph E,
26 subparagraph (2):
27 A. The categories of personal information it has collected about consumers;
28 B. The categories of sources from which the personal information is collected;
29 C. The business purpose or commercial purpose for collecting or selling personal
30 information;
31 D. The categories of 3rd parties with whom the business shares personal information;
32 and
33 E. That a consumer has the right to request the specific pieces of personal information
34 the business has collected about that consumer.
35 4. Actions not required. This section does not require a business to:
36 A. Retain any personal information about a consumer collected for a single, one-time
37 transaction if, in the ordinary course of business, that information about the consumer
38 is not retained; or
39 B. Reidentify or otherwise link any data that, in the ordinary course of business, is not
40 maintained in a manner that would be considered personal information.
Page 10 - 130LR2336(01)
1 §9606. Consumer's rights when information sold or disclosed
2 1. Request. A consumer has the right to request that a business that sells the
3 consumer's personal information, or that discloses it for a business purpose, disclose to that
4 consumer:
5 A. The categories of personal information about the consumer that the business
6 collected;
7 B. The categories of personal information about the consumer that the business sold
8 and the categories of 3rd parties to whom the personal information was sold, by
9 category or categories of personal information for each category of 3rd parties to whom
10 the personal information was sold; and
11 C. The categories of personal information about the consumer that the business
12 disclosed for a business purpose.
13 2. Verifiable consumer request. A business that sells personal information about a
14 consumer, or that discloses a consumer's personal information for a business purpose, shall
15 disclose, pursuant to section 9609, subsection 1, paragraph D, the information specified in
16 subsection 1 to the consumer upon receipt of a verifiable consumer request from the
17 consumer. A business is not obligated to provide information to the consumer pursuant to
18 this section and sections 9603, 9604 and 9605 if the business cannot verify, pursuant to this
19 subsection and rules adopted by the Attorney General pursuant to section 9619, subsection
20 1, paragraph G, that the consumer making the request is the consumer about whom the
21 business has collected information or is a person authorized by the consumer to act on such
22 consumer's behalf.
23 3. Information disclosed. A business that sells consumers' personal information, or
24 that discloses consumers' personal information for a business purpose, shall disclose,
25 pursuant to section 9609, subsection 1, paragraph E, subparagraph (2):
26 A. The category or categories of consumers' personal information it has sold, or if the
27 business has not sold consumers' personal information, it shall disclose that fact; and
28 B. The category or categories of consumers' personal information it has disclosed for
29 a business purpose, or if the business has not disclosed the consumers' personal
30 information for a business purpose, it shall disclose that fact.
31 4. Sale by 3rd party. A 3rd party may not sell personal information about a consumer
32 that has been sold to the 3rd party by a business unless the consumer has received explicit
33 notice and is provided an opportunity to exercise the right to opt out pursuant to section
34 9607.
35 5. Violations. A person covered by this section that violates any of the restrictions set
36 forth in this chapter is liable for the violations. A business that discloses personal
37 information to a person covered by this section in compliance with this section is not liable
38 under this chapter if the person receiving the personal information uses it in violation of
39 the restrictions set forth in this chapter, if, at the time of disclosing the personal information,
40 the business does not have actual knowledge, or reason to believe, that the person intends
41 to commit such a violation.
42 §9607. Consumer's right to prohibit sale; right to opt out
Page 11 - 130LR2336(01)
1 1. Right to opt out. A consumer has the right, at any time, to direct a business that
2 sells personal information about the consumer to 3rd parties not to sell the consumer's
3 personal information. This right may be referred to as "the right to opt out."
4 2. Notice of right to opt out. A business that sells consumers' personal information
5 to 3rd parties shall provide notice to consumers, pursuant to section 9610, subsection 1,
6 that this information may be sold and that consumers have the right to opt out of the sale
7 of their personal information.
8 3. Right to opt in; consumer's age. Notwithstanding subsection 1, a business may
9 not sell the personal information of a consumer if the business has actual knowledge that
10 the consumer is less than 16 years of age, unless the consumer, in the case of a consumer
11 at least 13 years of age and less than 16 years of age, or the consumer's parent or guardian,
12 in the case of a consumer less than 13 years of age, has affirmatively authorized the sale of
13 the consumer's personal information. A business that willfully disregards the consumer's
14 age is deemed to have had actual knowledge of the consumer's age.
15 The affirmative authorization required under this subsection may be referred to as "the right
16 to opt in."
17 4. Sale prohibited. A business that has received direction from a consumer not to sell
18 the consumer's personal information or, in the case of a minor consumer's personal
19 information has not received consent to sell the minor consumer's personal information, is
20 prohibited, pursuant to section 9610, subsection 1, paragraph D, from selling the
21 consumer's personal information after its receipt of the consumer's direction, unless the
22 consumer subsequently provides express authorization for the sale of the consumer's
23 personal information.
24 §9608. Business practices
25 1. Discrimination based on exercise of rights prohibited. A business may not
26 discriminate against a consumer because the consumer exercised any of the consumer's
27 rights under this chapter, including, but not limited to, by:
28 A. Denying goods or services to the consumer;
29 B. Charging different prices or rates for goods or services, including through the use
30 of discounts or other benefits or imposing penalties;
31 C. Providing a different level or quality of goods or services to the consumer; or
32 D. Suggesting that the consumer will receive a different price or rate for goods or
33 services or a different level or quality of goods or services.
34 Nothing in this subsection prohibits a business from charging a consumer a different price
35 or rate, or from providing a different level or quality of goods or services to the consumer,
36 if that difference is reasonably related to the value provided to the business by the
37 consumer's personal information.
38 2. Financial incentives. A business may offer financial incentives, including
39 payments to consumers as compensation, for the collection of personal information, the
40 sale of personal information or the deletion of personal information. A business may also
41 offer a different price, rate, level or quality of goods or services to the consumer if that
42 price or difference is directly related to the value provided to the business by the consumer's
43 personal information.
Page 12 - 130LR2336(01)
1 A. A business that offers any financial incentives pursuant to this subsection shall
2 notify consumers of the financial incentives pursuant to section 9609.
3 B. A business may enter a consumer into a financial incentive program only if the
4 consumer gives the business prior opt in consent pursuant to section 9609 that clearly
5 describes the material terms of the financial incentive program, and which may be
6 revoked by the consumer at any time.
7 C. A business may not use financial incentive practices that are unjust, unreasonable,
8 coercive or usurious in nature.
9 §9609. Requests for and disclosure of information
10 1. Requests; disclosure and delivery. In order to comply with sections 9603, 9604,
11 9605, 9606 and 9608, a business shall, in a form that is reasonably accessible to consumers:
12 A. Make available to consumers 2 or more designated methods for submitting requests
13 for information required to be disclosed pursuant to sections 9605 and 9606, including,
14 at a minimum, a toll-free telephone number.
15 (1) A business that operates exclusively online and has a direct relationship with
16 a consumer from whom it collects personal information is required to provide only
17 an e-mail address for submitting requests for information required to be disclosed
18 pursuant to sections 9605 and 9606.
19 (2) If the business maintains an Internet website, the business shall make the
20 Internet website available to consumers to submit requests for information required
21 to be disclosed pursuant to sections 9605 and 9606;
22 B. Disclose and deliver the required information to a consumer free of charge within
23 45 days of receiving a verifiable consumer request from the consumer. The business
24 shall promptly take steps to determine whether the request is a verifiable consumer
25 request, but this does not extend the business's duty to disclose and deliver the
26 information within 45 days of receipt of the consumer's request. The time period to
27 provide the required information may be extended once by an additional 45 days when
28 reasonably necessary, if the consumer is provided notice of the extension within the
29 first 45-day period. The disclosure must cover at least the 12-month period preceding
30 the business's receipt of the verifiable consumer request and must be made in writing
31 and delivered through the consumer's account with the business, if the consumer
32 maintains an account with the business, or by mail or electronically at the consumer's
33 option if the consumer does not maintain an account with the business, in a readily
34 useable format that allows the consumer to transmit this information from one entity
35 to another entity without hindrance. The business may require authentication of the
36 consumer that is reasonable in light of the nature of the personal information requested,
37 but may not require the consumer to create an account with the business in order to
38 make a verifiable consumer request. If the consumer maintains an account with the
39 business, the business may require the consumer to submit the request through that
40 account;
41 C. For purposes of section 9605, subsection 2:
42 (1) To identify the consumer, associate the information provided by the consumer
43 in the verifiable consumer request to any personal information previously collected
44 by the business about the consumer; and
Page 13 - 130LR2336(01)
1 (2) Identify by category or categories the personal information collected about the
2 consumer in the preceding 12 months by reference to the enumerated category or
3 categories in subsection 3 that most closely describe the personal information
4 collected;
5 D. For purposes of section 9606, subsection 2:
6 (1) Identify the consumer and associate the information provided by the consumer
7 in the verifiable consumer request to any personal information previously collected
8 by the business about the consumer;
9 (2) Identify by category or categories the personal information of the consumer
10 that the business sold in the preceding 12 months by reference to the enumerated
11 category or categories in subsection 3 that most closely describe the personal
12 information, and provide the categories of 3rd parties to whom the consumer's
13 personal information was sold in the preceding 12 months by reference to the
14 enumerated category or categories in subsection 3 that most closely describe the
15 personal information sold. The business shall disclose the information in a list that
16 is separate from a list generated for the purposes of subparagraph (3); and
17 (3) Identify by category or categories the personal information of the consumer
18 that the business disclosed for a business purpose in the preceding 12 months by
19 reference to the enumerated category or categories in subsection 3 that most closely
20 describes the personal information, and provide the categories of 3rd parties to
21 whom the consumer's personal information was disclosed for a business purpose
22 in the preceding 12 months by reference to the enumerated category or categories
23 in subsection 3 that most closely describe the personal information disclosed. The
24 business shall disclose the information in a list that is separate from a list generated
25 for the purposes of subparagraph (2);
26 E. Disclose the following information in its online privacy policy or policies if the
27 business has an online privacy policy or policies and in any description of consumers'
28 privacy rights specific to the State or, if the business does not maintain those policies,
29 on its Internet website and update that information at least once every 12 months:
30 (1) A description of a consumer's rights pursuant to sections 9603, 9604, 9605,
31 9606 and 9608 and one or more designated methods for submitting requests;
32 (2) For purposes of section 9605, subsection 3, a list of the categories of personal
33 information it has collected about consumers in the preceding 12 months by
34 reference to the enumerated category or categories in subsection 3 that most closely
35 describe the personal information collected;
36 (3) For purposes of section 9606, subsection 3, paragraphs A and B, 2 separate
37 lists:
38 (a) A list of the categories of personal information it has sold about consumers
39 in the preceding 12 months by reference to the enumerated category or
40 categories in subsection 3 that most closely describe the personal information
41 sold or, if the business has not sold consumers' personal information in the
42 preceding 12 months, the business shall disclose that fact; and
43 (b) A list of the categories of personal information it has disclosed about
44 consumers for a business purpose in the preceding 12 months by reference to 
Page 14 - 130LR2336(01)
45 the enumerated category or categories in subsection 3 that most closely
46 describe the personal information disclosed or, if the business has not disclosed
47 consumers' personal information for a business purpose in the preceding 12
48 months, the business shall disclose that fact;
5 (4) In the case of a business that sells or discloses deidentified patient information
6 not subject to this chapter pursuant to section 9612, subsection 1, paragraph D,
7 subparagraph (1), whether the business sells or discloses deidentified patient
8 information derived from patient information and, if so, whether that patient
9 information was deidentified pursuant to one or more of the following:
10 (a) The deidentification methodology described in of 45 Code of Federal
11 Regulations, Section 164.514(b)(1); and
12 (b) The deidentification methodology described in 45 Code of Federal
13 Regulations, Section 164.514(b)(2);
14 F. Ensure that all individuals responsible for handling consumer inquiries about the
15 business's privacy practices or the business's compliance with this chapter are informed
16 of all requirements in sections 9603, 9604, 9605, 9606 and 9608 and this section and
17 how to direct consumers to exercise their rights under those sections; and
18 G. Use any personal information collected from the consumer in connection with the
19 business's verification of the consumer's request solely for the purposes of verification.
20 2. Limit. A business is not obligated to provide the information required by sections
21 9605 and 9606 to the same consumer more than twice in a 12-month period.
22 3. Categories of personal information. The categories of personal information
23 required to be disclosed pursuant to sections 9605 and 9606 must follow the enumerated
24 categories set out in the definition of "personal information" in section 9602, subsection
25 13.
26 §9610. Posted links and information
27 1. Required links and information. A business that is required to comply with
28 section 9607 shall, in a form that is reasonably accessible to consumers:
29 A. Provide a clear and conspicuous link on the business's homepage, titled "Do Not
30 Sell My Personal Information," to an Internet webpage that enables a consumer, or a
31 person authorized by the consumer, to opt out of the sale of the consumer's personal
32 information. A business may not require a consumer to create an account in order to
33 direct the business not to sell the consumer's personal information;
34 B. Include a description of a consumer's rights pursuant to section 9607, along with a
35 separate link to the "Do Not Sell My Personal Information" Internet webpage in:
36 (1) Its online privacy policy or policies if the business has an online privacy policy
37 or policies; and
38 (2) Any description of consumers' privacy rights specific to this State;
39 C. Ensure that all individuals responsible for handling consumer inquiries about the
40 business's privacy practices or the business's compliance with this chapter are informed
41 of all requirements in section 9607 and this section and how to direct consumers to
42 exercise their rights under section 9607 and this section;
1
2
3
4
Page 15 - 130LR2336(01)
1 D. For consumers who exercise their right to opt out of the sale of their personal
2 information under section 9607, refrain from selling personal information collected by
3 the business about the consumer;
4 E. For a consumer who has opted out of the sale of the consumer's personal information
5 under section 9607, respect the consumer's decision to opt out for at least 12 months
6 before requesting that the consumer authorize the sale of the consumer's personal
7 information; and
8 F. Use any personal information collected from a consumer in connection with the
9 submission of the consumer's opt out request pursuant to section 9607 solely for the
10 purposes of complying with the opt out request.
11 2. Location of links and information. Nothing in this chapter may be construed to
12 require a business to comply with this chapter by including the required links and text on
13 the homepage that the business makes available to the public generally, if the business
14 maintains a separate and additional homepage that is dedicated to consumers in this State
15 and that includes the required links and text, and the business takes reasonable steps to
16 ensure that consumers in this State are directed to the homepage for consumers in this State
17 and not the homepage made available to the public generally.
18 3. Authorized person. A consumer may authorize another person solely to opt out of
19 the sale of the consumer's personal information on the consumer's behalf, and a business
20 shall comply with an opt out request received from a person authorized by the consumer to
21 act on the consumer's behalf, pursuant to rules adopted by the Attorney General pursuant
22 to section 9619.
23 §9611. Unaffected ability; continued application
24 1. Business ability not restricted. The obligations imposed on businesses by this
25 chapter do not restrict a business's, service provider's or 3rd party's ability to:
26 A. Comply with federal, state or local laws;
27 B. Comply with a civil, criminal or regulatory inquiry, investigation, subpoena or
28 summons by federal, state or local authorities;
29 C. Cooperate with law enforcement agencies concerning conduct or activity that the
30 business, service provider or 3rd party reasonably and in good faith believes may
31 violate federal, state or local law;
32 D. Exercise or defend legal claims;
33 E. Collect, use, retain, sell or disclose consumer information that is deidentified or
34 aggregate consumer information; or
35 F. Collect or sell a consumer's personal information if every aspect of that commercial
36 conduct takes place wholly outside of the State. For purposes of this chapter,
37 commercial conduct takes place wholly outside of the State if the business, service
38 provider or 3rd party collected that information while the consumer was outside of the
39 State, no part of the sale of the consumer's personal information occurred in the State
40 and no personal information collected while the consumer was in the State is sold. This
41 paragraph does not permit a business, service provider or 3rd party to store, including
42 on a device, personal information about a consumer when the consumer is in the State 
Page 16 - 130LR2336(01)
43 and then collect that personal information when the consumer and stored personal
44 information is outside of the State.
3 2. Violation of evidentiary privilege. The obligations imposed on businesses by
4 sections 9605 to 9610 do not apply when compliance by the business with those sections
5 would violate an evidentiary privilege under the law of this State and do not prevent a
6 business from providing the personal information of a consumer to a person covered by an
7 evidentiary privilege under the law of this State as part of a privileged communication.
8 3. Credit reporting agency information. Except for section 9614, this chapter does
9 not apply to an activity involving the collection, maintenance, disclosure, sale,
10 communication or use of any personal information bearing on a consumer's credit
11 worthiness, credit standing, credit capacity, character, general reputation, personal
12 characteristics or mode of living by a consumer reporting agency, as defined in 15 United
13 States Code, Section 1681a(f), by a furnisher of information, as set forth in 15 United States
14 Code, Section 1681s-2, who provides information for use in a consumer report, as defined
15 in 15 United States Code, Section 1681a(d), and by a user of a consumer report as set forth
16 in 15 United States Code, Section 1681b.
17 This subsection applies only to the extent that such activity involving the collection,
18 maintenance, disclosure, sale, communication or use of such information by that agency,
19 furnisher or user is subject to regulation under the Fair Credit Reporting Act, 15 United
20 States Code, Section 1681 et seq., and the information is not used, communicated, disclosed
21 or sold except as authorized by the Fair Credit Reporting Act.
22 Personal information covered by this subsection may be the subject of an action under
23 section 9614.
24 4. Gramm-Leach-Bliley Act information. Except for section 9614, this chapter does
25 not apply to personal information collected, processed, sold or disclosed pursuant to the
26 federal Gramm-Leach-Bliley Act, Public Law 106-102, and implementing regulations.
27 Personal information covered by this subsection may be the subject of an action under
28 section 9614.
29 5. Driver's Privacy Protection Act information. Except for section 9614, this
30 chapter does not apply to personal information collected, processed, sold or disclosed
31 pursuant to the Driver's Privacy Protection Act of 1994, 18 United States Code, Section
32 2721 et seq.
33 Personal information covered by this subsection may be the subject of an action under
34 section 9614.
35 6. Vehicle information or ownership information. Section 9607 does not apply to
36 vehicle information or ownership information retained or shared between a new motor
37 vehicle dealer, as defined in section 1171, subsection 12, and the vehicle's manufacturer,
38 as defined in section 1171, subsection 10, if the vehicle information or ownership
39 information is shared for the purpose of effectuating, or in anticipation of effectuating, a
40 vehicle repair covered by a vehicle warranty or a recall conducted pursuant to 49 United
41 States Code, Sections 30118 to 30120, as long as the new motor vehicle dealer or vehicle
42 manufacturer with which that vehicle information or ownership information is shared does
43 not sell, share or use that information for any other purpose.
44 For purposes of this subsection:
1
2
Page 17 - 130LR2336(01)
1 A. "Vehicle information" means the vehicle information number, make, model, year
2 and odometer reading; and
3 B. "Ownership information" means the name or names of the registered owner or
4 owners and the contact information for the owner or owners.
5 7. Extensions; reasons for not taking action; manifestly unfounded or excessive
6 requests. Notwithstanding a business's obligations to respond to and honor consumer
7 rights requests pursuant to this chapter:
8 A. A time period for a business to respond to any verifiable consumer request may be
9 extended by up to 90 additional days where necessary, taking into account the
10 complexity and number of the requests. The business shall inform the consumer of any
11 such extension within 45 days of receipt of the request, together with the reasons for
12 the delay;
13 B. If the business does not take action on the request of the consumer, the business
14 shall inform the consumer, without delay and at the latest within the time period
15 permitted for response by this section, of the reasons for not taking action and any
16 rights the consumer may have to appeal the decision to the business; and
17 C. If requests from a consumer are manifestly unfounded or excessive, in particular
18 because of their repetitive character, a business may either charge a reasonable fee,
19 taking into account the administrative costs of providing the information or
20 communication or taking the action requested, or refuse to act on the request and notify
21 the consumer of the reason for refusing the request. The business shall bear the burden
22 of demonstrating that any verifiable consumer request is manifestly unfounded or
23 excessive.
24 8. Disclosure to service provider. A business that discloses personal information to
25 a service provider is not liable under this chapter if the service provider receiving the
26 personal information uses it in violation of the restrictions set forth in this chapter, if, at the
27 time of disclosing the personal information, the business does not have actual knowledge
28 or reason to believe that the service provider intends to commit such a violation. A service
29 provider is likewise not liable under this chapter for the obligations of a business for which
30 it provides services as set forth in this chapter.
31 9. Ordinary course of business. This chapter may not be construed to require a
32 business to collect personal information that it would not otherwise collect in the ordinary
33 course of its business, retain personal information for longer than it would otherwise retain
34 such information in the ordinary course of its business or reidentify or otherwise link
35 information that is not maintained in a manner that would be considered personal
36 information.
37 10. Rights and freedoms of other consumers. The rights afforded to consumers and
38 the obligations imposed on a business under this chapter may not adversely affect the rights
39 and freedoms of other consumers.
40 §9612. Medical and health information
41 1. Nonapplicability. This chapter does not apply to any of the following:
42 A. Health care information governed by Title 22, section 1711-C or protected health
43 care information that is collected by a covered entity or business associate governed by 
Page 18 - 130LR2336(01)
44 the privacy, security and breach notification rules issued by the United States
45 Department of Health and Human Services, 45 Code of Federal Regulations, Parts 160
46 and 164, established pursuant to the federal Health Insurance Portability and
47 Accountability Act of 1996, Public Law 104-191, and the federal Health Information
48 Technology for Economic and Clinical Health Act, Title XIII of the federal American
49 Recovery and Reinvestment Act of 2009, Public Law 111-5;
7 B. A health care practitioner governed by Title 22, section 1711-C or a covered entity
8 governed by the privacy, security and breach notification rules issued by the United
9 States Department of Health and Human Services, 45 Code of Federal Regulations,
10 Parts 160 and 164, established pursuant to the federal Health Insurance Portability and
11 Accountability Act of 1996, Public Law 104-191, to the extent the provider or covered
12 entity maintains, uses and discloses patient information in the same manner as health
13 care information or protected health care information as described in paragraph A;
14 C. A business associate of a covered entity governed by the privacy, security and data
15 breach notification rules issued by the United States Department of Health and Human
16 Services, 45 Code of Federal Regulations, Parts 160 and 164, established pursuant to
17 the federal Health Insurance Portability and Accountability Act of 1996, Public Law
18 104-191, and the federal Health Information Technology for Economic and Clinical
19 Health Act, Title XIII of the federal American Recovery and Reinvestment Act of
20 2009, Public Law 111-5, to the extent that the business associate maintains, uses and
21 discloses patient information in the same manner as health care information or
22 protected health care information as described in paragraph A;
23 D. Information that meets both of the following conditions:
24 (1) It is deidentified in accordance with the requirements for deidentification set
25 forth in 45 Code of Federal Regulations, Section 164.514; and
26 (2) It is derived from patient information that was originally collected, created,
27 transmitted or maintained by an entity regulated by the federal Health Insurance
28 Portability and Accountability Act of 1996 or 45 Code of Federal Regulations, Part
29 46.
30 Information that met the requirements of subparagraphs (1) and (2) but is subsequently
31 reidentified is no longer eligible for the exemption in this paragraph and is subject to
32 applicable federal and state data privacy and security laws, including, but not limited
33 to, the federal Health Insurance Portability and Accountability Act of 1996 and this
34 chapter; and
35 E. Information that is collected, used or disclosed in research, as defined in 45 Code
36 of Federal Regulations, Section 164.501, including, but not limited to, a clinical trial,
37 and that is conducted in accordance with applicable ethics, confidentiality, privacy and
38 security rules of 45 Code of Federal Regulations, Parts 46 and 164, good clinical
39 practice guidelines issued by the International Council for Harmonisation of Technical
40 Requirements for Pharmaceuticals for Human Use, or successor organization, or
41 human subject protection requirements of the United States Food and Drug
42 Administration.
43 2. Terms. For purposes of this section:
1
2
3
4
5
6
Page 19 - 130LR2336(01)
1 A. "Business associate" has the same meaning as in 45 Code of Federal Regulations,
2 Section 160.103;
3 B. "Covered entity" has the same meaning as in 45 Code of Federal Regulations,
4 Section 160.103;
5 C. "Health care information" means any individually identifiable information, in
6 electronic or physical form, in possession of or derived from a provider of health care,
7 health care service plan, pharmaceutical company or contractor regarding a patient's
8 medical history, mental or physical condition or treatment. "Individually identifiable"
9 means that the medical information includes or contains any element of personal
10 identifying information sufficient to allow identification of the individual, such as the
11 patient's name, mailing address, e-mail address, telephone number or social security
12 number, or other information that, alone or in combination with other publicly available
13 information, reveals the individual's identity;
14 D. "Health care practitioner" means a person licensed by this State to provide or
15 otherwise lawfully providing health care or a partnership or corporation made up of
16 health care practitioners or an officer, employee, agent or contractor of a health care
17 practitioner acting in the course and scope of employment, agency or contract related
18 to or supportive of the provision of health care to individuals;
19 E. "Identifiable private information" has the same meaning as in 45 Code of Federal
20 Regulations, Section 46.102;
21 F. "Individually identifiable health information" has the same meaning as in 45 Code
22 of Federal Regulations, Section 160.103;
23 G. "Patient information" means identifiable private information, protected health
24 information, individually identifiable health information or health care information;
25 and
26 H. "Protected health information" has the same meaning as in 45 Code of Federal
27 Regulations, Section 160.103.
28 §9613. Reidentification of information
29 1. Reidentification prohibited; exceptions. A business or other person may not
30 reidentify, or attempt to reidentify, information that has met the requirements of section
31 9612, subsection 1, paragraph D, except for one or more of the following purposes:
32 A. Treatment, payment or health care operations conducted by a covered entity or
33 business associate acting on behalf of, and at the written direction of, the covered entity.
34 For purposes of this paragraph, "treatment," "payment" and "health care operations"
35 have the same meanings as in 45 Code of Federal Regulations, Section 164.501 and
36 "covered entity" and "business associate" have the same meanings as in 45 Code of
37 Federal Regulations, Section 160.103;
38 B. Public health activities or purposes as described in 45 Code of Federal Regulations,
39 Section 164.512;
40 C. Research, as defined in 45 Code of Federal Regulations, Section 164.501, that is
41 conducted in accordance with 45 Code of Federal Regulations, Part 46;
Page 20 - 130LR2336(01)
1 D. Pursuant to a contract where the lawful holder of the deidentified information that
2 met the requirements of section 9612, subsection 1, paragraph D expressly engages a
3 person or entity to attempt to reidentify the deidentified information in order to conduct
4 testing, analysis or validation of deidentification, or related statistical techniques, if the
5 contract bans any other use or disclosure of the reidentified information and requires
6 the return or destruction of the information that was reidentified upon completion of
7 the contract; and
8 E. If otherwise required by law.
9 2. Information reidentified subject to data privacy and security laws. In
10 accordance with section 9612, subsection 1, paragraph D, information reidentified pursuant
11 to this section is subject to applicable federal and state data privacy and security laws
12 including, but not limited to, the federal Health Insurance Portability and Accountability
13 Act of 1996 and this chapter.
14 3. Statements; further disclosure. A contract for the sale or license of deidentified
15 information that has met the requirements of section 9612, subsection 1, paragraph D, when
16 one of the parties is a person residing or doing business in the State, must include the
17 following, or substantially similar, provisions:
18 A. A statement that the deidentified information being sold or licensed includes
19 deidentified patient information;
20 B. A statement that reidentification, and attempted reidentification, of the deidentified
21 information by the purchaser or licensee of the information is prohibited pursuant to
22 this section; and
23 C. A requirement that, unless otherwise required by law, the purchaser or licensee of
24 the deidentified information may not further disclose the deidentified information to
25 any 3rd party unless the 3rd party is contractually bound by the same or stricter
26 restrictions and conditions.
27 §9614. Civil action based on violation
28 1. Civil action based on violation. A consumer whose nonencrypted and nonredacted
29 personal information is subject to an unauthorized access and exfiltration, theft or
30 disclosure as a result of the business's violation of the duty to implement and maintain
31 reasonable security procedures and practices appropriate to the nature of the information to
32 protect the personal information may institute a civil action for any of the following:
33 A. To recover damages in an amount not less than $100 and not greater than $750 per
34 consumer per incident or actual damages, whichever is greater;
35 B. Injunctive or declaratory relief; and
36 C. Any other relief the court determines proper.
37 In assessing the amount of statutory damages, the court shall consider any one or more of
38 the relevant circumstances presented by any of the parties to the case, including, but not
39 limited to, the nature and seriousness of the misconduct, the number of violations, the
40 persistence of the misconduct, the length of time over which the misconduct occurred, the
41 willfulness of the defendant's misconduct and the defendant's assets, liabilities and net
42 worth.
Page 21 - 130LR2336(01)
1 2. Notice of alleged violation; opportunity to cure; exception. Actions pursuant to
2 this section may be brought by a consumer if, prior to initiating any action against a
3 business for statutory damages on an individual or classwide basis, a consumer provides a
4 business 30 days' written notice identifying the specific provisions of this chapter the
5 consumer alleges have been or are being violated. In the event a cure is possible, if within
6 the 30 days the business actually cures the noticed violation and provides the consumer an
7 express written statement that the violations have been cured and that no further violations
8 will occur, no action for individual statutory damages or classwide statutory damages may
9 be initiated against the business. Notice is not required prior to an individual consumer's
10 initiating an action solely for actual pecuniary damages suffered as a result of the alleged
11 violation of this chapter. If a business continues to violate this chapter in breach of the
12 express written statement provided to the consumer under this section, the consumer may
13 initiate an action against the business to enforce the written statement and may pursue
14 statutory damages for each breach of the express written statement, as well as any other
15 violation of the chapter that postdates the written statement.
16 3. Defined violations; no basis for private right of action under other law. The
17 cause of action established by this section applies only to violations as defined in subsection
18 1 and may not be based on violations of any other section of this chapter. Nothing in this
19 chapter may be interpreted to serve as the basis for a private right of action under any other
20 law. This section may not be construed to relieve any party from any duties or obligations
21 imposed under other law or the United States Constitution or the Constitution of Maine.
22 §9615. Compliance; civil penalties
23 1. Guidance of Attorney General. Any business or 3rd party may seek the opinion
24 of the Attorney General for guidance on how to comply with the provisions of this chapter.
25 2. Failure to cure; civil penalties. A business violates this chapter if it fails to cure
26 any alleged violation within 30 days after being notified of alleged noncompliance. A
27 business, service provider or other person that violates this chapter is subject to an
28 injunction and liable for a civil penalty of not more than $2,500 for each violation or $7,500
29 for each intentional violation, which must be assessed and recovered in a civil action
30 brought in the name of the State by the Attorney General. The civil penalties provided for
31 in this section must be exclusively assessed and recovered in a civil action brought in the
32 name of the State by the Attorney General.
33 3. Consumer Privacy Fund. Any civil penalty assessed for a violation of this chapter
34 and the proceeds of any settlement of an action brought pursuant to subsection 2 must be
35 deposited in the Consumer Privacy Fund created pursuant to section 9616, subsection 1.
36 §9616. Consumer Privacy Fund
37 1. Consumer Privacy Fund created. A special fund to be known and referred to in
38 this section as "the Consumer Privacy Fund" is created within the General Fund in the State
39 Treasury. The purpose of the fund is to offset any costs incurred by the courts in connection
40 with actions brought to enforce this chapter and any costs incurred by the Attorney General
41 in carrying out the Attorney General's duties under this chapter.
42 2. Uses. Funds transferred to the Consumer Privacy Fund must be used exclusively to
43 offset any costs incurred by the state courts and the Attorney General in connection with
44 this chapter. These funds may not be subject to appropriation or transfer by the Legislature 
Page 22 - 130LR2336(01)
45 for any other purpose, unless the Commissioner of Administrative and Financial Services
46 determines that the funds are in excess of the funding needed to fully offset the costs
47 incurred by the state courts and the Attorney General in connection with this chapter, in
48 which case the Legislature may appropriate excess funds for other purposes.
5 §9617. Privacy protection
6 This chapter is intended to further the right of privacy and to supplement existing laws
7 relating to consumers' personal information. The provisions of this chapter are not limited
8 to information collected electronically or over the Internet, but apply to the collection and
9 sale of all personal information collected by a business from consumers. Wherever
10 possible, law relating to consumers' personal information must be construed to harmonize
11 with the provisions of this chapter, but in the event of a conflict between other laws and the
12 provisions of this chapter, the provisions of the law that afford the greatest protection for
13 the right of privacy for consumers control.
14 §9618. Preemption
15 This chapter is a matter of statewide concern and supersedes and preempts all rules,
16 regulations, codes, ordinances and other laws adopted by a municipality, county or local
17 agency regarding the collection and sale of consumers' personal information by a business.
18 §9619. Rules
19 1. Rules. On or before January 1, 2023, the Attorney General shall solicit broad public
20 participation and adopt rules to further the purposes of this chapter, including, but not
21 limited to, the following areas:
22 A. Updating as needed additional categories of personal information to those
23 enumerated in section 9602, subsection 13 in order to address changes in technology,
24 data collection practices, obstacles to implementation and privacy concerns;
25 B. Updating as needed the definition of "unique identifiers" to address changes in
26 technology, data collection, obstacles to implementation and privacy concerns, and to
27 add categories to the definition of "designated methods for submitting requests" to
28 facilitate a consumer's ability to obtain information from a business pursuant to section
29 9609;
30 C. Establishing any exceptions necessary to comply with state or federal law,
31 including, but not limited to, those relating to trade secrets and intellectual property
32 rights, within one year of passage of this chapter and as needed thereafter;
33 D. Establishing rules and procedures for the following:
34 (1) To facilitate and govern the submission of a request by a consumer to opt out
35 of the sale of personal information pursuant to section 9607;
36 (2) To govern business compliance with a consumer's opt out request; and
37 (3) For the development and use of a recognizable and uniform opt out logo or
38 button by all businesses to promote consumer awareness of the opportunity to opt
39 out of the sale of personal information;
40 E. Adjusting the monetary threshold in section 9602, subsection 3, paragraph A,
41 subparagraph (1) in January of every odd-numbered year to reflect any increase in the
42 United States Department of Labor, Bureau of Labor Statistics Consumer Price Index;
1
2
3
4
Page 23 - 130LR2336(01)
1 F. Establishing rules, procedures and any exceptions necessary to ensure that the
2 notices and information that businesses are required to provide pursuant to this chapter
3 are provided in a manner that may be easily understood by the average consumer, are
4 accessible to consumers with disabilities and are available in the language primarily
5 used to interact with the consumer, including establishing rules and guidelines
6 regarding financial incentive offerings, within one year of passage of this chapter and
7 as needed thereafter; and
8 G. Establishing rules and procedures to further the purposes of sections 9605 and 9606
9 and to facilitate a consumer's or the consumer's authorized agent's ability to obtain
10 information pursuant to section 9609, with the goal of minimizing the administrative
11 burden on consumers, taking into account available technology, security concerns and
12 the burden on the business, to govern a business's determination that a request for
13 information received from a consumer is a verifiable consumer request, including
14 treating a request submitted through a password-protected account maintained by the
15 consumer with the business while the consumer is logged into the account as a
16 verifiable consumer request and providing a mechanism for a consumer who does not
17 maintain an account with the business to request information through the business's
18 authentication of the consumer's identity, within one year of passage of this chapter
19 and as needed thereafter.
20 2. Additional rules. The Attorney General may adopt additional rules as follows:
21 A. To establish rules and procedures on how to process and comply with verifiable
22 consumer requests for specific pieces of personal information relating to a household
23 in order to address obstacles to implementation and privacy concerns; and
24 B. As necessary to further the purposes of this chapter.
25 3. Time of enforcement action. The Attorney General may not bring an enforcement
26 action under this chapter until 6 months after the adoption of the final rules issued pursuant
27 to this section or July 1, 2023, whichever is sooner.
28 4. Routine technical rules. Rules adopted under this section are routine technical
29 rules pursuant to Title 5, chapter 375, subchapter 2-A.
30 §9620. Series of steps or transactions
31 If a series of steps or transactions were component parts of a single transaction intended
32 from the beginning to be taken with the intention of avoiding the reach of this chapter,
33 including the disclosure of information by a business to a 3rd party in order to avoid the
34 definition of "sell," a court shall disregard the intermediate steps or transactions for
35 purposes of effectuating the purposes of this chapter.
36 §9621. Waiver or limit of rights void and unenforceable
37 Any provision of a contract or agreement of any kind that purports to waive or limit in
38 any way a consumer's rights under this chapter, including, but not limited to, any right to a
39 remedy or means of enforcement, is deemed contrary to public policy and is void and
40 unenforceable. This section does not prevent a consumer from declining to request
41 information from a business, declining to opt out of a business's sale of the consumer's
42 personal information or authorizing a business to sell the consumer's personal information
43 after previously opting out.
Page 24 - 130LR2336(01)
1 §9622. Liberal construction
2 This chapter must be liberally construed to effectuate its purposes.
3 §9623. Effective date
4 This chapter takes effect January 1, 2023.
5 Sec. A-2. 35-A MRSA c. 94, as amended, is repealed.
6 Sec. A-3. Effective date. That section of this Part that repeals the Maine Revised
7 Statutes, Title 35-A, chapter 94 takes effect January 1, 2023.
8 PART B
9 Sec. B-1. 9-A MRSA §2-701, as enacted by PL 2021, c. 297, §1, is amended to
10 read:
11 §2-701. Engaging in pretense to evade requirements of this Article prohibited
12 An entity covered by this Article may not engage in any device, subterfuge or pretense
13 to evade the requirements of this Article, including, but not limited to, making a loan
14 disguised as a personal property sale and leaseback transaction, or disguising loan proceeds
15 as a cash rebate for the pretextual installment sale of goods or services or making, offering,
16 assisting or arranging a debtor to obtain a loan with a greater rate of interest, consideration
17 or charge than is permitted by this Article through any method. A loan made in violation
18 of this Part is void and uncollectible as to any principal, fee, interest or charge.
19 Sec. B-2. 9-A MRSA §2-702, as enacted by PL 2021, c. 297, §1, is repealed.
20 Sec. B-3. 9-A MRSA Art. 2, Pt. 8 is enacted to read:
21 PART 8
22 MAINE SMALL DOLLAR CONSUMER PROTECTION ACT
23 §2-801. Short title
24 This Part may be known and cited as "the Maine Small Dollar Consumer Protection
25 Act."
26 §2-802. Definitions
27 As used in this Part, unless the context otherwise indicates, the following terms have
28 the following meanings.
29 1. Renew. "Renew" means to renew, repay, refinance or consolidate an existing small
30 dollar loan with the proceeds of another small dollar loan.
31 2. Small dollar loan. "Small dollar loan" means a loan made by a supervised lender
32 that:
33 A. Is made to one or more individuals for personal, family or household use;
34 B. Has a principal amount that does not exceed $2,500;
35 C. Is unsecured and payable in substantially equal installments;
Page 25 - 130LR2336(01)
1 D. Has a repayment schedule consisting of installment payments of substantially equal
2 amounts of principal and service fees amortizing over the term of the loan such that the
3 loan is repaid in full by the maturity date; and
4 E. Has a maturity date that is not less than 90 days or more than 365 days after the date
5 of the loan agreement.
6 §2-803. Small dollar loans
7 A supervised lender may make a small dollar loan to a consumer pursuant to this Part.
8 The administrator shall adopt rules to carry out the purposes of this Part. The rules must
9 include, at a minimum, the following:
10 1. Copy. A requirement for a supervised lender to immediately provide a consumer
11 with a signed copy of the small dollar loan agreement;
12 2. Payment. Provisions allowing for payment of a small dollar loan to a consumer by
13 check, money order, cash or other mutually agreed upon means, but that prohibit the
14 supervised lender from charging the consumer additional fees based on the method of
15 payment;
16 3. Right to cancel. The right of a consumer to cancel a small dollar loan agreement
17 by notifying the supervised lender and returning the total original loan amount within 3
18 business days after the date the consumer entered into the loan agreement;
19 4. Right to file complaint. The right of a consumer to file a complaint against a
20 supervised lender for a violation of any provision of this Part; and
21 5. Modification. An allowance for a supervised lender and a consumer to mutually
22 agree to modify the repayment schedule to allow for different payment amounts over the
23 term of the small dollar loan, as long as the modified repayment schedule does not include
24 a payment due at the date of maturity that is substantially larger than any previously
25 scheduled installment payment.
26 Rules adopted pursuant to this section are routine technical rules as defined in Title 5,
27 chapter 375, subchapter 2-A.
28 §2-804. Limitations on multiple small dollar loans
29 A supervised lender may not enter into a small dollar loan agreement with a consumer
30 if the consumer has an open small dollar loan agreement with the supervised lender or
31 another supervised lender in the State. A supervised lender may rely on a consumer's
32 representation of open small dollar loan agreements with any other supervised lender.
33 §2-805. Renewal of small dollar loans
34 A supervised lender may not renew a small dollar loan agreement unless:
35 1. Payments as scheduled. A consumer has made payments as scheduled; and
36 2. Percentage of payments. A consumer has made at least 30% of all scheduled
37 payments or has retired at least 50% of the principal amount of the small dollar loan.
38 §2-806. Consideration of ability to repay
39 1. Considerations. In determining the amount and duration of the loan as part of the
40 underwriting, making or negotiating of a small dollar loan, a supervised lender shall take
41 into consideration a consumer's financial ability to repay the loan in the time and manner 
Page 26 - 130LR2336(01)
42 provided in the prospective small dollar loan agreement. The supervised lender's
43 consideration must include, but is not limited to, a consumer's:
3 A. Credit and borrowing history;
4 B. Gross income;
5 C. Representation of major financial obligations; and
6 D. Estimated basic living expenses, including, but not limited to, expenses for food,
7 utilities, regular medical costs and other costs associated with the consumer's health,
8 welfare and ability to produce income and the health and welfare of members of the
9 consumer's household who are financially dependent on the consumer. A supervised
10 lender may rely on a consumer's representation of estimated basic living expenses when
11 determining the consumer's ability to repay a small dollar loan.
12 2. Ability to pay. A supervised lender may enter into a small dollar loan agreement
13 only with a consumer whom the supervised lender has determined pursuant to subsection
14 1 to be able to repay the small dollar loan in the time and manner provided in the
15 prospective small dollar loan agreement. A supervised lender may not enter into a small
16 dollar loan agreement with a consumer if the supervised lender determines that the
17 prospective small dollar loan agreement would result in a monthly payment that would
18 exceed 12% of the consumer's gross monthly income.
19 §2-807. No prepayment penalty
20 A consumer may prepay in full the unpaid balance of the small dollar loan at any time
21 without additional interest, fees or penalties.
22 §2-808. Limitations on collections
23 A supervised lender may not use or threaten to use criminal proceedings in order to
24 collect upon the terms of a small dollar loan agreement under this Part.
25 §2-809. Report to credit reporting agency required
26 A supervised lender shall report to a credit reporting agency or agencies the terms of a
27 small dollar loan agreement and a consumer's performance pursuant to those terms.
28 §2-810. Required consumer disclosures
29 1. Public notice. A supervised lender shall post prominently at its place of business
30 in a way designed to be seen by a consumer before the consumer enters into a small dollar
31 loan agreement and on any website designed to be seen by a consumer before the consumer
32 enters into a small dollar loan agreement a written notice that, at a minimum, informs the
33 consumer that:
34 A. State law prohibits a supervised lender from entering into a small dollar loan
35 agreement with a consumer who already has a small dollar loan in effect with the
36 supervised lender or with another supervised lender in the State;
37 B. If a consumer enters into a small dollar loan agreement, a copy of the signed
38 agreement must be immediately provided to the consumer;
39 C. The proceeds of a small dollar loan are payable to a consumer by check, money
40 order, cash or any other mutually acceptable means and that the consumer may not be
41 subjected to additional fees based on the method of payment;
1
2
Page 27 - 130LR2336(01)
1 D. State law guarantees to a consumer the right to cancel a small dollar loan agreement
2 and that, in order to cancel an agreement, the consumer must notify the supervised
3 lender and return the original dollar value received within 3 business days after the date
4 the consumer entered into the loan agreement;
5 E. State law prohibits a supervised lender from using or threatening to use any criminal
6 proceedings to collect on a small dollar loan agreement; and
7 F. State law entitles a consumer to information regarding how to file a complaint
8 against a supervised lender if the consumer has reason to believe that the supervised
9 lender has violated the law and that a consumer who believes the supervised lender is
10 acting unlawfully should contact the Bureau of Consumer Credit Protection within the
11 Department of Professional and Financial Regulation.
12 Notices posted at the place of business must be in at least 36-point type. Notices posted on
13 a website must be located in a prominent place easily located by a consumer.
14 2. Public notice of fees and charges. A supervised lender shall post prominently at
15 its place of business in a way designed to be seen by a consumer before the consumer enters
16 into a small dollar loan agreement and on any website designed to be seen by a consumer
17 before the consumer enters into a small dollar loan agreement a schedule of all fees and
18 charges to be imposed for small dollar loans. Notices posted at the place of business must
19 be in at least 36-point type. Notices posted on a website must be located in a prominent
20 place easily located by a consumer.
21 SUMMARY
22 Part A of this bill establishes the Maine Consumer Privacy Act. It applies to the
23 collection and sale of all personal information collected by a business from consumers.
24 Because the new Act covers Internet service providers, this bill repeals the Maine Revised
25 Statutes, Title 35-A, chapter 94 to keep all entities on equal footing, allowing consumers
26 to opt out of the sale of personal information.
27 The Maine Consumer Privacy Act takes effect January 1, 2023. The repeal of Title
28 35-A, chapter 94 is effective January 1, 2023.
29 Part B of this bill establishes the Maine Small Dollar Consumer Protection Act. It
30 outlines the process for supervised lenders to offer small dollar loans, which are defined as
31 loans not exceeding $2,500, and the rights that consumers have when entering into small
32 dollar loan agreements.
22
23
24
25
26