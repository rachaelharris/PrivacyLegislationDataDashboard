LRB-6068/1
KP:cjs
2021 - 2022 LEGISLATURE
2021 ASSEMBLY BILL 1050
February 17, 2022 - Introduced by Representatives BROSTOFF, HEBL, ANDERSON,
SINICKI, SHELTON, STUBBS and CABRERA, cosponsored by Senators LARSON,
CARPENTER, ROYS, AGARD and SMITH. Referred to Committee on Consumer
Protection.
***AUTHORS SUBJECT TO CHANGE***
AN ACT to create 134.985 of the statutes; relating to: the privacy of consumer
data, granting rule-making authority, and providing a penalty.
Analysis by the Legislative Reference Bureau
Generally, this bill establishes requirements for businesses related to personal
information collected about consumers. The bill's requirements apply to
“businesses,” which is defined in the bill to mean a sole proprietorship, limited
liability company, corporation, association, or other entity operated for profit that
satisfies all of the following: 1) collects consumers' personal information or alone or
jointly with others determines the purposes and means of the processing of personal
information; 2) does business in this state; and 3) either has annual gross revenues
exceeding $25,000,000; buys, receives, sells, or shares the personal information of
50,000 or more consumers annually; or derives 50 percent or more of its annual
revenues from selling consumers' personal information. The bill defines “personal
information” as information that identifies, relates to, describes, or is capable of
being associated or linked with a particular consumer or household other than
certain information that is lawfully made available from federal, state, or local
government records.
The bill requires a business to disclose certain information to consumers if the
business has an online privacy policy or a Wisconsin-specific description of
consumers' privacy rights, including the following: 1) information about how a
consumer can make a request for a copy of the personal information collected about
the consumer; 2) the categories of personal information collected by the business in
the past twelve months; 3) the categories of sources from which the business collected
1
2
2021 - 2022 Legislature - 2 - LRB-6068/1
KP:cjs
 ASSEMBLY BILL 1050
personal information in the past twelve months; 4) the business's purposes for
collecting consumers' personal information; and 5) if the business sells consumers'
personal information, the purpose for selling the personal information. If the
business has an Internet site but not an online privacy policy or a Wisconsin-specific
description of consumers' privacy rights, the business must disclose the above
information on the Internet site.
Under the bill, a consumer may request a business to disclose certain
information if the business collects personal information about the consumer,
including the following: 1) the categories of personal information about the
consumer collected by the business in the past twelve months; 2) the categories of
sources from which the business collected personal information about the consumer
in the past twelve months; 3) the purposes for collecting the personal information
about the consumer; 4) if the business has sold the consumer's personal information
in the past twelve months, the purpose for selling the personal information; and 5)
the specific pieces of personal information about the consumer that the business
collected in the past twelve months. In addition, the business must deliver this
information within 45 days or within 90 days if the longer duration is reasonably
necessary and the business notifies the consumer about the delay within 45 days.
The business must disclose the information in a portable and readily useable format.
A consumer may request this information twice in a twelve-month period.
A consumer may request a business that sells the consumer's personal
information to disclose certain information, including the categories of personal
information collected about the consumer in the past twelve months, the categories
of personal information about the consumer that the business sold in the past twelve
months, and the categories of personal information about the consumer sold to each
third party in the past twelve months. The business must disclose the information
in a portable and readily useable format and within 45 days or within 90 days if the
longer duration is reasonably necessary.
The bill also requires a business, before collecting a consumer's personal
information, to inform the consumer about the categories of personal information
that the business will collect and the purpose for which the business will use the
personal information collected. Under the bill, in order for a business to sell a
consumer's personal information, certain requirements apply, including the
following: 1) if the business has an Internet site, it must provide a link titled “Do Not
Sell My Personal Information” that enables consumers to object to the selling of the
consumer's personal information; 2) if the business has an online privacy policy, the
business must include the link described above in that policy; 3) a business may not
sell the personal information if a consumer is 16 or older and the consumer directs
the business not to sell the consumer's personal information; 4) a business may sell
the personal information of a consumer aged 13 to 16 only if the consumer
affirmatively authorizes selling the personal information; 5) a business may sell the
personal information of a consumer under the age of 13 only if the consumer's parent
or guardian affirmatively authorizes it; and 6) a third party must notify a consumer
before selling the consumer's personal information. A business must also implement
reasonable security procedures to protect the personal information of consumers.
2021 - 2022 Legislature - 3 - LRB-6068/1
KP:cjs
 ASSEMBLY BILL 1050
The bill also requires that if a consumer requests that a business delete the
personal information that the business has collected about the consumer, the
business must delete that personal information. The bill provides certain exceptions
to that requirement, including the cases in which it is necessary for the business to
maintain the personal information to do any of the following: 1) complete a
transaction or contract with a consumer; 2) detect security incidents; 3) identify
errors; 4) exercise free speech or ensure the right of another consumer to exercise free
speech; 5) comply with a legal obligation; or 6) otherwise use the personal
information internally in a lawful manner.
The bill provides that a business may not discriminate against a consumer
because the consumer requests information about the business's collection or sale of
personal information, requests the business not to sell the consumer's personal
information, or requests that the business delete the consumer's personal
information. Under the bill, a business is allowed to charge a consumer a different
price or provide a different level of services if the difference is reasonably related to
the value provided to the consumer by the consumer's personal data, and a business
may offer financial incentives to a consumer for collecting the consumer's personal
information, subject to certain requirements described in the bill.
The bill requires the Department of Justice to promulgate various rules to
implement the bill's requirements. The bill also authorizes businesses to request
advice from the attorney general on how to comply with the bill's requirements and
requires the attorney general to respond to those requests.
Additionally, a provision in a contract is void and unenforceable if it would
waive or limit one or more of the bill's requirements. The bill also provides a
consumer with a private cause of action against a business if the business does not
implement reasonable security procedures to protect the consumer's personal
information and the personal information is subject to unauthorized access. A
business, service provider, or person that violates the bill is subject to a forfeiture of
up to $2,500 for each violation and a forfeiture of up to $7,500 for each intentional
violation.
For further information see the state fiscal estimate, which will be printed as
an appendix to this bill.
The people of the state of Wisconsin, represented in senate and assembly, do
enact as follows:
SECTION 1. 134.985 of the statutes is created to read:
134.985 Consumer data. (1) DEFINITIONS. In this section:
(a) “Aggregate consumer information” means information that relates to a
group or category of consumers, from which individual consumer identities have
1
2
3
4
2021 - 2022 Legislature - 4 - LRB-6068/1
KP:cjs
 ASSEMBLY BILL 1050 SECTION 1
been removed, and that is not linked or reasonably linkable to any consumer or
household.
(b) “Biometric information” means an individual's physiological, biological, or
behavioral characteristics, including deoxyribonucleic acid, that can be used singly
or in combination with each other or with other identifying data to establish
individual identity. “Biometric information” includes imagery of the iris, retina,
fingerprint, face, hand, palm, vein patterns, voice recordings, keystroke patterns or
rhythms, gait patterns or rhythms, and sleep, health, or exercise data that contain
identifying information.
(c) “Business” means any of the following:
1. A sole proprietorship, partnership, limited liability company, corporation,
association, or other legal entity that is organized or operated for the profit or
financial benefit of its shareholders or other owners, that collects consumers'
personal information or, on the behalf of consumers, alone or jointly with others
determines the purposes and means of the processing of consumers' personal
information, that does business in this state, and that satisfies any of the following:
a. Has annual gross revenues exceeding $25,000,000.
b. Annually, alone or jointly with others buys, receives for commercial
purposes, sells, or shares for commercial purposes the personal information of 50,000
or more consumers, households, or devices.
c. Derives 50 percent or more of its annual revenues from selling consumers'
personal information.
2. An entity that controls or is controlled by an entity described in subd. 1. and
that shares a name, service mark, or trademark with that entity.
1
2
3
4
5
6
7
8
9
10
11
12
13
14
15
16
17
18
19
20
21
22
23
24
2021 - 2022 Legislature - 5 - LRB-6068/1
KP:cjs
SECTION 1 ASSEMBLY BILL 1050
(d) “Business purpose” means a use of personal information for a business's or
a service provider's operational purposes or other notified purposes that is
reasonably necessary and proportionate to achieving the operational purpose for
which the personal information was collected or processed or for another operational
purpose that is compatible with the context in which the personal information was
collected.
(e) “Collect” means to gather, obtain, receive, buy, rent, or access personal
information pertaining to a consumer by any means, including by receiving
information from the consumer or by observing the consumer's behavior.
(f) “Consumer” means an individual who is a resident of this state.
(g) “Deidentified” means information to which all of the following apply:
1. The information does not reasonably identify, relate to, or describe a
consumer and is not capable of being associated with or linked to an individual
consumer.
2. Technical safeguards and business processes implemented by the person
possessing the information prohibit identifying an individual consumer to whom the
information pertains.
(h) “Device” means an object that is capable of directly or indirectly connecting
to the Internet or to another device.
(i) 1. “Personal information” means information that identifies, relates to,
describes, or is capable of being associated or linked with a particular consumer or
household. “Personal information” includes all of the following that identify, relate
to, describe, or are capable of being associated or linked with a particular individual
consumer or household:
1
2
3
4
5
6
7
8
9
10
11
12
13
14
15
16
17
18
19
20
21
22
23
24
2021 - 2022 Legislature - 6 - LRB-6068/1
KP:cjs
 ASSEMBLY BILL 1050 SECTION 1
a. Identifiers such as a real name, alias, postal address, unique personal
identifier, online identifier, Internet Protocol address, e-mail address, account
name, social security number, driver's license number, or passport number.
b. A signature, telephone number, state identification card number, insurance
policy number, employment history, bank account number, credit card number, or
debit card number or medical information or health insurance information.
c. Characteristics of protected classifications under state or federal law.
d. Commercial information such as records of personal property, records of
products or services purchased, obtained, or considered, or other purchasing or
consuming histories or tendencies.
e. Biometric information.
f. Internet or other electronic network activity information, including browsing
history, search history, and information regarding a consumer's interaction with an
Internet site, application, or advertisement.
g. Geolocation data.
h. Audio, electronic, visual, thermal, olfactory, or similar information.
i. Professional or employment-related information.
j. Education information that is not publicly available personally identifiable
information under the federal Family Educational Rights and Privacy Act, 20 USC
1232g.
k. Inferences drawn from personal information that create a profile about a
consumer reflecting the consumer's preferences, characteristics, psychological
trends, predispositions, behavior, attitudes, intelligence, abilities, and aptitudes.
2. “Personal information” does not include information that is lawfully made
available from federal, state, or local government records if the information is used
1
2
3
4
5
6
7
8
9
10
11
12
13
14
15
16
17
18
19
20
21
22
23
24
25
2021 - 2022 Legislature - 7 - LRB-6068/1
KP:cjs
SECTION 1 ASSEMBLY BILL 1050
for a purpose that is compatible with the purpose for which the information is
maintained and made available.
(j) “Sell” means to transfer, disseminate, disclose, release, rent, make available,
or otherwise communicate for monetary or other valuable consideration.
(k) “Service provider” means a sole proprietorship, partnership, limited
liability company, corporation, association, or other legal entity that is organized or
operated for the profit of its shareholders or other owners and that processes
information on behalf of a business and to which the business discloses a consumer's
personal information for a business purpose pursuant to a written contract that
prohibits the recipient of the information from retaining, using, or disclosing the
information for any purpose other than for the specific purpose of performing the
services specified in the contract.
(L) “Third party” means a person that is not any of the following:
1. A business that collects personal information from consumers.
2. A person to whom a business discloses a consumer's personal information
for a business purpose pursuant to a written contract that prohibits the person
receiving the personal information from selling, retaining, using, or disclosing the
personal information for any purpose other than for the specific purpose of
performing the services specified in the contract.
(m) “Verifiable consumer request” means a request, received by a business that
has collected personal information about a consumer, that the business can
reasonably verify to be from the consumer or the consumer's authorized
representative or, if the consumer is under 13 years of age, the consumer's parent or
guardian.
1
2
3
4
5
6
7
8
9
10
11
12
13
14
15
16
17
18
19
20
21
22
23
24
2021 - 2022 Legislature - 8 - LRB-6068/1
KP:cjs
 ASSEMBLY BILL 1050 SECTION 1
(2) REQUIRED NOTICES. (a) If a business has an online privacy policy or a
Wisconsin-specific description of consumers' privacy rights, the business shall
disclose all of the following information in the policy or description in a form that is
reasonably accessible to consumers:
1. The right of a consumer to request a disclosure under sub. (3) (a) and (c) and
one or more methods that a consumer is able to use to make a request.
2. The categories of consumers' personal information collected in the preceding
12 months.
3. The categories of sources from which consumers' personal information was
collected in the preceding 12 months.
4. The business or commercial purposes for collecting consumers' personal
information.
5. If the business has sold consumers' personal information in the preceding
12 months, the business or commercial purposes for selling the personal information.
6. If the business has shared consumers' personal information with a 3rd party
in the preceding 12 months, the categories of 3rd parties with whom the business has
shared personal information.
7. A list of the categories of consumers' personal information sold, if the
business has sold consumers' personal information in the preceding 12 months, or
if the business has not sold consumers' personal information in the preceding 12
months, a disclosure of that fact.
8. A list of the categories of consumers' personal information disclosed for a
business purpose, if the business has disclosed consumers' personal information for
a business purpose in the preceding 12 months, or if the business has not disclosed
1
2
3
4
5
6
7
8
9
10
11
12
13
14
15
16
17
18
19
20
21
22
23
24
2021 - 2022 Legislature - 9 - LRB-6068/1
KP:cjs
SECTION 1 ASSEMBLY BILL 1050
consumers' personal information for a business purpose in the preceding 12 months,
a disclosure of that fact.
(b) If a business does not have an online privacy policy or a Wisconsin-specific
description of consumers' privacy rights under par. (a) and the business has an
Internet site, the business shall disclose the information listed in par. (a) 1. to 8. on
its Internet site in a form that is reasonably accessible to consumers.
(c) A business that makes a disclosure under par. (a) or (b) shall update the
information in the disclosure at least once every 12 months.
(3) DISCLOSURE OF INFORMATION COLLECTED. (a) 1. Upon receiving a verifiable
consumer request from a consumer, a business that has collected personal
information about that consumer shall promptly disclose and deliver free of charge
to the consumer all of the following:
a. The categories of personal information it has collected about the consumer
in the preceding 12 months.
b. The categories of sources from which the consumer's personal information
was collected in the preceding 12 months.
c. The business or commercial purposes for collecting the consumer's personal
information.
d. If the business has sold the consumer's personal information in the preceding
12 months, the business or commercial purposes for selling the personal information.
e. If the business has shared the consumer's personal information with 3rd
parties in the preceding 12 months, the categories of 3rd parties with whom the
business has shared personal information.
f. The specific pieces of personal information that the business has collected
about the consumer in the preceding 12 months.
1
2
3
4
5
6
7
8
9
10
11
12
13
14
15
16
17
18
19
20
21
22
23
24
25
2021 - 2022 Legislature - 10 - LRB-6068/1
KP:cjs
 ASSEMBLY BILL 1050 SECTION 1
2. A business may disclose and deliver personal information to a consumer
under this paragraph only after receiving a verifiable consumer request from the
consumer.
3. A business shall make available to consumers 2 or more methods for
submitting verifiable consumer requests for a disclosure under this paragraph
including, at a minimum, a toll-free telephone number and, if the business
maintains an Internet site, an Internet address.
4. a. Except as provided in subd. 4. b., a business shall deliver the disclosure
required under this paragraph within 45 days of receiving a verifiable consumer
request from a consumer. A business shall promptly take steps to determine whether
a request received is a verifiable consumer request. The time that a business spends
determining whether a request is a verifiable consumer request is included in the
45-day deadline under this subd. 4. a.
b. A business may deliver the disclosure required under this paragraph within
90 days after receiving a verifiable consumer request if reasonably necessary and if
the business notifies the consumer of the delayed delivery before the time period
under subd. 4. a. expires.
5. A business shall deliver personal information under this paragraph in
writing and through the consumer's account with the business, if the consumer
maintains an account with the business. If the consumer does not maintain an
account with the business, the business shall deliver personal information under this
paragraph by mail or electronically, at the choice of the consumer. If the business
provides personal information under this paragraph electronically, the business
shall provide the information in a portable and, to the extent technically feasible, a
1
2
3
4
5
6
7
8
9
10
11
12
13
14
15
16
17
18
19
20
21
22
23
24
2021 - 2022 Legislature - 11 - LRB-6068/1
KP:cjs
SECTION 1 ASSEMBLY BILL 1050
readily useable format that allows the consumer to transmit the information to
another entity without hindrance.
6. A business may not require a consumer to create an account in order to
submit a verifiable consumer request for a disclosure of personal information
required under this paragraph.
7. A business is not required to provide personal information to a consumer
under this paragraph more than 2 times in a 12-month period.
(b) Paragraph (a) does not require any of the following:
1. That a business retain any personal information collected for a single,
onetime transaction, if the personal information is not sold or retained by the
business.
2. That a business reidentify or otherwise link information that is not
maintained in a manner that would be considered personal information.
(c) 1. Upon receiving a verifiable consumer request from a consumer, a business
that has sold or disclosed for a business purpose personal information about that
consumer shall disclose to the consumer the following information:
a. If the business has collected personal information in the preceding 12
months, the categories of personal information that the business collected about the
consumer.
b. If the business sold the consumer's personal information in the preceding 12
months, the categories of the personal information that the business sold.
c. If the business sold the consumer's personal information in the preceding 12
months, for each 3rd party to whom the business sold the personal information, the
categories of personal information that the business sold to the 3rd party.
1
2
3
4
5
6
7
8
9
10
11
12
13
14
15
16
17
18
19
20
21
22
23
24
2021 - 2022 Legislature - 12 - LRB-6068/1
KP:cjs
 ASSEMBLY BILL 1050 SECTION 1
d. If the business disclosed the consumer's personal information for a business
purpose in the preceding 12 months, the categories of the personal information that
the business disclosed.
2. A business shall identify the information required to be disclosed under
subd. 1. c. separately from the information required to be disclosed under subd. 1. d.
3. A business shall make available to consumers 2 or more methods for
submitting verifiable consumer requests for a disclosure under this paragraph
including, at a minimum, a toll-free telephone number and, if the business
maintains an Internet site, an Internet address.
4. a. Except as provided in subd. 4. b., a business shall deliver the disclosure
required under this paragraph within 45 days of receiving a verifiable consumer
request from a consumer. A business shall promptly take steps to determine whether
a request received is a verifiable consumer request. The time that a business spends
determining whether a request received is a verifiable consumer request is included
in the 45-day deadline under this subd. 4. a.
b. A business may deliver the disclosure required under this paragraph within
90 days after receiving a verifiable consumer request if reasonably necessary and if
the business notifies the consumer of the delayed delivery before the time period
under subd. 4. a. expires.
5. A business shall deliver personal information under this paragraph in
writing and through the consumer's account with the business, if the consumer
maintains an account with the business. If the consumer does not maintain an
account with the business, the business shall deliver personal information under this
paragraph by mail or electronically, at the choice of the consumer. If the business
provides information under this paragraph electronically, the business shall provide
1
2
3
4
5
6
7
8
9
10
11
12
13
14
15
16
17
18
19
20
21
22
23
24
25
2021 - 2022 Legislature - 13 - LRB-6068/1
KP:cjs
SECTION 1 ASSEMBLY BILL 1050
the information in a portable and, to the extent technically feasible, a readily useable
format that allows the consumer to transmit the information to another entity
without hindrance.
6. A business may not require a consumer to create an account in order to
submit a verifiable consumer request for a disclosure of personal information
required under this paragraph.
(4) SELLING OR COLLECTING INFORMATION. (a) A business may not collect a
consumer's personal information or use the personal information for a particular
purpose unless the business informs the consumer, at or before the point of collecting
the information, about all of the following:
1. That the business will collect that category of personal information about the
consumer.
2. The purpose for which the business will use the category of personal
information collected.
(b) 1. A business may not sell a consumer's personal information to 3rd parties
unless the business satisfies all of the following:
a. If the business has an Internet site, the business provides a clear and
conspicuous link on the homepage of the Internet site, titled “Do Not Sell My
Personal Information,” to an Internet page that enables a consumer or person
authorized by the consumer to object to the sale of the consumer's personal
information.
b. The business includes a statement explaining that a consumer may object
to the sale of the consumer's personal information and a link to the Internet page
required under subd. 1. a. in its online privacy policy or policies, if the business has
1
2
3
4
5
6
7
8
9
10
11
12
13
14
15
16
17
18
19
20
21
22
23
24
2021 - 2022 Legislature - 14 - LRB-6068/1
KP:cjs
 ASSEMBLY BILL 1050 SECTION 1
an online privacy policy or policies, and any Wisconsin-specific description of
consumers' privacy rights.
c. The business ensures that each individual responsible for handling
consumer inquiries about the business's privacy practices or compliance with this
section is informed of the requirements in this subdivision and par. (c) and of how to
direct consumers to object to the sale of their personal information under par. (c).
2. Subdivision 1. does not require a business to provide the link described in
subd. 1. a. on the homepage of an Internet site that the business makes available to
the public generally, if all of the following apply:
a. The business maintains a separate and additional Internet site that the
business dedicates to consumers.
b. The Internet site under subd. 2. a. satisfies the requirements under subd.
1. a.
c. The business takes reasonable steps to ensure that consumers in this state
are directed to the Internet site under subd. 2. a. and not the homepage of the
Internet site made available to the public generally.
(c) 1. a. A business may not sell personal information that the business collects
about a consumer to a 3rd party if the consumer is 16 years of age or older and the
consumer directs the business not to sell the consumer's personal information,
unless the consumer subsequently provides express authorization for the business
to sell the personal information.
b. A consumer may authorize another person to, as provided in subd. 1. a.,
direct a business not to sell the consumer's personal information. A representative
authorized under this subd. 1. b. may not subsequently provide express
1
2
3
4
5
6
7
8
9
10
11
12
13
14
15
16
17
18
19
20
21
22
23
24
2021 - 2022 Legislature - 15 - LRB-6068/1
KP:cjs
SECTION 1 ASSEMBLY BILL 1050
authorization on behalf of the consumer for the business to sell the consumer's
personal information.
2. A business may not sell personal information that the business collects about
a consumer to a 3rd party if the business has actual knowledge that the consumer
is under 16 years of age unless any of the following applies:
a. The consumer is at least 13 years of age and under 16 years of age and the
consumer affirmatively authorizes the business to sell the consumer's personal
information.
b. The consumer is under 13 years of age and the consumer's parent or guardian
affirmatively authorizes the business to sell the consumer's personal information.
3. A business that willfully disregards the age of a consumer is considered to
have actual knowledge of the consumer's age.
4. A business may not request authorization from a consumer or the consumer's
parent or guardian to sell the consumer's personal information within 12 months of
the most recent occasion that the consumer, parent, or guardian directed the
business not to sell the personal information or denied the business's request for
authorization to sell the personal information.
5. A business that collects the personal information of a consumer in connection
with receiving a direction under subd. 1. to not sell the consumer's personal
information may use the personal information only for the purposes of implementing
the consumer's direction not to sell the personal information.
(d) 1. A 3rd party may not sell personal information about a consumer that has
been sold to the 3rd party by a business unless the 3rd party provides explicit notice
to the consumer that the consumer's personal information has been sold to the 3rd
1
2
3
4
5
6
7
8
9
10
11
12
13
14
15
16
17
18
19
20
21
22
23
24
2021 - 2022 Legislature - 16 - LRB-6068/1
KP:cjs
 ASSEMBLY BILL 1050 SECTION 1
party and that the 3rd party intends to sell the information and the 3rd party
satisfies par. (b) 1. a. to c.
2. A 3rd party may not sell information about a consumer that has been sold
to the 3rd party by a business if the consumer directs the 3rd party not to sell the
consumer's personal information, unless the consumer subsequently provides
express authorization for the 3rd party to sell the personal information.
(e) A business may not require a consumer to create an account in order to direct
the business not to sell the consumer's personal information under par. (c).
(f) A business shall, to protect the personal information of consumers,
implement and maintain reasonable security procedures and practices appropriate
to the nature of the personal information.
(5) DELETION OF INFORMATION. (a) Except as provided in par. (b), a business that
receives a verifiable consumer request from a consumer to delete the consumer's
personal information shall delete the personal information that the business has
collected from the consumer from its records and direct any of the business's service
providers to delete the consumer's personal information from their records.
(b) A business or its service provider is not required to delete a consumer's
personal information if it is necessary for the business or its service provider to
maintain the consumer's personal information for any of the following purposes:
1. To complete the transaction for which the personal information was
collected, to provide a good or service requested by the consumer or reasonably
anticipated to be requested by the consumer within the context of the business's
ongoing relationship with the consumer, or to otherwise perform a contract between
the business and the consumer.
1
2
3
4
5
6
7
8
9
10
11
12
13
14
15
16
17
18
19
20
21
22
23
24
2021 - 2022 Legislature - 17 - LRB-6068/1
KP:cjs
SECTION 1 ASSEMBLY BILL 1050
2. To detect security incidents, to protect against malicious, deceptive,
fraudulent, or illegal activity, or to prosecute a person responsible for that activity.
3. To debug to identify and repair errors that impair existing or intended
functionality.
4. To exercise free speech, to ensure the right of another consumer to exercise
free speech, or to exercise another right provided by law.
5. If the consumer provides informed consent, to engage in public or
peer-reviewed scientific, historical, or statistical research in the public interest that
adheres to all other applicable ethics and privacy laws, if the business's deletion of
the personal information is likely to render impossible or seriously impair the
achievement of that research.
6. To enable solely internal uses that are reasonably aligned with the
expectations of the consumer based on the consumer's relationship with the
business.
7. To comply with a legal obligation.
8. To otherwise use the consumer's personal information internally in a lawful
manner that is compatible with the context in which the consumer provided the
information.
(6) DISCRIMINATION PROHIBITED. (a) 1. A business may not discriminate against
a consumer because the consumer makes a verifiable consumer request under sub.
(3) (a) or (c) or (5) or because under sub. (4) (c) the personal information of the
consumer was not permitted to be sold by the business, including by doing any of the
following:
a. Denying goods or services to the consumer.
1
2
3
4
5
6
7
8
9
10
11
12
13
14
15
16
17
18
19
20
21
22
23
24
2021 - 2022 Legislature - 18 - LRB-6068/1
KP:cjs
 ASSEMBLY BILL 1050 SECTION 1
b. Charging different prices or rates for goods or services, including through the
use of discounts or other benefits or imposing penalties.
c. Providing a different level or quality of goods or services to the consumer.
d. Suggesting that the consumer will receive a different price or rate for goods
or services or a different level or quality of goods or services.
2. This paragraph does not prohibit a business from charging a consumer a
different price or rate, or from providing a different level or quality of goods or
services to the consumer, if the difference is reasonably related to the value provided
to the consumer by the consumer's data.
(b) 1. A business may offer financial incentives, including payments to
consumers as compensation, for the collection of a consumer's personal information,
the sale of a consumer's personal information, or the deletion of a consumer's
personal information. A business may also offer a different price, rate, level, or
quality of goods or services to a consumer if that difference is directly related to the
value provided to the consumer by the consumer's data.
2. A business may enter a consumer into a financial incentive program
described in subd. 1. only if the consumer or the consumer's parent or guardian
affirmatively authorizes entry into the program after receiving a notice that clearly
describes the material terms of the program.
3. A consumer or a consumer's parent or guardian may revoke entry into a
financial incentive program described in subd. 1. at any time.
4. If a business offers a financial incentive program described in subd. 1., the
business shall include a description of the program on the Internet page described
in sub. (4) (b) 1. a. and in the policies described in sub. (4) (b) 1. b.
1
2
3
4
5
6
7
8
9
10
11
12
13
14
15
16
17
18
19
20
21
22
23
24
2021 - 2022 Legislature - 19 - LRB-6068/1
KP:cjs
SECTION 1 ASSEMBLY BILL 1050
5. A business may not use financial incentive practices that are unjust,
unreasonable, coercive, or usurious in nature.
(7) GUIDANCE; RULES. (a) A business or 3rd party may request advice from the
attorney general on how to comply with this section, and the attorney general shall
respond to the request.
(b) The department of justice shall promulgate rules to implement this section,
including the following:.
1. Rules that specify additional categories of personal information to those
enumerated in sub. (1) (i) 1.
2. Rules that specify unique personal identifiers to address changes in
technology, changes in data collection, obstacles to implementing this section, and
privacy concerns.
3. Rules that specify additional methods for consumers to make requests and
businesses to provide disclosures under this section.
4. Rules that establish any exceptions necessary to comply with other state or
federal law, including exceptions relating to trade secrets and intellectual property
rights.
5. Rules that establish procedures for the following:
a. The submission of a direction under sub. (4) (c) 1. a.
b. Business compliance with a direction submitted under sub. (4) (c) 1. a.
c. The use of a recognizable and uniform logo or button by all businesses to
promote consumer awareness of the option to make a direction under sub. (4) (c) 1.
a.
6. Rules that ensure that the notices and information that business are
required to provide under this section are provided in a manner that may be easily
1
2
3
4
5
6
7
8
9
10
11
12
13
14
15
16
17
18
19
20
21
22
23
24
25
2021 - 2022 Legislature - 20 - LRB-6068/1
KP:cjs
 ASSEMBLY BILL 1050 SECTION 1
understood by the average consumer, are accessible to consumers with disabilities,
and are available in the language primarily used to interact with the consumer.
7. Rules that facilitate a consumer's or, under sub. (4) (c) 1. b., a representative's
ability to make a request or submit a direction under this section, with the goal of
minimizing the administrative burden on consumers, taking into account available
technology, security concerns, and the burden on the business, to govern a business's
determination that a request by a consumer is a verifiable consumer request,
including by treating a request submitted through a password-protected account
maintained by the consumer with the business while the consumer is logged into the
account as a verifiable consumer request and providing a mechanism for a business
to authenticate the identity of a consumer who does not maintain an account with
the business and requests information or submits a direction under this section.
(c) The department of justice shall adjust the monetary threshold amount in
sub. (1) (c) 1. a. in January of every odd-numbered year by the percentage change
in the U.S. consumer price index for all urban consumers, U.S. city average, as
determined by the federal department of labor for the period since the last
adjustment under this paragraph.
(8) CONTRACTS IN VIOLATION. A provision in a contract or agreement that
purports to waive or limit a requirement under this section is void and
unenforceable.
(9) PRIVATE CAUSE OF ACTION. (a) 1. A consumer may initiate an action against
a business to enforce a written statement under subd. 2. and may pursue injunctive
or declaratory relief, damages in an amount not less than $100 and not more than
$750 per consumer per incident or actual damages, whichever is greater, or any other
relief the court deems proper if all of the following apply:
1
2
3
4
5
6
7
8
9
10
11
12
13
14
15
16
17
18
19
20
21
22
23
24
25
2021 - 2022 Legislature - 21 - LRB-6068/1
KP:cjs
SECTION 1 ASSEMBLY BILL 1050
a. The consumer, on an individual or class-wide basis, provides the business
with written notice identifying that the consumer's nonencrypted or nonredacted
personal information is subject to an unauthorized access and exfiltration, theft, or
disclosure as a result of the business's violation of sub. (4) (f).
b. The business continues to violate sub. (4) (f) more than 30 days after
receiving the written notice under subd. 1. a.
2. No action may be brought under subd. 1. if within 30 days of receiving a
written notice under subd. 1. a., a business cures the noticed violation of sub. (4) (f)
and provides the consumer that provided the written notice with an express written
statement that the violation has been cured.
3. In assessing the amount of damages under subd. 1., a court shall consider
the relevant circumstances presented by any of the parties to the case, including the
nature and seriousness of the misconduct, the number of violations, the persistence
of the misconduct, the length of time over which the misconduct occurred, the
willfulness of the defendant's misconduct, and the defendant's assets, liabilities, and
net worth.
(b) A consumer may initiate an action against a business solely for actual
pecuniary damages suffered as a result of the business's violation of sub. (4) (f).
(10) INAPPLICABILITY. (a) This section does not do any of the following:
1. Restrict a business from complying with federal or state laws or local
ordinances.
2. Restrict a business from complying with a civil, criminal, or regulatory
inquiry, investigation, subpoena, or summons by federal, state, or local authorities.
3. Restrict a business, service provider, or 3rd party from cooperating with law
enforcement agencies concerning conduct or activity that the business, service
1
2
3
4
5
6
7
8
9
10
11
12
13
14
15
16
17
18
19
20
21
22
23
24
25
2021 - 2022 Legislature - 22 - LRB-6068/1
KP:cjs
 ASSEMBLY BILL 1050 SECTION 1
provider, or 3rd party reasonably and in good faith believes might violate federal,
state, or local law.
4. Restrict a business from exercising or defending legal claims.
5. Restrict the collection, use, retention, sale, or disclosure of consumer
information that is deidentified or aggregate consumer information.
6. Restrict the collection or sale of a consumer's personal information if the
information is collected while the consumer was outside of this state, no part of any
sale of the consumer's personal information occurs in this state, and no personal
information collected from a consumer while the consumer is in this state is sold.
(b) This section does not apply to any of the following:
1. Medical information that is collected by a health care provider or entity and
covered by federal law.
2. Information collected as part of a clinical trial subject to the Federal Policy
for the Protection of Human Subjects while following standards developed by
international organizations of federal agencies.
3. Personal information sold to or from a consumer reporting agency, as defined
in s. 422.501 (1m), if the information is reported in or used to generate a consumer
report, as defined in s. 100.54 (1) (b), and the use of the information complies with
the federal Fair Credit Reporting Act, 15 USC 1681 et seq.
4. Personal information collected, processed, sold, or disclosed pursuant to the
federal Gramm-Leach-Bliley Act, Public Law 106-102.
5. Personal information collected, processed, sold, or disclosed pursuant to the
the federal Driver's Privacy Protection Act, 18 USC 2721 et seq.
(11) VIOLATIONS; PENALTY. (a) If a series of steps or transactions are component
parts of a single transaction intended from the beginning to be taken with the
1
2
3
4
5
6
7
8
9
10
11
12
13
14
15
16
17
18
19
20
21
22
23
24
25
2021 - 2022 Legislature - 23 - LRB-6068/1
KP:cjs
SECTION 1 ASSEMBLY BILL 1050
intention of avoiding the requirements of this section, including the disclosure of
information by a business to a 3rd party in order to avoid constituting a sale, a court
shall disregard the intermediate steps or transactions for purposes of this section.
(b) The department of justice may, if a business, service provider, or person
violates this section more than 30 days after receiving notification of the violation
from the department of justice, commence an action in the name of the state against
the business, service provider, or other person to recover a forfeiture to the state of
not more than $2,500 for each violation or a forfeiture of not more than $7,500 for
each intentional violation.
SECTION 2.0Initial applicability.
(1) This act first applies to a contract that is entered into, renewed, or modified
on the effective date of this subsection.
SECTION 3.0Effective date.
(1) This act takes effect on the first day of the 7th month beginning after
publication.
(END)
1
2
3
4
5
6
7
8
9
10
11
12
13
14
15
16