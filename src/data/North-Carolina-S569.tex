GENERAL ASSEMBLY OF NORTH CAROLINA
SESSION 2021
S 1
SENATE BILL 569
Short Title: Consumer Privacy Act. (Public)
Sponsors: Senators Salvador, Clark, and Waddell (Primary Sponsors).
Referred to: Rules and Operations of the Senate
April 7, 2021
*S569-v-1*
1 A BILL TO BE ENTITLED
2 AN ACT TO PROTECT CONSUMERS BY ENACTING THE CONSUMER PRIVACY ACT
3 OF NORTH CAROLINA.
4 The General Assembly of North Carolina enacts:
5 SECTION 1. This act shall be known and may be cited as the "Consumer Privacy
6 Act of North Carolina."
7 SECTION 2. Chapter 75 of the General Statutes is amended by adding a new Article
8 to read:
9 "Article 2B.
10 "Consumer Privacy Act.
11 "§ 75-70. Definitions; scope; exemptions.
12 (a) Definitions. – The following definitions apply in this Article:
13 (1) Affiliate. – A legal entity that controls, is controlled by, or is under common
14 control with another legal entity or shares common branding with another
15 legal entity. For the purposes of this definition, "control" or "controlled"
16 means (i) ownership of, or the power to vote, more than fifty percent (50%)
17 of the outstanding shares of any class of voting security of a company; (ii)
18 control in any manner over the election of a majority of the directors or of
19 individuals exercising similar functions; or (iii) the power to exercise
20 controlling influence over the management of a company.
21 (2) Authenticate. – Verifying through reasonable means that the consumer
22 entitled to exercise his consumer rights in G.S. 75-71 is the same consumer
23 exercising such consumer rights with respect to the personal data at issue.
24 (3) Biometric data. – Data generated by automatic measurements of an
25 individual's biological characteristics such as a fingerprint, voiceprint, eye
26 retinas, irises, or other unique biological patterns or characteristics that is used
27 to identify a specific individual. "Biometric data" does not include a physical
28 or digital photograph, a video or audio recording or data generated therefrom,
29 or information collected, used, or stored for health care treatment, payment,
30 or operations under HIPAA.
31 (4) Business associate. – The same meaning as the term established by HIPAA.
32 (5) Child. – Any natural person younger than 13 years of age.
33 (6) Consent. – A clear affirmative act signifying a consumer's freely given,
34 specific, informed, and unambiguous agreement to process personal data
35 relating to the consumer. Consent may include a written statement, including 
General Assembly Of North Carolina Session 2021
Page 2 Senate Bill 569-First Edition
1 a statement written by electronic means, or any other unambiguous affirmative
2 action.
3 (7) Consumer. – A natural person who is a resident of this State acting only in an
4 individual or household context. It does not include a natural person acting in
5 a commercial or employment context.
6 (8) Controller. – The natural or legal person that, alone or jointly with others,
7 determines the purpose and means of processing personal data.
8 (9) Covered entity. – The same as the term established by HIPAA.
9 (10) Decisions that produce legal or similarly significant effects concerning a
10 consumer. – A decision made by the controller that results in the provision or
11 denial by the controller of financial and lending services, housing, insurance,
12 education enrollment, criminal justice, employment opportunities, health care
13 services, or access to basic necessities, such as food and water.
14 (11) De-identified data. – Data that cannot reasonably be linked to an identified or
15 identifiable natural person, or a device linked to such person. A controller that
16 possesses "de-identified data" shall comply with the requirements of
17 subsection (a) of G.S. 74-75.
18 (12) Fund. – The Consumer Privacy Fund established in this Article.
19 (13) Health record. – Any written, printed, or electronically recorded material
20 maintained by a health care entity in the course of providing health services
21 to an individual concerning the individual and the services provided. "Health
22 record" also includes the substance of any communication made by an
23 individual to a health care entity in confidence during or in connection with
24 the provision of health services or information otherwise acquired by the
25 health care entity about an individual in confidence and in connection with the
26 provision of health services to the individual.
27 (14) Health care provider. – Includes the following persons licensed, certified, or
28 otherwise permitted to conduct business or practice in this State: (i) a hospital,
29 (ii) a nursing home or nursing facility, (iii) any person practicing medicine,
30 osteopathy, or dentistry, or (iv) any person furnishing health care policies or
31 plans.
32 (15) HIPAA. – The federal Health Insurance Portability and Accountability Act of
33 1996 (42 U.S.C. § 1320d, et seq.).
34 (16) Identified or identifiable natural person. – A person who can be readily
35 identified, directly or indirectly.
36 (17) Institution of higher education. – A public or private college or university.
37 (18) Nonprofit organization. – Any corporation exempt from taxation under
38 sections 501(c)(3), 501(c)(6), or 501 (c)(12) of the Internal Revenue Code.
39 (19) Personal data. – Any information that is linked or reasonably linkable to an
40 identified or identifiable natural person. The term does not include
41 de-identified data or publicly available information.
42 (20) Precise geolocation data. – Information derived from technology, including,
43 but not limited to, global positioning system level latitude and longitude
44 coordinates or other mechanisms that directly identify the specific location of
45 a natural person with precision and accuracy within a radius of 1,750 feet.
46 "Precise geolocation data" does not include the content of communications or
47 any data generated by or connected to advanced utility metering infrastructure
48 systems or equipment for use by a utility.
49 (21) Process or processing. – Any operation or set of operations performed,
50 whether by manual or automated means, on personal data or on sets of 
General Assembly Of North Carolina Session 2021
Senate Bill 569-First Edition Page 3
1 personal data, such as the collection, use, storage, disclosure, analysis,
2 deletion, or modification of personal data.
3 (22) Processor. – A natural or legal entity that processes personal data on behalf of
4 a controller.
5 (23) Profiling. – Any form of automated processing performed on personal data to
6 evaluate, analyze, or predict personal aspects related to an identified or
7 identifiable natural person's economic situation, health, personal preferences,
8 interests, reliability, behavior, location, or movements.
9 (24) Protected health information. – The same as the term established by HIPAA.
10 (25) Pseudonymous data. – Personal data that cannot be attributed to a specific
11 natural person without the use of additional information, provided that such
12 additional information is kept separately and is subject to appropriate
13 technical and organizational measures to ensure that the personal data is not
14 attributed to an identified or identifiable natural person.
15 (26) Publicly available information. – Information that is lawfully made available
16 through federal, State, or local government records, or information that a
17 business has a reasonable basis to believe is lawfully made available to the
18 general public through widely distributed media, by the consumer, or by a
19 person to whom the consumer has disclosed the information, unless the
20 consumer has restricted the information to a specific audience.
21 (27) Sale of personal data. – The exchange of personal data for monetary
22 consideration by the controller to a third party. "Sale of personal data" does
23 not include any of the following:
24 a. The disclosure of personal data to a processor that processes the
25 personal data on behalf of the controller.
26 b. The disclosure of personal data to a third party for purposes of
27 providing a product or service requested by the consumer.
28 c. The disclosure or transfer of personal data to an affiliate of the
29 controller.
30 d. The disclosure of information that the consumer (i) intentionally made
31 available to the general public via a channel of mass media and (ii) did
32 not restrict to a specific audience.
33 e. The disclosure or transfer of personal data to a third party as an asset
34 that is part of a merger, acquisition, bankruptcy, or other transaction
35 in which the third party assumes control of all or part of the controller's
36 assets.
37 (28) Sensitive data. – A category of personal data that includes the following:
38 a. Personal data revealing racial or ethnic origin, religious beliefs, mental
39 or physical health diagnosis, sexual orientation, or citizenship or
40 immigration status.
41 b. The processing of genetic or biometric data for the purpose of uniquely
42 identifying a natural person.
43 c. The personal data collected from a known child.
44 d. Precise geolocation data.
45 (29) State agency. – A State agency, board, bureau, council, department,
46 institution, or other instrumentality of State government in the executive
47 branch. The term also includes county human services agencies; local
48 departments of social services; county health departments; district health
49 departments; local emergency management agencies; and area mental health,
50 developmental disabilities, and substance abuse authorities.
General Assembly Of North Carolina Session 2021
Page 4 Senate Bill 569-First Edition
1 (30) Targeted advertising. – Displaying advertisements to a consumer where the
2 advertisement is selected based on personal data obtained from that
3 consumer's activities over time and across nonaffiliated websites or online
4 applications to predict such consumer's preferences or interests. "Targeted
5 advertising" does not include any of the following:
6 a. Advertisements based on activities within a controller's own websites
7 or online applications.
8 b. Advertisements based on the context of a consumer's current search
9 query, visit to a website, or online application.
10 c. Advertisements directed to a consumer in response to the consumer's
11 request for information or feedback.
12 d. Processing personal data processed solely for measuring or reporting
13 advertising performance, reach, or frequency.
14 (31) Third party. – A natural or legal person, public authority, agency, or body
15 other than the consumer, controller, processor, or an affiliate of the processor
16 or the controller.
17 (b) Scope. – This Article applies to persons that conduct business in the State or produce
18 products or services that are targeted to residents of this State and that either (i) during a calendar
19 year, control or process personal data of at least 100,000 consumers or (ii) control or process
20 personal data of at least 25,000 consumers and derive over fifty percent (50%) of gross revenue
21 from the sale of personal data.
22 (c) Coverage Exemptions. – This Article does not apply to any of the following:
23 (1) Political subdivisions of the State.
24 (2) Financial institutions or data subject to Title V of the federal
25 Gramm-Leach-Bliley Act, 15 U.S.C. § 6801, et seq.
26 (3) A covered entity or business associate governed by the privacy, security, and
27 breach notification rules issued by the U.S. Department of Health and Human
28 Services, 45 C.F.R. Parts 160 and 164 established pursuant to HIPAA, and the
29 Health Information Technology for Economic and Clinical Health Act, P.L.
30 111-5.
31 (4) A nonprofit organization.
32 (5) An institution of higher education.
33 (6) A public school unit, as defined in G.S. 115C-5(7a).
34 (7) Controller or processor that complies with the verifiable parental consent
35 requirements of the Children's Online Privacy Protection Act, 15 U.S.C.
36 section 6501, et seq., but only to the extent that the controller or processor
37 shall be deemed compliant with any obligation to obtain parental consent
38 under this Article.
39 (d) Data Exemption. – The following information and data are exempt from this Article:
40 (1) Protected health information under HIPAA.
41 (2) Health records for the purpose of carrying out the duties and responsibilities
42 of the North Carolina Department of Health and Human Services.
43 (3) Patient identifying information for purposes of 42 U.S.C. § 290dd-2.
44 (4) Identifiable private information for purposes of the federal policy for the
45 protection of human subjects under 45 C.F.R. Part 46; identifiable private
46 information that is otherwise information collected as part of human subjects
47 research pursuant to the good clinical practice guidelines issued by the
48 International Council for Harmonisation of Technical Requirements for
49 Pharmaceuticals for Human Use; the protection of human subjects under 21
50 C.F.R. Parts 6, 50, and 56, or personal data used or shared in research 
General Assembly Of North Carolina Session 2021
Senate Bill 569-First Edition Page 5
1 conducted in accordance with the requirements set forth in this Article, or
2 other research conducted in accordance with applicable law.
3 (5) Information and documents created for purposes of the federal Health Care
4 Quality Improvement Act of 1986, 42 U.S.C. § 11101, et seq.
5 (6) Patient safety work products for purposes of the federal Patient Safety and
6 Quality Improvement Act, 42 U.S.C. § 299b-21, et seq.
7 (7) Information derived from any of the health care-related information listed in
8 this subsection that is de-identified in accordance with the requirements for
9 de-identification pursuant to HIPAA.
10 (8) Information originating from, and intermingled to be indistinguishable with,
11 or information treated in the same manner as information exempt under this
12 subsection that is maintained by a covered entity or business associate as
13 defined by HIPAA or a program or a qualified service organization as defined
14 by 42 U.S.C. § 290dd-2.
15 (9) Information used only for public health activities and purposes as authorized
16 by HIPAA.
17 (10) The collection, maintenance, disclosure, sale, communication, or use of any
18 personal information bearing on a consumer's credit worthiness, credit
19 standing, credit capacity, character, general reputation, personal
20 characteristics, or mode of living by a consumer reporting agency or furnisher
21 that provides information for use in a consumer report, and by a user of a
22 consumer report, but only to the extent that such activity is regulated by and
23 authorized under the federal Fair Credit Reporting Act, 15 U.S.C. § 168.
24 (11) Personal data collected, processed, sold, or disclosed in compliance with the
25 federal Driver's Privacy Protection Act of 1994, 18 U.S.C. § 2721, et seq.
26 (12) Personal data regulated by the federal Family Educational Rights and Privacy
27 Act, 20 U.S.C. § 1232g, et seq.
28 (13) Personal data collected, processed, sold, or disclosed in compliance with the
29 federal Farm Credit Act, 12 U.S.C. § 2001, et seq.
30 (14) Data processed or maintained (i) in the course of an individual applying to,
31 employed by, or acting as an agent or independent contractor of a controller,
32 processor, or third party, to the extent that the data is collected and used within
33 the context of that role, (ii) as the emergency contact information of an
34 individual under this Article used for emergency contact purposes, or (iii) that
35 is necessary to retain to administer benefits for another individual relating to
36 the individual under clause (iii) and used for the purposes of administering
37 those benefits.
38 "§ 75-71. Consumer privacy rights.
39 (a) A consumer may invoke the consumer rights granted under this subsection at any time
40 by submitting a request to a controller specifying the consumer rights the consumer wishes to
41 invoke. A known child's parent or legal guardian may invoke such consumer rights on behalf of
42 the child regarding processing personal data belonging to the known child. A controller shall
43 comply with an authenticated consumer request to exercise the consumer's right to the following:
44 (1) To confirm whether or not a controller is processing the consumer's personal
45 data and to access such personal data.
46 (2) To correct inaccuracies in the consumer's personal data, taking into account
47 the nature of the personal data and the purposes of the processing of the
48 consumer's personal data.
49 (3) To delete personal data provided by or obtained about the consumer.
50 (4) To obtain a copy of the consumer's personal data that the consumer previously
51 provided to the controller in a portable and, to the extent technically feasible, 
General Assembly Of North Carolina Session 2021
Page 6 Senate Bill 569-First Edition
1 readily usable format that allows the consumer to transmit the data to another
2 controller without hindrance, where the processing is carried out by automated
3 means.
4 (5) To opt out of the processing of the personal data for purposes of (i) targeted
5 advertising, (ii) the sale of personal data, or (iii) profiling in furtherance of
6 decisions that produce legal or similarly significant effects concerning the
7 consumer.
8 (b) Except as otherwise provided in this Article, a controller shall comply with a request
9 by a consumer to exercise the consumer rights authorized pursuant to subsection (a) of this
10 section as follows:
11 (1) A controller shall respond to the consumer without undue delay, but in all
12 cases within 45 days of receipt of the request submitted pursuant to
13 G.S. 75-71. The response period may be extended once by 45 additional days
14 when reasonably necessary, taking into account the complexity and number
15 of the consumer's requests, so long as the controller informs the consumer of
16 any such extension within the initial 45-day response period, together with the
17 reason for the extension.
18 (2) If a controller declines to take action regarding the consumer's request, the
19 controller shall inform the consumer without undue delay, but in all cases and
20 at the latest within 45 days of receipt of the request, of the justification for
21 declining to take action and instructions for how to appeal the decision
22 pursuant to subsection (c) of this section.
23 (3) Information provided in response to a consumer request shall be provided by
24 a controller free of charge, up to twice annually per consumer. If requests from
25 a consumer are manifestly unfounded, excessive, or repetitive, the controller
26 may charge the consumer a reasonable fee to cover the administrative costs of
27 complying with the request or decline to act on the request. The controller
28 bears the burden of demonstrating the manifestly unfounded, excessive, or
29 repetitive nature of the request.
30 (4) If a controller is unable to authenticate the request using commercially
31 reasonable efforts, the controller shall not be required to comply with a request
32 to initiate an action under subsection (a) of this section and may request that
33 the consumer provide additional information reasonably necessary to
34 authenticate the consumer and the consumer's request.
35 (c) A controller shall establish a process for a consumer to appeal the controller's refusal
36 to take action on a request within a reasonable period of time after the consumer's receipt of the
37 decision pursuant to subdivision (b)(2) of this section. The appeal process shall be conspicuously
38 available and similar to the process for submitting requests to initiate action pursuant to
39 subsection (a) of this section. Within 60 days of receipt of an appeal, a controller shall inform the
40 consumer in writing of any action taken or not taken in response to the appeal, including a written
41 explanation of the reasons for the decisions. If the appeal is denied, the controller shall also
42 provide the consumer with an online mechanism, if available, or other method through which the
43 consumer may contact the Attorney General to submit a complaint.
44 "§ 75-72. Data controller responsibilities; transparency.
45 (a) A controller shall do all of the following:
46 (1) Limit the collection of personal data to what is adequate, relevant, and
47 reasonably necessary in relation to the purposes for which such data is
48 processed, as disclosed to the consumer.
49 (2) Except as otherwise provided in this Article, not process personal data for
50 purposes that are neither reasonably necessary to nor compatible with the 
General Assembly Of North Carolina Session 2021
Senate Bill 569-First Edition Page 7
1 disclosed purposes for which such personal data is processed, as disclosed to
2 the consumer, unless the controller obtains the consumer's consent.
3 (3) Establish, implement, and maintain reasonable administrative, technical, and
4 physical data security practices to protect the confidentiality, integrity, and
5 accessibility of personal data. Such data security practices shall be appropriate
6 to the volume and nature of the personal data at issue.
7 (4) Not process personal data in violation of State and federal laws that prohibit
8 unlawful discrimination against consumers. A controller shall not
9 discriminate against a consumer for exercising any of the consumer rights
10 contained in this Article, including denying goods or services, charging
11 different prices or rates for goods or services, or providing a different level of
12 quality of goods and services to the consumer. However, nothing in this
13 subdivision shall be construed to require a controller to provide a product or
14 service that requires the personal data of a consumer that the controller does
15 not collect or maintain or to prohibit a controller from offering a different
16 price, rate, level, quality, or selection of goods or services to a consumer,
17 including offering goods or services for no fee, if the consumer has exercised
18 the right to opt out pursuant to G.S. 75-71, or the offer is related to a
19 consumer's voluntary participation in a bona fide loyalty, rewards, premium
20 features, discounts, or club card program.
21 (5) Not process sensitive data concerning a consumer without obtaining the
22 consumer's consent, or, in the case of the processing of sensitive data
23 concerning a known child, without processing such data in accordance with
24 the federal Children's Online Privacy Protection Act, 15 U.S.C. § 6501, et seq.
25 (b) Any provision of a contract or agreement of any kind that purports to waive or limit
26 in any way consumer rights pursuant to G.S. 75-71 shall be deemed contrary to public policy and
27 shall be void and unenforceable.
28 (c) Controllers shall provide consumers with a reasonably accessible, clear, and
29 meaningful privacy notice that includes all of the following:
30 (1) The categories of personal data processed by the controller.
31 (2) The purpose for processing personal data.
32 (3) How consumers may exercise their consumer rights pursuant to G.S. 75-71,
33 including how a consumer may appeal a controller's decision with regard to
34 the consumer's request.
35 (4) The categories of personal data that the controller shares with third parties, if
36 any.
37 (5) The categories of third parties, if any, with whom the controller shares
38 personal data.
39 (d) If a controller sells personal data to third parties or processes personal data for targeted
40 advertising, the controller shall clearly and conspicuously disclose such processing, as well as
41 the manner in which a consumer may exercise the right to opt out of such processing.
42 (e) A controller shall establish, and shall describe in a privacy notice, one or more secure
43 and reliable means for consumers to submit a request to exercise their consumer rights under this
44 Article. Such means shall take into account the ways in which consumers normally interact with
45 the controller, the need for secure and reliable communication of such requests, and the ability
46 of the controller to authenticate the identity of the consumer making the request. Controllers shall
47 not require a consumer to create a new account in order to exercise consumer rights pursuant to
48 G.S. 75-71 but may require a consumer to use an existing account.
49 "§ 75-73. Responsibilities according to role; controllers and processors.
General Assembly Of North Carolina Session 2021
Page 8 Senate Bill 569-First Edition
1 (a) A processor shall adhere to the instructions of a controller and shall assist the
2 controller in meeting its obligations under this Article. The assistance shall include all of the
3 following:
4 (1) Taking into account the nature of processing and the information available to
5 the processor, by appropriate technical and organizational measures, insofar
6 as this is reasonably practicable, to fulfill the controller's obligation to respond
7 to consumer rights requests pursuant to G.S. 75-71.
8 (2) Taking into account the nature of processing and the information available to
9 the processor, by assisting the controller in meeting the controller's obligations
10 in relation to the security of processing the personal data and in relation to the
11 notification of a breach of security of the system of the processor pursuant to
12 in order to meet the controller's obligations.
13 (3) Providing necessary information to enable the controller to conduct and
14 document data protection assessments pursuant to G.S. 75-74.
15 (b) A contract between a controller and a processor shall govern the processor's data
16 processing procedures with respect to processing performed on behalf of the controller. The
17 contract shall be binding and clearly set forth instructions for processing data, the nature and
18 purpose of processing, the type of data subject to processing, the duration of processing, and the
19 rights and obligations of both parties. The contract shall also include requirements that the
20 processor shall do all of the following:
21 (1) Ensure that each person processing personal data is subject to a duty of
22 confidentiality with respect to the data.
23 (2) At the controller's direction, delete or return all personal data to the controller
24 as requested at the end of the provision of services, unless retention of the
25 personal data is required by law.
26 (3) Upon the reasonable request of the controller, make available to the controller
27 all information in its possession necessary to demonstrate the processor's
28 compliance with the obligations in this Article.
29 (4) Allow, and cooperate with, reasonable assessments by the controller or the
30 controller's designated assessor; alternatively, the processor may arrange for
31 a qualified and independent assessor to conduct an assessment of the
32 processor's policies and technical and organizational measures in support of
33 the obligations under this Article using an appropriate and accepted control
34 standard or framework and assessment procedure for such assessments. The
35 processor shall provide a report of such assessment to the controller upon
36 request.
37 (5) Engage any subcontractor pursuant to a written contract in accordance with
38 subsection (b) of this section that requires the subcontractor to meet the
39 obligations of the processor with respect to the personal data.
40 (c) Nothing in this section shall be construed to relieve a controller or a processor from
41 the liabilities imposed on it by virtue of its role in the processing relationship as defined by this
42 Article.
43 (d) Determining whether a person is acting as a controller or processor with respect to a
44 specific processing of data is a fact-based determination that depends upon the context in which
45 personal data is to be processed. A processor that continues to adhere to a controller's instructions
46 with respect to a specific processing of personal data remains a processor.
47 "§ 75-74. Data protection assessments.
48 (a) At least annually, a controller shall conduct and document a data protection
49 assessment of each of the following processing activities involving personal data:
50 (1) The processing of personal data for purposes of targeted advertising.
51 (2) The sale of personal data.
General Assembly Of North Carolina Session 2021
Senate Bill 569-First Edition Page 9
1 (3) The processing of personal data for purposes of profiling, where such profiling
2 presents a reasonably foreseeable risk of (i) unfair or deceptive treatment of,
3 or unlawful disparate impact on, consumers, (ii) financial, physical, or
4 reputational injury to consumers, (iii) a physical or other intrusion upon the
5 solitude or seclusion, or the private affairs or concerns, of consumers, where
6 such intrusion would be offensive to a reasonable person, or (iv) other
7 substantial injury to consumers.
8 (4) The processing of sensitive data.
9 (5) Any processing activities involving personal data that present a heightened
10 risk of harm to consumers.
11 (b) Data protection assessments conducted pursuant to subsection (a) of this section shall
12 identify the following:
13 (1) The benefits that may flow, directly and indirectly, from the processing to the
14 controller, the consumer, or other stakeholders.
15 (2) The potential risks to the rights of the consumer associated with the
16 processing.
17 (3) The mitigation by safeguards that will be employed by the controller to reduce
18 any risks to the consumer.
19 (4) An analysis of the use of de-identified data, factoring in the reasonable
20 expectations of consumers, as well as the context of the processing and the
21 relationship between the controller and the consumer whose personal data will
22 be processed and the extent to which de-identified data can be used in the
23 place of other data.
24 (5) A cybersecurity analysis, including established processes to identify potential
25 risks to the security of personal information and an action plan to remedy
26 deficiencies.
27 (c) The Attorney General may request that a controller disclose any data protection
28 assessment that is relevant to an investigation conducted by the Attorney General, and the
29 controller shall make the data protection assessment available to the Attorney General. The
30 Attorney General may evaluate the data protection assessment for compliance with the
31 responsibilities set forth in G.S. 75-72. The disclosure of a data protection assessment pursuant
32 to a request from the Attorney General shall not constitute a waiver of attorney-client privilege
33 or work product protection with respect to the assessment and any information contained in the
34 assessment. Data protection assessments shall be confidential and exempt from public inspection
35 and copying under Chapter 132 of the General Statutes.
36 (d) A single data protection assessment may address a comparable set of processing
37 operations that include similar activities.
38 (e) Data protection assessments conducted by a controller for the purpose of compliance
39 with other laws or regulations may comply under this section if the assessments have a reasonably
40 comparable scope and effect.
41 (f) Data protection assessment requirements shall apply to processing activities created
42 or generated after January 1, 2023, and are not retroactive.
43 "§ 75-75. Processing de-identified data; exemptions.
44 (a) The controller in possession of de-identified data shall do all of the following:
45 (1) Take reasonable measures to ensure that the data cannot be associated with a
46 natural person.
47 (2) Publicly commit to maintaining and using de-identified data without
48 attempting to re-identify the data.
49 (3) Contractually obligate any recipients of the de-identified data to comply with
50 all provisions of this Article.
General Assembly Of North Carolina Session 2021
Page 10 Senate Bill 569-First Edition
1 (b) Nothing in this Article shall be construed to (i) require a controller or processor to
2 re-identify de-identified data or pseudonymous data or (ii) maintain data in identifiable form, or
3 collect, obtain, retain, or access any data or technology, in order to be capable of associating an
4 authenticated consumer request with personal data.
5 (c) Nothing in this Article shall be construed to require a controller or processor to
6 comply with an authenticated consumer rights request, pursuant to G.S. 75-71, if all of the
7 following circumstances exist:
8 (1) The controller is not reasonably capable of associating the request with the
9 personal data or it would be unreasonably burdensome for the controller to
10 associate the request with the personal data.
11 (2) The controller does not use the personal data to recognize or respond to the
12 specific consumer who is the subject of the personal data, or associate the
13 personal data with other personal data about the same specific consumer; and
14 (3) The controller does not sell the personal data to any third party or otherwise
15 voluntarily disclose the personal data to any third party other than a processor,
16 except as otherwise permitted in this section.
17 (d) The consumer rights contained in subdivisions (a)(1) through (a)(4) of G.S. 75-71 and
18 G.S. 75-72 shall not apply to pseudonymous data in cases where the controller is able to
19 demonstrate any information necessary to identify the consumer is kept separately and is subject
20 to effective technical and organizational controls that prevent the controller from accessing such
21 information.
22 (e) A controller that discloses pseudonymous data or de-identified data shall exercise
23 reasonable oversight to monitor compliance with any contractual commitments to which the
24 pseudonymous data or de-identified data is subject and shall take appropriate steps to address
25 any breaches of those contractual commitments.
26 "§ 75-76. Limitations.
27 (a) Nothing in this Article shall be construed to restrict a controller's or processor's ability
28 to do any of the following:
29 (1) Comply with federal, State, or local laws, rules, or regulations.
30 (2) Comply with a civil, criminal, or regulatory inquiry, investigation, subpoena,
31 or summons by federal, State, local, or other governmental authorities.
32 (3) Cooperate with law enforcement agencies concerning conduct or activity that
33 the controller or processor reasonably and in good faith believes may violate
34 federal, State, or local law, rule, or regulation.
35 (4) Investigate, establish, exercise, prepare for, or defend legal claims.
36 (5) Provide a product or service specifically requested by a consumer, perform a
37 contract to which the consumer is a party, including fulfilling the terms of a
38 written warranty, or take steps at the request of the consumer prior to entering
39 into a contract.
40 (6) Take immediate steps to protect an interest that is essential for the life or
41 physical safety of the consumer or of another natural person, and where the
42 processing cannot be manifestly based on another legal basis.
43 (7) Prevent, detect, protect against, or respond to security incidents, identity theft,
44 fraud, harassment, malicious or deceptive activities, or any illegal activity;
45 preserve the integrity or security of systems; or investigate, report, or
46 prosecute those responsible for any such action.
47 (8) Engage in public or peer-reviewed scientific or statistical research in the
48 public interest that adheres to all other applicable ethics and privacy laws and
49 is approved, monitored, and governed by an institutional review board, or
50 similar independent oversight entities that determine (i) if the deletion of the
51 information is likely to provide substantial benefits that do not exclusively 
General Assembly Of North Carolina Session 2021
Senate Bill 569-First Edition Page 11
1 accrue to the controller, (ii) the expected benefits of the research outweigh the
2 privacy risks, and (iii) if the controller has implemented reasonable safeguards
3 to mitigate privacy risks associated with research, including any risks
4 associated with reidentification.
5 (9) Assist another controller, processor, or third party with any of the obligations
6 under this subsection.
7 (b) The obligations imposed on controllers or processors under this Article shall not
8 restrict a controller's or processor's ability to collect, use, or retain data to do any of the following:
9 (1) Conduct internal research to develop, improve, or repair products, services, or
10 technology.
11 (2) Effectuate a product recall.
12 (3) Identify and repair technical errors that impair existing or intended
13 functionality.
14 (4) Perform internal operations that are reasonably aligned with the expectations
15 of the consumer or reasonably anticipated based on the consumer's existing
16 relationship with the controller or are otherwise compatible with processing
17 data in furtherance of the provision of a product or service specifically
18 requested by a consumer or the performance of a contract to which the
19 consumer is a party.
20 (c) The obligations imposed on controllers or processors under this Article shall not apply
21 where compliance by the controller or processor with this Article would violate an evidentiary
22 privilege under the laws of this State. Nothing in this Article shall be construed to prevent a
23 controller or processor from providing personal data concerning a consumer to a person covered
24 by an evidentiary privilege under the laws of this State as part of a privileged communication.
25 (d) A controller or processor that discloses personal data to a third-party controller or
26 processor, in compliance with the requirements of this Article, is not in violation of this Article
27 if the third-party controller or processor that receives and processes such personal data is in
28 violation of this Article, provided that, at the time of disclosing the personal data, the disclosing
29 controller or processor did not have actual knowledge that the recipient intended to commit a
30 violation. A third-party controller or processor receiving personal data from a controller or
31 processor in compliance with the requirements of this Article is likewise not in violation of this
32 Article for the transgressions of the controller or processor from which it receives the personal
33 data.
34 (e) Nothing in this Article shall be construed as an obligation imposed on controllers and
35 processors that adversely affects the rights or freedoms of any persons, such as exercising the
36 right of free speech pursuant to the First Amendment to the United States Constitution, or applies
37 to the processing of personal data by a person in the course of a purely personal or household
38 activity.
39 (f) Personal data processed by a controller pursuant to this section shall not be processed
40 for any purpose other than those expressly listed in this section unless otherwise allowed by this
41 Article. Personal data processed by a controller pursuant to this section may be processed to the
42 extent that such processing is:
43 (1) Reasonably necessary and proportionate to the purposes listed in this section;
44 and
45 (2) Adequate, relevant, and limited to what is necessary in relation to the specific
46 purposes listed in this section. Personal data collected, used, or retained
47 pursuant to subsection (b) of this section shall, where applicable, take into
48 account the nature and purpose or purposes of such collection, use, or
49 retention. The data shall be subject to reasonable administrative, technical,
50 and physical measures to protect the confidentiality, integrity, and
51 accessibility of the personal data and to reduce reasonably foreseeable risks 
General Assembly Of North Carolina Session 2021
Page 12 Senate Bill 569-First Edition
1 of harm to consumers relating to such collection, use, or retention of personal
2 data.
3 (g) If a controller processes personal data pursuant to an exemption in this section, the
4 controller bears the burden of demonstrating that such processing qualifies for the exemption and
5 complies with the requirements in subsection (f) of this section.
6 (h) Processing personal data for the purposes expressly identified in subdivisions (a)(1)
7 through (a)(9) of this section shall not solely make an entity a controller with respect to such
8 processing.
9 "§ 75-77. Investigations; enforcement; civil penalties; expenses.
10 (a) A violation of this Article is a violation of G.S. 75-1.1.
11 (b) The Attorney General shall enforce this Article, except if a private right of action for
12 a violation of this Article arises for any person injured as a result of the violation.
13 (c) Prior to initiating any action under this Article, the Attorney General may provide a
14 controller or processor 30 days' written notice identifying the specific provisions of this Article
15 the Attorney General alleges have been or are being violated. If within the 30-day period, the
16 controller or processor cures the noticed violation and provides the Attorney General an express
17 written statement that the alleged violations have been cured and that no further violations shall
18 occur, no action shall be initiated against the controller or processor.
19 (d) If a controller or processor continues to violate this Article following the cure period
20 or breaches an express written statement provided to the Attorney General under subsection (b)
21 of this section, the Attorney General may initiate an action seeking an injunction to restrain any
22 violations of this Article and civil penalties of up to five thousand dollars ($5,000) for each
23 violation of this Article. The Attorney General may recover reasonable expenses incurred in
24 investigating and preparing the case, including attorney fees, in any action initiated under this
25 Article.
26 (e) An injured person seeking damages under this Article may also institute a civil action
27 to enjoin and restrain future acts that would constitute a violation of this Article. The court, in an
28 action brought under this section, may award reasonable attorneys' fees to the prevailing party.
29 A deceased person's estate shall have the right to recover damages pursuant to this section.
30 (f) The venue for any civil action brought under this section shall be the county in which
31 the plaintiff resides or any county in which any part of the alleged violation of this Article took
32 place, regardless of whether the defendant was ever actually present in that county.
33 (g) A civil action under this Article must be brought within three years from the date on
34 which the violation was discovered or reasonably should have been discovered.
35 "§ 77-78. Miscellaneous provisions.
36 (a) There is hereby created the Consumer Privacy Fund (Fund). Moneys in the Fund shall
37 be used to support the work of the Attorney General to enforce the provisions of this Article,
38 subject to appropriation by the General Assembly. Interest earned on moneys in the Fund shall
39 remain in the Fund and be credited to it. Any moneys remaining in the Fund, including interest
40 thereon, at the end of each fiscal year shall not revert to the General Fund but shall remain in the
41 Fund.
42 (b) The Joint Legislative Oversight Committee on Information Technology (Oversight
43 Committee) shall create a work group to review the provisions of this Article and issues related
44 to its implementation. The ex officio members of the working group shall include the following
45 official or the official's designee: Commissioners of Agriculture and Insurance, State Chief
46 Information Officer, and the Secretary of Commerce and may include industry representatives
47 and members of the general public. The working group shall report to the Oversight Committee
48 by no later than October 1, 2021, and then at least annually thereafter.
49 (c) Any reference to federal law or statute in this Article shall be deemed to include any
50 accompanying rules or regulations or exemptions thereto."
51 SECTION 3. G.S. 114-2 is amended by adding a new subdivision to read:
General Assembly Of North Carolina Session 2021
Senate Bill 569-First Edition Page 13
1 "(11) To conduct investigative and enforcement activities under Article 2B of
2 Chapter 75 of the General Statutes, the Consumer Privacy Act of North
3 Carolina."
4 SECTION 4. Sections 1, 2, and 3 of this act become effective January 1, 2023. The
5 remainder of this act is effective when it becomes law.