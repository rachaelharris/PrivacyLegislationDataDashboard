PRINTER'S NO. 2507
THE GENERAL ASSEMBLY OF PENNSYLVANIA
HOUSE BILL
No. 2202 Session of
2021
INTRODUCED BY MERCURI, N. NELSON, GROVE, ARMANINI, CIRESI, COOK,
SCHLEGEL CULVER, DRISCOLL, GAYDOS, HILL-EVANS, JAMES,
KAUFFMAN, KEEFER, KLUNK, LEWIS DELROSSO, MAJOR, MILLARD,
B. MILLER, MIZGORSKI, OBERLANDER, PISCIOTTANO, PYLE, ROSSI,
RYAN, STAATS, STAMBAUGH, THOMAS, TWARDZIK, WHEELAND AND
D. WILLIAMS, DECEMBER 13, 2021
REFERRED TO COMMITTEE ON CONSUMER AFFAIRS, DECEMBER 13, 2021
AN ACT
Providing for consumer data privacy, for rights of consumers and
duties of businesses relating to the collection of personal
information and for duties of the Attorney General.
The General Assembly of the Commonwealth of Pennsylvania
hereby enacts as follows:
Section 1. Short title.
This act shall be known and may be cited as the Consumer Data
Privacy Act.
Section 2. Definitions.
The following words and phrases when used in this act shall
have the meanings given to them in this section unless the
context clearly indicates otherwise:
"Biometric information." Personal information generated from
the measurement or specific technological processing of an
individual's unique biological, physical or physiological
characteristics, including any fingerprint, voice print, iris or
1
2
3
4
5
6
7
8
9
10
11
12
13
14
15
16
retina scan, facial scan or template, deoxyribonucleic acid
(DNA) information or gait. The term does not include any writing
sample, written signature, photograph, voice recording, video,
demographic data or physical characteristics, including height,
weight, hair color or eye color, if the information is not used
for the purpose of identifying an individual's unique
biological, physical or physiological characteristics.
"Business." The following:
(1) A sole proprietorship, partnership, limited
liability company, corporation, association or other legal
entity that is organized or operated for the profit or
financial benefit of its shareholders or other owners, that
collects consumers' personal information, or on the behalf of
which such information is collected, that alone, or jointly
with others, determines the purposes and means of the
processing of consumers' personal information, that does
business in this Commonwealth and that satisfies one or more
of the following thresholds:
(i) Has annual gross revenues in excess of
$20,000,000.
(ii) Alone or in combination, annually buys,
receives for the business's commercial purposes, sells or
shares for commercial purposes, alone or in combination,
the personal information of 100,000 or more consumers.
(iii) Derives 50% or more of annual revenues from
selling consumers' personal information.
(2) An entity that controls, is controlled by or is
under common control with a business under paragraph (1) or
shares common branding with the business.
"Common branding." A shared name, servicemark or trademark.
20210HB2202PN2507 - 2 -
1
2
3
4
5
6
7
8
9
10
11
12
13
14
15
16
17
18
19
20
21
22
23
24
25
26
27
28
29
30
"Consent." A clear and affirmative act, including a written
or electronic statement, signifying a consumer's freely given,
specific, informed and unambiguous agreement to the processing
of personal information. The term does not include any of the
following:
(1) Acceptance of general or broad terms of use or a
similar document that contains descriptions of personal
information processing with other unrelated information.
(2) Hovering over, muting, pausing or closing a piece of
content.
(3) An agreement obtained through use of a design,
modification or manipulation of a user interface with the
purpose or substantial effect of obscuring, subverting or
impairing user autonomy, decision making or choice as
specified in the regulations promulgated under section 3(n).
"Consumer." An individual who is a resident of this
Commonwealth acting only in the context of the individual or the
individual's household. The term does not include an individual
acting in a commercial or employment context, as a job applicant
or as a beneficiary of an individual acting in an employment
context.
"Control." Ownership of or the power to vote on more than
50% of the outstanding shares of any class of voting security of
a business, control in any manner over the election of a
majority of the directors, or of individuals exercising similar
functions, or the power to exercise a controlling influence over
the management of a company.
"Decisions that produce legal or similarly significant
effects." Decisions that result in the provision or denial of
financial and lending services, housing, insurance, education
20210HB2202PN2507 - 3 -
1
2
3
4
5
6
7
8
9
10
11
12
13
14
15
16
17
18
19
20
21
22
23
24
25
26
27
28
29
30
enrollment, criminal justice, employment opportunities, health
care services or access to basic necessities, including food or
water.
"Deidentified data." Data that cannot reasonably be used to
infer information about, or otherwise be linked to, an
identified or identifiable individual or a device linked to the
individual and is possessed by a business that:
(1) takes reasonable measures to ensure that the data
cannot be associated with the individual;
(2) publicly commits to maintain and use the data only
in a deidentified manner and not attempt to reidentify the
data; and
(3) contractually obligates a recipient of the data to
meet the criteria specified in this definition.
"Personal information." Information that identifies or could
reasonably be linked, directly or indirectly, with a particular
consumer, household or consumer device. The term does not
include any of the following:
(1) Information that is lawfully made available from
Federal, state or local government records.
(2) Consumer information that is deidentified or
aggregate consumer information.
"Process" or "processing." Any operation or set of
operations that are performed on personal information or on sets
of personal information, whether or not by automated means,
including the collection, use, storage, disclosure, analysis,
deletion or modification of personal information.
"Profiling." A form of automated processing of personal
information to evaluate, analyze or predict personal aspects
concerning an identified individual or identifiable individual,
20210HB2202PN2507 - 4 -
1
2
3
4
5
6
7
8
9
10
11
12
13
14
15
16
17
18
19
20
21
22
23
24
25
26
27
28
29
30
including the individual's economic situation, health, personal
preferences, interests, reliability, behavior, location or
movements.
"Publicly available." Information that is lawfully made
available from Federal, State or local government records or
information that a business has a reasonable basis to believe is
lawfully made available to the general public through widely
distributed media, by the consumer or by a person to whom the
consumer has disclosed the information, unless the consumer has
restricted the information to a specific audience. The term does
not include biometric information collected by a business about
a consumer without the consumer's knowledge or consumer
information that is deidentified or aggregate consumer
information.
"Sale," "sell" or "sold." The exchange of personal
information for monetary or other valuable consideration by a
business to a third party. The term does not include any of the
following:
(1) The disclosure of personal information to a service
provider that processes the personal information on behalf of
a business.
(2) The disclosure of personal information to a third
party for the purpose of providing a product or service
requested by a consumer.
(3) The disclosure or transfer of personal information
to an affiliate of a business.
(4) The disclosure or transfer to a third party of
personal information as an asset that is part of a proposed
or actual merger, acquisition, bankruptcy or other
transaction in which the third party assumes control of all
20210HB2202PN2507 - 5 -
1
2
3
4
5
6
7
8
9
10
11
12
13
14
15
16
17
18
19
20
21
22
23
24
25
26
27
28
29
30
or part of a business's assets.
(5) The disclosure of personal information that:
(i) a consumer directs a business to disclose or
intentionally discloses by using the business to interact
with a third party; or
(ii) is intentionally made available by a consumer
to the general public via a channel of mass media unless
the consumer has restricted the information to a specific
audience.
"Service provider." A person that processes personal
information on behalf of a business.
"Targeted advertising." Displaying to a consumer an
advertisement that is selected based on personal information
obtained or inferred during a period of time from the consumer's
activities across nonaffiliated Internet websites, applications
or online services to predict consumer preferences or interests.
The term does not include any of the following:
(1) Advertising to a consumer in response to the
consumer's request for information or feedback.
(2) Advertising based on activities within a business's
own Internet website or online applications.
(3) Advertising based on the context of a consumer's
current search query or visit to an Internet website or
online application.
"Third party." Any person, public authority, public agency,
entity or body other than a consumer, business, service provider
or an affiliate of the business or service provider.
Section 3. Consumer data privacy.
(a) General rule.--A consumer shall have the right to:
(1) Know whether a business is processing personal
20210HB2202PN2507 - 6 -
1
2
3
4
5
6
7
8
9
10
11
12
13
14
15
16
17
18
19
20
21
22
23
24
25
26
27
28
29
30
information about the consumer.
(2) Know whether the consumer's personal information is
processed for the purpose of targeted advertising or the sale
of personal information.
(3) Decline or opt out of the processing of the
consumer's personal information for the purpose of any of the
following:
(i) Targeted advertising.
(ii) The sale of personal information.
(iii) Profiling in furtherance of decisions that
produce legal or similarly significant effects concerning
a consumer.
(4) Access the consumer's personal information.
(5) Correct inaccurate personal information concerning
the consumer, taking into account the nature of the personal
information and the purpose of the processing of the personal
information.
(6) Request that a business delete personal information
that the business processes about the consumer. The following
shall apply to this paragraph:
(i) A business that collects personal information
about a consumer shall disclose under subsection (l) the
consumer's right to request the deletion of the
consumer's personal information.
(ii) Except as otherwise provided under this act, a
business that receives a verifiable request from a
consumer to delete the consumer's personal information
shall delete the consumer's personal information from its
records and direct a service provider who processes the
consumer's personal information on the business's behalf
20210HB2202PN2507 - 7 -
1
2
3
4
5
6
7
8
9
10
11
12
13
14
15
16
17
18
19
20
21
22
23
24
25
26
27
28
29
30
to delete the personal information within 45 calendar
days.
(7) Obtain personal information previously provided by
the consumer to the business in a portable and, to the extent
technically feasible, readily usable format that allows the
consumer to transmit the personal information to another
business without hindrance, when the processing of the
personal information is carried out by automated means.
(b) Disclosure by businesses.--A business shall provide a
consumer with a reasonably accessible, clear and meaningful
privacy notice, including the following:
(1) The categories of personal information the business
processes.
(2) The categories of sources from which the personal
information is collected.
(3) The purpose for processing the categories of
personal information.
(4) The categories of personal information that the
business shares with a third party, if applicable.
(5) The specific pieces of personal information the
business has collected about the consumer.
(6) How and where the consumer may exercise the
consumers' rights provided under this act.
(7) If the business sells personal information to a
third party or processes personal information for targeted
advertising, the sale or processing and the manner in which a
consumer may exercise the consumer's right to opt out of the
sale or processing.
(c) Request from consumer.--Nothing in this section shall be
construed to require a business to:
20210HB2202PN2507 - 8 -
1
2
3
4
5
6
7
8
9
10
11
12
13
14
15
16
17
18
19
20
21
22
23
24
25
26
27
28
29
30
(1) retain any personal information about a consumer
collected for a single one-time transaction if, in the
ordinary course of business, that information about the
consumer is not retained; or
(2) reidentify or otherwise link any data that, in the
ordinary course of business, is not maintained in a manner
that would be considered personal information.
(d) Consumers of young age.--A business may not process a
consumer's personal information for the purpose of targeted
advertising or the sale of personal information if the business
has actual knowledge that the consumer is less than 16 years of
age, unless the consumer, in the case of a consumer who is
between 13 and 16 years of age, or the consumer's parent or
guardian, in the case of a consumer who is less than 13 years of
age, has consented to the processing. A business that willfully
disregards the consumer's age shall be deemed to have had actual
knowledge of the consumer's age.
(e) Duties of care.--A business or service provider shall
implement and maintain reasonable security procedures and
practices, including administrative, physical and technical
safeguards, appropriate to the nature of the personal
information and the purposes for which the personal information
will be used, to protect consumers' personal information from
unauthorized use, disclosure, access, destruction or
modification.
(f) Duties of data minimization.--A business's collection of
personal information shall be adequate, relevant and limited to
what is reasonably necessary regarding the purpose for which the
personal information is processed.
(g) Duties to avoid secondary use.--Except as provided under
20210HB2202PN2507 - 9 -
1
2
3
4
5
6
7
8
9
10
11
12
13
14
15
16
17
18
19
20
21
22
23
24
25
26
27
28
29
30
this act, a business may not process personal information for a
purpose that is not reasonably necessary to, or compatible with,
the purpose for which the personal information is processed
unless the business obtains the consumer's consent.
(h) Duties to avoid unlawful discrimination.--A business may
not process personal information in violation of a Federal or
State law that prohibits unlawful discrimination against
consumers.
(i) Discrimination prohibited.--
(1) A business shall not discriminate against a consumer
because the consumer exercised any of the consumer's rights
under this section, including, but not limited to, by:
(i) Denying goods or services to the consumer.
(ii) Charging different prices or rates for goods or
services, including through the use of discounts or other
benefits or imposing penalties.
(iii) Providing a different level or quality of
goods or services to the consumer.
(iv) Suggesting that the consumer will receive a
different price or rate for goods or services or a
different level or quality of goods or services.
(2) Nothing in this subsection shall prohibit a business
from charging a consumer a different price or rate, or from
providing a different level or quality of goods or services
to the consumer, if that difference is reasonably related to
the value provided to the consumer by the consumer's data.
(j) Exercise of rights.--A business shall:
(1) In a form that is reasonably accessible to
consumers, make available to consumers two or more designated
methods for submitting verifiable requests to exercise the
20210HB2202PN2507 - 10 -
1
2
3
4
5
6
7
8
9
10
11
12
13
14
15
16
17
18
19
20
21
22
23
24
25
26
27
28
29
30
rights specified under subsection (a), including, but not
limited to, a publicly accessible Internet website.
(2) Respond to a consumer's verifiable request under
paragraph (1) free of charge within 45 days of receiving the
verifiable request from the consumer. The time period to
respond to the verifiable request may be extended once by an
additional 45 days when reasonably necessary, provided the
consumer is provided notice of the extension within the first
45-day period. A business shall not be required to provide
the information required under subsection (l) to a consumer
more than once during a 12-month period.
(3) Ensure that all individuals responsible for handling
consumer inquiries about the business's privacy practices are
informed of the requirements of this section and how to
direct consumers to exercise their rights.
(4) For a consumer who exercises the consumer's right to
opt out of the processing of the consumer's personal
information for the purpose of targeted advertising or the
sale of personal information, refrain from processing the
personal information for the purpose of targeted advertising
or the sale of personal information unless the consumer
subsequently consents to the processing. This paragraph shall
apply to a consumer who communicates or signals the
consumer's right to opt out via user-enabled global privacy
controls, including browser plug-in or privacy settings,
device settings or any other mechanism.
(5) For a consumer who exercises the consumer's right to
opt out of the processing of the consumer's personal
information for the purpose of targeted advertising or the
sale of personal information, respect the consumer's decision
20210HB2202PN2507 - 11 -
1
2
3
4
5
6
7
8
9
10
11
12
13
14
15
16
17
18
19
20
21
22
23
24
25
26
27
28
29
30
to opt out for a period of no less than 12 months before
requesting the consumer's consent to the processing.
(6) Use personal information collected from the consumer
in relation to the consumer's verifiable request under
paragraph (1) for the sole purpose of complying with the
verifiable request.
(k) Obligations on business.--
(1) The obligations imposed on a business or service
provider under this section shall not restrict the ability of
a business or service provider to:
(i) Comply with Federal, State or local laws.
(ii) Comply with a civil, criminal or regulatory
inquiry, investigation, subpoena or summons by Federal,
State or local authorities.
(iii) Cooperate with law enforcement agencies
concerning conduct or activity that the business, service
provider or third party reasonably and in good faith
believes may violate Federal, State or local laws.
(iv) Exercise or defend legal claims.
(v) Collect, use, retain, sell or disclose consumer
information that is deidentified.
(vi) Collect or sell a consumer's personal
information if every aspect of that commercial conduct
takes place wholly outside of this Commonwealth. For
purposes of this section, commercial conduct takes place
wholly outside of this Commonwealth if the business
collected that information while the consumer was outside
of this Commonwealth, no part of the sale of the
consumer's personal information occurred in this
Commonwealth and no personal information collected while
20210HB2202PN2507 - 12 -
1
2
3
4
5
6
7
8
9
10
11
12
13
14
15
16
17
18
19
20
21
22
23
24
25
26
27
28
29
30
the consumer was in this Commonwealth is sold. This
subparagraph shall not permit a business to store,
including on a device, personal information about a
consumer when the consumer is in this Commonwealth and
then collecting that personal information when the
consumer and stored personal information is outside of
this Commonwealth.
(vii) Provide a product or service specifically
requested by a consumer, perform a contract to which the
consumer is a party or take steps at the request of the
consumer before entering into the contract or offer a
voluntary bona fide loyalty or rewards program.
(viii) Take immediate steps to protect an interest
that is essential for the life of the consumer or another
individual if the processing cannot otherwise be
authorized under this act.
(ix) Prevent, detect, protect against or respond to
a security incident, identity theft, fraud, harassment, a
malicious or deceptive activity or an illegal activity to
preserve the integrity or security of the system or to
investigate, report or prosecute a person responsible for
an activity specified under this subparagraph.
(x) Engage in public or peer-reviewed scientific,
historical or statistical research in the public interest
that adheres to applicable Federal and State laws and is
approved, monitored and governed by an institutional
review board, human subjects research ethics review board
or a similar independent oversight entity, which
determines all of the following:
(A) If the research is likely to provide
20210HB2202PN2507 - 13 -
1
2
3
4
5
6
7
8
9
10
11
12
13
14
15
16
17
18
19
20
21
22
23
24
25
26
27
28
29
30
substantial benefits that do not exclusively accrue
to the controller.
(B) If the expected benefits of the research
outweigh the privacy risks.
(C) If the controller has implemented reasonable
safeguards to mitigate privacy risks associated with
the research, including any risks associated with
reidentification.
(2) The obligations imposed on a business or service
provider under this section shall not restrict the ability of
a business or service provider to collect, use or retain
information for any of the following purposes:
(i) Conducting internal research to improve, repair
or develop products, services or technology.
(ii) Performing internal operations that are
reasonably aligned with the expectations of the consumer
based on the consumer's existing relationship with the
business.
(3) The obligations imposed on a business or service
provider under this section shall not do any of the
following:
(i) Apply when compliance by the business or service
provider would violate an evidentiary privilege provided
under the laws of this Commonwealth.
(ii) Prevent a business or service provider from
providing personal information concerning a consumer to
an individual covered by an evidentiary privilege
provided under the laws of this Commonwealth as part of a
privileged communication.
(iii) Adversely affect the rights of an individual
20210HB2202PN2507 - 14 -
1
2
3
4
5
6
7
8
9
10
11
12
13
14
15
16
17
18
19
20
21
22
23
24
25
26
27
28
29
30
provided under the United States Constitution or the
Constitution of Pennsylvania.
(iv) Apply to the processing of personal information
by an individual in the course of only a personal or
household activity.
(4) If a business or service provider processes personal
information in accordance with this subsection, the business
or service provider shall have the burden of demonstrating
that the processing meets the requirements under this
subsection.
(5) Personal information that is processed by a business
or service provider under this act may not be processed for
any purpose other than a purpose authorized under this
subsection.
(6) Personal information that is processed by a business
or service provider under this act may be processed only to
the extent that the processing:
(i) is necessary, reasonable and proportionate for a
purpose authorized under this subsection;
(ii) is adequate, relevant and limited to a purpose
authorized under this subsection; and
(iii) to the extent possible, adheres to reasonable
administrative, technical and physical measures to
protect the confidentiality, integrity and accessibility
of the personal information and to reduce reasonably
foreseeable risks of harm to the consumer.
(l) Duties of businesses and service providers.--
(1) A business or service provider shall meet the
obligations imposed under this act.
(2) A service provider shall adhere to the instructions
20210HB2202PN2507 - 15 -
1
2
3
4
5
6
7
8
9
10
11
12
13
14
15
16
17
18
19
20
21
22
23
24
25
26
27
28
29
30
of a business and assist the business to meet the business's
obligations under this act. Based on the nature of the
processing and the information available to the service
provider, the service provider shall assist the business by
engaging in all of the following:
(i) To the extent possible, taking appropriate
technical and organizational measures to satisfy the
business's obligation to respond to a consumer request to
exercise the consumer's rights under subsection (a).
(ii) Assisting the business in meeting the
business's obligations regarding the security of
processing personal information and notice of a breach of
the security of the system in accordance with the act of
December 22, 2005 (P.L.474, No.94), known as the Breach
of Personal Information Notification Act.
(3) Notwithstanding the instructions of a business, a
service provider shall have the following duties:
(i) Ensuring each person processing personal
information is subject to a duty of confidentiality with
respect to the information.
(ii) Engaging a subcontractor, after providing the
business with an opportunity to object in accordance with
a written contract under paragraph (5), that requires the
subcontractor to meet the obligations of the service
provider regarding the personal information.
(4) Based on the context of the processing, a business
and a service provider shall implement appropriate technical
and organizational measures to ensure a level of security
appropriate to the risk and clearly allocate the duties to
implement the measures.
20210HB2202PN2507 - 16 -
1
2
3
4
5
6
7
8
9
10
11
12
13
14
15
16
17
18
19
20
21
22
23
24
25
26
27
28
29
30
(5) The processing by a service provider shall be
governed by a binding written contract between the business
and the service provider that provides for all of the
following provisions:
(i) The processing instructions for the service,
including the nature and purpose of the processing.
(ii) The type of personal information subject to the
processing and the duration of the processing.
(iii) The requirements imposed under this paragraph
and paragraphs (3) and (4).
(iv) At the request of the business, the service
provider shall delete or return the personal information
to the business at the end of the provision of services,
unless retention of the personal information is required
by the laws of this Commonwealth.
(v) The service provider shall make available to the
business all information necessary to demonstrate
compliance with the obligations under this act.
(vi) Except as provided under subparagraph (vii),
the service provider shall allow for and contribute to
reasonable audits and inspections by the business or the
business's designated auditor.
(vii) In lieu of complying with subparagraph (vi),
the service provider may, with the business's consent,
arrange for a qualified and independent auditor to
conduct, at least annually and at the service provider's
expense, an audit of the service provider's policies and
technical and organizational measures in support of the
obligations under this act. An auditor shall use an
appropriate and accepted control standard or framework
20210HB2202PN2507 - 17 -
1
2
3
4
5
6
7
8
9
10
11
12
13
14
15
16
17
18
19
20
21
22
23
24
25
26
27
28
29
30
and audit procedure for an audit under this subparagraph.
Upon request by the business, the service provider shall
provide a report of an audit under this subparagraph to
the business.
(viii) The contract may not relieve the business or
service provider from the liabilities imposed on the
business or service provider regarding processing under
this act.
(ix) The determination whether a person is acting as
the business or service provider regarding processing is
a fact-based determination that depends on the context in
which personal information is processed. A person who is
not limited in the processing of personal information, in
accordance with the business's instructions, or who fails
to adhere to the business's instructions, shall be
considered a business regarding the processing of the
personal information. A service provider that continues
to adhere to the business's instructions regarding the
processing of personal information shall remain the
service provider. If a service provider determines, by
itself or collaboration with another person, the purpose
and means of the processing of personal information, the
service provider shall be considered a business regarding
the processing.
(6) A business or service provider that discloses
personal information to another business or service provider
in compliance with this act shall not be in violation of this
act if all of the following apply:
(i) The recipient processes the personal information
in violation of this act.
20210HB2202PN2507 - 18 -
1
2
3
4
5
6
7
8
9
10
11
12
13
14
15
16
17
18
19
20
21
22
23
24
25
26
27
28
29
30
(ii) At the time of disclosing the personal
information, the business or service provider did not
have actual knowledge that the recipient intended to
commit a violation of this act.
(7) A business or service provider that receives
personal information from another business or service
provider in compliance with this act as specified under
paragraph (6) shall not be in violation of this act if
another business or service provider fails to comply with
applicable obligations under this act.
(m) Violation.--A business shall be in violation of this
section if the business fails to cure an alleged violation
within 60 days after being notified of alleged noncompliance. A
business that fails to cure an alleged violation within 60 days
after being notified of alleged noncompliance shall be in
violation of the act of December 17, 1968 (P.L.1224, No.387),
known as the Unfair Trade Practices and Consumer Protection Law.
A business, service provider or any other person that violates
this section shall be subject to an injunction and liable for a
civil penalty of not more than $2,500 for each unintentional
violation and not more than $7,500 for each intentional
violation. Nothing in this act shall be construed to create or
imply a private cause of action.
(n) Rules and regulations.--The Attorney General shall
promulgate rules and regulations to implement this section and
may provide publicly available opinions for the purpose of
promoting the effective compliance with this act.
Section 4. Effective date.
This act shall take effect in one year.
20210HB2202PN2507 - 19 -
1
2
3
4
5
6
7
8
9
10
11
12
13
14
15
16
17
18
19
20
21
22
23
24
25
26
27
28
29