HOUSE . . . . . . . . No. 4514
The Commonwealth of Massachusetts
________________________________________
HOUSE OF REPRESENTATIVES, March 3, 2022.
The committee on Advanced Information Technology, the Internet and
Cybersecurity to whom was referred the petition (accompanied by bill,
House, No. 142) of Andres X. Vargas, David M. Rogers and others
relative to consumer data privacy, reports recommending that the
accompanying bill (House, No. 4514) ought to pass.
For the committee,
LINDA DEAN CAMPBELL.
1 of 65
 FILED ON: 2/1/2022
HOUSE . . . . . . . . . . . . . . . No. 4514
The Commonwealth of Massachusetts
_______________
In the One Hundred and Ninety-Second General Court
(2021-2022)
_______________
An Act establishing the Massachusetts Information Privacy and Security Act..
Be it enacted by the Senate and House of Representatives in General Court assembled, and by the authority
of the same, as follows:
1 SECTION 1. The General Laws are hereby amended by inserting after Chapter 93L the
2 following chapter:
3 CHAPTER 93M. The Massachusetts Information Privacy and Security Act.
4 SECTION 1. Title
5 This chapter shall be known as the “Massachusetts Information Privacy and Security
6 Act.”
7 SECTION 2. Definitions
8 As used in this chapter, the following words shall have the following meanings unless the
9 context clearly requires otherwise:
10 “Advertising” means a communication in any medium by a controller or an entity acting
11 on the controller’s behalf intended to induce an individual to obtain goods, services, or
12 employment.
2 of 65
13 “Affiliate” means an entity that controls, is controlled by, or is under common control or
14 shares common branding with another entity. For the purposes of this definition, “control” or
15 “controlled” shall mean: (1) ownership of more than fifty per cent of the outstanding shares of
16 any class of voting security of the entity; (2) control in any manner over the election of a
17 majority of the entity’s directors or of persons exercising similar functions; or (3) the power to
18 otherwise exercise a controlling influence over the management of the entity.
19 “Authorized agent” means an entity or natural person that an individual has designated
20 pursuant to subsection (d) of section 15 of this chapter.
21 “Biometric information” means a retina or iris scan, fingerprint, voiceprint, map or scan
22 of hand or face geometry, vein patterns, gait patterns, or other measurements of unique
23 biological patterns or characteristics used to identify a specific individual; provided, however,
24 that “biometric information” shall not include: (i) writing samples; (ii) written signatures; (iii)
25 photographs; (iv) video or audio recordings or data generated therefrom; (v) human biological
26 samples used for valid scientific testing or screening; (vi) demographic data; (vii) tattoo
27 descriptions; (viii) physical descriptions such as height, weight, hair color or eye color; (ix)
28 donated organs, tissues, or parts as defined in chapter 113A of the General Laws; (x) blood or
29 serum stored on behalf of recipients or potential recipients of living or cadaveric transplants and
30 obtained or stored by a federally designated organ procurement agency; (xi) biological materials
31 regulated under section 70G of chapter 111 of the General Laws; (xii) information captured from
32 a patient in a health care setting; (xiii) information collected, used, or stored for health care
33 treatment, payment or operations under HIPAA; or (xiv) an X-ray, roentgen process, computed
34 tomography, MRI, PET scan, mammography, or other image or film of the human anatomy used
3 of 65
35 to diagnose, prognose, or treat an illness or other medical condition or to further validate
36 scientific testing or screening.
37 “Business associate” shall have the same meaning as in 45 C.F.R. 160.103.
38 “Child” means an individual who a controller knows or reasonably should know is under
39 the age of 13.
40 “Collects,” “collected,” or “collection” means buying, renting, gathering, obtaining,
41 receiving, or otherwise accessing any personal information pertaining to an individual by any
42 means. This includes, but is not limited to, obtaining information from the individual, either
43 actively or passively, or by observing the individual’s behavior.
44 “Common branding” means a shared name, servicemark, trademark, or other indicator
45 that an individual would reasonably understand to indicate that two or more entities are
46 commonly owned.
47 “Consent” means a clear affirmative act signifying an individual’s freely given, specific,
48 informed, and unambiguous agreement to allow the processing of personal information relating
49 to the individual for a narrowly defined particular purpose. Consent may include a written
50 statement, including a statement written by electronic means, or any other unambiguous
51 affirmative action. The following shall not constitute consent: (1) acceptance of a general or
52 broad terms of use or similar document that contains descriptions of personal information
53 processing along with other, unrelated information; (2) hovering over, muting, pausing, or
54 closing a given piece of content; or (3) agreement obtained through dark patterns.
4 of 65
55 “Controller” means the entity that, alone or jointly with others, determines the purposes
56 and means of the processing of personal information of an individual.
57 “Covered entity” shall have the same meaning as in 45 C.F.R. 160.103.
58 “Dark pattern” means a user interface designed or manipulated with the substantial effect
59 of subverting or impairing user autonomy, decision-making, or choice.
60 “Data broker” means a controller that knowingly collects and sells to third parties:
61 (1) The sensitive information of not less than 10,000 individuals; or
62 (2) The personal information of not less than 10,000 individuals with whom the controller
63 does not have a direct relationship, including, but not limited to, a relationship in which an
64 individual is a past or present: (i) customer, client, subscriber, user, or registered user of the
65 controller’s goods or services; (ii) an employee, contractor, or agent of the controller; (iii) an
66 investor in the controller; or (iv) a donor to the controller.
67 The following activities conducted by a controller, and the collection and sale of personal
68 information incidental to conducting these activities, shall not qualify the controller as a data
69 broker: (i) providing 411 directory assistance or directory information services, including name,
70 address, and telephone number, on behalf of or as a function of a telecommunications carrier; (ii)
71 providing publicly available information related to an individual’s business or profession; or (iii)
72 providing publicly available information via real-time or near-real-time alert services for health
73 or safety purposes.
74 “De-identified information” means information, derived from personal information, that
75 cannot reasonably be used to infer information about, or otherwise be linked to, an identified or
5 of 65
76 identifiable individual or household, or a device linked to such individual or household. De77 identification means the creation of de-identified information from personal information.
78 “Designated methods for submitting a request” means a mailing address, email address,
79 Internet web page, Internet web portal, toll-free telephone number, or other applicable contact
80 information, whereby an individual may submit a request or direction under this chapter,
81 provided that: (1) the designated methods shall be reasonably accessible to individuals and take
82 into account the ways in which individuals interact with the controller, the need for secure and
83 reliable communication of the request, and the ability of the controller to determine that the
84 request is a verifiable request; and (2) a controller shall not require an individual to create a new
85 account in order to exercise a right under this chapter, but a controller may require an individual
86 to use an existing account to exercise the individual’s rights under this chapter.
87 “Device” means any physical object that is capable of connecting to the Internet, directly
88 or indirectly, or to another device.
89 “Entity” means a sole proprietorship, or a corporation, association, partnership or other
90 legal entity.
91 “Health care facility” shall have the same meaning as defined in section 25B of chapter
92 111 of the General Laws.
93 “Health care provider” shall have the same meaning as defined in section 1 of chapter
94 111 of the General Laws.
95 “Health record” means an individual’s health-related record, as kept pursuant to section
96 70 of chapter 111 of the General Laws.
6 of 65
97 “HIPAA” means the federal Health Insurance Portability and Accountability Act of 1996,
98 42 U.S.C. 1320d et seq., as amended from time to time.
99 “Homepage” means the introductory page of an Internet website and any Internet web
100 page where personal information is collected; provided, however, that in the case of an online
101 service, such as a mobile application, homepage shall mean: (i) the application’s platform page
102 or download page; (ii) a link within the application, such as from the application configuration,
103 “About,” “Information,” or settings page; or (iii) any other location that allows individuals to
104 review the notices required by this chapter, including, but not limited to, before downloading the
105 application.
106 “Identified or identifiable individual household” is a group of individuals who: (i)
107 cohabitate with one another at the same residential address in the Commonwealth; (ii) use
108 common devices or services; and (iii) can be readily identified, directly or indirectly.
109 “Identified or identifiable individual” means an individual who can be readily identified,
110 directly or indirectly.
111 “Individual” means a natural person who is a resident of the Commonwealth; provided,
112 however, that “individual” shall not include a natural person acting as a sole proprietorship.
113 “Infer” or “inference” means the derivation of information, data, assumptions or
114 conclusions from facts, evidence, or another source of information or data.
115 “Institution of higher education” means any college, junior college, university or other
116 public or private educational institution that has been authorized to grant degrees pursuant to
117 sections 30, 30A, and 31A of chapter 69 of the General Laws.
7 of 65
118 “Intentionally interacts” means when an individual intends to interact with an entity, or
119 disclose personal information to an entity, via one or more deliberate interactions, including
120 visiting the entity’s website or purchasing a good or service from the entity; provided, however,
121 that hovering over, muting, pausing, or closing a given piece of content does not constitute an
122 individual’s intent to interact with an entity.
123 “Minor” means an individual who a controller knows or reasonably should know is not
124 less than 13 years of age and not more than 16 years of age.
125 “Nonpersonalized advertising” means advertising that is based solely on an individual’s
126 personal information, except for the individual’s specific geolocation information, derived from
127 the individual’s current interaction with the controller.
128 “Nonprofit organization” means any organization that is exempt from taxation under 26
129 U.S.C. 501(c), as amended from time to time.
130 “Personal information” means information that identifies, relates to, describes, is
131 reasonably capable of being associated with, or could reasonably be linked, directly or indirectly,
132 with an identified or identifiable individual; provided, however, that personal information shall
133 not include de-identified information or publicly available information.
134 For the following purposes, personal information shall also include information that
135 identifies, relates to, describes, is reasonably capable of being associated with, or could
136 reasonably be linked, directly or indirectly, with an identified or identifiable household:
137 (1) As “personal information” is used in the definition of “sale,” “sell” or “sold” in this
138 section;
8 of 65
139 (2) In any other reference to the sale of personal information in this chapter; or
140 (3) As “personal information” is used in subsection (b) of section 3 of this chapter.
141 “Process” or “processing” means any operation or set of operations which are performed
142 on personal information or on sets of personal information, whether or not by automated means,
143 such as the collection, use, storage, disclosure, analysis, prediction, deletion, or modification of
144 personal information. Process or processing includes the actions of a controller directing a
145 processor to process personal information.
146 “Processor” means an entity that processes personal information on behalf of a controller.
147 “Protected health information” shall have the same meaning as defined in 45 C.F.R.
148 160.103, established pursuant to HIPAA.
149 “Publicly available information” means information about an individual that is: (1)
150 lawfully made available from federal, state, or local government records; or (2) information that
151 a controller has a reasonable basis to believe is lawfully and intentionally made available by the
152 individual to the general public through widely distributed media.
153 “Research” means a systematic investigation, including research development, testing,
154 and evaluation, designed to develop or contribute to generalizable knowledge and that is
155 conducted in accordance with other applicable ethics and privacy laws.
156 “Sale, “sell,” or “sold” means sharing, renting, releasing, disclosing, disseminating,
157 making available, transferring, or otherwise communicating orally, in writing, or by electronic or
158 other means, an individual’s personal information by the controller to a third party for monetary
9 of 65
159 or other valuable consideration in a bargained-for exchange, or for the purposes of targeted
160 advertising. “Sale,” “sell,” or “sold” does not include the following:
161 (1) The disclosure of personal information to a processor where the processor only
162 processes such personal information on behalf of the controller;
163 (2) The controller’s use or sharing of an identifier for an individual who has opted out of
164 the sale of the individual’s personal information or limited the use of the individual’s sensitive
165 information for the purposes of alerting entities that the individual has opted out of the sale of the
166 individual’s personal information or limited the use of the individual’s sensitive information;
167 (3) The disclosure or transfer of personal information to an affiliate of the controller;
168 (4) The disclosure or transfer of personal information to a third party as an asset that is
169 part of a proposed or actual merger, acquisition, bankruptcy, or other transaction in which the
170 third party assumes control of all or part of the controller’s assets;
171 (5) The disclosure of personal information to a third party for purposes of providing a
172 product or service specifically requested by the individual; or
173 (6) When the individual uses or expressly directs the controller to disclose personal
174 information to a third party or otherwise interact with a third party, not including disclosures or
175 interactions for the purposes of targeted advertising; provided, however, that the individual’s
176 direction was not obtained through dark patterns.
177 “Security and integrity” means the ability of:
10 of 65
178 (1) Networks or information systems to detect security incidents that compromise the
179 availability, authenticity, integrity, and confidentiality of stored or transmitted personal
180 information;
181 (2) Controllers to detect security incidents, resist malicious, deceptive, fraudulent or
182 illegal actions and to help prosecute those responsible for those actions; or
183 (3) Controllers to ensure the physical safety of natural persons.
184 “Sensitive information” means:
185 (1) Personal information that reveals an individual’s: (i) racial or ethnic origin, (ii)
186 religious beliefs; or (iii) citizenship or immigration status;
187 (2) Biometric information or genetic information processed for the purpose of uniquely
188 identifying an individual;
189 (3) Personal information processed concerning an individual’s mental or physical health
190 diagnosis or treatment;
191 (4) Personal information processed concerning an individual’s sex life or sexual
192 orientation;
193 (5) An individual’s specific geolocation information;
194 (6) The personal information from a child;
195 (7) Personal information that reveals an individual’s philosophical beliefs or union
196 membership; or
11 of 65
197 (8) Personal information that reveals: (i) an individual’s social security number, driver’s
198 license number, military identification number, passport number, or state-issued identification
199 card number; or (ii) financial account number, or credit or debit card number, with or without
200 any required security code, access code, personal identification number or password, that would
201 permit access to an individual’s financial account.
202 Sensitive information is a form of personal information. Sensitive information that is
203 “publicly available information” shall not be considered sensitive information or personal
204 information.
205 “Specific geolocation information” means information derived from technology
206 including, but not limited to, global positioning system level latitude and longitude coordinates
207 or other mechanisms that directly identify the specific location of an individual within a
208 geographic area that is equal to or less than the area of a circle with a radius of 1,850 feet.
209 Specific geolocation information excludes the content of communications or any information
210 generated by or connected to advanced utility metering infrastructure systems or equipment for
211 use by a utility.
212 “Targeted advertising” means the targeting of advertising to an individual based on the
213 individual’s personal information obtained from the individual’s activity across controllers,
214 distinctly-branded websites, applications, or services, other than the controller, distinctly215 branded website, application, or service with which the individual intentionally interacts.
216 Targeted advertising shall not include:
217 (1) Advertising to an individual in response to the individual’s request for information
218 and feedback;
12 of 65
219 (2) Advertising based on the context of an individual’s current search query, visit to a
220 website, or online application; or
221 (3) Processing personal information solely for measuring or reporting advertising
222 performance, reach, or frequency.
223 “Third party” means a natural person, entity, public authority, agency, or body other than
224 the applicable individual, controller, processor, or affiliate of the controller or the processor.
225 “Verifiable request” means a request: (i) to exercise any of the rights set forth in sections
226 8 through 11 of this chapter; and (ii) that a controller can use commercially reasonable means to
227 determine is being made by the individual or by a person authorized to exercise rights on behalf
228 of such individual with respect to the personal information at issue, pursuant to subsections (b)
229 and (c) of section 15 of this chapter.
230 SECTION 3. Scope and Applicability
231 (a) This chapter shall apply to:
232 (1) A controller or processor that conducts business in the Commonwealth; and
233 (2) The processing of personal information by a controller or processor not physically
234 established in the Commonwealth, where the processing activities are related to: (i) the offering
235 of goods or services that are targeted to individuals; or (ii) the monitoring of behavior of
236 individuals where such behavior takes place in the Commonwealth.
237 (b) Notwithstanding subsection (a) of this section, sections 7 through 17 and section 20 of
238 this chapter shall only apply to a controller that satisfies at least 1 of the following additional
239 thresholds or is an entity that is an affiliate of and shares common branding with such a
13 of 65
240 controller, in which case sections 7 through 17 and section 20 shall apply only to the personal
241 information processed by the affiliate on behalf of the controller:
242 (1) The controller, as of January 1 of the calendar year, had annual global gross revenues
243 in excess of 25,000,000 dollars in the preceding calendar year;
244 (2) The controller determines the purposes and means of processing of the personal
245 information of not less than 100,000 individuals; or
246 (3) The controller is a data broker.
247 (c) Payment-only credit, check, or cash transactions where no information is retained
248 about an individual entering into the transaction do not count as “individuals” for the purposes of
249 subsection (b).
250 (d) The provisions of this chapter are not limited to personal information collected
251 electronically or over the Internet, but apply to the processing of all personal information
252 processed by a controller.
253 (e) This chapter shall not apply to:
254 (1) Any agency, executive office, department, board, commission, bureau, division or
255 authority of the Commonwealth, or any of its branches, or any political subdivision thereof.
256 (2) Any national securities association that is registered under 15 U.S.C. 78o-3 of the
257 Securities Exchange Act of 1934, as amended from time to time.
258 (3) Any registered futures association that is so designated pursuant to 7 U.S.C. 21, as
259 amended from time to time.
14 of 65
260 (f) The following information shall be exempt from the provisions of this chapter:
261 (1) Protected health information that is processed by a covered entity or business
262 associate pursuant to 45 C.F.R. 160, 162, and 164.
263 (2) Health records for the purposes of section 70 of chapter 111 of the General Laws, to
264 the extent that the records are maintained pursuant to 45 C.F.R. 160, 162, and 164.
265 (3) Information and documents that are created by a covered entity for purposes of
266 complying with HIPAA and its implementing regulations.
267 (4) Information used only for public health activities and purposes as authorized by
268 HIPAA.
269 (5) Patient identifying information for purposes of 42 C.F.R. 2, established pursuant to 42
270 U.S.C. 290dd-2, as amended from time to time.
271 (6) Information that is: (i) collected for a clinical trial subject to the Federal Policy for the
272 Protection of Human Subjects (also known as the Common Rule) under 45 C.F.R. 46; (ii)
273 collected pursuant to good clinical practice guidelines issued by the International Council for
274 Harmonisation of Technical Requirements for Pharmaceuticals for Human Use; (iii) collected
275 pursuant to the human subject protection requirements under 21 C.F.R. 50 and 56; or (iv)
276 personal information used or disclosed in research conducted in accordance with one or more of
277 the requirements set forth in this paragraph.
278 (7) Information and documents created for purposes of the federal Health Care Quality
279 Improvement Act of 1986, 42 U.S.C. 11101 et seq., as amended from time to time.
15 of 65
280 (8) Patient safety work product for purposes of the federal Patient Safety and Quality
281 Improvement Act, 42 U.S.C. 299b-21 et seq., as amended from time to time.
282 (9) Information that is: (i) derived from any of the health care-related information listed
283 in this subsection; and (ii) de-identified in accordance with the requirements for de-identification
284 pursuant to 45 C.F.R. 164.
285 (10) Information that is treated in the same manner as, or that originates from and is
286 intermingled to be indistinguishable with, information exempt under this subsection that is
287 maintained by: (i) a covered entity or business associate; (ii) a health care facility or health care
288 provider; or (iii) a program of a qualified service organization as defined by 42 U.S.C. 290dd-2.
289 (11) (i) An activity involving the processing of any personal information bearing on an
290 individual’s credit worthiness, credit standing, credit capacity, character, general reputation,
291 personal characteristics, or mode of living by: (A) a consumer reporting agency, as defined in 15
292 U.S.C. 1681a(f); (B) a furnisher of information, as set forth in 15 U.S.C. 1681s-2, that provides
293 information for use in a consumer report, as defined in 15 U.S.C. 1681a(d); and (C) a user of a
294 consumer report, as set forth in 15 U.S.C. 1681b.
295 (ii) Clause (i) of this paragraph shall apply only to the extent that: (A) the activity is
296 regulated by the federal Fair Credit Reporting Act, 15 U.S.C. 1681 et seq., as amended from time
297 to time; and (B) the personal information is processed solely as authorized by the federal Fair
298 Credit Reporting Act.
299 (12) Personal information processed in compliance with the federal Driver’s Privacy
300 Protection Act of 1994, 18 U.S.C. 2721 et seq. as amended from time to time, and implementing
301 regulations.
16 of 65
302 (13) Personal information regulated by the federal Family Educational Rights and
303 Privacy Act, 20 U.S.C. 1232g et seq. as amended from time to time, and its implementing
304 regulations.
305 (14) Personal information processed in compliance with the federal Farm Credit Act, 12
306 U.S.C. 2001 et seq. as amended from time to time, and its implementing regulations, 12 C.F.R.
307 600 et seq.
308 (15) Personal information processed in compliance with the federal Gramm-Leach-Bliley
309 Act, 15 U.S.C. 6801 et seq. as amended from time to time, and its implementing regulations.
310 (16) Personal information processed in compliance with chapter 175I of the General
311 Laws.
312 (17) Personal information processed in relation to price, route or service, as such terms
313 are used in the Airline Deregulation Act, 49 U.S.C. 40101 et seq. as amended from time to time,
314 by an air carrier subject to said act, to the extent that this chapter is preempted by section 41713
315 of the Airline Deregulation Act.
316 (18) Personal information processed for purposes of chapter 176Q of the General Laws.
317 (g) Sections 8 through 11 and section 13 of this chapter shall not apply to information
318 that is processed: (1) in the course of an individual acting in a commercial context, to the extent
319 that the information is collected and used within that context; (2) in the course of an individual
320 acting as a job applicant to, an employee of, or an agent or independent contractor of a controller,
321 processor, or third party, to the extent that the information is collected and used within the
322 context of that role; (3) as the emergency contact information of an individual under paragraph
17 of 65
323 (2), provided that the information is used solely for emergency contact purposes; or (4) in order
324 to administer benefits for another natural person relating to the individual under paragraph (2),
325 provided that the information is used solely for the purposes of administering those benefits.
326 (h) The provisions of this chapter relating to individuals under 16 years of age shall only
327 apply to the extent not in conflict with the federal Children's Online Privacy Protection Act, 15
328 U.S.C. 6501 et seq., and its implementing regulations. Controllers and processors that comply
329 with the Children's Online Privacy Protection Act and its implementing regulations shall be in
330 compliance with any obligation to obtain parental consent under this chapter.
331 (i) This chapter shall also apply in full to an entity that voluntarily certifies to the
332 Attorney General that it is in compliance with, and agrees to be bound by, this chapter; provided,
333 however, that the entity processes the personal information of one or more individuals but does
334 not meet the applicability criteria set forth in subsection (b) of this section.
335 SECTION 4. Conflicting Provisions
336 Wherever possible, law relating to individuals’ personal information should be construed
337 to harmonize with the provisions of this chapter, but in the event of a conflict between the
338 provisions of other laws and the provisions of this chapter, the provisions that afford the greatest
339 protection for the right of privacy for individuals shall control.
340 SECTION 5. General Principles for Processing Personal Information
341 (a) Personal information shall be:
342 (1) Processed lawfully, fairly, and in a transparent manner in relation to the individual
343 and in compliance with this chapter;
18 of 65
344 (2) Collected for specified, explicit and legitimate purposes and not further processed in a
345 manner that is incompatible with those purposes;
346 (3) Processed in a manner that is adequate, relevant and limited to what is necessary in
347 relation to the purposes for which it is processed;
348 (4) Maintained in a manner such that the information is accurate and, where necessary,
349 kept up to date;
350 (5) Maintained in a form which permits identification of individuals for no longer than is
351 necessary for the purposes for which the personal information is processed; and
352 (6) Processed in a manner that ensures that the information remains appropriately secure.
353 (b) A controller shall be responsible for, and capable of demonstrating compliance with,
354 the above subsection (a), including by implementing procedures to comply with the subsection
355 that are reasonable and appropriate taking into consideration:
356 (1) The size, scope, and type of the controller;
357 (2) The amount of resources available to the controller;
358 (3) The amount and nature of personal information processed by the controller, including,
359 but not limited to, whether the personal information is sensitive information; and
360 (4) The need for the security and confidentiality of the personal information processed by
361 the controller.
362 (c) A controller that is compliant with the regulations promulgated pursuant to chapter
363 93H of the General Laws with respect to “personal information,” as that term is defined in
19 of 65
364 section 1 of said chapter 93H, shall be in compliance with the principle set forth in paragraph (6)
365 of subsection (a) of this section with respect to such personal information.
366 SECTION 6. Lawful Bases For Processing Personal Information
367 (a) Processing shall be lawful and in compliance with this chapter only if and to the
368 extent that at least 1 of the following applies:
369 (1) The individual has given consent to the processing of their personal information for
370 one or more specific purposes;
371 (2) Processing is necessary for the performance of a contract to which the individual is
372 party or in order to take steps at the request of the individual prior to entering into a contract;
373 (3) Processing is necessary for compliance with a legal obligation to which the controller
374 is subject;
375 (4) Processing is necessary in order to protect the vital interests of the individual or of
376 another natural person; provided, however, that the processing cannot be manifestly based on
377 another legal basis and that the individual or other natural person is at risk or danger of death or
378 serious physical injury; or
379 (5) Processing is necessary for the purposes of the legitimate interests pursued by the
380 controller or by a third party, except where such interests are overridden by the individual’s
381 reasonable expectations of privacy or other legal rights.
382 (b) Processing pursuant to paragraph (5) of subsection (a) shall be consistent with the
383 reasonable expectations of the individual based on the individual’s relationship with the
384 controller, and such processing shall be conspicuously disclosed to the individual in advance;
20 of 65
385 provided, however, that the controller shall also assess the following factors to determine
386 whether there is a legitimate interest for the processing:
387 (1) The possible consequences and cognizable harms for the individual whose personal
388 information would be processed;
389 (2) The amount and nature of personal information that would be processed;
390 (3) The need for the security and confidentiality of the personal information that would
391 be processed;
392 (4) The context in which the personal information would be collected; and
393 (5) Whether the processing is necessary and proportionate in relation to the purposes, or
394 whether the controller or third party can achieve their legitimate interests in another less intrusive
395 way.
396 (c) A controller shall not rely on paragraph (5) of subsection (a) as a lawful basis for
397 processing sensitive information unless the controller meets a heightened standard of proof,
398 under which a controller shall conduct a documented risk assessment in accordance with section
399 21 of this chapter that shows that the legitimate interests pursued by the controller or by a third
400 party substantially outweigh the individual’s reasonable expectations of privacy or other legal
401 rights. In particular, a controller shall not rely on paragraph (5) of subsection (a) to sell sensitive
402 information that meets any of the subcategories set forth in paragraphs (1) through (5) in the
403 definition of sensitive information in section 2 of this chapter.
404 (d) A controller shall not sell the personal information of a child unless the controller has
405 obtained the consent of the parent or guardian of the child.
21 of 65
406 (e) A controller shall not sell the personal information of a minor unless the controller has
407 obtained the minor’s consent.
408 (f) If a minor does not consent to the sale of the minor’s personal information, a
409 controller shall: (1) wait for not less than 12 months before making a subsequent request for the
410 minor’s consent to sell the minor’s personal information; or (2) wait until the individual attains
411 16 years of age, whichever occurs sooner.
412 SECTION 7. Right to Privacy Notice
413 (a) At or before the point of the collection of an individual’s personal information,
414 controllers shall provide the individual with a reasonably accessible, clear, and meaningful
415 privacy notice that shall include:
416 (1) A clear and conspicuous description of: (i) whether the controller sells personal
417 information to third parties or processes personal information for the purposes of targeted
418 advertising; (ii) what categories of sensitive information, if any, the controller processes and for
419 what purposes; (iii) an individual’s rights pursuant to sections 8 through 13 of this chapter; (iv)
420 how and where individuals may request to exercise these rights, pursuant to section 16 of this
421 chapter; and (v) a link to the Attorney General’s online mechanism through which the individual
422 may contact the Attorney General to submit a complaint, pursuant to section 25 of this chapter.
423 (2) The categories of personal information processed by the controller;
424 (3) The controller’s purposes for processing the personal information;
425 (4) The categories of personal information that the controller sells to third parties,
426 specifying the categories of sensitive information that the controller sells to third parties, if any; 
22 of 65
427 (5) The categories of third parties, if any, to whom the controller sells personal
428 information;
429 (6) A contact method, such as an email address, that the individual may use to contact the
430 controller; and
431 (7) The length of time the controller intends to retain each category of personal
432 information, or if that is not possible, the criteria used to determine such period, provided that a
433 controller shall retain personal information for a duration consistent with paragraph (5) of
434 subsection (a) of section 5 of this chapter.
435 (b) A controller shall not collect additional categories of personal information or process
436 personal information collected for additional purposes that are incompatible with the disclosed
437 purposes for which the personal information was collected, without providing the individual with
438 notice consistent with subsection (a) of this section.
439 (c) An entity that, acting as a third party, controls the collection of an individual’s
440 personal information may satisfy its obligation under this section by providing the required
441 information prominently and conspicuously on the homepage of its Internet website; provided,
442 however, that if an entity, acting as a third party, controls the collection of personal information
443 about an individual on its premises, including in a vehicle, then the entity shall, at or before the
444 point of collection, satisfy its obligation under subsection (a) of this section by providing the
445 required information in a clear and conspicuous manner at such location.
446 (d) Nothing in this section shall require a controller to provide the information required in
447 a manner that would disclose the controller’s trade secrets.
23 of 65
448 (e) The categories of sensitive information required to be disclosed by a controller
449 pursuant to this section shall specifically include each applicable subcategory set forth in
450 paragraphs (1) through (8) in the definition of sensitive information in section 2 of this chapter.
451 SECTION 8. The Right to Know and Access Personal Information
452 An individual shall have the right to request that a controller that collects personal
453 information about the individual disclose to the individual:
454 (1) The specific pieces of personal information that the controller has collected about the
455 individual; and
456 (2) The categories of sources from which the personal information has been collected.
457 SECTION 9. Right to Data Portability
458 (a) In response to a verifiable request pursuant to section 8 of this chapter, a controller
459 shall disclose to the individual the information requested in the following manner:
460 (1) The controller shall provide to the individual the specific pieces of personal
461 information that the controller has collected about the individual in a portable format that is
462 easily understandable to the average individual and, to the extent technically feasible, in a readily
463 usable format that allows the individual to transmit the information to another controller without
464 hindrance. For the purposes of this subsection, “specific pieces of information” do not include
465 any data generated to help ensure security and integrity.
466 (2) The controller shall also disclose the information specified in paragraph (2) of section
467 8 of this chapter, if so requested by the individual. 
24 of 65
468 (3) The disclosure of the required information pursuant to paragraphs (1) and (2) of this
469 subsection shall cover the 12 month period preceding the controller’s receipt of the verifiable
470 request; provided, however, that an individual may request that the controller disclose the
471 required information beyond the 12 month period and the controller shall be required to provide
472 such information unless doing so proves impossible or would constitute an undue burden for the
473 controller. An individual’s ability to request information beyond the 12 month period shall be
474 clearly disclosed in a controller’s privacy notice pursuant to clause (iii) of paragraph (1) of
475 subsection (a) of section 7 of this chapter.
476 (b) Nothing in this section shall require a controller to provide the information requested
477 in a manner that would disclose the controller’s trade secrets.
478 SECTION 10. Right to Delete Personal Information
479 (a) An individual shall have the right to request that a controller delete any personal
480 information provided by or obtained about the individual.
481 (b) A controller that receives a verifiable request to delete the individual’s personal
482 information shall, pursuant to section 17 of this chapter, delete the individual’s personal
483 information from its records, notify any processors to delete the individual’s personal
484 information from their records, and notify all third parties to whom the controller has sold the
485 personal information to delete the individual’s personal information unless doing so proves
486 impossible or would constitute an undue burden for the controller.
487 (c) The controller may maintain a confidential record of deletion requests solely for the
488 purpose of preventing the personal information of an individual who has submitted a deletion
25 of 65
489 request from being sold, for compliance with laws, or for other purposes solely to the extent
490 permissible under this chapter.
491 (d) A controller, or a processor acting pursuant to its contract with the controller, shall
492 not be required to comply with an individual’s request to delete the individual’s personal
493 information if it is reasonably necessary for the controller or processor to maintain the
494 individual’s personal information in order to:
495 (1) Complete the transaction for which the personal information was collected, provide a
496 good or service requested by the individual or reasonably anticipated by the individual within the
497 context of a controller’s ongoing relationship with the individual, or otherwise perform a contract
498 between the controller and the individual;
499 (2) Enable solely internal uses that are reasonably aligned with the expectations of the
500 individual based on the individual’s relationship with the controller and compatible with the
501 context in which the individual provided the information; or
502 (3) Comply with a legal obligation.
503 (e) The controller or processor shall retain personal information pursuant to subsection
504 (d) solely for the applicable purposes under that subsection.
505 SECTION 11. Right to Correct Personal Information
506 (a) An individual shall have the right to request that a controller correct inaccurate
507 personal information concerning the individual, taking into account the nature of the personal
508 information and the purposes of the processing of the personal information.
26 of 65
509 (b) A controller that receives a verifiable request to correct inaccurate personal
510 information shall correct the inaccurate personal information as directed by the individual,
511 pursuant to section 17 of this chapter.
512 SECTION 12. Right to Opt Out of the Sale of Personal Information
513 (a) An individual shall have the right to opt out of the processing of the individual’s
514 personal information for the purposes of the sale of such personal information. This shall also be
515 known as the right to opt out of the sale of personal information.
516 (b) A controller shall comply with a request to exercise the right to opt out of the sale of
517 personal information as soon as reasonably possible, but not later than 30 days after receipt of
518 the request. A controller that has received direction from an individual not to sell the individual’s
519 personal information shall be prohibited from selling the individual’s personal information
520 unless the individual subsequently provides consent for the sale of the individual’s personal
521 information pursuant to subsection (c).
522 (c) After complying with an individual’s request to exercise the right to opt out of the sale
523 of their personal information, a controller shall wait for not less than 12 months before
524 requesting the individual’s consent to sell the individual’s personal information.
525 (d) A data broker shall not sell an individual’s personal information unless the individual
526 has received explicit notice and is provided an opportunity to exercise the right to opt out of the
527 sale of their personal information.
528 (e) If a controller communicates to any entity authorized by the controller to collect
529 personal information that an individual has requested to exercise the right to opt out of the sale of
27 of 65
530 their personal information, that entity shall thereafter only use that individual’s personal
531 information for purposes specified by the controller, or as otherwise permitted by this chapter,
532 and shall be prohibited from:
533 (1) Selling the personal information; and
534 (2) Retaining, using, or disclosing that individual’s personal information: (i) for any
535 purpose other than for the specific purpose of performing the services offered to the controller;
536 (ii) outside of the direct relationship between the entity and the controller; or (iii) for a
537 commercial purpose other than providing the services to the controller.
538 (f) A controller that communicates an individual’s opt-out request to an entity pursuant to
539 subsection (e) shall not be liable under this chapter if the entity receiving the opt-out request
540 violates the restrictions set forth in this chapter; provided, however, that at the time of
541 communicating the opt-out request, the controller does not know or should not reasonably know
542 that the entity intends to commit such a violation.
543 SECTION 13. Right to Limit Use and Disclosure of Sensitive Information
544 (a) An individual shall have the right to direct a controller that collects sensitive
545 information about the individual to limit its use of the individual’s sensitive information to that
546 use which is necessary to perform the services or provide the goods reasonably expected by an
547 average individual who requests those goods or services or to perform the following services:
548 (1) Short-term, transient use, including, but not limited to, nonpersonalized advertising
549 shown as part of an individual’s current interaction with the controller, provided that the
550 individual’s sensitive information is not disclosed to another third party and is not used to build a
28 of 65
551 profile about the individual or otherwise alter the individual’s experience outside the current
552 interaction with the controller;
553 (2) The performance of services on behalf of the controller, including maintaining or
554 servicing accounts, providing customer service, processing or fulfilling orders and transactions,
555 verifying customer information, processing payments, providing financing, providing analytic
556 services, providing storage, or providing similar services on behalf of the controller;
557 (3) Undertaking activities to verify or maintain the quality or safety of a service or device
558 that is owned, manufactured, manufactured for, or controlled by the controller, and to improve,
559 upgrade, or enhance the service or device that is owned, manufactured, manufactured for, or
560 controlled by the controller; or
561 (4) Helping to ensure security and integrity, to the extent the use of the individual’s
562 personal information is reasonably necessary and proportionate for those purposes.
563 (b) A controller shall comply with a request to exercise the right in subsection (a) as soon
564 as reasonably possible, but not later than 30 days after receipt of the request. A controller that
565 has received direction from an individual not to use or disclose the individual’s sensitive
566 information, except as authorized under this section, shall be prohibited from using or disclosing
567 the sensitive information for any other purpose, unless the individual subsequently provides
568 consent for the use or disclosure of the individual’s sensitive information for additional purposes
569 pursuant to subsection (c).
570 (c) For an individual who exercises the right in subsection (a), a controller shall wait for
571 not less than 12 months before requesting the individual’s consent to use and disclose the
572 individual’s sensitive information for additional purposes.
29 of 65
573 SECTION 14. Non-Discrimination Against Individuals’ Good Faith Exercise of Privacy
574 Rights
575 (a) A controller shall not discriminate against an individual for exercising, in good faith,
576 any of the rights set forth in this chapter, including, but not limited to, by:
577 (1) Denying goods or services to the individual;
578 (2) Charging different prices or rates for goods or services, including through the use of
579 discounts or other benefits or imposing penalties;
580 (3) Providing a different level of quality of goods or services to the individual;
581 (4) Suggesting that the individual will receive a different price or rate for goods or
582 services or a different level of quality or goods or services; or
583 (5) Retaliating against a job applicant to, an employee of, or an agent or independent
584 contractor of the controller for exercising their rights under this chapter.
585 (b) This section shall not prohibit a controller from offering a different price, rate, level,
586 quality, or selection of goods or services to an individual, including offering goods or services
587 for no fee, if the offering is in connection with an individual’s voluntary participation in a bona
588 fide loyalty, rewards, premium features, discounts, or club card program.
589 SECTION 15. Exercising Privacy Rights
590 (a) An individual may exercise the rights set forth in sections 8 through 13 of this chapter
591 by submitting a request, at any time, to a controller specifying which rights the individual wishes
592 to exercise.
30 of 65
593 (b) With respect to the processing of personal information of a child, the parent or legal
594 guardian of the child may exercise the rights of this chapter on the child’s behalf.
595 (c) With respect to the processing of personal information concerning an individual
596 subject to guardianship, conservatorship, or other protective arrangement under article V or
597 article 5A of chapter 190B of the General Laws, the guardian or the conservator of the individual
598 may exercise the rights of this chapter on the individual’s behalf.
599 (d) An individual may also designate an authorized agent to exercise, on behalf of that
600 individual, the rights set forth in sections 12 and 13 of this chapter; provided, however, that:
601 (1) Unless the individual has provided the authorized agent with power of attorney
602 pursuant to sections 5-501 through sections 5-507 of article V of chapter 190B of the General
603 Laws, a controller receiving a request from an authorized agent to exercise these rights may
604 require the authorized agent to provide proof that the individual gave the agent permission to
605 submit the request; provided, further, that if the controller has a reasonable basis to believe that
606 the proof submitted by the agent is insufficient or invalid, the controller may also require the
607 individual to do either of the following: (i) verify the individual’s own identity directly with the
608 controller; or (ii) directly confirm with the controller that the individual provided the authorized
609 agent with permission to submit the request; and
610 (2) An authorized agent shall not use an individual’s personal information, or any
611 information collected from or about the individual, for any purposes other than to fulfill the
612 individual’s requests, for verification, or for fraud prevention and shall implement and maintain
613 reasonable security procedures and practices to protect the individual’s personal information.
614 SECTION 16. Disclosure of Methods for Exercising Privacy Rights
31 of 65
615 (a) A controller shall make available, and shall describe in a privacy notice pursuant to
616 section 7 of this chapter, not less than 2 designated methods for submitting a request to exercise
617 the rights set forth in sections 8 through 13 of this chapter. If a controller maintains an Internet
618 website, the controller shall make its Internet website available as one such designated method
619 for submitting a request to exercise the rights set forth in said sections 8 through 13.
620 (b) A controller that sells individuals’ personal information shall also provide a clear and
621 conspicuous link on the controller’s Internet homepages to an Internet web page that enables an
622 individual, or an individual’s authorized agent, to exercise their right to opt out of the sale of the
623 individual’s personal information.
624 (c) A controller that uses or discloses individuals’ sensitive information for purposes
625 other than those specified by section 13 of this chapter shall also provide a clear and conspicuous
626 link on the controller’s Internet homepages that enables an individual, or an individual’s
627 authorized agent, to limit the use or disclosure of the individual’s sensitive information to those
628 purposes authorized under said section 13.
629 (d) A controller that is subject to both subsections (b) and (c), in lieu of complying with
630 both of those subsections, may utilize a single, clearly labeled link on the controller’s Internet
631 homepages, if that link easily allows an individual, or an individual’s authorized agent, to
632 exercise their right to opt out of the sale of the individual’s personal information and to limit the
633 use or disclosure of the individual’s sensitive information.
634 (e) A controller shall:
635 (1) Ensure that all persons responsible for handling individuals’ inquiries about the
636 controller’s privacy practices or the controller’s compliance with this chapter are informed of: (i)
32 of 65
637 all requirements set forth under this chapter; and (ii) how to direct individuals to exercise their
638 rights under sections 8 through 13 of this chapter;
639 (2) Include a separate link to the applicable web pages required under subsections (b), (c),
640 or (d) of this section in any privacy notice that the controller is required to provide to individuals
641 pursuant to section 7 of this chapter;
642 (3) Use any personal information collected from the individual in connection with the
643 submission of the individual’s request to exercise any of the rights set forth in sections 8 through
644 13 of this chapter solely for the purposes of complying with the request;
645 (4) Use any personal information collected in connection with the controller’s
646 verification of the individual’s request solely for the purposes of verification and shall not further
647 disclose the personal information, retain it longer than necessary for purposes of verification, or
648 use it for unrelated purposes; and
649 (5) Not require an individual to provide additional information beyond what is necessary
650 to direct the controller to not sell the individual’s personal information pursuant to section 12 of
651 this chapter, or to limit use or disclosure of the individual’s sensitive information pursuant to
652 section 13 of this chapter.
653 SECTION 17. Responding to an Individual’s Request
654 (a) Except as otherwise provided in this chapter, a controller shall comply with a request
655 to exercise the rights set forth in sections 8 through 11 of this chapter.
656 (b) A controller shall inform the individual of any action taken on a request to exercise
657 any of the rights set forth in sections 8 through 11 of this chapter without undue delay and in any
33 of 65
658 event within 45 days of receipt of the request; provided, however, that the period may be
659 extended once by 45 additional days where reasonably necessary, taking into account the
660 complexity and number of the requests. The controller shall notify the individual of any such
661 extension within 45 days of receipt of the request, together with the reasons for the delay.
662 (c) A controller shall not be obligated to comply with a request to exercise the rights set
663 forth in sections 8 through 11 of this chapter if the request is not a verifiable request. In such a
664 case, the controller shall notify the individual that it is unable to act on the request until it
665 receives additional information reasonably necessary to verify that the request is being made by
666 the individual or by another person who is entitled to exercise such rights on behalf of the
667 individual pursuant to subsections (b) and (c) of section 15 of this chapter.
668 (d) A verifiable request to exercise the rights set forth in sections 8 through 11 of this
669 chapter shall not extend to personal information about the individual that belongs to, or the
670 controller maintains on behalf of, another natural person. A controller may rely on
671 representations made in a verifiable request as to rights with respect to personal information and
672 shall not be required to seek out other persons that may have or claim to have rights to personal
673 information or to take any action under this chapter in the event of a dispute between or among
674 persons claiming rights to personal information in the controller’s possession.
675 (e) A request to exercise any of the rights in sections 12 or 13 of this chapter shall not
676 need to be a verifiable request. If a controller, however, has a good-faith, reasonable, and
677 documented belief that the request is fraudulent, the controller may deny the request. The
678 controller shall inform the requestor that it will not comply with the request and shall provide an
679 explanation why it believes the request is fraudulent.
34 of 65
680 (f) When a controller, pursuant to subsection (b) of section 23 of this chapter, is incapable
681 of complying with an individual’s verifiable request, the controller shall, if possible, notify the
682 individual that it is not in a position to identify the individual. The individual, or a person entitled
683 to exercise the rights of this chapter on behalf of the individual pursuant to subsections (b) and
684 (c) of section 15 of this chapter, may provide additional information to the controller enabling
685 the individual’s identification for the purposes of exercising their rights set forth in sections 8
686 through 11 of this chapter.
687 (g) If a controller declines to take action regarding an individual’s request, the controller
688 shall notify the individual of the justification for declining to take action and provide the
689 individual with instructions on how to submit a complaint pursuant to subsection (j) of this
690 section. Such notification shall occur without undue delay, but not later than 45 days after the
691 initial receipt of the request or not later than 45 days after notifying the individual of the
692 applicability of an extension pursuant to subsection (b) of this section.
693 (h) A controller shall not be obligated to provide the information required by section 9 of
694 this chapter to the same individual more than twice in a 12 month period. Information provided
695 in response to a request shall be provided by the controller to the individual free of charge.
696 (i) If requests from an individual, or from a person entitled to exercise the rights of this
697 chapter on behalf of such individual pursuant to subsections (b) and (c) of section 15 of this
698 chapter, are manifestly unfounded or excessive, in particular because of their repetitive character,
699 the controller may: (1) charge a reasonable fee to cover the administrative costs of complying
700 with the request; or (2) refuse to act on the request. The controller shall bear the burden of
701 demonstrating the manifestly unfounded or excessive nature of the request.
35 of 65
702 (j) When informing an individual of any action taken or not taken in response to a
703 request, the controller shall provide the individual with a link to the Attorney General’s online
704 mechanism through which the individual may contact the Attorney General to submit a
705 complaint. The controller shall maintain records of all rejected requests for not less than 24
706 months and shall compile and provide a copy of such records to the Attorney General upon the
707 Attorney General’s request.
708 SECTION 18. No Waiver
709 Any provision of a contract or agreement of any kind that purports to waive or limit in
710 any way individual rights under this chapter shall be deemed contrary to public policy and shall
711 be void and unenforceable.
712 SECTION 19. Relationship Between Controllers and Processors
713 (a) A processor shall not be required to comply with a request pursuant to sections 8
714 through 13 of this chapter that the processor receives directly from an individual or from a
715 person entitled to exercise such rights on behalf of the individual, to the extent that the processor
716 has processed the individual’s personal information on behalf of the controller. A processor shall
717 adhere to the instructions of the controller and shall assist the controller in meeting its
718 obligations under this chapter. Such assistance shall include, but not be limited to, the following:
719 (1) Taking into account the nature of the processing and the information available to the
720 processor, the processor shall assist the controller by taking appropriate technical and
721 organizational measures, if possible, to fulfill the controller’s obligation to respond to
722 individuals’ requests to exercise their rights pursuant to sections 8 through 13 of this chapter,
723 including by:
36 of 65
724 (i) Providing to the controller the individual’s personal information, or correcting
725 inaccurate personal information, in the processor’s possession that the processor obtained as a
726 result of providing services to the controller, or enabling the controller to do the same;
727 (ii) At the direction of the controller in response to a verifiable request pursuant to section
728 10 of this chapter, deleting or enabling the controller to delete personal information about the
729 individual processed by the processor on behalf of the controller; provided, however, that the
730 processor shall notify any processors or third parties who may have accessed personal
731 information from or through the processor to delete the individual’s personal information, unless
732 the information was accessed at the direction of the controller or unless doing so proves
733 impossible or would constitute an undue burden; or
734 (iii) Not using sensitive information, after it has received instructions from the controller
735 and to the extent it has actual knowledge that the personal information is sensitive information,
736 for any purpose other than those authorized by section 13 of this chapter; provided, however, that
737 the processor shall only be required to limit its use of sensitive information received pursuant to
738 a written contract with the controller in response to instructions from the controller and only with
739 respect to its relationship with that controller;
740 (2) Taking into account the nature of the processing and the information available to the
741 processor, the processor shall assist the controller in meeting the controller’s obligations in
742 relation to the security of processing the personal information and in relation to the notification
743 of a breach of security of the system of the processor, pursuant to chapter 93H of the General
744 Laws; and
37 of 65
745 (3) The processor shall provide information to the controller necessary to enable the
746 controller to conduct and document any risk assessments required by section 21 of this chapter.
747 (b) Notwithstanding the instructions of the controller, a processor shall ensure that each
748 person processing personal information is subject to a duty of confidentiality with respect to the
749 information.
750 (c) If a processor engages another entity to assist the processor in processing personal
751 information on behalf of the controller, the processor shall provide the controller with an
752 opportunity to object and the engagement shall be pursuant to a written contract, in accordance
753 with subsection (e), that requires the entity to meet the obligations of the processor with respect
754 to the personal information.
755 (d) The controller and the processor shall implement appropriate technical and
756 organizational measures to ensure a level of security appropriate to the risk and establish a clear
757 allocation of the responsibilities between them to implement such measures.
758 (e) A contract between a controller and a processor shall govern the processor’s
759 procedures with respect to processing individuals’ personal information that the processor
760 receives from or on behalf of the controller. The contract shall be binding on both parties and
761 clearly set forth the processing instructions to which the processor is bound, including:
762 (1) The nature and purpose of the processing;
763 (2) The type of personal information subject to the processing;
764 (3) The duration of the processing;
765 (4) The rights and obligations of both parties;
38 of 65
766 (5) The requirements imposed by subsections (b) and (c); and
767 (6) The following requirements:
768 (i) At the controller’s direction, the processor shall delete or return all personal
769 information to the controller as requested at the end of the provision of services, unless retention
770 of the personal information is required by law;
771 (ii) Upon the reasonable request of the controller, the processor shall make available to
772 the controller all information in its possession necessary to demonstrate compliance with the
773 obligations under this chapter;
774 (iii) The processor shall: (A) allow for, and cooperate with, reasonable audits and
775 inspections by the controller or the controller’s designated auditor; or (B) arrange for, with the
776 controller’s consent, a qualified and independent auditor to conduct, at least annually and at the
777 processor’s expense, an audit of the processor’s policies and technical and organizational
778 measures in support of the obligations under this chapter using an appropriate and accepted
779 control standard or framework and audit procedure for such audits, provided that the processor
780 shall disclose a report of the audit to the controller upon request; and
781 (iv) The processor shall be prohibited from: (A) selling the personal information; (B)
782 retaining, using, or disclosing personal information other than for the purposes specified in the
783 contract or as otherwise permitted by this chapter; (C) retaining, using, or disclosing personal
784 information outside of the direct relationship between the processor and the controller; or (D)
785 combining, for the purpose of targeted advertising, the personal information with the personal
786 information that the processor receives from, or on behalf of, another entity or entities or that it
787 collects from its own interaction with the individual.
39 of 65
788 (f) In no event may any contract relieve a controller or a processor from the liabilities
789 imposed on it by this chapter.
790 (g) Determining whether an entity is acting as a controller or processor with respect to a
791 specific processing of information is a fact-based determination that depends upon the context in
792 which personal information is to be processed. A processor that continues to adhere to a
793 controller’s instructions with respect to a specific processing of personal information remains a
794 processor. If a processor begins, alone or jointly with others, determining the purposes and
795 means of the processing of personal information, it is a controller with respect to the processing.
796 An entity that is not limited in its processing of personal information pursuant to a controller’s
797 instruction, or that fails to adhere to such instructions, is a controller and not a processor with
798 respect to a specific processing.
799 SECTION 20. Data Broker Registration
800 (a) Not later than January 31 following each year in which a controller meets the
801 definition of a data broker under this chapter, the controller shall register with the Attorney
802 General pursuant to the requirements of this section.
803 (b) When registering with the Attorney General, a data broker shall:
804 (1) Pay a registration fee of 200 dollars; and
805 (2) Provide the following information:
806 (i) The name of the data broker and its primary physical, email, and Internet website
807 addresses;
40 of 65
808 (ii) Any privacy notice that a data broker discloses to individuals pursuant to section 7 of
809 this chapter;
810 (iii) How and where individuals may request to exercise the rights under sections 12 and
811 13 of this chapter;
812 (iv) Whether the data broker implements a purchaser credentialing process;
813 (v) Whether the data broker sells the personal information of individuals with whom the
814 data broker does not have a direct relationship;
815 (vi) Whether the data broker sells the sensitive information of at least 10,000 individuals;
816 (vii) Whether the data broker processes the personal information of minors or children;
817 and
818 (viii) Any additional information or explanation the data broker may wish to provide.
819 SECTION 21. Risk Assessments
820 (a) If a type of processing, taking into account the nature, scope, context and purposes of
821 the processing and whether the processing involves new technologies, is likely to result in a high
822 risk of harm to the individual, the controller shall, prior to the processing, carry out a risk
823 assessment of the impact of the envisioned processing operations on the protection of personal
824 information. A single assessment may address a set of similar processing operations that present
825 similar high risks.
826 (b) In particular, a controller shall conduct a risk assessment in the case of:
827 (1) The processing of sensitive information; 
41 of 65
828 (2) The sale of personal information; or
829 (3) A systematic and extensive evaluation of personal aspects relating to individuals that
830 is based on automated processing, on which decisions are based that present a reasonably
831 foreseeable risk of: (i) unfair or deceptive treatment of, or unlawful disparate impact on, certain
832 individuals; (ii) financial, physical, or reputational harm to individuals; (iii) a physical or other
833 intrusion upon the solitude or seclusion, or the private affairs or concerns, of individuals, where
834 such intrusion would be offensive to a reasonable person; or (iv) other substantial cognizable
835 harms to individuals.
836 (c) The assessment shall contain at a minimum:
837 (1) A systematic description of the envisioned processing operations and the purposes of
838 the processing, including, where applicable, the legitimate interest pursued by the controller or
839 third party;
840 (2) An assessment of the necessity of the processing operations in relation to the
841 purposes, taking into account whether the controller or third party can achieve their legitimate
842 interests in another less intrusive way;
843 (3) An assessment of the proportionality of the processing operations in relation to the
844 purposes, taking into account the amount and nature of the personal information to be processed;
845 (4) An assessment of the risks to individuals;
846 (5) The measures envisioned to address the risks, including safeguards such as de847 identification, security measures and mechanisms to ensure the protection of personal
42 of 65
848 information and to demonstrate compliance with this chapter taking into account the individuals’
849 reasonable expectations of privacy or other legal rights; and
850 (6) A description of: (i) the context of the processing; (ii) the relationship between the
851 controller and the individual whose personal information would be processed; and (iii) whether
852 the controller is processing an individual’s personal information in ways in which the individual
853 would reasonably expect.
854 (d) Subsections (a) through (c) shall not apply to processing pursuant to paragraph (3) of
855 section 6 of this chapter that has a legal basis in any federal or state law to which the controller is
856 subject; provided, however, that the law regulates the specific processing operation or set of
857 operations in question and the controller has already carried out a risk assessment that has
858 reasonably comparable scope and effect for the purpose of compliance with that law.
859 (e) Where necessary, the controller shall carry out a review to assess if processing is
860 performed in accordance with the risk assessment at least when there is a change of the risk
861 represented by processing operations.
862 (f) A controller shall implement procedures to comply with this section that are
863 reasonable and appropriate taking into consideration:
864 (1) The size, scope, and type of the controller;
865 (2) The amount of resources available to the controller;
866 (3) The amount and nature of personal information processed by the controller, including,
867 but not limited to, whether the personal information is sensitive information; and
43 of 65
868 (4) The need for the security and confidentiality of the personal information processed by
869 the controller.
870 (g) The Attorney General may require, pursuant to a civil investigative demand, that a
871 controller disclose any risk assessment that is relevant to an investigation conducted by the
872 Attorney General. The controller shall accordingly make the risk assessment available to the
873 Attorney General, and the Attorney General may evaluate the risk assessment for compliance
874 with the responsibilities in this chapter. Risk assessments shall be confidential and exempt from
875 public inspection and copying under chapter 66 of the General Laws. The disclosure of a risk
876 assessment pursuant to a civil investigative demand from the Attorney General shall not
877 constitute a waiver of attorney-client privilege or work product protection with respect to the
878 assessment and any information contained in the assessment.
879 (h) Risk assessments shall apply to processing activities created or generated after this
880 chapter is enacted and shall not be retroactive.
881 SECTION 22. Processing That Unlawfully Discriminates
882 (a) A controller that processes personal information in a manner that violates chapter
883 151B of the General Laws or any other state or federal law prohibiting unlawful discrimination
884 against individuals shall also be in violation of this chapter.
885 (b) Nothing in this section shall be construed to limit controllers from processing
886 personal information for legitimate testing to prevent unlawful discrimination or otherwise
887 determine the extent or effectiveness of the controller’s compliance with this section.
888 SECTION 23. De-identified Information
44 of 65
889 (a) A controller that possesses de-identified information shall:
890 (1) Take reasonable technical and organizational measures to ensure that the information
891 cannot be associated with an identified or identifiable individual or household;
892 (2) Not attempt to re-identify the information, provided that the controller may attempt to
893 re-identify the information solely for the purpose of determining whether its de-identification
894 procedures satisfy the requirements of this subsection; and
895 (3) Contractually require any recipients of the information to comply with all the
896 requirements of this subsection.
897 (b) This chapter shall not be construed to require a controller or processor to do any of
898 the following solely for the purpose of complying with this chapter:
899 (1) Maintain information in an identifiable, linkable, or associable form, or collect,
900 obtain, retain, or access any information or technology, in order to be capable of linking or
901 associating a verifiable request with personal information; or
902 (2) Reidentify or otherwise link de-identified information, provided that the controller
903 provides applicable notice to the individual pursuant to subsection (f) of section 17 of this
904 chapter.
905 SECTION 24. Limitations.
906 (a) The obligations imposed on controllers or processors under this chapter shall not
907 restrict a controller’s or a processor’s ability to:
908 (1) Comply with federal, state, or local laws, rules or regulations;
45 of 65
909 (2) Comply with a civil, criminal, or regulatory inquiry, subpoena, or summons by
910 federal, state, local, or other governmental authorities;
911 (3) Cooperate with law enforcement agencies concerning conduct or activity that the
912 controller or processor reasonably and in good faith believes may violate federal, state, or local
913 laws, rules, or regulations;
914 (4) Investigate, establish, exercise, prepare for, or defend legal claims;
915 (5) Take immediate steps to protect the security or protection of an individual or another
916 natural person, if that individual or other natural person is at risk or danger of death or serious
917 physical injury; or
918 (6) Assist another controller, processor, or third party with any of the obligations under
919 this subsection.
920 (b) The obligations imposed on controllers or processors under sections 8 through 13 of
921 this chapter shall not restrict a controller or processor’s ability to retain or process information
922 for the following purposes, provided that the use of the individual’s personal information is
923 reasonably necessary and proportionate for the purposes:
924 (1) Helping to ensure security and integrity;
925 (2) Debugging to identify and repair errors that impair existing intended functionality;
926 (3) Fulfilling the terms of a written warranty or product recall conducted in accordance
927 with federal law;
46 of 65
928 (4) Engaging in public or peer-reviewed scientific, historical, or statistical research in the
929 public interest that conforms or adheres to all other applicable ethics and privacy laws; provided,
930 however, that:
931 (i) Such research is approved, monitored, and governed by an institutional review board,
932 human subjects research ethics review board, or a similar independent oversight entity that
933 determines: (A) if the research is likely to provide substantial benefits that do not exclusively
934 accrue to the controller; (B) the expected benefits of the research outweigh the privacy risks; and
935 (C) if the controller has implemented reasonable safeguards to mitigate privacy risks associated
936 with research, including any risks associated with reidentification; or
937 (ii) A controller’s deletion of the personal information pursuant to a request under section
938 10 of this chapter is likely to render impossible or seriously impair the ability to complete such
939 research.
940 (d) Obligations imposed on controllers or processors under this chapter shall not:
941 (1) Apply to the processing of personal information by a natural person in the course of a
942 purely personal or household activity;
943 (2) Apply where compliance by the controller or processor would violate an evidentiary
944 privilege under the laws of the Commonwealth or be construed to prevent a controller or
945 processor from providing personal information concerning an individual to a person covered by
946 an evidentiary privilege under the laws of the Commonwealth as part of a privileged
947 communication;
47 of 65
948 (3) Adversely affect the right of an individual or any other person to exercise free speech,
949 pursuant to the First Amendment to the United States Constitution, or to exercise another right
950 provided for by law; or
951 (4) Apply to an entity’s publication of entity-based member or employee contact
952 information where such publication is intended to allow members of the public to contact such
953 member or employee in the ordinary course of the entity’s operations.
954 (e) Personal information that is processed by a controller pursuant to an exemption under
955 subsections (a) through (d) of this section:
956 (1) Shall not be processed for any purpose other than those expressly listed in subsections
957 (a) through (d), unless otherwise allowed by this chapter; and
958 (2) Notwithstanding anything in this section to the contrary, shall be processed in
959 accordance with section 5 of this chapter and subject to reasonable administrative, technical, and
960 physical measures to reduce reasonably foreseeable risks of harm to individuals.
961 (f) If a controller processes personal information pursuant to an exemption in subsections
962 (a) through (d) of this section, the controller bears the burden of demonstrating that such
963 processing qualifies for the exemption and complies with the requirements in subsection (e).
964 (g) A controller or processor that discloses personal information to a processor or third
965 party in compliance with the requirements of this chapter is not in violation of this chapter if the
966 recipient processes such personal information in violation of this chapter; provided, however,
967 that at the time of disclosing the personal information, the disclosing controller or processor did
968 not know or should not reasonably have known that the recipient intended to commit a violation.
48 of 65
969 (h) A processor or third party receiving personal information from a controller or
970 processor in compliance with the requirements of this chapter is not in violation of this chapter if
971 the controller or processor from which it receives the personal information fails to comply with
972 applicable obligations under this chapter; provided, however, that the processor or third party
973 shall be liable for its own violations of this chapter.
974 (i) If an individual has already consented to a controller’s use, disclosure, or sale of their
975 personal information to produce a physical item, such as a school yearbook, sections 10 through
976 13 of this chapter shall not apply to the controller’s use, disclosure, or sale of the particular
977 pieces of the individual’s personal information for the production of that physical item; provided,
978 however, that:
979 (1) The controller has incurred significant expense in reliance on the individual’s consent;
980 (2) Compliance with the individual’s request to exercise any of the rights in sections 10
981 through 13 would not be commercially reasonable; and
982 (3) The controller complies with the individual’s request as soon as it is commercially
983 reasonable to do so.
984 SECTION 25. Powers of the Attorney General
985 (a) Whenever the Attorney General of the Commonwealth has reasonable cause to
986 believe that an entity has engaged in, is engaging in, or is about to engage in a violation of this
987 chapter, the Attorney General may issue a civil investigative demand. The provisions of section 6
988 of chapter 93A of the General Laws shall apply mutatis mutandis to civil investigative demands
989 issued under this chapter.
49 of 65
990 (b) The Attorney General shall have the authority to enforce the provisions of this
991 chapter. A violation of this chapter shall not serve as the basis for or be subject to a private right
992 of action under this chapter. Nothing in this chapter shall be construed as creating a new private
993 right of action or serving as the basis for a private right of action that would not otherwise have
994 had a basis under any other law but for the enactment of this chapter. This chapter neither
995 relieves any party from any duties or obligations imposed, nor alters any independent rights that
996 individuals have, under chapter 93A of the General Laws, other state or federal laws, the
997 Massachusetts Constitution, or the United States Constitution.
998 (c) Prior to initiating any civil action under this chapter, the Attorney General shall
999 provide an entity written notice identifying the specific provisions of this chapter that the
1000 Attorney General alleges have been or are being violated.
1001 (d) (1) The entity shall have a period of 30 days in which to cure a violation after being
1002 provided notice by the Attorney General. If within that time period the entity cures the noticed
1003 violation and provides the Attorney General an express written statement that the alleged
1004 violations have been cured and that no further violations shall occur, no action shall be initiated
1005 against the entity.
1006 (2) Paragraph (1) shall not apply when:
1007 (i) The court has previously issued a temporary restraining order, preliminary injunction,
1008 or permanent injunction or assessed civil penalties against the entity for a violation of this
1009 chapter;
50 of 65
1010 (ii) The Attorney General and the entity have previously reached a settlement relating to
1011 this chapter that includes an admission by the entity that it has violated this chapter, not
1012 including any express written statement provided pursuant to paragraph (1);
1013 (iii) The Attorney General has clear and convincing evidence that the entity willfully and
1014 wantonly violated this chapter;
1015 (iv) The violation is a data broker’s failure to register pursuant to section 20 of this
1016 chapter; or
1017 (v) The violation occurs more than twenty four months after the effective date of this
1018 section and the violating entity: (A) as of January 1 of the calendar year, had annual global gross
1019 revenues in excess of 1,000,000,000 dollars in the preceding calendar year; and (B) determines
1020 the purposes and means of processing of the personal information of not less than 100,000
1021 individuals.
1022 (3) In its notice pursuant to subsection (c), the Attorney General shall specify the length,
1023 if any, of the period in which the entity can cure the noticed violation.
1024 (e) If an entity continues to violate this chapter following the cure period in subsection
1025 (d), breaches an express written statement provided to the Attorney General under that
1026 subsection, or is not eligible for a cure period pursuant to that subsection, the Attorney General
1027 may initiate a civil action against the entity in the name of the Commonwealth or as parens
1028 patriae on behalf of individuals. The Attorney General may seek a temporary restraining order,
1029 preliminary injunction, or permanent injunction to restrain any violations of this chapter and may
1030 seek civil penalties of up to 7,500 dollars for each violation under this chapter, not including
1031 violations of section 20 of this chapter.
51 of 65
1032 (f) The superior court shall have jurisdiction of actions brought under this section. Such
1033 actions may be brought in any county where a defendant resides or has its principal place of
1034 business or in which the violation occurred in whole or in part, or, with the consent of a
1035 defendant, in the superior court for Suffolk County.
1036 (g) In determining the overall amount of civil penalties to seek or assess against an entity,
1037 the Attorney General or the court shall include, but not be limited to, the following in its
1038 consideration:
1039 (1) The size, scope, and type of the entity;
1040 (2) The amount of resources available to the entity;
1041 (3) The amount and nature of personal information processed by the entity;
1042 (4) The number of violations;
1043 (5) The number of violations affecting minors or children;
1044 (6) The nature and severity of the violation;
1045 (7) The risks caused by the violation;
1046 (8) Whether the entity’s violation was not an isolated instance but instead part of a
1047 pattern of violations and noncompliance with this chapter;
1048 (9) Whether the entity is a data broker that did not register pursuant to section 20 of this
1049 chapter;
1050 (10) Whether the violation was willful and not the result of error;
52 of 65
1051 (11) The length of time over which the violation occurred;
1052 (12) The precautions taken by the entity to prevent a violation;
1053 (13) The good faith cooperation of the entity with any investigations conducted by the
1054 Attorney General pursuant to this section;
1055 (14) Efforts undertaken by the entity to cure the violation; and
1056 (15) The entity’s past violations of information privacy rules, regulations, codes,
1057 ordinances, and laws in other jurisdictions.
1058 (h) A data broker that fails to register as required by section 20 of this chapter may be
1059 subject to injunction and liable for civil penalties, fees, and costs in a civil action brought on
1060 behalf of the Commonwealth by the Attorney General as follows:
1061 (1) A civil penalty of up to 500 dollars for each day, not to exceed a total of 100,000
1062 dollars for each year, the data broker fails to register as required by this section; and
1063 (2) Fees equal to the fees that were due during the period the data broker failed to
1064 register.
1065 (i) Any entity that violates the terms of an injunction or other order issued under this
1066 section shall forfeit and pay a civil penalty of up to 10,000 dollars for each violation. For the
1067 purposes of this section, the court issuing such an injunction or order shall retain jurisdiction, and
1068 the cause shall be continued, and in such case the Attorney General acting in the name of the
1069 Commonwealth may petition for recovery of such civil penalty.
53 of 65
1070 (j) The Attorney General may recover reasonable expenses incurred in investigating and
1071 preparing the case, including attorney fees, in any action initiated under this chapter.
1072 (k) If two or more entities are involved in the same processing that violates this chapter,
1073 the liability shall be allocated among the parties according to principles of comparative fault.
1074 (l) Notwithstanding any general or special law to the contrary, the court may require that
1075 the amount of a civil penalty imposed pursuant to this section exceeds the economic benefit
1076 realized by an entity for noncompliance.
1077 (m) If a series of steps or transactions were component parts of a single transaction
1078 intended to avoid the reach of this chapter, the Attorney General and the court shall disregard the
1079 intermediate steps or transactions and consider everything one transaction for purposes of
1080 effectuating the purposes of this chapter.
1081 (n) Not later than 30 days after the end of each calendar year, the Attorney General shall
1082 publish a public, easily accessible report that provides, for that calendar year, the following
1083 information:
1084 (1) Anonymized examples of alleged violations that have been cured by an entity
1085 pursuant to subsection (d); provided, however, that these examples shall protect the
1086 confidentiality of the entity;
1087 (2) The number of written notices issued pursuant to subsection (c);
1088 (3) The number of entities that received written notices issued pursuant to subsection (c);
1089 and
1090 (4) The categories of violations of this chapter and the number of violations per category.
54 of 65
1091 (o) The Attorney General shall receive and may investigate sworn complaints from an
1092 individual or other natural person that an entity has engaged in, is engaging in, or is about to
1093 engage in any violation of this chapter. The Attorney General shall notify the individual or other
1094 natural person who made the complaint of the action, if any, the Attorney General has taken or
1095 plans to take on the complaint, together with the reasons for that action or nonaction.
1096 (p) The Attorney General shall maintain the following Internet web pages: (1) a web page
1097 that includes an online mechanism through which any individual or other natural person may
1098 contact the Attorney General to submit a sworn complaint; (2) a web page that enables data
1099 brokers to register pursuant to section 20 of this chapter; and (3) a web page that makes publicly
1100 accessible the information provided by data brokers pursuant to section 20 of this chapter.
1101 (q) The Attorney General shall promote public awareness and understanding of the risks,
1102 rules, responsibilities, safeguards, and rights in relation to the processing of personal
1103 information, including the rights of individuals under the age of 16 with respect to their own
1104 information. The Attorney General shall provide guidance to individuals regarding what to do if
1105 they believe their rights under this chapter have been violated.
1106 (r) The Attorney General shall create and make publicly accessible the following
1107 templates: (1) a template privacy policy that meets the requirements of section 7 of this chapter;
1108 (2) a template contract between a controller and a processor that meets the requirements of
1109 section 19 of this chapter; and (3) a template risk assessment that meets the requirements of
1110 section 21 of this chapter.
55 of 65
1111 (s) The Attorney General shall have the power to determine, pursuant to section 27 of this
1112 chapter, whether the provisions of a personal information privacy law in another jurisdiction are
1113 equally or more protective of personal information than the provisions in this chapter.
1114 (t) The Attorney General shall establish a mechanism pursuant to which an entity that
1115 processes the personal information of one or more individuals but does not meet the applicability
1116 criteria set forth in subsection (b) of section 3 of this chapter may voluntarily certify that it is in
1117 compliance with, and agrees to be bound by, this chapter. The Attorney General shall make a list
1118 of those entities available to the public.
1119 (u) The Attorney General shall adopt regulations for the purposes of carrying out this
1120 chapter, including, but not limited to, the following areas:
1121 (1) Supplementing any of the definitions used in this chapter or adding in new definitions
1122 for terms that are used but not otherwise defined in this chapter, in order to address changes in
1123 technology, data collection, obstacles to implementation, and privacy concerns; and
1124 (2) Ensuring that the notices and information that controllers are required to provide
1125 pursuant to section 7 of this chapter are provided in a manner that may be easily understood by
1126 the average individual, are accessible to individuals with disabilities, and are available in the
1127 language primarily used to interact with the individual.
1128 (v) The Attorney General shall conduct research and monitor relevant developments
1129 relating to the protection of personal information, the development of information and
1130 communication technologies and commercial practices, and the enactment and implementation
1131 of privacy laws in other states, territories, and countries or by the federal government. Specific
56 of 65
1132 topics for research by the Attorney General shall include, but are not limited to, the following
1133 areas:
1134 (1) The available best methods for an individual to exercise the rights set forth in sections
1135 8 through 13 of this chapter, including: (i) the development of technology, such as a browser
1136 setting, browser extension, or global device setting, indicating an individual’s affirmative, freely
1137 given, and unambiguous choice to opt out of the sale of the individual’s personal information or
1138 to limit the use or disclosure of the individual’s sensitive information; (ii) the development of
1139 technology that enables an individual to opt out of the sale of the individual’s personal
1140 information by all data brokers that have registered pursuant to section 20 of this chapter; and
1141 (iii) ways for entities to conspicuously and clearly disclose how to exercise the rights set forth in
1142 sections 8 through 13;
1143 (2) Access and opt-out rights with respect to controllers’ use of automated decision1144 making technology;
1145 (3) Eye-tracking technology and targeted advertising based on information collected
1146 through eye-tracking technology;
1147 (4) Financial incentive programs offered by controllers for the processing of personal
1148 information;
1149 (5) The targeting of advertising based on a profile of an individual created by an
1150 individual’s activity over time with regard to an entity’s own businesses, distinctly-branded
1151 websites, applications, or services;
57 of 65
1152 (6) The data broker industry, including data brokers that have registered pursuant to
1153 section 20 of this chapter;
1154 (7) The effectiveness of allowing an individual to designate an authorized agent to
1155 exercise a right on their behalf pursuant to subsection (d) of section 15 of this chapter; and
1156 (8) Whether to change or eliminate the cure period established in subsection (d) of
1157 section 25 of this chapter.
1158 (w) At least once per calendar year, the Attorney General shall provide a full written
1159 report to the Legislature’s Joint Committee on Advanced Information Technology, the Internet
1160 and Cybersecurity. The report shall summarize the Attorney General’s research and any
1161 recommendations with respect to privacy-related legislation. The first such report provided by
1162 the Attorney General shall be submitted within 12 months of the effective date of this subsection
1163 and shall include a summary of the Attorney General’s research and recommendations pursuant
1164 to paragraphs (1) through (5) of subsection (v).
1165 (x) The monetary amounts referred to in this chapter shall be indexed for inflation by the
1166 Attorney General, who, not later than December 31 of each even numbered year, shall calculate
1167 and publish such indexed amounts, using the federal consumer price index for the Boston
1168 statistical area and rounding to the nearest dollar.
1169 SECTION 26. Massachusetts Privacy Fund.
1170 (a) There shall be established upon the books of the Commonwealth a separate special
1171 fund to be known as the Massachusetts Privacy Fund.
58 of 65
1172 (b) All civil penalties, expenses, attorney fees, and registration fees collected pursuant to
1173 sections 20 and 25 of this chapter shall be paid into the state treasury and credited to the
1174 Massachusetts Privacy Fund. Interest earned on moneys in the Fund shall remain in the Fund and
1175 be credited to it. Any moneys remaining in the Fund, including interest thereon, at the end of
1176 each fiscal year shall not revert to the general fund but shall remain in the Fund.
1177 (c) The Attorney General shall have discretion to allocate the proceeds of any settlement
1178 of a civil action pursuant to this chapter to: (1) the Massachusetts Privacy Fund; (2) the general
1179 fund; or (3) where possible, directly to individuals impacted by the violation of the chapter.
1180 (d) Moneys in the Massachusetts Privacy Fund shall be used to support the work of the
1181 Attorney General pursuant to section 25 of this chapter. Moneys in the Massachusetts Privacy
1182 Fund shall be subject to appropriation and shall not be used to supplant general fund
1183 appropriations to the Attorney General.
1184 SECTION 27. Reciprocity and Interoperability
1185 (a) A controller or processor shall be in compliance with provisions of this chapter if: (1)
1186 it complies with comparable provisions of a personal information privacy law in another
1187 jurisdiction; (2) the controller or processor applies the provisions of that law to its processing
1188 activities concerning individuals; and (3) the Attorney General determines that the provisions of
1189 that law in the other jurisdiction are equally or more protective of personal information than the
1190 provisions of this chapter.
1191 (b) The Attorney General may charge a fee to a controller or processor that asserts
1192 compliance with a comparable law under subsection (a); provided, however, that the fee shall
59 of 65
1193 reflect costs reasonably expected to be incurred by the Attorney General to determine whether
1194 the provisions of said law are equally or more protective than the provisions of this chapter.
1195 SECTION 28. Delayed Implementation for Nonprofits and Institutions of Higher
1196 Education
1197 This chapter shall not apply to institutions of higher education or nonprofit organizations
1198 until 24 months after the effective date of this section.
1199 SECTION 29. Severability
1200 (a) The provisions of this chapter are severable. If any provision of this chapter, or the
1201 application of any provision of this chapter, is held invalid, the remaining provisions, or
1202 applications of provisions, shall remain in full force and not be affected.
1203 (b) If a court were to find in a final, unreviewable judgment that the exclusion of one or
1204 more entities or activities from the applicability of this chapter renders the chapter
1205 unconstitutional, those exceptions shall be rendered null and invalid and the exemption shall not
1206 continue.
1207 SECTION 2. Chapter 93H of the General Laws is hereby amended by inserting after
1208 section 6 of said chapter the following section:
1209 SECTION 7. Private Right of Action and Safe Harbor
1210 (a) For the purposes of this section, the term “personal information” shall have the same
1211 meaning as defined in section 1 of this chapter, except that for the purposes of subsections (c)
1212 and (d) of this section, the term “personal information” shall have the same meaning as in section
1213 2 of chapter 93M of the General Laws. 
60 of 65
1214 (b) For the purposes of this section, the following terms shall have the same meanings as
1215 such terms are defined in section 2 of chapter 93M of the General Laws: “controller”; “data
1216 broker,” “individual”; “process”; and “sell.”
1217 (c) This section shall apply to a controller that:
1218 (1) Conducts business in the Commonwealth or is not physically established in the
1219 Commonwealth but processes personal information where such processing activities are related
1220 to: (i) the offering of goods or services that are targeted to individuals; or (ii) the monitoring of
1221 behavior of individuals where such behavior takes place in the Commonwealth; and
1222 (2) Meets 1 of the following additional thresholds:
1223 (i) The controller, as of January 1 of the calendar year, had annual global gross revenues
1224 in excess of 25,000,000 dollars in the preceding calendar year;
1225 (ii) The controller determines the purposes and means of processing of the personal
1226 information of not less than 100,000 individuals; or
1227 (iii) The controller is a data broker.
1228 This section shall also apply to an entity that is an affiliate of and shares common
1229 branding with such a controller, with respect to the personal information processed by the
1230 affiliate on behalf of the controller.
1231 (d) This section shall not apply to controllers and information that are fully exempt from
1232 the provisions of chapter 93M of the General Laws pursuant to section 3 of that chapter;
1233 provided, however, that this section shall apply to an activity involving the processing of any
1234 personal information bearing on an individual’s credit worthiness, credit standing, credit
61 of 65
1235 capacity, character, general reputation, personal characteristics, or mode of living by: (A) a
1236 consumer reporting agency, as defined in 15 U.S.C. 1681a(f); (B) a furnisher of information, as
1237 set forth in 15 U.S.C. 1681s-2, that provides information for use in a consumer report, as defined
1238 in 15 U.S.C. 1681a(d); and (C) a user of a consumer report, as set forth in 15 U.S.C. 1681b.
1239 (e) Any individual whose personal information is subject to a breach of security, as
1240 defined in section 1 of this chapter, as a result of a controller’s failure to implement and maintain
1241 reasonable cybersecurity controls may institute a civil action for any of the following:
1242 (1) Damages from the controller in an amount up to 500 dollars per individual per
1243 incident or actual damages, whichever is greater;
1244 (2) Injunctive or declaratory relief;
1245 (3) Any other relief the court deems proper.
1246 (f) In assessing the amount of statutory damages against the controller, the court shall
1247 include, but not be limited to, the following in its consideration:
1248 (1) The size, scope, and type of the entity;
1249 (2) The amount of resources available to the entity;
1250 (3) The amount and nature of personal information processed by the entity;
1251 (4) The number of violations;
1252 (5) The number of violations affecting minors or children;
1253 (6) The nature and severity of the violation; 
62 of 65
1254 (7) The risks caused by the violation;
1255 (8) Whether the entity’s violation was not an isolated instance but instead part of a
1256 pattern of violations and noncompliance with this chapter;
1257 (9) Whether the entity is a data broker that did not register pursuant to section 20 of
1258 chapter 93M of the General Laws;
1259 (10) Whether the violation was willful and not the result of error;
1260 (11) The length of time over which the violation occurred;
1261 (12) The precautions taken by the entity to prevent a violation;
1262 (13) The good faith cooperation of the entity;
1263 (14) Efforts undertaken by the entity to cure the violation; and
1264 (15) The entity’s past violations of rules, regulations, codes, ordinances, and laws in other
1265 jurisdictions regarding breaches of security.
1266 (g) In any cause of action founded in tort that is brought pursuant to this section and that
1267 alleges that the controller’s failure to implement reasonable cybersecurity controls resulted in a
1268 breach of security concerning personal information, the court shall not assess punitive damages
1269 against a controller if such controller:
1270 (1) Created, maintained and complied with a written cybersecurity program that contains
1271 administrative, technical and physical safeguards for the protection of personal information and
1272 that conforms to an industry recognized cybersecurity framework, as described in subsection (i);
1273 and
63 of 65
1274 (2) Designed its cybersecurity program in accordance with the provisions of subsections
1275 (k) and (l).
1276 (h) Subsection (g) shall not apply if the controller’s failure to implement reasonable
1277 cybersecurity controls was the result of gross negligence or willful or wanton conduct.
1278 (i) A controller’s cybersecurity program, as described in subsection (g), shall conform to
1279 an industry recognized cybersecurity framework if:
1280 (1) The cybersecurity program conforms to the current version of or any combination of
1281 the current versions of:
1282 (i) The “Framework for Improving Critical Infrastructure Cybersecurity” published by
1283 the National Institute of Standards and Technology;
1284 (ii) The National Institute of Standards and Technology's special publication 800-171;
1285 (iii) The National Institute of Standards and Technology's special publications 800-53
1286 and 800-53a;
1287 (iv) The Federal Risk and Authorization Management Program's “FedRAMP Security
1288 Assessment Framework”;
1289 (v) The Center for Internet Security’s “Center for Internet Security Critical Security
1290 Controls for Effective Cyber Defense”; or
1291 (vi) The “ISO/IEC 27000-series” information security standards published by the
1292 International Organization for Standardization and the International Electrotechnical
1293 Commission; or
64 of 65
1294 (2) The cybersecurity program complies with the current version of the “Payment Card
1295 Industry Data Security Standard” and the current version of another applicable industry
1296 recognized cybersecurity framework described in paragraph (1) of this subsection.
1297 (j) When a revision to a document listed in paragraph (1) or (2) of subsection (i) is
1298 published, a controller whose cybersecurity program conforms to a prior version of that
1299 document shall be said to conform to the current version of that document if the controller
1300 conforms to such revision not later than six months after the publication date of the revision.
1301 (k) For the purposes of complying with this section, a controller’s cybersecurity program
1302 shall be implemented in accordance with the regulations adopted pursuant to chapter 93H of the
1303 General Laws.
1304 (l) The scale and scope of a controller’s cybersecurity program shall be based on:
1305 (1) The size, scope and type of controller obligated to safeguard the personal information
1306 under such program;
1307 (2) The amount of resources available to the controller;
1308 (3) The amount and nature of personal information processed by the controller; and
1309 (4) The reasonably foreseeable risks to the security and confidentiality of the personal
1310 information processed by the controller.
1311 (m) The cause of action established by this section shall apply only to violations as
1312 defined in this section. This chapter neither relieves any party from any duties or obligations
1313 imposed, nor alters any independent rights that individuals have under chapter 93A of the
65 of 65
1314 General Laws, other state or federal laws, the Massachusetts Constitution or the United States
1315 Constitution.
1316 (n) Nothing in this section shall limit the authority of the Attorney General to initiate
1317 actions as otherwise allowed in this section or pursuant to any other general law.
1318 SECTION 4. Chapter 93M of the General Laws shall take effect 18 months after the
1319 passage of this act, except that section 2 and subsections (p) through (w) of section 25 of said
1320 chapter shall take effect upon the passage of this act.
1321 SECTION 5. Section 2 of this act shall take effect 18 months after the passage of this act.