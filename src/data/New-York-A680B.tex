Introduced by M. of A. L. ROSENTHAL, QUART, WEPRIN, D. ROSENTHAL, SIMON,
   DINOWITZ, PAULIN -- read once and referred to the Committee on Consum-
   er  Affairs  and  Protection  --  committee  discharged, bill amended,
   ordered reprinted as amended and  recommitted  to  said  committee  --
   recommitted  to  the  Committee  on Consumer Affairs and Protection in
   accordance with Assembly Rule 3, sec. 2 -- committee discharged,  bill
   amended,  ordered reprinted as amended and recommitted to said commit-
   tee
 
 AN ACT to amend the general business law, in relation to the  management
   and oversight of personal data
 
   THE  PEOPLE OF THE STATE OF NEW YORK, REPRESENTED IN SENATE AND ASSEM-
 BLY, DO ENACT AS FOLLOWS:
 
   Section 1. Short title. This act shall be known and may  be  cited  as
 the "New York privacy act".
   §  2.  Legislative  intent.  1.  Privacy is a fundamental right and an
 essential element of freedom. Advances in technology have produced ramp-
 ant growth in the amount and categories of personal  data  being  gener-
 ated,   collected,  stored,  analyzed,  and  potentially  shared,  which
 presents both promise and peril. Companies collect, use  and  share  our
 personal  data  in  ways that can be difficult for ordinary consumers to
 understand. Opaque data processing policies make it impossible to evalu-
 ate risks  and  compare  privacy-related  protections  across  services,
 stifling  competition.  Algorithms  quietly make decisions with critical
 consequences for New York consumers, often with no human accountability.
 Behavioral advertising generates profits by turning people into products
 and their activity into assets. New York consumers deserve  more  notice
 and more control over their data and their digital privacy.
   2. This act seeks to help New York consumers regain their privacy.  It
 gives New York consumers the ability to exercise more control over their
 personal data and requires businesses to be responsible, thoughtful, and

  EXPLANATION--Matter in ITALICS (underscored) is new; matter in brackets
                       [ ] is old law to be omitted.
                                                            LBD00516-05-1
 A. 680--B                           2
 
 accountable  managers  of  that  information.  To achieve this, this act
 provides New York consumers a number  of  new  rights,  including  clear
 notice of how their data is being used, processed and shared; the abili-
 ty  to  access  and obtain a copy of their data in a commonly used elec-
 tronic format, with the ability to transfer  it  between  services;  the
 ability  to  correct  inaccurate  data and to delete their data; and the
 ability to challenge certain automated decisions. This act also  imposes
 obligations  upon  businesses  to  maintain reasonable data security for
 personal data, to notify New York consumers of foreseeable harms arising
 from use of their data and to obtain specific consent for that use,  and
 to conduct regular assessments to ensure that data is not being used for
 unacceptable purposes. These data assessments can be obtained and evalu-
 ated  by the New York State Attorney General, who is empowered to obtain
 penalties for violations of this act and prevent future violations. This
 act also grants New York consumers who have been injured as  the  result
 of  a  violation  a  private  right of action, which includes reasonable
 attorneys' fees to a prevailing plaintiff.
   § 3. The general business law is amended by adding a new article 42 to
 read as follows:
                                ARTICLE 42
                           NEW YORK PRIVACY ACT
 SECTION 1100. DEFINITIONS.
         1101. JURISDICTIONAL SCOPE.
         1102. CONSUMER RIGHTS.
         1103. CONTROLLER, PROCESSOR, AND THIRD PARTY RESPONSIBILITIES.
         1104. DATA BROKERS.
         1105. LIMITATIONS.
         1106. ENFORCEMENT AND PRIVATE RIGHT OF ACTION.
         1107. MISCELLANEOUS.
   § 1100. DEFINITIONS. THE FOLLOWING DEFINITIONS APPLY  THROUGHOUT  THIS
 ARTICLE UNLESS THE CONTEXT CLEARLY REQUIRES OTHERWISE:
   1.  "AUTOMATED DECISION-MAKING" OR "AUTOMATED DECISION" MEANS A COMPU-
 TATIONAL PROCESS, INCLUDING ONE DERIVED FROM MACHINE  LEARNING,  ARTIFI-
 CIAL  INTELLIGENCE,  OR  ANY OTHER AUTOMATED PROCESS, INVOLVING PERSONAL
 DATA THAT RESULTS IN A DECISION AFFECTING A CONSUMER.
   2. "BIOMETRIC INFORMATION" MEANS ANY PERSONAL DATA GENERATED FROM  THE
 MEASUREMENT  OR  SPECIFIC TECHNOLOGICAL PROCESSING OF A NATURAL PERSON'S
 BIOLOGICAL, PHYSICAL, OR PHYSIOLOGICAL CHARACTERISTICS, INCLUDING  FING-
 ERPRINTS, VOICE PRINTS, IRIS OR RETINA SCANS, FACIAL SCANS OR TEMPLATES,
 DEOXYRIBONUCLEIC ACID (DNA) INFORMATION, AND GAIT.
   3.  "BUSINESS  ASSOCIATE"  HAS  THE SAME MEANING AS IN TITLE 45 OF THE
 C.F.R., ESTABLISHED PURSUANT TO THE FEDERAL HEALTH INSURANCE PORTABILITY
 AND ACCOUNTABILITY ACT OF 1996.
   4. "CONSENT" MEANS A CLEAR AFFIRMATIVE ACT SIGNIFYING A FREELY  GIVEN,
 SPECIFIC, INFORMED, AND UNAMBIGUOUS INDICATION OF A CONSUMER'S AGREEMENT
 TO  THE  PROCESSING  OF  DATA RELATING TO THE CONSUMER.   CONSENT MAY BE
 WITHDRAWN AT ANY TIME, AND A CONTROLLER MUST PROVIDE CLEAR, CONSPICUOUS,
 AND CONSUMER-FRIENDLY MEANS TO WITHDRAW CONSENT. THE  BURDEN  OF  ESTAB-
 LISHING  CONSENT IS ON THE CONTROLLER.  CONSENT DOES NOT INCLUDE: (A) AN
 AGREEMENT OF GENERAL TERMS OF USE OR A SIMILAR DOCUMENT THAT  REFERENCES
 UNRELATED  INFORMATION  IN  ADDITION TO PERSONAL DATA PROCESSING; (B) AN
 AGREEMENT OBTAINED THROUGH FRAUD, DECEIT OR DECEPTION; (C) ANY ACT  THAT
 DOES  NOT CONSTITUTE A USER'S INTENT TO INTERACT WITH ANOTHER PARTY SUCH
 AS HOVERING OVER, PAUSING OR CLOSING ANY CONTENT; OR (D)  A  PRE-CHECKED
 BOX OR SIMILAR DEFAULT.
 A. 680--B                           3
 
   5. "CONSUMER" MEANS A NATURAL PERSON WHO IS A NEW YORK RESIDENT ACTING
 ONLY  IN  AN  INDIVIDUAL  OR  HOUSEHOLD  CONTEXT.  IT DOES NOT INCLUDE A
 NATURAL PERSON KNOWN TO  BE  ACTING  IN  A  PROFESSIONAL  OR  EMPLOYMENT
 CONTEXT.
   6.  "CONTROLLER"  MEANS  THE PERSON WHO, ALONE OR JOINTLY WITH OTHERS,
 DETERMINES THE PURPOSES AND MEANS OF THE PROCESSING OF PERSONAL DATA.
   7. "COVERED ENTITY" HAS THE SAME MEANING AS IN TITLE 45 OF THE C.F.R.,
 ESTABLISHED PURSUANT TO THE FEDERAL  HEALTH  INSURANCE  PORTABILITY  AND
 ACCOUNTABILITY ACT OF 1996.
   8.  "DATA  BROKER" MEANS A PERSON, OR UNIT OR UNITS OF A LEGAL ENTITY,
 SEPARATELY OR TOGETHER, THAT DOES BUSINESS IN THE STATE OF NEW YORK  AND
 KNOWINGLY  COLLECTS,  AND  SELLS  TO  CONTROLLERS  OR THIRD PARTIES, THE
 PERSONAL DATA OF A  CONSUMER  WITH  WHOM  IT  DOES  NOT  HAVE  A  DIRECT
 RELATIONSHIP. "DATA BROKER" DOES NOT INCLUDE ANY OF THE FOLLOWING:
   (A)  A  CONSUMER  REPORTING AGENCY TO THE EXTENT THAT IT IS COVERED BY
 THE FEDERAL FAIR CREDIT REPORTING ACT (15 U.S.C. SEC. 1681 ET SEQ.); OR
   (B) A FINANCIAL INSTITUTION TO THE EXTENT THAT IT IS  COVERED  BY  THE
 GRAMM-LEACH-BLILEY  ACT  (PUBLIC  LAW  106-102)  AND  IMPLEMENTING REGU-
 LATIONS.
   9. "DEIDENTIFIED DATA" MEANS DATA THAT CANNOT REASONABLY  BE  USED  TO
 INFER  INFORMATION ABOUT, OR OTHERWISE BE LINKED TO A PARTICULAR CONSUM-
 ER, HOUSEHOLD OR DEVICE, PROVIDED THAT THE PROCESSOR OR CONTROLLER  THAT
 POSSESSES THE DATA:
   (A) IMPLEMENTS REASONABLE TECHNICAL SAFEGUARDS TO ENSURE THAT THE DATA
 CANNOT BE ASSOCIATED WITH A CONSUMER, HOUSEHOLD OR DEVICE;
   (B) PUBLICLY COMMITS TO PROCESS THE DATA ONLY AS DEIDENTIFIED DATA AND
 NOT  ATTEMPT  TO  REIDENTIFY  THE  DATA,  EXCEPT  THAT THE CONTROLLER OR
 PROCESSOR MAY ATTEMPT TO  REIDENTIFY  THE  INFORMATION  SOLELY  FOR  THE
 PURPOSE  OF  DETERMINING  WHETHER ITS DEIDENTIFICATION PROCESSES SATISFY
 THE REQUIREMENTS OF THIS SUBDIVISION; AND
   (C) CONTRACTUALLY OBLIGATES ANY RECIPIENTS OF THE DATA TO COMPLY  WITH
 ALL PROVISIONS OF THIS ARTICLE.
   10.  "DEVICE"  MEANS ANY PHYSICAL OBJECT THAT IS CAPABLE OF CONNECTING
 TO THE INTERNET, DIRECTLY OR INDIRECTLY, OR TO  ANOTHER  DEVICE  AND  IS
 INTENDED  FOR  USE  BY A NATURAL PERSON OR HOUSEHOLD OR, IF USED OUTSIDE
 THE HOME, FOR USE BY THE GENERAL PUBLIC.
   11. "MEANINGFUL HUMAN REVIEW" MEANS REVIEW OR OVERSIGHT BY ONE OR MORE
 INDIVIDUALS WHO (A) ARE TRAINED IN THE CAPABILITIES AND  LIMITATIONS  OF
 THE  ALGORITHM  AT  ISSUE AND THE PROCEDURES TO INTERPRET AND ACT ON THE
 OUTPUT OF THE ALGORITHM, AND (B) HAVE THE AUTHORITY TO ALTER  THE  AUTO-
 MATED DECISION UNDER REVIEW.
   12. "NATURAL PERSON" MEANS A NATURAL PERSON ACTING ONLY IN AN INDIVID-
 UAL  OR HOUSEHOLD CONTEXT. IT DOES NOT INCLUDE A NATURAL PERSON KNOWN TO
 BE ACTING IN A PROFESSIONAL OR EMPLOYMENT CONTEXT.
   13. "PERSON" MEANS A NATURAL PERSON OR A LEGAL ENTITY,  INCLUDING  BUT
 NOT  LIMITED  TO  A  PROPRIETORSHIP,  PARTNERSHIP,  LIMITED PARTNERSHIP,
 CORPORATION, COMPANY, LIMITED LIABILITY COMPANY OR CORPORATION,  ASSOCI-
 ATION,  OR  OTHER  FIRM  OR SIMILAR BODY, OR ANY UNIT, DIVISION, AGENCY,
 DEPARTMENT, OR SIMILAR SUBDIVISION THEREOF.
   14. "PERSONAL DATA" MEANS ANY DATA THAT IDENTIFIES OR COULD REASONABLY
 BE LINKED, DIRECTLY OR  INDIRECTLY,  WITH  A  SPECIFIC  NATURAL  PERSON,
 HOUSEHOLD, OR DEVICE.  PERSONAL DATA DOES NOT INCLUDE DEIDENTIFIED DATA.
   15.  "IDENTIFIED  OR  IDENTIFIABLE"  MEANS A NATURAL PERSON WHO CAN BE
 IDENTIFIED, DIRECTLY OR INDIRECTLY, SUCH AS BY REFERENCE TO AN IDENTIFI-
 ER SUCH AS A NAME, AN IDENTIFICATION NUMBER, LOCATION DATA, OR AN ONLINE
 OR DEVICE IDENTIFIER.
 A. 680--B                           4
 
   16. "PROCESS", "PROCESSES" OR "PROCESSING" MEANS AN OPERATION  OR  SET
 OF  OPERATIONS WHICH ARE PERFORMED ON DATA OR ON SETS OF DATA, INCLUDING
 BUT NOT LIMITED TO THE COLLECTION, USE, ACCESS,  SHARING,  MONETIZATION,
 ANALYSIS, RETENTION, CREATION, GENERATION, DERIVATION, RECORDING, ORGAN-
 IZATION,   STRUCTURING,  STORAGE,  DISCLOSURE,  TRANSMISSION,  ANALYSIS,
 DISPOSAL, LICENSING, DESTRUCTION, DELETION, MODIFICATION, OR DEIDENTIFI-
 CATION OF DATA.
   17. "PROCESSOR" MEANS A PERSON THAT PROCESSES DATA ON  BEHALF  OF  THE
 CONTROLLER.
   18.  "PROFILING"  MEANS  ANY FORM OF AUTOMATED PROCESSING PERFORMED ON
 PERSONAL DATA TO EVALUATE, ANALYZE, OR PREDICT PERSONAL ASPECTS  RELATED
 TO  AN  IDENTIFIED  OR IDENTIFIABLE NATURAL PERSON'S ECONOMIC SITUATION,
 HEALTH,  PERSONAL   PREFERENCES,   INTERESTS,   RELIABILITY,   BEHAVIOR,
 LOCATION, OR MOVEMENTS.
   19. "PROTECTED HEALTH INFORMATION" HAS THE SAME MEANING AS IN TITLE 45
 C.F.R., ESTABLISHED PURSUANT TO THE FEDERAL HEALTH INSURANCE PORTABILITY
 AND ACCOUNTABILITY ACT OF 1996.
   20.  "SALE", "SELL", OR "SOLD" MEANS THE DISCLOSURE, TRANSFER, CONVEY-
 ANCE, SHARING, LICENSING,  MAKING  AVAILABLE,  PROCESSING,  GRANTING  OF
 PERMISSION  OR  AUTHORIZATION  TO PROCESS, OR OTHER EXCHANGE OF PERSONAL
 DATA, OR PROVIDING ACCESS TO PERSONAL DATA FOR MONETARY OR  OTHER  VALU-
 ABLE  CONSIDERATION BY THE CONTROLLER TO A THIRD PARTY.  "SALE" INCLUDES
 ENABLING, FACILITATING OR PROVIDING ACCESS TO A  CONSUMER  FOR  TARGETED
 ADVERTISING. "SALE" DOES NOT INCLUDE THE FOLLOWING:
   (A)  THE  DISCLOSURE  OF DATA TO A PROCESSOR WHO PROCESSES THE DATA ON
 BEHALF OF THE CONTROLLER AND  WHICH  IS  CONTRACTUALLY  PROHIBITED  FROM
 USING IT FOR ANY PURPOSE OTHER THAN AS INSTRUCTED BY THE CONTROLLER; OR
   (B)  THE  DISCLOSURE OR TRANSFER OF DATA AS AN ASSET THAT IS PART OF A
 MERGER, ACQUISITION, BANKRUPTCY, OR OTHER TRANSACTION IN  WHICH  ANOTHER
 ENTITY ASSUMES CONTROL OR OWNERSHIP OF ALL OR A MAJORITY OF THE CONTROL-
 LER'S ASSETS.
   21. "TARGETED ADVERTISING" MEANS DISPLAYING ONLINE ADVERTISEMENTS TO A
 CONSUMER  WHERE  THE  ADVERTISEMENT  IS  SELECTED BASED ON PERSONAL DATA
 OBTAINED OR INFERRED FROM A CONSUMER'S ACTIVITIES OVER TIME  AND  ACROSS
 ONE   OR  MORE  DISTINCTLY-BRANDED  WEBSITES,  ONLINE  APPLICATIONS,  OR
 SERVICES, TO PREDICT THE CONSUMER'S PREFERENCES OR INTERESTS.   IT  DOES
 NOT  INCLUDE  ADVERTISING (A) BASED SOLELY ON THE CONTEXT OF THE CONSUM-
 ER'S CURRENT SEARCH QUERY OR VISIT TO A WEBSITE OR ONLINE APPLICATION OR
 (B) TO A CONSUMER IN DIRECT  RESPONSE  TO  THE  CONSUMER'S  REQUEST  FOR
 INFORMATION OR FEEDBACK.
   22.  "THIRD  PARTY" MEANS, WITH RESPECT TO A PARTICULAR INTERACTION OR
 OCCURRENCE, A PERSON, PUBLIC AUTHORITY, AGENCY, OR BODY OTHER  THAN  THE
 CONSUMER, THE CONTROLLER, OR PROCESSOR OF THE CONTROLLER.  A THIRD PARTY
 MAY  ALSO  BE  A  CONTROLLER  IF  THE THIRD PARTY, ALONE OR JOINTLY WITH
 OTHERS, DETERMINES THE PURPOSES AND MEANS OF THE PROCESSING OF  PERSONAL
 DATA.
   23. "VERIFIED REQUEST" MEANS A REQUEST BY A CONSUMER OR THEIR AGENT TO
 EXERCISE  A  RIGHT AUTHORIZED BY THIS ARTICLE, THE AUTHENTICITY OF WHICH
 HAS BEEN ASCERTAINED BY THE CONTROLLER IN ACCORDANCE WITH PARAGRAPH  (C)
 OF SUBDIVISION EIGHT OF SECTION ELEVEN HUNDRED TWO OF THIS ARTICLE.
   § 1101. JURISDICTIONAL SCOPE. 1. THIS ARTICLE APPLIES TO LEGAL PERSONS
 THAT  CONDUCT  BUSINESS IN NEW YORK OR PRODUCE PRODUCTS OR SERVICES THAT
 ARE TARGETED TO RESIDENTS OF NEW YORK, AND THAT SATISFY ONE OR  MORE  OF
 THE FOLLOWING THRESHOLDS:
   (A) HAVE ANNUAL GROSS REVENUE OF TWENTY-FIVE MILLION DOLLARS OR MORE;
 A. 680--B                           5
 
   (B)  CONTROLS  OR  PROCESSES  PERSONAL  DATA  OF  ONE HUNDRED THOUSAND
 CONSUMERS OR MORE;
   (C)  CONTROLS  OR  PROCESSES  PERSONAL  DATA  OF FIVE HUNDRED THOUSAND
 NATURAL PERSONS OR MORE NATIONWIDE, AND CONTROLS OR  PROCESSES  PERSONAL
 DATA OF TEN THOUSAND CONSUMERS OR MORE; OR
   (D)  DERIVES  OVER  FIFTY  PERCENT  OF  GROSS REVENUE FROM THE SALE OF
 PERSONAL DATA, AND CONTROLS OR PROCESSES PERSONAL  DATA  OF  TWENTY-FIVE
 THOUSAND CONSUMERS OR MORE.
   2. THIS ARTICLE DOES NOT APPLY TO:
   (A) PERSONAL DATA PROCESSED BY STATE AND LOCAL GOVERNMENTS, AND MUNIC-
 IPAL  CORPORATIONS, FOR PROCESSES OTHER THAN SALE (FILING AND PROCESSING
 FEES ARE NOT SALE);
   (B) A NATIONAL SECURITIES ASSOCIATION REGISTERED PURSUANT  TO  SECTION
 15A  OF  THE SECURITIES EXCHANGE ACT OF 1934, AS AMENDED, OR REGULATIONS
 ADOPTED THEREUNDER OR A REGISTERED  FUTURES  ASSOCIATION  SO  DESIGNATED
 PURSUANT TO SECTION 17 OF THE COMMODITY EXCHANGE ACT, AS AMENDED, OR ANY
 REGULATIONS ADOPTED THEREUNDER;
   (C) INFORMATION THAT MEETS THE FOLLOWING CRITERIA:
   (I) PERSONAL DATA COLLECTED, PROCESSED, SOLD, OR DISCLOSED PURSUANT TO
 AND   IN  COMPLIANCE  WITH  THE  FEDERAL  GRAMM-LEACH-BLILEY  ACT  (P.L.
 106-102), AND IMPLEMENTING REGULATIONS;
   (II) PERSONAL DATA COLLECTED, PROCESSED, SOLD, OR  DISCLOSED  PURSUANT
 TO  THE  FEDERAL DRIVER'S PRIVACY PROTECTION ACT OF 1994 (18 U.S.C. SEC.
 2721 ET SEQ.), IF THE COLLECTION, PROCESSING, SALE, OR DISCLOSURE IS  IN
 COMPLIANCE WITH THAT LAW;
   (III) PERSONAL DATA REGULATED BY THE FEDERAL FAMILY EDUCATIONAL RIGHTS
 AND PRIVACY ACT, U.S.C. SEC. 1232G AND ITS IMPLEMENTING REGULATIONS;
   (IV)  PERSONAL  DATA COLLECTED, PROCESSED, SOLD, OR DISCLOSED PURSUANT
 TO THE FEDERAL FARM CREDIT ACT OF 1971 (AS AMENDED  IN  12  U.S.C.  SEC.
 2001-2279CC)  AND  ITS  IMPLEMENTING  REGULATIONS (12 C.F.R. PART 600 ET
 SEQ.) IF THE COLLECTION, PROCESSING, SALE, OR DISCLOSURE IS  IN  COMPLI-
 ANCE WITH THAT LAW;
   (V) PERSONAL DATA REGULATED BY SECTION TWO-D OF THE EDUCATION LAW;
   (VI)  DATA  MAINTAINED  AS EMPLOYMENT RECORDS, FOR PURPOSES OTHER THAN
 SALE;
   (VII) PROTECTED HEALTH INFORMATION THAT IS  LAWFULLY  COLLECTED  BY  A
 COVERED  ENTITY  OR  BUSINESS  ASSOCIATE AND IS GOVERNED BY THE PRIVACY,
 SECURITY, AND BREACH NOTIFICATION RULES  ISSUED  BY  THE  UNITED  STATES
 DEPARTMENT  OF  HEALTH AND HUMAN SERVICES, PARTS 160 AND 164 OF TITLE 45
 OF THE CODE OF FEDERAL REGULATIONS, ESTABLISHED PURSUANT TO  THE  HEALTH
 INSURANCE  PORTABILITY  AND  ACCOUNTABILITY  ACT  OF  1996  (PUBLIC  LAW
 104-191) ("HIPAA") AND THE HEALTH INFORMATION  TECHNOLOGY  FOR  ECONOMIC
 AND CLINICAL HEALTH ACT (PUBLIC LAW 111-5);
   (VIII)  PATIENT IDENTIFYING INFORMATION FOR PURPOSES OF 42 C.F.R. PART
 2, ESTABLISHED PURSUANT TO 42 U.S.C. SEC. 290DD-2, AS LONG AS SUCH  DATA
 IS NOT SOLD IN VIOLATION OF HIPAA OR ANY STATE OR FEDERAL LAW;
   (IX)  INFORMATION  AND  DOCUMENTS LAWFULLY CREATED FOR PURPOSES OF THE
 FEDERAL HEALTH CARE QUALITY IMPROVEMENT ACT OF 1986, AND  RELATED  REGU-
 LATIONS;
   (X) PATIENT SAFETY WORK PRODUCT CREATED FOR PURPOSES OF 42 C.F.R. PART
 3, ESTABLISHED PURSUANT TO 42 U.S.C. SEC. 299B-21 THROUGH 299B-26;
   (XI)  INFORMATION  THAT  IS  TREATED IN THE SAME MANNER AS INFORMATION
 EXEMPT UNDER SUBPARAGRAPH (VII) OF THIS PARAGRAPH THAT IS MAINTAINED  BY
 A  COVERED ENTITY OR BUSINESS ASSOCIATE AS DEFINED BY HIPAA OR A PROGRAM
 OR A QUALIFIED SERVICE ORGANIZATION AS DEFINED BY 42 U.S.C.  §  290DD-2,
 A. 680--B                           6
 
 AS  LONG  AS SUCH DATA IS NOT SOLD IN VIOLATION OF HIPAA OR ANY STATE OR
 FEDERAL LAW;
   (XII)  DEIDENTIFIED HEALTH INFORMATION THAT MEETS ALL OF THE FOLLOWING
 CONDITIONS:
   (A) IT IS DEIDENTIFIED IN ACCORDANCE WITH THE REQUIREMENTS FOR DEIDEN-
 TIFICATION SET FORTH IN SECTION 164.514 OF PART 164 OF TITLE 45  OF  THE
 CODE OF FEDERAL REGULATIONS;
   (B)  IT  IS  DERIVED  FROM  PROTECTED HEALTH INFORMATION, INDIVIDUALLY
 IDENTIFIABLE HEALTH INFORMATION,  OR  IDENTIFIABLE  PRIVATE  INFORMATION
 COMPLIANT  WITH THE FEDERAL POLICY FOR THE PROTECTION OF HUMAN SUBJECTS,
 ALSO KNOWN AS THE COMMON RULE; AND
   (C) A COVERED ENTITY OR BUSINESS ASSOCIATE DOES NOT ATTEMPT TO REIDEN-
 TIFY THE INFORMATION NOR DO THEY  ACTUALLY  REIDENTIFY  THE  INFORMATION
 EXCEPT AS OTHERWISE ALLOWED UNDER STATE OR FEDERAL LAW;
   (XIII)  PATIENT INFORMATION MAINTAINED BY A COVERED ENTITY OR BUSINESS
 ASSOCIATE GOVERNED BY THE PRIVACY,  SECURITY,  AND  BREACH  NOTIFICATION
 RULES  ISSUED  BY  THE  UNITED  STATES  DEPARTMENT  OF  HEALTH AND HUMAN
 SERVICES, PARTS 160 AND 164 OF TITLE 45 OF THE  CODE  OF  FEDERAL  REGU-
 LATIONS,  ESTABLISHED  PURSUANT  TO THE HEALTH INSURANCE PORTABILITY AND
 ACCOUNTABILITY ACT OF 1996 (PUBLIC  LAW  104-191),  TO  THE  EXTENT  THE
 COVERED  ENTITY  OR BUSINESS ASSOCIATE MAINTAINS THE PATIENT INFORMATION
 IN THE SAME MANNER AS  PROTECTED  HEALTH  INFORMATION  AS  DESCRIBED  IN
 SUBPARAGRAPH (VII) OF THIS PARAGRAPH;
   (XIV)  DATA  COLLECTED AS PART OF HUMAN SUBJECTS RESEARCH, INCLUDING A
 CLINICAL TRIAL, CONDUCTED IN ACCORDANCE WITH THE FEDERAL POLICY FOR  THE
 PROTECTION OF HUMAN SUBJECTS, ALSO KNOWN AS THE COMMON RULE, PURSUANT TO
 GOOD  CLINICAL  PRACTICE  GUIDELINES ISSUED BY THE INTERNATIONAL COUNCIL
 FOR HARMONISATION OR PURSUANT TO HUMAN SUBJECT  PROTECTION  REQUIREMENTS
 OF THE UNITED STATES FOOD AND DRUG ADMINISTRATION; OR
   (XV)  PERSONAL  DATA  PROCESSED  ONLY FOR ONE OR MORE OF THE FOLLOWING
 PURPOSES:
   (A) PRODUCT  REGISTRATION  AND  TRACKING  CONSISTENT  WITH  APPLICABLE
 UNITED STATES FOOD AND DRUG ADMINISTRATION REGULATIONS AND GUIDANCE;
   (B)  PUBLIC  HEALTH  ACTIVITIES  AND  PURPOSES AS DESCRIBED IN SECTION
 164.512 OF TITLE 45 OF THE CODE OF FEDERAL REGULATIONS; AND/OR
   (C) ACTIVITIES RELATED TO QUALITY, SAFETY, OR EFFECTIVENESS  REGULATED
 BY THE UNITED STATES FOOD AND DRUG ADMINISTRATION;
   (D) (I) AN ACTIVITY INVOLVING THE COLLECTION, MAINTENANCE, DISCLOSURE,
 SALE, COMMUNICATION, OR USE OF ANY PERSONAL DATA BEARING ON A CONSUMER'S
 CREDIT  WORTHINESS, CREDIT STANDING, CREDIT CAPACITY, CHARACTER, GENERAL
 REPUTATION, PERSONAL CHARACTERISTICS, OR MODE OF LIVING  BY  A  CONSUMER
 REPORTING  AGENCY,  AS  DEFINED  IN  TITLE 15 U.S.C. SEC. 1681A(F), BY A
 FURNISHER OF INFORMATION, AS SET FORTH IN TITLE 15 U.S.C. SEC.  1681S-2,
 WHO PROVIDES INFORMATION FOR USE IN A CONSUMER  REPORT,  AS  DEFINED  IN
 TITLE  15  U.S.C.  SEC. 1861A(D), AND BY A USER OF A CONSUMER REPORT, AS
 SET FORTH IN TITLE 15 U.S.C. SEC. 1681B.; AND
   (II) THIS PARAGRAPH SHALL APPLY ONLY TO THE EXTENT THAT SUCH  ACTIVITY
 INVOLVING  THE COLLECTION, MAINTENANCE, DISCLOSURE, SALE, COMMUNICATION,
 OR USE OF SUCH DATA BY THAT AGENCY, FURNISHER, OR  USER  IS  SUBJECT  TO
 REGULATION  UNDER  THE  FAIR  CREDIT REPORTING ACT, TITLE 15 U.S.C. SEC.
 1681 ET SEQ., AND THE DATA IS NOT COLLECTED, MAINTAINED, USED,  COMMUNI-
 CATED,  DISCLOSED,  OR  SOLD  EXCEPT  AS  AUTHORIZED  BY THE FAIR CREDIT
 REPORTING ACT.
   § 1102. CONSUMER RIGHTS. 1. RIGHT TO NOTICE. (A) NOTICE. EACH CONTROL-
 LER THAT PROCESSES A CONSUMER'S PERSONAL DATA  MUST  MAKE  PUBLICLY  AND
 A. 680--B                           7
 
 PERSISTENTLY  AVAILABLE, IN A CONSPICUOUS AND READILY ACCESSIBLE MANNER,
 A NOTICE CONTAINING THE FOLLOWING:
   (I)  A  DESCRIPTION  OF  THE  CONSUMER'S RIGHTS UNDER SUBDIVISIONS TWO
 THROUGH SIX OF THIS SECTION  AND  HOW  A  CONSUMER  MAY  EXERCISE  THOSE
 RIGHTS, INCLUDING HOW TO WITHDRAW CONSENT;
   (II)  THE  CATEGORIES OF PERSONAL DATA PROCESSED BY THE CONTROLLER AND
 BY ANY PROCESSOR WHO PROCESSES PERSONAL DATA ON BEHALF OF  THE  CONTROL-
 LER;
   (III) THE SOURCES FROM WHICH PERSONAL DATA IS COLLECTED;
   (IV) THE PURPOSES FOR PROCESSING PERSONAL DATA;
   (V) THE IDENTITY OF EACH THIRD PARTY TO WHOM THE CONTROLLER DISCLOSED,
 SHARED, TRANSFERRED OR SOLD PERSONAL DATA AND, FOR EACH IDENTIFIED THIRD
 PARTY,  (A)  THE  CATEGORIES  OF  PERSONAL DATA BEING SHARED, DISCLOSED,
 TRANSFERRED, OR SOLD TO THE THIRD PARTY,  (B)  THE  PURPOSES  FOR  WHICH
 PERSONAL  DATA  IS  BEING SHARED, DISCLOSED, TRANSFERRED, OR SOLD TO THE
 THIRD PARTY, (C) THE THIRD PARTY'S RETENTION PERIOD FOR EACH CATEGORY OF
 PERSONAL DATA PROCESSED BY THE THIRD PARTY OR PROCESSED ON THEIR BEHALF,
 OR IF THAT IS NOT POSSIBLE, THE CRITERIA USED TO DETERMINE  THE  PERIOD,
 AND  (D)  WHETHER  THE  THIRD  PARTY USES THE PERSONAL DATA FOR TARGETED
 ADVERTISING;
   (VI) THE CONTROLLER'S RETENTION PERIOD FOR EACH CATEGORY  OF  PERSONAL
 DATA  THAT  THEY  PROCESS OR IS PROCESSED ON THEIR BEHALF, OR IF THAT IS
 NOT POSSIBLE, THE CRITERIA USED TO DETERMINE THAT PERIOD; AND
   (VII)  FOR  CONTROLLERS  ENGAGING  IN  TARGETED  ADVERTISING,  AVERAGE
 EXPECTED REVENUE PER USER (ARPU) OR A SIMILAR METRIC FOR THE MOST RECENT
 FISCAL YEAR FOR THE REGION THAT COVERS NEW YORK.
   (B) NOTICE REQUIREMENTS.
   (I)  THE  NOTICE  MUST BE WRITTEN IN EASY-TO-UNDERSTAND LANGUAGE AT AN
 EIGHTH GRADE READING LEVEL OR BELOW.
   (II) THE CATEGORIES OF PERSONAL DATA PROCESSED AND PURPOSES FOR  WHICH
 EACH CATEGORY OF PERSONAL DATA IS PROCESSED MUST BE DESCRIBED AT A LEVEL
 SPECIFIC ENOUGH TO ENABLE A CONSUMER TO EXERCISE MEANINGFUL CONTROL OVER
 THEIR  PERSONAL DATA BUT NOT SO SPECIFIC AS TO RENDER THE NOTICE UNHELP-
 FUL TO A REASONABLE CONSUMER.
   (III) THE NOTICE MUST BE DATED WITH ITS EFFECTIVE DATE AND UPDATED  AT
 LEAST  ANNUALLY.    WHEN  THE  INFORMATION REQUIRED TO BE DISCLOSED TO A
 CONSUMER PURSUANT TO PARAGRAPH (A) OF THIS SUBDIVISION HAS  NOT  CHANGED
 SINCE  THE  IMMEDIATELY  PREVIOUS  NOTICE  (WHETHER  INITIAL, ANNUAL, OR
 REVISED) PROVIDED TO THE CONSUMER, A CONTROLLER MAY  ISSUE  A  STATEMENT
 THAT NO CHANGES HAVE BEEN MADE.
   (IV)  THE  NOTICE,  AS WELL AS EACH VERSION OF THE NOTICE IN EFFECT IN
 THE PRECEDING SIX YEARS,   MUST BE EASILY ACCESSIBLE  TO  CONSUMERS  AND
 CAPABLE OF BEING VIEWED BY CONSUMERS AT ANY TIME.
   2. OPT-IN CONSENT.  (A) A CONTROLLER MUST OBTAIN FREELY GIVEN, SPECIF-
 IC, INFORMED, AND UNAMBIGUOUS OPT-IN CONSENT FROM A CONSUMER TO:
   (I)  PROCESS  THE  CONSUMER'S PERSONAL DATA FOR ANY PURPOSE OTHER THAN
 THOSE IN SUBDIVISION TWO OF SECTION ELEVEN HUNDRED FIVE OF THIS ARTICLE;
 OR
   (II) MAKE  ANY  CHANGES  TO  THE  EXISTING  PROCESSING  OR  PROCESSING
 PURPOSE,  INCLUDING  THOSE REGARDING THE METHOD AND SCOPE OF COLLECTION,
 OF THE CONSUMER'S PERSONAL DATA THAT  MAY  BE  LESS  PROTECTIVE  OF  THE
 CONSUMER'S  PERSONAL  DATA THAN THE PROCESSING TO WHICH THE CONSUMER HAS
 PREVIOUSLY GIVEN THEIR FREELY GIVEN, SPECIFIC, INFORMED, AND UNAMBIGUOUS
 OPT-IN CONSENT.
   (B) ANY REQUEST FOR CONSENT MUST BE PROVIDED TO THE CONSUMER, PRIOR TO
 PROCESSING THEIR PERSONAL DATA, IN A STANDALONE DISCLOSURE THAT IS SEPA-
 A. 680--B                           8
 
 RATE AND APART FROM ANY CONTRACT OR  PRIVACY  POLICY.  THE  REQUEST  FOR
 CONSENT MUST:
   (I)  INCLUDE  A  CLEAR AND CONSPICUOUS DESCRIPTION OF EACH CATEGORY OF
 DATA AND PROCESSING PURPOSE FOR WHICH CONSENT IS SOUGHT;
   (II) CLEARLY IDENTIFY AND DISTINGUISH BETWEEN CATEGORIES OF  DATA  AND
 PROCESSING  PURPOSES THAT ARE NECESSARY TO PROVIDE THE SERVICES OR GOODS
 REQUESTED BY THE CONSUMER AND CATEGORIES OF DATA AND PROCESSING PURPOSES
 THAT ARE NOT NECESSARY TO PROVIDE THE SERVICES OR GOODS REQUESTED BY THE
 CONSUMER;
   (III) ENABLE A REASONABLE CONSUMER TO EASILY IDENTIFY  THE  CATEGORIES
 OF DATA AND PROCESSING PURPOSES FOR WHICH CONSENT IS SOUGHT;
   (IV)  CLEARLY  PRESENT  AS  THE  MOST  CONSPICUOUS CHOICE AN OPTION TO
 PROVIDE ONLY THE CONSENT NECESSARY TO  PROVIDE  THE  SERVICES  OR  GOODS
 REQUESTED BY THE CONSUMER;
   (V) CLEARLY PRESENT AN OPTION TO DENY CONSENT; AND
   (VI) WHERE THE REQUEST SEEKS CONSENT TO SHARING, DISCLOSURE, TRANSFER,
 OR  SALE  OF  PERSONAL  DATA  TO THIRD PARTIES, IDENTIFY EACH SUCH THIRD
 PARTY, THE CATEGORIES OF DATA SOLD OR SHARED WITH THEM,  THE  PROCESSING
 PURPOSES, THE RETENTION PERIOD, OR IF THAT IS NOT POSSIBLE, THE CRITERIA
 USED  TO  DETERMINE  THE  PERIOD, AND FOR EACH THIRD PARTY STATE IF SUCH
 SHARING, DISCLOSURE, TRANSFER, OR  SALE  ENABLES  OR  INVOLVES  TARGETED
 ADVERTISING.  THE  DETAILS  OF IDENTITIES OF SUCH THIRD PARTIES, AND THE
 CATEGORIES OF DATA, PROCESSING PURPOSES, AND THE RETENTION  PERIOD,  MAY
 BE  SET  FORTH  IN A DIFFERENT DISCLOSURE, PROVIDED THAT THE REQUEST FOR
 CONSENT CONTAINS A CONSPICUOUS AND  DIRECTLY  ACCESSIBLE  LINK  TO  THAT
 DISCLOSURE.
   (C)  TARGETED  ADVERTISING  AND  SALE  OF  PERSONAL  DATA SHALL NOT BE
 CONSIDERED PROCESSING PURPOSES THAT ARE NECESSARY TO PROVIDE SERVICES OR
 GOODS REQUESTED BY A CONSUMER.
   (D) ONCE A CONSUMER HAS PROVIDED FREELY GIVEN, SPECIFIC, INFORMED, AND
 UNAMBIGUOUS OPT-IN CONSENT TO PROCESS THEIR PERSONAL DATA FOR A PROCESS-
 ING PURPOSE, A CONTROLLER MAY RELY ON SUCH CONSENT  UNTIL  IT  IS  WITH-
 DRAWN.
   (E)  A  CONTROLLER MUST PROVIDE A MECHANISM FOR A CONSUMER TO WITHDRAW
 PREVIOUSLY GIVEN CONSENT AT ANY TIME. SUCH MECHANISM SHALL  MAKE  IT  AS
 EASY FOR A CONSUMER TO WITHDRAW THEIR CONSENT AS IT IS FOR SUCH CONSUMER
 TO PROVIDE CONSENT.
   (F)  A  CONTROLLER  MUST NOT INFER THAT A CONSUMER HAS PROVIDED FREELY
 GIVEN, SPECIFIC, INFORMED,  AND  UNAMBIGUOUS  OPT-IN  CONSENT  FROM  THE
 CONSUMER'S  INACTION  OR  THE  CONSUMER'S  CONTINUED USE OF A SERVICE OR
 PRODUCT PROVIDED BY THE CONTROLLER.
   (G) TO THE EXTENT THAT A CONTROLLER  MUST  PROCESS  INTERNET  PROTOCOL
 ADDRESSES,  SYSTEM  CONFIGURATION  INFORMATION, URLS OF REFERRING PAGES,
 LOCALE AND LANGUAGE PREFERENCES, KEYSTROKES,  OR  ANY  OTHER  DATA  THAT
 INDIVIDUALLY  OR  COLLECTIVELY  MAY  COMPRISE  PERSONAL DATA IN ORDER TO
 OBTAIN A CONSUMER'S FREELY GIVEN, SPECIFIC,  INFORMED,  AND  UNAMBIGUOUS
 OPT-IN CONSENT, THE CONTROLLER MUST:
   (I)  PROCESS ONLY THE PERSONAL DATA NECESSARY TO REQUEST FREELY GIVEN,
 SPECIFIC, INFORMED, AND UNAMBIGUOUS OPT-IN CONSENT;
   (II) PROCESS THE PERSONAL DATA SOLELY TO REQUEST FREELY GIVEN, SPECIF-
 IC, INFORMED, AND UNAMBIGUOUS OPT-IN CONSENT; AND
   (III) PROMPTLY DELETE  THE  PERSONAL  DATA  IF  CONSENT  IS  WITHHELD,
 DENIED, OR WITHDRAWN.
   (H)  CONTROLLERS  MUST  NOT  REQUEST  CONSENT  FROM A CONSUMER WHO HAS
 PREVIOUSLY WITHHELD OR DENIED CONSENT, UNLESS CONSENT  IS  NECESSARY  TO
 PROVIDE THE SERVICES OR GOODS REQUESTED BY THE CONSUMER.
 A. 680--B                           9
 
   (I) CONTROLLERS MUST TREAT USER-ENABLED PRIVACY CONTROLS IN A BROWSER,
 BROWSER   PLUG-IN,  SMARTPHONE  APPLICATION,  OPERATING  SYSTEM,  DEVICE
 SETTING, OR OTHER MECHANISM THAT COMMUNICATES OR SIGNALS THE  CONSUMER'S
 CHOICE  NOT  TO  BE SUBJECT TO TARGETED ADVERTISING OR THE SALE OF THEIR
 PERSONAL  DATA AS A DENIAL OF CONSENT UNDER THIS ACT. TO THE EXTENT THAT
 THE PRIVACY CONTROL CONFLICTS WITH A  CONSUMER'S  CONSENT,  THE  PRIVACY
 CONTROL  SETTINGS  GOVERN,  UNLESS  THE  CONSUMER PROVIDES FREELY GIVEN,
 SPECIFIC, INFORMED, AND  UNAMBIGUOUS  OPT-IN  CONSENT  TO  OVERRIDE  THE
 PRIVACY CONTROL.
   (J)  A  CONTROLLER  MUST NOT DISCRIMINATE AGAINST A CONSUMER FOR WITH-
 HOLDING OR DENYING CONSENT, INCLUDING, BUT NOT LIMITED TO, BY:
   (I) DENYING SERVICES OR GOODS TO THE  CONSUMER,  UNLESS  THE  CONSUMER
 DOES  NOT  CONSENT  TO  PROCESSING  NECESSARY TO PROVIDE THE SERVICES OR
 GOODS REQUESTED BY THE CONSUMER;
   (II) CHARGING  DIFFERENT  PRICES  FOR  GOODS  OR  SERVICES,  INCLUDING
 THROUGH  THE  USE OF DISCOUNTS OR OTHER BENEFITS, IMPOSING PENALTIES, OR
 PROVIDING A DIFFERENT LEVEL OR QUALITY  OF  SERVICES  OR  GOODS  TO  THE
 CONSUMER; OR
   (III)  SUGGESTING  THAT THE CONSUMER WILL RECEIVE A DIFFERENT PRICE OR
 RATE FOR GOODS OR SERVICES OR A DIFFERENT LEVEL OR QUALITY  OF  SERVICES
 OR GOODS.
   (K)  A  CONTROLLER  MAY,  WITH  THE CONSUMER'S FREELY GIVEN, SPECIFIC,
 INFORMED, AND UNAMBIGUOUS OPT-IN CONSENT GIVEN PURSUANT TO THIS SECTION,
 OPERATE A PROGRAM IN WHICH INFORMATION, PRODUCTS, OR  SERVICES  SOLD  TO
 THE  CONSUMER  ARE  DISCOUNTED  BASED  SOLELY  ON  SUCH CONSUMER'S PRIOR
 PURCHASES FROM THE CONTROLLER, PROVIDED THAT THE PERSONAL DATA  USED  TO
 OPERATE  SUCH  PROGRAM  IS PROCESSED SOLELY FOR THE PURPOSE OF OPERATING
 SUCH PROGRAM.
   (L) IN THE EVENT OF A MERGER, ACQUISITION, BANKRUPTCY, OR OTHER TRANS-
 ACTION IN WHICH ANOTHER ENTITY ASSUMES CONTROL OR OWNERSHIP  OF  ALL  OR
 MAJORITY  OF  THE  CONTROLLER'S  ASSETS,  ANY  CONSENT  PROVIDED  TO THE
 CONTROLLER BY A CONSUMER PRIOR TO SUCH TRANSACTION SHALL BE DEEMED WITH-
 DRAWN.
   3. RIGHT TO ACCESS.  UPON  THE  VERIFIED  REQUEST  OF  A  CONSUMER,  A
 CONTROLLER SHALL:
   (A)  CONFIRM  WHETHER OR NOT THE CONTROLLER IS PROCESSING OR HAS PROC-
 ESSED PERSONAL DATA OF THAT CONSUMER, AND PROVIDE ACCESS TO  A  COPY  OF
 ANY  SUCH  PERSONAL  DATA  IN  A  MANNER  UNDERSTANDABLE TO A REASONABLE
 CONSUMER WHEN REQUESTED; AND
   (B) PROVIDE THE IDENTITY OF EACH PROCESSOR OR THIRD PARTY TO WHOM  THE
 CONTROLLER  DISCLOSED, TRANSFERRED, OR SOLD THE CONSUMER'S PERSONAL DATA
 AND, FOR EACH IDENTIFIED PROCESSOR OR THIRD PARTY, (I) THE CATEGORIES OF
 THE CONSUMER'S PERSONAL DATA DISCLOSED, TRANSFERRED,  OR  SOLD  TO  EACH
 PROCESSOR  OR  THIRD PARTY AND (II) THE PURPOSES FOR WHICH EACH CATEGORY
 OF THE CONSUMER'S PERSONAL DATA WAS DISCLOSED, TRANSFERRED, OR  SOLD  TO
 EACH PROCESSOR OR THIRD PARTY.
   4. RIGHT TO PORTABLE DATA.  UPON A VERIFIED REQUEST, AND TO THE EXTENT
 TECHNICALLY FEASIBLE, THE CONTROLLER MUST: (A) PROVIDE TO THE CONSUMER A
 COPY  OF  ALL  OF, OR A PORTION OF, AS DESIGNATED IN A VERIFIED REQUEST,
 THE  CONSUMER'S  PERSONAL  DATA  IN  A  STRUCTURED,  COMMONLY  USED  AND
 MACHINE-READABLE  FORMAT  AND (B) TRANSMIT THE DATA TO ANOTHER PERSON OF
 THE CONSUMER'S OR THEIR AGENT'S DESIGNATION WITHOUT HINDRANCE.
   5. RIGHT TO CORRECT. (A) UPON THE VERIFIED REQUEST OF  A  CONSUMER  OR
 THEIR  AGENT,  A  CONTROLLER  MUST CONDUCT A REASONABLE INVESTIGATION TO
 DETERMINE WHETHER PERSONAL DATA, THE ACCURACY OF WHICH  IS  DISPUTED  BY
 THE  CONSUMER,  IS  INACCURATE,  WITH SUCH INVESTIGATION TO BE CONCLUDED
 A. 680--B                          10
 
 WITHIN THE TIME PERIOD SET FORTH IN PARAGRAPH (A) OF  SUBDIVISION  EIGHT
 OF THIS SECTION.
   (B)  NOTWITHSTANDING  PARAGRAPH  (A) OF THIS SUBDIVISION, A CONTROLLER
 MAY TERMINATE AN INVESTIGATION INITIATED PURSUANT TO SUCH  PARAGRAPH  IF
 THE  CONTROLLER REASONABLY AND IN GOOD FAITH DETERMINES THAT THE DISPUTE
 BY THE CONSUMER IS WHOLLY WITHOUT MERIT, INCLUDING BY REASON OF A  FAIL-
 URE  BY  A CONSUMER TO PROVIDE SUFFICIENT INFORMATION TO INVESTIGATE THE
 DISPUTED PERSONAL DATA. UPON MAKING ANY DETERMINATION IN ACCORDANCE WITH
 THIS PARAGRAPH THAT A DISPUTE IS  WHOLLY  WITHOUT  MERIT,  A  CONTROLLER
 MUST,  WITHIN  THE TIME PERIOD SET FORTH IN PARAGRAPH (A) OF SUBDIVISION
 EIGHT OF THIS SECTION, PROVIDE THE  AFFECTED  CONSUMER  A  STATEMENT  IN
 WRITING THAT INCLUDES, AT A MINIMUM, THE SPECIFIC REASONS FOR THE DETER-
 MINATION,  AND IDENTIFICATION OF ANY INFORMATION REQUIRED TO INVESTIGATE
 THE DISPUTED PERSONAL DATA, WHICH MAY CONSIST  OF  A  STANDARDIZED  FORM
 DESCRIBING THE GENERAL NATURE OF SUCH INFORMATION.
   (C)  IF,  AFTER ANY INVESTIGATION UNDER PARAGRAPH (A) OF THIS SUBDIVI-
 SION OF ANY PERSONAL DATA  DISPUTED  BY  A  CONSUMER,  AN  ITEM  OF  THE
 PERSONAL  DATA  IS  FOUND  TO  BE INACCURATE OR INCOMPLETE, OR CANNOT BE
 VERIFIED, THE CONTROLLER MUST:
   (I) CORRECT THE INACCURATE OR INCOMPLETE PERSONAL DATA OF THE  CONSUM-
 ER; AND
   (II)  UNLESS IT PROVES IMPOSSIBLE OR INVOLVES DISPROPORTIONATE EFFORT,
 COMMUNICATE SUCH REQUEST TO EACH PROCESSOR OR THIRD PARTY  TO  WHOM  THE
 CONTROLLER  DISCLOSED, TRANSFERRED, OR SOLD THE PERSONAL DATA WITHIN ONE
 YEAR PRECEDING THE CONSUMER'S REQUEST, AND TO REQUIRE  THOSE  PROCESSORS
 OR  THIRD  PARTIES  TO  DO  THE SAME FOR ANY FURTHER PROCESSORS OR THIRD
 PARTIES THEY DISCLOSED, TRANSFERRED, OR SOLD THE PERSONAL DATA TO.
   (D) IF THE INVESTIGATION DOES NOT RESOLVE THE  DISPUTE,  THE  CONSUMER
 MAY  FILE WITH THE CONTROLLER A BRIEF STATEMENT SETTING FORTH THE NATURE
 OF THE DISPUTE. WHENEVER A STATEMENT OF A DISPUTE IS FILED, UNLESS THERE
 EXISTS REASONABLE GROUNDS TO BELIEVE THAT IT IS  WHOLLY  WITHOUT  MERIT,
 THE CONTROLLER MUST NOTE THAT IT IS DISPUTED BY THE CONSUMER AND INCLUDE
 EITHER  THE CONSUMER'S STATEMENT OR A CLEAR AND ACCURATE CODIFICATION OR
 SUMMARY  THEREOF  WITH  THE  DISPUTED  PERSONAL  DATA  WHENEVER  IT   IS
 DISCLOSED, TRANSFERRED, OR SOLD TO ANY PROCESSOR OR THIRD PARTY.
   6.  RIGHT  TO  DELETE.  (A) UPON THE VERIFIED REQUEST OF A CONSUMER, A
 CONTROLLER MUST:
   (I) WITHIN FORTY-FIVE  DAYS  AFTER  RECEIVING  THE  VERIFIED  REQUEST,
 DELETE  ANY  OR  ALL PERSONAL DATA, AS DIRECTED BY THE CONSUMER OR THEIR
 AGENT,  THAT THE CONTROLLER POSSESSES OR CONTROLS; AND
   (II) UNLESS IT PROVES IMPOSSIBLE OR INVOLVES  DISPROPORTIONATE  EFFORT
 THAT  IS  DOCUMENTED  IN  WRITING  BY  THE  CONTROLLER, COMMUNICATE SUCH
 REQUEST TO  EACH  PROCESSOR  OR  THIRD  PARTY  TO  WHOM  THE  CONTROLLER
 DISCLOSED, TRANSFERRED OR SOLD THE PERSONAL DATA WITHIN ONE YEAR PRECED-
 ING  THE  CONSUMER'S  REQUEST  AND  TO REQUIRE THOSE PROCESSORS OR THIRD
 PARTIES TO DO THE SAME FOR ANY FURTHER PROCESSORS OR THIRD PARTIES  THEY
 DISCLOSED, TRANSFERRED, OR SOLD THE PERSONAL DATA TO.
   (B) FOR PERSONAL DATA THAT IS NOT POSSESSED BY THE CONTROLLER BUT BY A
 PROCESSOR  OF  THE CONTROLLER, THE CONTROLLER MAY CHOOSE TO (I) COMMUNI-
 CATE THE CONSUMER'S REQUEST FOR  DELETION  TO  THE  PROCESSOR,  OR  (II)
 REQUEST  THAT  THE  PROCESSOR RETURN TO THE CONTROLLER THE PERSONAL DATA
 THAT IS THE SUBJECT OF THE CONSUMER'S REQUEST AND DELETE  SUCH  PERSONAL
 DATA UPON RECEIPT OF THE REQUEST.
   (C) A CONSUMER'S DELETION OF THEIR ONLINE ACCOUNT MUST BE TREATED AS A
 REQUEST  TO  THE  CONTROLLER  TO  DELETE ALL OF THAT CONSUMER'S PERSONAL
 DATA.
 A. 680--B                          11
 
   (D) A CONTROLLER  MUST  MAINTAIN  REASONABLE  PROCEDURES  DESIGNED  TO
 PREVENT  THE  REAPPEARANCE IN ITS SYSTEMS, AND IN ANY DATA IT DISCLOSES,
 TRANSFERS, OR SELLS TO ANY PROCESSOR OR THIRD PARTY, THE  PERSONAL  DATA
 THAT IS DELETED PURSUANT TO THIS SUBDIVISION.
   (E)  A  CONTROLLER IS NOT REQUIRED TO COMPLY WITH A CONSUMER'S REQUEST
 TO DELETE PERSONAL DATA IF:
   (I) COMPLYING WITH THE  REQUEST  WOULD  PREVENT  THE  CONTROLLER  FROM
 PERFORMING  ACCOUNTING  FUNCTIONS,  PROCESSING  REFUNDS,  EFFECTUATING A
 PRODUCT RECALL PURSUANT TO FEDERAL OR STATE LAW, OR FULFILLING  WARRANTY
 CLAIMS,  PROVIDED  THAT  THE  PERSONAL  DATA  THAT IS THE SUBJECT OF THE
 REQUEST IS NOT PROCESSED FOR ANY PURPOSE OTHER THAN SUCH SPECIFIC ACTIV-
 ITIES; OR
   (II) IT IS NECESSARY FOR THE CONTROLLER  TO  MAINTAIN  THE  CONSUMER'S
 PERSONAL  DATA  TO ENGAGE IN PUBLIC OR PEER-REVIEWED SCIENTIFIC, HISTOR-
 ICAL, OR STATISTICAL RESEARCH IN THE PUBLIC INTEREST THAT ADHERES TO ALL
 OTHER APPLICABLE ETHICS AND PRIVACY LAWS, WHEN THE CONTROLLER'S DELETION
 OF THE INFORMATION IS LIKELY TO RENDER IMPOSSIBLE  OR  SERIOUSLY  IMPAIR
 THE  ACHIEVEMENT  OF SUCH RESEARCH, PROVIDED THAT THE CONSUMER HAS GIVEN
 INFORMED CONSENT AND THE PERSONAL DATA IS NOT PROCESSED FOR ANY  PURPOSE
 OTHER THAN SUCH RESEARCH.
   7. AUTOMATED DECISION-MAKING. (A) WHENEVER A CONTROLLER MAKES AN AUTO-
 MATED  DECISION  INVOLVING  SOLELY  AUTOMATED PROCESSING THAT MATERIALLY
 CONTRIBUTES TO A DENIAL  OF  FINANCIAL  OR  LENDING  SERVICES,  HOUSING,
 PUBLIC  ACCOMMODATION,  INSURANCE,  HEALTH  CARE  SERVICES, OR ACCESS TO
 BASIC NECESSITIES, SUCH AS FOOD AND WATER, THE CONTROLLER MUST:
   (I) DISCLOSE IN A CLEAR,  CONSPICUOUS,  AND  CONSUMER-FRIENDLY  MANNER
 THAT THE DECISION WAS MADE BY A SOLELY AUTOMATED PROCESS;
   (II)  PROVIDE  AN AVENUE FOR THE AFFECTED CONSUMER TO APPEAL THE DECI-
 SION, WHICH MUST AT MINIMUM ALLOW THE AFFECTED CONSUMER TO (A)  FORMALLY
 CONTEST THE DECISION, (B) PROVIDE INFORMATION TO SUPPORT THEIR POSITION,
 AND (C) OBTAIN MEANINGFUL HUMAN REVIEW OF THE DECISION; AND
   (III) EXPLAIN THE PROCESS TO APPEAL THE DECISION.
   (B) A CONTROLLER MUST RESPOND TO A CONSUMER'S APPEAL WITHIN FORTY-FIVE
 DAYS  OF  RECEIPT  OF  THE  APPEAL.  THAT PERIOD MAY BE EXTENDED ONCE BY
 FORTY-FIVE ADDITIONAL  DAYS  WHERE  REASONABLY  NECESSARY,  TAKING  INTO
 ACCOUNT THE COMPLEXITY AND NUMBER OF APPEALS. THE CONTROLLER MUST INFORM
 THE  CONSUMER OF ANY SUCH EXTENSION WITHIN FORTY-FIVE DAYS OF RECEIPT OF
 THE APPEAL, TOGETHER WITH THE REASONS FOR THE DELAY.
   (C) (I) A CONTROLLER OR PROCESSOR ENGAGED IN AUTOMATED DECISION-MAKING
 AFFECTING FINANCIAL OR LENDING SERVICES, HOUSING, PUBLIC  ACCOMMODATION,
 INSURANCE,  EDUCATION  ENROLLMENT,  EMPLOYMENT, HEALTH CARE SERVICES, OR
 ACCESS TO BASIC NECESSITIES, SUCH AS  FOOD  AND  WATER,  OR  ENGAGED  IN
 ASSISTING  OTHERS  IN  AUTOMATED  DECISION-MAKING  IN THOSE FIELDS, MUST
 ANNUALLY CONDUCT AN IMPACT ASSESSMENT OF SUCH AUTOMATED  DECISION-MAKING
 THAT:
   (A)  DESCRIBES  AND  EVALUATES  THE  OBJECTIVES AND DEVELOPMENT OF THE
 AUTOMATED DECISION-MAKING PROCESSES INCLUDING THE  DESIGN  AND  TRAINING
 DATA  USED  TO  DEVELOP  THE  AUTOMATED DECISION-MAKING PROCESS, HOW THE
 AUTOMATED DECISION-MAKING PROCESS WAS  TESTED  FOR  ACCURACY,  FAIRNESS,
 BIAS AND DISCRIMINATION; AND
   (B)  ASSESSES  WHETHER  THE  AUTOMATED DECISION-MAKING SYSTEM PRODUCES
 DISCRIMINATORY RESULTS ON THE BASIS OF A CONSUMER'S OR CLASS OF  CONSUM-
 ERS'  ACTUAL  OR  PERCEIVED  RACE,  COLOR, ETHNICITY, RELIGION, NATIONAL
 ORIGIN, SEX,  GENDER,  GENDER  IDENTITY,  SEXUAL  ORIENTATION,  FAMILIAL
 STATUS, BIOMETRIC INFORMATION, LAWFUL SOURCE OF INCOME, OR DISABILITY.
 A. 680--B                          12
 
   (II)  A  CONTROLLER OR PROCESSOR MUST UTILIZE AN EXTERNAL, INDEPENDENT
 AUDITOR OR RESEARCHER TO CONDUCT SUCH ASSESSMENTS.
   (III)  A  CONTROLLER  OR  PROCESSOR  MUST MAKE PUBLICLY AVAILABLE IN A
 MANNER ACCESSIBLE ONLINE ALL IMPACT  ASSESSMENTS  PREPARED  PURSUANT  TO
 THIS SECTION, RETAIN ALL SUCH IMPACT ASSESSMENTS FOR AT LEAST SIX YEARS,
 AND  MAKE  ANY  SUCH RETAINED IMPACT ASSESSMENTS AVAILABLE TO ANY STATE,
 FEDERAL, OR LOCAL GOVERNMENT AUTHORITY UPON REQUEST.
   (IV) FOR PURPOSES OF THIS PARAGRAPH, THE LIMITATIONS TO JURISDICTIONAL
 SCOPE SET FORTH IN PARAGRAPHS (B) AND (C) OF SUBDIVISION TWO OF  SECTION
 ELEVEN HUNDRED ONE OF THIS ARTICLE SHALL NOT APPLY.
   8.  RESPONDING  TO  REQUESTS.  (A) A CONTROLLER MUST TAKE ACTION UNDER
 SUBDIVISIONS THREE THROUGH SIX OF THIS SECTION AND INFORM  THE  CONSUMER
 OF  ANY ACTIONS TAKEN WITHOUT UNDUE DELAY AND IN ANY EVENT WITHIN FORTY-
 FIVE DAYS OF RECEIPT OF THE REQUEST. THAT PERIOD MAY BE EXTENDED ONCE BY
 FORTY-FIVE ADDITIONAL  DAYS  WHERE  REASONABLY  NECESSARY,  TAKING  INTO
 ACCOUNT  THE  COMPLEXITY AND NUMBER OF THE REQUESTS. THE CONTROLLER MUST
 INFORM THE CONSUMER OF ANY SUCH  EXTENSION  WITHIN  FORTY-FIVE  DAYS  OF
 RECEIPT  OF THE REQUEST, TOGETHER WITH THE REASONS FOR THE DELAY. WHEN A
 CONTROLLER DENIES ANY SUCH REQUEST, IT MUST WITHIN THIS PERIOD  DISCLOSE
 TO  THE  CONSUMER A STATEMENT IN WRITING OF THE SPECIFIC REASONS FOR THE
 DENIAL.
   (B) A CONTROLLER SHALL PERMIT THE EXERCISE OF RIGHTS AND CARRY OUT ITS
 OBLIGATIONS SET FORTH IN SUBDIVISIONS THREE THROUGH SIX OF THIS  SECTION
 FREE  OF CHARGE, AT LEAST TWICE ANNUALLY TO THE CONSUMER. WHERE REQUESTS
 FROM A CONSUMER ARE MANIFESTLY UNFOUNDED  OR  EXCESSIVE,  IN  PARTICULAR
 BECAUSE  OF  THEIR  REPETITIVE  CHARACTER, THE CONTROLLER MAY EITHER (I)
 CHARGE A REASONABLE FEE TO COVER THE ADMINISTRATIVE COSTS  OF  COMPLYING
 WITH  THE  REQUEST  OR  (II) REFUSE TO ACT ON THE REQUEST AND NOTIFY THE
 CONSUMER OF THE REASON FOR REFUSING THE REQUEST.  THE  CONTROLLER  BEARS
 THE  BURDEN OF DEMONSTRATING THE MANIFESTLY UNFOUNDED OR EXCESSIVE CHAR-
 ACTER OF THE REQUEST.
   (C) (I)  A  CONTROLLER  SHALL  PROMPTLY  ATTEMPT,  USING  COMMERCIALLY
 REASONABLE  EFFORTS,  TO VERIFY THAT ALL REQUESTS TO EXERCISE ANY RIGHTS
 SET FORTH IN ANY SECTION OF THIS ARTICLE REQUIRING  A  VERIFIED  REQUEST
 WERE MADE BY THE CONSUMER WHO IS THE SUBJECT OF THE DATA, OR BY A PERSON
 LAWFULLY  EXERCISING  THE  RIGHT  ON  BEHALF  OF THE CONSUMER WHO IS THE
 SUBJECT OF THE DATA. COMMERCIALLY REASONABLE EFFORTS SHALL BE DETERMINED
 BASED ON THE TOTALITY OF THE CIRCUMSTANCES, INCLUDING THE NATURE OF  THE
 DATA IMPLICATED BY THE REQUEST.
   (II)  A  CONTROLLER  MAY  REQUIRE  THE  CONSUMER TO PROVIDE ADDITIONAL
 INFORMATION ONLY IF THE REQUEST CANNOT REASONABLY  BE  VERIFIED  WITHOUT
 THE  PROVISION  OF  SUCH  ADDITIONAL  INFORMATION. A CONTROLLER MUST NOT
 TRANSFER OR PROCESS ANY SUCH ADDITIONAL INFORMATION PROVIDED PURSUANT TO
 THIS SECTION FOR ANY OTHER PURPOSE AND MUST DELETE ANY  SUCH  ADDITIONAL
 INFORMATION  WITHOUT UNDUE DELAY AND IN ANY EVENT WITHIN FORTY-FIVE DAYS
 AFTER THE CONTROLLER HAS NOTIFIED THE CONSUMER THAT IT HAS TAKEN  ACTION
 ON  A  REQUEST  UNDER  SUBDIVISIONS  TWO THROUGH FIVE OF THIS SECTION AS
 DESCRIBED IN PARAGRAPH (A) OF THIS SUBDIVISION.
   (III) IF A CONTROLLER DISCLOSES THIS  ADDITIONAL  INFORMATION  TO  ANY
 PROCESSOR  OR  THIRD  PARTY  FOR  THE  PURPOSE  OF  VERIFYING A CONSUMER
 REQUEST, IT MUST NOTIFY THE RECEIVING PROCESSOR OR THIRD  PARTY  AT  THE
 TIME  OF  SUCH  DISCLOSURE,  OR AS CLOSE IN TIME TO THE DISCLOSURE AS IS
 REASONABLY PRACTICABLE,  THAT  SUCH  INFORMATION  WAS  PROVIDED  BY  THE
 CONSUMER  FOR  THE  SOLE PURPOSE OF VERIFICATION AND CANNOT BE PROCESSED
 FOR ANY PURPOSE OTHER THAN VERIFICATION.
 A. 680--B                          13

   9. IMPLEMENTATION OF RIGHTS. CONTROLLERS MUST PROVIDE EASILY  ACCESSI-
 BLE  AND  CONVENIENT  MEANS FOR CONSUMERS TO EXERCISE THEIR RIGHTS UNDER
 THIS ARTICLE.
   10.  NON-WAIVER OF RIGHTS. ANY PROVISION OF A CONTRACT OR AGREEMENT OF
 ANY KIND THAT PURPORTS TO WAIVE OR LIMIT IN ANY WAY A CONSUMER'S  RIGHTS
 UNDER  THIS  ARTICLE  IS CONTRARY TO PUBLIC POLICY AND IS VOID AND UNEN-
 FORCEABLE.
   § 1103.  CONTROLLER, PROCESSOR, AND THIRD PARTY  RESPONSIBILITIES.  1.
 CONTROLLER  RESPONSIBILITIES. (A) DATA PROTECTION ASSESSMENT. A CONTROL-
 LER SHALL REGULARLY CONDUCT AND DOCUMENT A  DATA  PROTECTION  ASSESSMENT
 FOR  PROCESSING ACTIVITIES THAT PRESENT A HEIGHTENED RISK OF HARM TO THE
 CONSUMER. SUCH ASSESSMENT MUST IDENTIFY AND WEIGH THE BENEFITS THAT  MAY
 FLOW,  DIRECTLY  AND  INDIRECTLY, FROM THE PROCESSING TO THE CONTROLLER,
 THE CONSUMER, OTHER STAKEHOLDERS, AND THE PUBLIC AGAINST  THE  POTENTIAL
 RISKS  TO  THE RIGHTS OF THE CONSUMER, OR CLASS OF CONSUMERS, ASSOCIATED
 WITH THE PROCESSING, AS MITIGATED BY SAFEGUARDS THAT THE CONTROLLER  CAN
 EMPLOY  TO  REDUCE  THE  RISKS.  THE  CONTROLLER  SHALL FACTOR INTO THIS
 ASSESSMENT THE USE OF DEIDENTIFIED DATA AND THE REASONABLE  EXPECTATIONS
 OF CONSUMERS, AS WELL AS THE CONTEXT OF THE PROCESSING AND THE RELATION-
 SHIP BETWEEN THE CONTROLLER AND THE CONSUMER WHOSE PERSONAL DATA WILL BE
 PROCESSED,  WITH  THE GOAL OF RESTRICTING OR PROHIBITING SUCH PROCESSING
 IF THE RISKS OF HARM TO THE CONSUMER  OUTWEIGH  THE  BENEFITS  RESULTING
 FROM THE PROCESSING TO THE CONSUMER.  PROCESSING THAT PRESENTS A HEIGHT-
 ENED RISK OF HARM TO THE CONSUMER INCLUDES THE FOLLOWING:
   (I) PROCESSING THAT MAY BENEFIT THE CONTROLLER TO THE DETRIMENT OF THE
 CONSUMER;
   (II)  PROCESSING  THAT  WOULD  BE UNEXPECTED AND HIGHLY OFFENSIVE TO A
 REASONABLE CONSUMER;
   (III) PROCESSING PERSONAL DATA FOR PURPOSES OF TARGETED ADVERTISING;
   (IV) SALE OF PERSONAL DATA; AND
   (V) PROCESSING OF PERSONAL DATA FOR PURPOSES OF PROFILING, WHERE  SUCH
 PROFILING PRESENTS A REASONABLY FORESEEABLE RISK OF:
   (A)  UNFAIR  OR  DECEPTIVE TREATMENT, OR UNLAWFUL DISPARATE IMPACT ON,
 CONSUMERS OR A CLASS OF CONSUMERS;
   (B) FINANCIAL,  PHYSICAL,  PSYCHOLOGICAL  OR  REPUTATIONAL  INJURY  TO
 CONSUMERS, OR A CLASS OF CONSUMERS;
   (C)  A PHYSICAL OR OTHERWISE INTRUSION UPON THE SOLITUDE OR SECLUSION,
 OR THE PRIVATE AFFAIRS OR CONCERNS, OF CONSUMERS, WHERE  SUCH  INTRUSION
 WOULD BE OFFENSIVE TO A REASONABLE PERSON; OR
   (D) OTHER SUBSTANTIAL INJURY TO CONSUMERS.
   (B)  DUTY  OF  LOYALTY.  (I) A CONTROLLER MUST NOTIFY THE CONSUMER, OR
 CLASS OF CONSUMERS, OF THE INTEREST THAT MAY BE  HARMED  IN  ADVANCE  OF
 REQUESTING CONSENT AND AS CLOSE IN TIME TO THE PROCESSING AS PRACTICABLE
 WHERE  IT  IS  REASONABLY  FORESEEABLE  TO THE CONTROLLER THAT A PROCESS
 PRESENTS A HEIGHTENED RISK OF HARM TO THE CONSUMER OR CLASS  OF  CONSUM-
 ERS.
   (II) CONTROLLERS MUST NOT ENGAGE IN UNFAIR, DECEPTIVE, OR ABUSIVE ACTS
 OR  PRACTICES WITH RESPECT TO OBTAINING CONSUMER CONSENT, THE PROCESSING
 OF PERSONAL DATA, AND A CONSUMER'S EXERCISE OF  ANY  RIGHTS  UNDER  THIS
 ARTICLE, INCLUDING WITHOUT LIMITATION:
   (A)  DESIGNING A USER INTERFACE WITH THE PURPOSE OR SUBSTANTIAL EFFECT
 OF DECEIVING CONSUMERS, OBSCURING CONSUMERS' RIGHTS UNDER THIS  ARTICLE,
 OR  SUBVERTING OR IMPAIRING USER AUTONOMY, DECISION-MAKING, OR CHOICE IN
 ORDER TO OBTAIN CONSENT; OR
   (B) OBTAINING CONSENT IN A MANNER DESIGNED TO OVERPOWER  A  CONSUMER'S
 RESISTANCE; FOR EXAMPLE, BY MAKING EXCESSIVE REQUESTS FOR CONSENT.
 A. 680--B                          14
 
   (C)  DUTY  OF  CARE.  (I)  (A) CONTROLLERS MUST, ON AT LEAST AN ANNUAL
 BASIS, CONDUCT AND DOCUMENT RISK ASSESSMENTS OF ALL  CURRENT  PROCESSING
 OF PERSONAL DATA.
   (B) RISK ASSESSMENTS MUST ASSESS AT A MINIMUM:
   (I)  THE NATURE, SENSITIVITY AND CONTEXT OF THE PERSONAL DATA THAT THE
 CONTROLLER PROCESSES;
   (II) THE NATURE, PURPOSE, AND VALUE OF THE PROCESSES;
   (III) ANY RISKS OR HARMS TO CONSUMERS ACTUALLY OR POTENTIALLY  ARISING
 OUT  OF  THE PROCESSES, INCLUDING PHYSICAL, FINANCIAL, PSYCHOLOGICAL, OR
 REPUTATIONAL HARMS;
   (IV) THE ADEQUACY AND EFFECT OF SAFEGUARDS IMPLEMENTED BY THE CONTROL-
 LERS;
   (V) THE SUFFICIENCY  OF  THE  CONTROLLER'S  NOTICES  TO  CONSUMERS  AT
 DESCRIBING AND OBTAINING CONSENT CONCERNING THE PROCESSES; AND
   (VI)  THE  ADEQUACY  OF  THE  SAFEGUARDS  AND  MONITORING PRACTICES OF
 PROCESSORS AND  THIRD  PARTIES  TO  WHOM  THE  CONTROLLER  HAS  PROVIDED
 PERSONAL DATA.
   (C) THE CONTROLLER MUST RETAIN RISK ASSESSMENTS FOR AT LEAST SIX YEARS
 AND  MAKE  RISK  ASSESSMENTS  AVAILABLE  TO  THE  ATTORNEY  GENERAL UPON
 REQUEST.
   (II) CONTROLLERS MUST  DEVELOP,  IMPLEMENT,  AND  MAINTAIN  REASONABLE
 SAFEGUARDS TO PROTECT THE SECURITY, CONFIDENTIALITY AND INTEGRITY OF THE
 PERSONAL DATA OF CONSUMERS INCLUDING ADOPTING REASONABLE ADMINISTRATIVE,
 TECHNICAL  AND  PHYSICAL SAFEGUARDS APPROPRIATE TO THE VOLUME AND NATURE
 OF THE PERSONAL DATA AT ISSUE.
   (III) (A) A CONTROLLER SHALL LIMIT THE USE AND RETENTION OF A  CONSUM-
 ER'S  PERSONAL  DATA  TO  WHAT IS NECESSARY TO PROVIDE A SERVICE OR GOOD
 REQUESTED BY A CONSUMER OR FOR  PURPOSES  FOR  WHICH  THE  CONSUMER  HAS
 PROVIDED  FREELY  GIVEN,  SPECIFIC,  INFORMED,  AND  UNAMBIGUOUS  OPT-IN
 CONSENT.
   (B) AT LEAST ANNUALLY, A CONTROLLER SHALL REVIEW ITS  RETENTION  PRAC-
 TICES  FOR  THE  PURPOSE  OF ENSURING THAT IT IS MAINTAINING THE MINIMUM
 AMOUNT OF PERSONAL DATA AS IS NECESSARY FOR THE OPERATION OF  ITS  BUSI-
 NESS.  A  CONTROLLER MUST DISPOSE OF ALL PERSONAL DATA THAT IS NO LONGER
 (I) NECESSARY TO PROVIDE THE SERVICES OR GOODS REQUESTED BY THE  CONSUM-
 ER,  (II) NECESSARY FOR THE INTERNAL BUSINESS OPERATIONS OF THE CONTROL-
 LER AND CONSISTENT WITH THE DISCLOSURES MADE TO THE CONSUMER PURSUANT TO
 SECTION ELEVEN HUNDRED TWO OF THIS ARTICLE, OR (III) NECESSARY TO COMPLY
 WITH THE LEGAL OBLIGATIONS OF THE CONTROLLER.
   (IV) CONTROLLERS SHALL BE UNDER A CONTINUING OBLIGATION TO  ENGAGE  IN
 REASONABLE  MEASURES  TO  REVIEW THEIR ACTIVITIES FOR CIRCUMSTANCES THAT
 MAY HAVE ALTERED THEIR ABILITY TO IDENTIFY A SPECIFIC NATURAL PERSON AND
 TO UPDATE THEIR CLASSIFICATIONS OF DATA AS  IDENTIFIED  OR  IDENTIFIABLE
 ACCORDINGLY.
   (D) NON-DISCRIMINATION. (I) A CONTROLLER MUST NOT DISCRIMINATE AGAINST
 A  CONSUMER  FOR  EXERCISING  RIGHTS  UNDER  THIS ACT, INCLUDING BUT NOT
 LIMITED TO, BY:
   (A) DENYING SERVICES OR GOODS TO CONSUMERS;
   (B) CHARGING DIFFERENT PRICES FOR SERVICES OR GOODS, INCLUDING THROUGH
 THE USE OF DISCOUNTS OR OTHER BENEFITS; IMPOSING PENALTIES; OR PROVIDING
 A DIFFERENT LEVEL OR QUALITY OF SERVICES OR GOODS TO THE CONSUMER; OR
   (C) SUGGESTING THAT THE CONSUMER WILL RECEIVE  A  DIFFERENT  PRICE  OR
 RATE  FOR  SERVICES OR GOODS OR A DIFFERENT LEVEL OR QUALITY OF SERVICES
 OR GOODS.
 A. 680--B                          15
 
   (II) THIS PARAGRAPH DOES NOT APPLY  TO  A  CONTROLLER'S  CONDUCT  WITH
 RESPECT  TO  OPT-IN  CONSENT, IN WHICH CASE PARAGRAPH (J) OF SUBDIVISION
 TWO OF SECTION ELEVEN HUNDRED TWO OF THIS ARTICLE GOVERNS.
   (E)  AGREEMENTS  WITH  PROCESSORS.  (I)  BEFORE MAKING ANY DISCLOSURE,
 TRANSFER, OR SALE OF PERSONAL DATA TO ANY PROCESSOR, THE CONTROLLER MUST
 ENTER INTO A WRITTEN, SIGNED CONTRACT WITH THAT PROCESSOR. SUCH CONTRACT
 MUST BE BINDING AND CLEARLY SET FORTH INSTRUCTIONS FOR PROCESSING  DATA,
 THE  NATURE AND PURPOSE OF PROCESSING, THE TYPE OF DATA SUBJECT TO PROC-
 ESSING, THE DURATION OF PROCESSING, AND THE RIGHTS  AND  OBLIGATIONS  OF
 BOTH  PARTIES.  THE  CONTRACT  MUST  ALSO  INCLUDE REQUIREMENTS THAT THE
 PROCESSOR MUST:
   (A) ENSURE THAT EACH PERSON PROCESSING PERSONAL DATA IS SUBJECT  TO  A
 DUTY OF CONFIDENTIALITY WITH RESPECT TO THE DATA;
   (B)  PROTECT  THE DATA IN A MANNER CONSISTENT WITH THE REQUIREMENTS OF
 THIS ACT AND AT LEAST EQUAL TO THE SECURITY REQUIREMENTS OF THE CONTROL-
 LER SET FORTH IN THEIR PUBLICLY AVAILABLE POLICIES, NOTICES, OR  SIMILAR
 STATEMENTS;
   (C)  PROCESS  THE DATA ONLY WHEN AND TO THE EXTENT NECESSARY TO COMPLY
 WITH ITS LEGAL OBLIGATIONS TO THE CONTROLLER UNLESS OTHERWISE EXPLICITLY
 AUTHORIZED BY THE CONTROLLER;
   (D) NOT COMBINE THE PERSONAL DATA WHICH THE PROCESSOR RECEIVES FROM OR
 ON BEHALF OF THE CONTROLLER  WITH  PERSONAL  DATA  WHICH  THE  PROCESSOR
 RECEIVES  FROM  OR  ON BEHALF OF ANOTHER PERSON OR COLLECTS FROM ITS OWN
 INTERACTION WITH CONSUMERS;
   (E) COMPLY WITH ANY EXERCISES OF A  CONSUMER'S  RIGHTS  UNDER  SECTION
 ELEVEN  HUNDRED  TWO OF THIS ARTICLE UPON THE REQUEST OF THE CONTROLLER,
 SUBJECT TO THE LIMITATIONS SET FORTH IN SECTION ELEVEN HUNDRED  FIVE  OF
 THIS ARTICLE;
   (F)  AT THE CONTROLLER'S DIRECTION, DELETE OR RETURN ALL PERSONAL DATA
 TO THE CONTROLLER AS REQUESTED AT THE END OF THE PROVISION OF  SERVICES,
 UNLESS RETENTION OF THE PERSONAL DATA IS REQUIRED BY LAW;
   (G)  UPON  THE REASONABLE REQUEST OF THE CONTROLLER, MAKE AVAILABLE TO
 THE CONTROLLER ALL DATA IN ITS POSSESSION NECESSARY TO  DEMONSTRATE  THE
 PROCESSOR'S COMPLIANCE WITH THE OBLIGATIONS IN THIS ACT;
   (H)  ALLOW, AND COOPERATE WITH, REASONABLE ASSESSMENTS BY THE CONTROL-
 LER OR THE CONTROLLER'S DESIGNATED ASSESSOR; ALTERNATIVELY, THE PROCESS-
 OR MAY ARRANGE FOR A QUALIFIED AND INDEPENDENT ASSESSOR  TO  CONDUCT  AN
 ASSESSMENT  OF THE PROCESSOR'S POLICIES AND TECHNICAL AND ORGANIZATIONAL
 MEASURES IN SUPPORT OF THE  OBLIGATIONS  UNDER  THIS  ARTICLE  USING  AN
 APPROPRIATE  AND  ACCEPTED  CONTROL STANDARD OR FRAMEWORK AND ASSESSMENT
 PROCEDURE FOR SUCH ASSESSMENTS. THE PROCESSOR SHALL PROVIDE A REPORT  OF
 SUCH ASSESSMENT TO THE CONTROLLER UPON REQUEST;
   (I) A REASONABLE TIME IN ADVANCE BEFORE DISCLOSING OR TRANSFERRING THE
 DATA TO ANY FURTHER PROCESSORS, NOTIFY THE CONTROLLER OF SUCH A PROPOSED
 DISCLOSURE  OR  TRANSFER  AND  PROVIDE  THE CONTROLLER AN OPPORTUNITY TO
 APPROVE OR REJECT THE PROPOSAL; AND
   (J) ENGAGE  ANY  FURTHER  PROCESSOR  PURSUANT  TO  A  WRITTEN,  SIGNED
 CONTRACT  THAT  INCLUDES  THE  CONTRACTUAL REQUIREMENTS PROVIDED IN THIS
 PARAGRAPH, CONTAINING AT MINIMUM THE SAME OBLIGATIONS THAT THE PROCESSOR
 HAS ENTERED INTO WITH REGARD TO THE DATA.
   (II) A CONTROLLER MUST NOT AGREE  TO  INDEMNIFY,  DEFEND,  OR  HOLD  A
 PROCESSOR  HARMLESS,  OR  AGREE  TO  A  PROVISION THAT HAS THE EFFECT OF
 INDEMNIFYING, DEFENDING, OR HOLDING THE PROCESSOR HARMLESS, FROM  CLAIMS
 OR  LIABILITY  ARISING  FROM  THE  PROCESSOR'S  BREACH  OF  THE CONTRACT
 REQUIRED BY CLAUSE (A) OF  SUBPARAGRAPH  (I)  OF  THIS  PARAGRAPH  OR  A
 A. 680--B                          16
 
 VIOLATION  OF THIS ACT. ANY PROVISION OF AN AGREEMENT THAT VIOLATES THIS
 SUBPARAGRAPH IS CONTRARY TO PUBLIC POLICY AND IS VOID AND UNENFORCEABLE.
   (III)  NOTHING  IN THIS PARAGRAPH RELIEVES A CONTROLLER OR A PROCESSOR
 FROM THE LIABILITIES IMPOSED ON IT BY VIRTUE OF ITS ROLE IN THE PROCESS-
 ING RELATIONSHIP AS DEFINED BY THIS ARTICLE.
   (IV) DETERMINING WHETHER A PERSON IS ACTING AS A CONTROLLER OR PROCES-
 SOR WITH RESPECT TO A SPECIFIC PROCESSING OF DATA IS A FACT-BASED DETER-
 MINATION THAT DEPENDS UPON THE CONTEXT IN WHICH PERSONAL DATA IS  TO  BE
 PROCESSED.  A  PROCESSOR  THAT  CONTINUES  TO  ADHERE  TO A CONTROLLER'S
 INSTRUCTIONS WITH RESPECT TO A  SPECIFIC  PROCESSING  OF  PERSONAL  DATA
 REMAINS A PROCESSOR.
   (F)  THIRD  PARTIES. (I) A CONTROLLER MUST NOT SHARE, DISCLOSE, TRANS-
 FER, OR SELL PERSONAL DATA, OR  FACILITATE  OR  ENABLE  THE  PROCESSING,
 DISCLOSURE,  TRANSFER,  OR  SALE  OF  PERSONAL DATA TO A THIRD PARTY FOR
 WHICH CONSENT OF THE CONSUMER PURSUANT TO  SUBDIVISION  TWO  OF  SECTION
 ELEVEN  HUNDRED  TWO  OF  THIS  ARTICLE, HAS NOT BEEN OBTAINED OR IS NOT
 CURRENTLY IN EFFECT. ANY REQUEST FOR CONSENT TO SHARE, DISCLOSE,  TRANS-
 FER,  OR  SELL PERSONAL DATA, OR TO FACILITATE OR ENABLE THE PROCESSING,
 DISCLOSURE, TRANSFER, OR SALE OF PERSONAL DATA TO  A  THIRD  PARTY  MUST
 CLEARLY  INCLUDE  THE  IDENTITY  OF  THE  THIRD PARTY AND THE PROCESSING
 PURPOSES FOR WHICH THE THIRD PARTY MAY USE THE PERSONAL DATA.
   (II) A CONTROLLER MUST NOT SHARE, DISCLOSE, TRANSFER, OR SELL PERSONAL
 DATA, OR FACILITATE OR ENABLE THE PROCESSING, DISCLOSURE,  TRANSFER,  OR
 SALE OF PERSONAL DATA IF IT CAN REASONABLY EXPECT THE PERSONAL DATA OF A
 CONSUMER  TO BE USED FOR PURPOSES THAT THE CONSUMER HAS NOT CONSENTED TO
 PURSUANT TO SUBDIVISION TWO OF SECTION ELEVEN HUNDRED TWO OF THIS  ARTI-
 CLE,  OR  IF  IT  CAN  REASONABLY EXPECT THAT ANY RIGHTS OF THE CONSUMER
 PROVIDED IN THIS ARTICLE WOULD BE COMPROMISED AS A RESULT OF SUCH TRANS-
 ACTION.
   (III) BEFORE MAKING ANY DISCLOSURE, TRANSFER, OR SALE OF PERSONAL DATA
 TO ANY THIRD PARTY, THE CONTROLLER MUST ENTER  INTO  A  WRITTEN,  SIGNED
 CONTRACT.  SUCH  CONTRACT  MUST  BE  BINDING  AND THE SCOPE, NATURE, AND
 PURPOSE OF PROCESSING, THE TYPE OF DATA SUBJECT TO PROCESSING, THE DURA-
 TION OF PROCESSING, AND THE RIGHTS  AND  OBLIGATIONS  OF  BOTH  PARTIES.
 SUCH CONTRACT MUST INCLUDE REQUIREMENTS THAT THE THIRD PARTY:
   (A)  PROCESS  THAT  DATA ONLY TO THE EXTENT PERMITTED BY THE AGREEMENT
 ENTERED INTO WITH THE CONTROLLER; AND
   (B) PROVIDE A MECHANISM TO COMPLY WITH ANY EXERCISES OF  A  CONSUMER'S
 RIGHTS UNDER SECTION ELEVEN HUNDRED TWO OF THIS ARTICLE UPON THE REQUEST
 OF  THE  CONTROLLER, SUBJECT TO ANY LIMITATIONS THEREON AS AUTHORIZED BY
 THIS ARTICLE; AND
   (C) TO THE EXTENT THE DISCLOSURE, TRANSFER, OR SALE  OF  THE  PERSONAL
 DATA  CAUSES  THE  THIRD  PARTY  TO BECOME A CONTROLLER, COMPLY WITH ALL
 OBLIGATIONS IMPOSED ON CONTROLLERS UNDER THIS ARTICLE.
   2. PROCESSOR RESPONSIBILITIES. (A)  FOR  ANY  PERSONAL  DATA  THAT  IS
 OBTAINED,  RECEIVED,  PURCHASED,  OR  OTHERWISE ACQUIRED BY A PROCESSOR,
 WHETHER DIRECTLY FROM A CONTROLLER OR INDIRECTLY FROM ANOTHER PROCESSOR,
 THE PROCESSOR MUST COMPLY WITH THE REQUIREMENTS SET FORTH IN CLAUSES (A)
 THROUGH (J) OF SUBPARAGRAPH (I) OF PARAGRAPH (E) OF SUBDIVISION  ONE  OF
 THIS SECTION.
   (B)  A  PROCESSOR  IS  NOT  REQUIRED  TO  COMPLY WITH A REQUEST BY THE
 CONSUMER SUBMITTED PURSUANT TO THIS ARTICLE BY A  CONSUMER  DIRECTLY  TO
 THE PROCESSOR TO THE EXTENT THAT THE PROCESSOR HAS PROCESSED THE CONSUM-
 ER'S PERSONAL DATA SOLELY IN ITS ROLE AS A PROCESSOR FOR A CONTROLLER.
   (C)  PROCESSORS  SHALL  BE  UNDER A CONTINUING OBLIGATION TO ENGAGE IN
 REASONABLE MEASURES TO REVIEW THEIR ACTIVITIES  FOR  CIRCUMSTANCES  THAT
 A. 680--B                          17
 
 MAY HAVE ALTERED THEIR ABILITY TO IDENTIFY A SPECIFIC NATURAL PERSON AND
 TO  UPDATE  THEIR  CLASSIFICATIONS OF DATA AS IDENTIFIED OR IDENTIFIABLE
 ACCORDINGLY.
   (D)  A  PROCESSOR  SHALL NOT ENGAGE IN ANY SALE OF PERSONAL DATA OTHER
 THAN ON BEHALF OF THE CONTROLLER PURSUANT TO ANY AGREEMENT ENTERED  INTO
 WITH THE CONTROLLER.
   3.  THIRD  PARTY  RESPONSIBILITIES.  (A) FOR ANY PERSONAL DATA THAT IS
 OBTAINED, RECEIVED, PURCHASED, OR OTHERWISE ACQUIRED OR  ACCESSED  BY  A
 THIRD PARTY FROM A CONTROLLER OR PROCESSOR, THE THIRD PARTY MUST:
   (I)  PROCESS  THAT DATA ONLY TO THE EXTENT PERMITTED BY ANY AGREEMENTS
 ENTERED INTO WITH THE CONTROLLER;
   (II) PROCESS ONLY THE PERSONAL DATA NECESSARY FOR PURPOSES  FOR  WHICH
 FREELY  GIVEN,  SPECIFIC, INFORMED, AND UNAMBIGUOUS OPT-IN CONSENT IS IN
 EFFECT, AS CONVEYED BY THE CONTROLLER, LIMIT THE USE  AND  RETENTION  OF
 THAT  DATA TO WHAT IS NECESSARY FOR SUCH PURPOSES, AND SHALL IMMEDIATELY
 DELETE SUCH PERSONAL DATA WHEN NOTIFIED THAT THE  CONSENT  IS  WITHHELD,
 DENIED, OR WITHDRAWN;
   (III)  COMPLY  WITH ANY EXERCISES OF A CONSUMER'S RIGHTS UNDER SECTION
 ELEVEN HUNDRED TWO OF THIS ARTICLE UPON THE REQUEST OF THE CONTROLLER OR
 PROCESSOR, SUBJECT TO ANY LIMITATIONS  THEREON  AS  AUTHORIZED  BY  THIS
 ARTICLE; AND
   (IV)  TO  THE EXTENT THE THIRD PARTY BECOMES A CONTROLLER FOR PERSONAL
 DATA, COMPLY WITH ALL OBLIGATIONS  IMPOSED  ON  CONTROLLERS  UNDER  THIS
 ARTICLE.
   4. EXCEPTIONS. THE REQUIREMENTS OF THIS SECTION SHALL NOT APPLY WHERE:
   (A) THE PROCESSING IS REQUIRED BY LAW;
   (B)  THE PROCESSING IS MADE PURSUANT TO A REQUEST BY A FEDERAL, STATE,
 OR LOCAL GOVERNMENT OR GOVERNMENT ENTITY; OR
   (C) THE PROCESSING SIGNIFICANTLY ADVANCES PROTECTION AGAINST  CRIMINAL
 OR TORTIOUS ACTIVITY.
   § 1104. DATA BROKERS. 1. A DATA BROKER, AS DEFINED UNDER THIS ARTICLE,
 MUST:
   (A)  ANNUALLY,  ON  OR BEFORE JANUARY THIRTY-FIRST FOLLOWING A YEAR IN
 WHICH A PERSON MEETS THE DEFINITION OF DATA BROKER IN THIS ARTICLE:
   (I) REGISTER WITH THE ATTORNEY GENERAL;
   (II) PAY A REGISTRATION FEE OF ONE HUNDRED  DOLLARS  OR  AS  OTHERWISE
 DETERMINED  BY THE ATTORNEY GENERAL PURSUANT TO THE REGULATORY AUTHORITY
 GRANTED TO THE ATTORNEY GENERAL UNDER THIS ARTICLE, NOT  TO  EXCEED  THE
 REASONABLE  COST OF ESTABLISHING AND MAINTAINING THE DATABASE AND INFOR-
 MATIONAL WEBSITE DESCRIBED IN THIS SECTION; AND
   (III) PROVIDE THE FOLLOWING INFORMATION:
   (A) THE NAME AND PRIMARY PHYSICAL, EMAIL, AND INTERNET WEBSITE ADDRESS
 OF THE DATA BROKER;
   (B) THE NAME AND BUSINESS ADDRESS OF AN OFFICER OR REGISTERED AGENT OF
 THE DATA BROKER AUTHORIZED TO ACCEPT LEGAL PROCESS ON BEHALF OF THE DATA
 BROKER;
   (C) A STATEMENT DESCRIBING THE METHOD FOR EXERCISING CONSUMERS  RIGHTS
 UNDER SECTION ELEVEN HUNDRED TWO OF THIS ARTICLE;
   (D) A STATEMENT WHETHER THE DATA BROKER IMPLEMENTS A PURCHASER CREDEN-
 TIALING PROCESS; AND
   (E)  ANY ADDITIONAL INFORMATION OR EXPLANATION THE DATA BROKER CHOOSES
 TO PROVIDE CONCERNING ITS DATA COLLECTION PRACTICES.
   2. NOTWITHSTANDING ANY OTHER PROVISION OF THIS ARTICLE, ANY CONTROLLER
 THAT CONDUCTS BUSINESS IN THE STATE OF NEW YORK MUST:
   (A) ANNUALLY, ON OR BEFORE JANUARY THIRTY-FIRST FOLLOWING  A  YEAR  IN
 WHICH  A  PERSON MEETS THE DEFINITION OF CONTROLLER IN THIS ACT, PROVIDE
 A. 680--B                          18

 TO THE ATTORNEY GENERAL A LIST OF ALL DATA BROKERS OR PERSONS REASONABLY
 BELIEVED TO BE DATA BROKERS TO WHICH THE  CONTROLLER  PROVIDED  PERSONAL
 DATA IN THE PRECEDING YEAR; AND
   (B)  NOT  SELL A CONSUMER'S PERSONAL DATA TO A DATA BROKER THAT IS NOT
 REGISTERED WITH THE ATTORNEY GENERAL.
   3. THE ATTORNEY GENERAL SHALL ESTABLISH, MANAGE AND MAINTAIN A  STATE-
 WIDE  REGISTRY  ON ITS INTERNET WEBSITE, WHICH SHALL LIST ALL REGISTERED
 DATA BROKERS AND MAKE ACCESSIBLE  TO  THE  PUBLIC  ALL  THE  INFORMATION
 PROVIDED  BY  DATA BROKERS PURSUANT TO THIS SECTION. PRINTED HARD COPIES
 OF SUCH REGISTRY SHALL BE MADE AVAILABLE UPON REQUEST AND PAYMENT  OF  A
 FEE TO BE DETERMINED BY THE ATTORNEY GENERAL.
   4. A DATA BROKER THAT FAILS TO REGISTER AS REQUIRED BY THIS SECTION OR
 SUBMITS  FALSE  INFORMATION  IN  ITS REGISTRATION IS, IN ADDITION TO ANY
 OTHER INJUNCTION, PENALTY, OR LIABILITY THAT MAY BE IMPOSED  UNDER  THIS
 ARTICLE,  LIABLE  FOR  CIVIL  PENALTIES,  FEES,  AND  COSTS IN AN ACTION
 BROUGHT BY THE ATTORNEY GENERAL AS FOLLOWS: (A) A CIVIL PENALTY  OF  ONE
 THOUSAND  DOLLARS  FOR  EACH  DAY  THE  DATA BROKER FAILS TO REGISTER AS
 REQUIRED BY THIS SECTION OR FAILS TO CORRECT FALSE INFORMATION,  (B)  AN
 AMOUNT  EQUAL  TO  THE FEES THAT WERE DUE DURING THE PERIOD IT FAILED TO
 REGISTER, AND (C) EXPENSES INCURRED  BY  THE  ATTORNEY  GENERAL  IN  THE
 INVESTIGATION AND PROSECUTION OF THE ACTION AS THE COURT DEEMS APPROPRI-
 ATE.
   §  1105. LIMITATIONS. 1. THIS ARTICLE DOES NOT REQUIRE A CONTROLLER OR
 PROCESSOR TO DO ANY OF THE FOLLOWING SOLELY FOR  PURPOSES  OF  COMPLYING
 WITH THIS ARTICLE:
   (A) REIDENTIFY DEIDENTIFIED DATA;
   (B)  COMPLY  WITH  A  VERIFIED CONSUMER REQUEST TO ACCESS, CORRECT, OR
 DELETE PERSONAL DATA PURSUANT TO THIS ARTICLE IF ALL  OF  THE  FOLLOWING
 ARE TRUE:
   (I)  THE  CONTROLLER  IS  NOT  REASONABLY  CAPABLE  OF ASSOCIATING THE
 REQUEST WITH THE PERSONAL DATA;
   (II) THE CONTROLLER DOES NOT ASSOCIATE THE PERSONAL  DATA  WITH  OTHER
 PERSONAL  DATA  ABOUT  THE  SAME SPECIFIC CONSUMER AS PART OF ITS NORMAL
 BUSINESS PRACTICE; AND
   (III) THE CONTROLLER DOES NOT SELL THE  PERSONAL  DATA  TO  ANY  THIRD
 PARTY OR OTHERWISE VOLUNTARILY DISCLOSE OR TRANSFER THE PERSONAL DATA TO
 ANY  PROCESSOR  OR  THIRD  PARTY,  EXCEPT AS OTHERWISE PERMITTED IN THIS
 ARTICLE; OR
   (C) MAINTAIN PERSONAL DATA IN IDENTIFIABLE FORM, OR  COLLECT,  OBTAIN,
 RETAIN,  OR ACCESS ANY PERSONAL DATA OR TECHNOLOGY, IN ORDER TO BE CAPA-
 BLE OF ASSOCIATING A VERIFIED CONSUMER REQUEST WITH PERSONAL DATA.
   2. THE OBLIGATIONS IMPOSED ON CONTROLLERS AND  PROCESSORS  UNDER  THIS
 ARTICLE  DO NOT RESTRICT A CONTROLLER'S OR PROCESSOR'S ABILITY TO DO ANY
 OF THE FOLLOWING, TO THE EXTENT THAT THE USE OF THE CONSUMER'S  PERSONAL
 DATA IS REASONABLY NECESSARY AND PROPORTIONATE FOR THESE PURPOSES:
   (A) COMPLY WITH FEDERAL, STATE, OR LOCAL LAWS, RULES, OR REGULATIONS;
   (B)  COMPLY  WITH  A  CIVIL, CRIMINAL, OR REGULATORY INQUIRY, INVESTI-
 GATION, SUBPOENA, OR SUMMONS BY FEDERAL, STATE, LOCAL, OR OTHER  GOVERN-
 MENTAL AUTHORITIES;
   (C)  COOPERATE  WITH  LAW  ENFORCEMENT  AGENCIES CONCERNING CONDUCT OR
 ACTIVITY THAT THE CONTROLLER OR PROCESSOR REASONABLY AND IN  GOOD  FAITH
 BELIEVES  MAY  VIOLATE  FEDERAL,  STATE,  OR LOCAL LAWS, RULES, OR REGU-
 LATIONS;
   (D) INVESTIGATE, ESTABLISH, EXERCISE, PREPARE  FOR,  OR  DEFEND  LEGAL
 CLAIMS;
 A. 680--B                          19
 
   (E)  PROCESS  PERSONAL DATA NECESSARY TO PROVIDE THE SERVICES OR GOODS
 REQUESTED BY A CONSUMER; PERFORM A CONTRACT TO WHICH THE CONSUMER  IS  A
 PARTY;  OR  TAKE  STEPS AT THE REQUEST OF THE CONSUMER PRIOR TO ENTERING
 INTO A CONTRACT;
   (F) TAKE IMMEDIATE STEPS TO PROTECT THE LIFE OR PHYSICAL SAFETY OF THE
 CONSUMER  OR  OF ANOTHER NATURAL PERSON, AND WHERE THE PROCESSING CANNOT
 BE MANIFESTLY BASED ON ANOTHER LEGAL BASIS;
   (G) PREVENT, DETECT, PROTECT AGAINST, OR  RESPOND  TO  SECURITY  INCI-
 DENTS,  IDENTITY THEFT, FRAUD, HARASSMENT, MALICIOUS OR DECEPTIVE ACTIV-
 ITIES, OR ANY ILLEGAL ACTIVITY; PRESERVE THE INTEGRITY  OR  SECURITY  OF
 SYSTEMS;  OR INVESTIGATE, REPORT, OR PROSECUTE THOSE RESPONSIBLE FOR ANY
 SUCH ACTION;
   (H) IDENTIFY AND REPAIR  TECHNICAL  ERRORS  THAT  IMPAIR  EXISTING  OR
 INTENDED FUNCTIONALITY; OR
   (I) PROCESS BUSINESS CONTACT INFORMATION, INCLUDING A NATURAL PERSON'S
 NAME,  POSITION  NAME  OR  TITLE,  BUSINESS  TELEPHONE  NUMBER, BUSINESS
 ADDRESS, BUSINESS ELECTRONIC MAIL ADDRESS, BUSINESS FAX NUMBER, OR QUAL-
 IFICATIONS AND ANY OTHER SIMILAR INFORMATION ABOUT THE NATURAL PERSON.
   3. THE OBLIGATIONS IMPOSED ON CONTROLLERS  OR  PROCESSORS  UNDER  THIS
 ARTICLE  DO  NOT  APPLY  WHERE COMPLIANCE BY THE CONTROLLER OR PROCESSOR
 WITH THIS ARTICLE WOULD VIOLATE AN EVIDENTIARY PRIVILEGE UNDER NEW  YORK
 LAW AND DO NOT PREVENT A CONTROLLER OR PROCESSOR FROM PROVIDING PERSONAL
 DATA  CONCERNING A CONSUMER TO A PERSON COVERED BY AN EVIDENTIARY PRIVI-
 LEGE UNDER NEW YORK LAW AS PART OF A PRIVILEGED COMMUNICATION.
   4. A CONTROLLER THAT RECEIVES A REQUEST PURSUANT TO SUBDIVISIONS THREE
 THROUGH SIX OF SECTION ELEVEN HUNDRED TWO OF THIS ARTICLE, OR A PROCESS-
 OR OR THIRD PARTY TO WHOM A CONTROLLER COMMUNICATES SUCH A REQUEST,  MAY
 DECLINE TO FULFILL THE RELEVANT PART OF SUCH REQUEST IF:
   (A)  THE CONTROLLER, PROCESSOR, OR THIRD PARTY IS UNABLE TO VERIFY THE
 REQUEST USING COMMERCIALLY REASONABLE EFFORTS, AS DESCRIBED IN PARAGRAPH
 (C) OF SUBDIVISION EIGHT OF SECTION ELEVEN HUNDRED TWO OF THIS ARTICLE;
   (B) COMPLYING WITH THE REQUEST WOULD BE DEMONSTRABLY  IMPOSSIBLE  (FOR
 PURPOSES  OF  THIS  PARAGRAPH, THE RECEIPT OF A LARGE NUMBER OF VERIFIED
 REQUESTS, ON ITS OWN, IS NOT SUFFICIENT  TO  RENDER  COMPLIANCE  WITH  A
 REQUEST DEMONSTRABLY IMPOSSIBLE);
   (C)  COMPLYING  WITH  THE  REQUEST WOULD IMPAIR THE PRIVACY OF ANOTHER
 INDIVIDUAL OR THE RIGHTS OF ANOTHER TO EXERCISE FREE SPEECH; OR
   (D) THE PERSONAL DATA WAS CREATED BY A NATURAL PERSON OTHER  THAN  THE
 CONSUMER  MAKING  THE  REQUEST AND IS BEING PROCESSED FOR THE PURPOSE OF
 FACILITATING INTERPERSONAL RELATIONSHIPS OR PUBLIC DISCUSSION.
   § 1106. ENFORCEMENT AND  PRIVATE  RIGHT  OF  ACTION.  1.  WHENEVER  IT
 APPEARS  TO  THE  ATTORNEY  GENERAL, EITHER UPON COMPLAINT OR OTHERWISE,
 THAT ANY PERSON OR PERSONS HAS ENGAGED IN OR IS ABOUT TO ENGAGE  IN  ANY
 OF  THE  ACTS OR PRACTICES STATED TO BE UNLAWFUL UNDER THIS ARTICLE, THE
 ATTORNEY GENERAL MAY BRING AN ACTION OR SPECIAL PROCEEDING IN  THE  NAME
 AND  ON  BEHALF  OF  THE  PEOPLE  OF THE STATE OF NEW YORK TO ENJOIN ANY
 VIOLATION OF THIS ARTICLE, TO OBTAIN RESTITUTION OF ANY MONEYS OR  PROP-
 ERTY  OBTAINED  DIRECTLY  OR INDIRECTLY BY ANY SUCH VIOLATION, TO OBTAIN
 DISGORGEMENT OF ANY PROFITS OBTAINED DIRECTLY OR INDIRECTLY BY ANY  SUCH
 VIOLATION,  TO  OBTAIN CIVIL PENALTIES OF NOT MORE THAN FIFTEEN THOUSAND
 DOLLARS PER VIOLATION, AND TO OBTAIN ANY SUCH OTHER AND  FURTHER  RELIEF
 AS THE COURT MAY DEEM PROPER, INCLUDING PRELIMINARY RELIEF.
   (A)  ANY  ACTION OR SPECIAL PROCEEDING BROUGHT BY THE ATTORNEY GENERAL
 PURSUANT TO THIS SECTION MUST BE COMMENCED WITHIN SIX YEARS.
   (B)  EACH  INSTANCE  OF  UNLAWFUL  PROCESSING  COUNTS  AS  A  SEPARATE
 VIOLATION.  UNLAWFUL  PROCESSING  OF  THE PERSONAL DATA OF MORE THAN ONE
 A. 680--B                          20
 
 CONSUMER COUNTS AS A  SEPARATE  VIOLATION  AS  TO  EACH  CONSUMER.  EACH
 PROVISION  OF  THIS  ARTICLE  THAT  IS  VIOLATED  COUNTS  AS  A SEPARATE
 VIOLATION.
   (C)  IN ASSESSING THE AMOUNT OF PENALTIES, THE COURT MUST CONSIDER ANY
 ONE OR MORE OF THE  RELEVANT  CIRCUMSTANCES  PRESENTED  BY  ANY  OF  THE
 PARTIES,  INCLUDING,  BUT  NOT LIMITED TO, THE NATURE AND SERIOUSNESS OF
 THE MISCONDUCT, THE NUMBER OF VIOLATIONS, THE PERSISTENCE OF THE MISCON-
 DUCT, THE LENGTH OF TIME OVER WHICH THE MISCONDUCT OCCURRED,  THE  WILL-
 FULNESS  OF  THE  VIOLATOR'S  MISCONDUCT,  AND  THE VIOLATOR'S FINANCIAL
 CONDITION.
   2. IN CONNECTION WITH ANY PROPOSED ACTION OR SPECIAL PROCEEDING  UNDER
 THIS  SECTION, THE ATTORNEY GENERAL IS AUTHORIZED TO TAKE PROOF AND MAKE
 A DETERMINATION OF THE RELEVANT FACTS, AND TO ISSUE SUBPOENAS IN ACCORD-
 ANCE WITH THE CIVIL PRACTICE LAW AND RULES.   THE ATTORNEY  GENERAL  MAY
 ALSO REQUIRE SUCH OTHER DATA AND INFORMATION AS HE OR SHE MAY DEEM RELE-
 VANT  AND  MAY  REQUIRE WRITTEN RESPONSES TO QUESTIONS UNDER OATH.  SUCH
 POWER OF SUBPOENA AND EXAMINATION SHALL NOT ABATE OR TERMINATE BY REASON
 OF ANY ACTION OR SPECIAL PROCEEDING  BROUGHT  BY  THE  ATTORNEY  GENERAL
 UNDER THIS ARTICLE.
   3.  ANY  PERSON, WITHIN OR OUTSIDE THE STATE, WHO THE ATTORNEY GENERAL
 BELIEVES MAY BE IN POSSESSION, CUSTODY, OR CONTROL OF ANY BOOKS, PAPERS,
 OR OTHER THINGS, OR MAY HAVE INFORMATION, RELEVANT TO ACTS OR  PRACTICES
 STATED  TO  BE  UNLAWFUL  IN THIS ARTICLE IS SUBJECT TO THE SERVICE OF A
 SUBPOENA ISSUED BY  THE  ATTORNEY  GENERAL  PURSUANT  TO  THIS  SECTION.
 SERVICE  MAY  BE  MADE IN ANY MANNER THAT IS AUTHORIZED FOR SERVICE OF A
 SUBPOENA OR A SUMMONS BY THE STATE IN WHICH SERVICE IS MADE.
   4. (A) FAILURE TO   COMPLY WITH A SUBPOENA  ISSUED  PURSUANT  TO  THIS
 SECTION  WITHOUT REASONABLE CAUSE TOLLS THE APPLICABLE STATUTES OF LIMI-
 TATIONS IN ANY ACTION OR SPECIAL  PROCEEDING  BROUGHT  BY  THE  ATTORNEY
 GENERAL  AGAINST THE NONCOMPLIANT PERSON THAT ARISES OUT OF THE ATTORNEY
 GENERAL'S INVESTIGATION.
   (B) IF A PERSON FAILS TO COMPLY WITH A  SUBPOENA  ISSUED  PURSUANT  TO
 THIS  SECTION,  THE  ATTORNEY  GENERAL  MAY MOVE IN THE SUPREME COURT TO
 COMPEL COMPLIANCE.  IF THE COURT FINDS THAT THE SUBPOENA WAS AUTHORIZED,
 IT SHALL ORDER COMPLIANCE AND MAY IMPOSE A CIVIL PENALTY OF UP  TO  FIVE
 HUNDRED DOLLARS PER DAY OF NONCOMPLIANCE.
   (C)  SUCH  TOLLING AND CIVIL PENALTY SHALL BE IN ADDITION TO ANY OTHER
 PENALTIES OR REMEDIES PROVIDED BY LAW FOR NONCOMPLIANCE WITH A SUBPOENA.
   5. THIS SECTION SHALL APPLY TO ALL ACTS DECLARED TO BE UNLAWFUL  UNDER
 THIS ARTICLE, WHETHER OR NOT SUBJECT TO ANY OTHER LAW OF THIS STATE, AND
 SHALL  NOT  SUPERSEDE, AMEND OR REPEAL ANY OTHER LAW OF THIS STATE UNDER
 WHICH THE ATTORNEY GENERAL IS AUTHORIZED TO TAKE ANY ACTION  OR  CONDUCT
 ANY INQUIRY.
   6.  ANY  CONSUMER  WHO  HAS BEEN INJURED BY A VIOLATION OF SUBDIVISION
 TWO, SEVEN OR EIGHT OF SECTION ELEVEN HUNDRED TWO OF  THIS  ARTICLE  MAY
 BRING  AN  ACTION  IN HIS OR HER OWN NAME TO ENJOIN SUCH UNLAWFUL ACT OR
 PRACTICE AND TO RECOVER HIS  OR  HER  ACTUAL  DAMAGES  OR  ONE  THOUSAND
 DOLLARS,  WHICHEVER  IS  GREATER.  THE  COURT  MAY ALSO AWARD REASONABLE
 ATTORNEYS' FEES TO A PREVAILING PLAINTIFF.   ACTIONS  PURSUANT  TO  THIS
 SECTION MAY BE BROUGHT ON A CLASS-WIDE BASIS.
   §  1107.  MISCELLANEOUS.  1.  PREEMPTION: THIS ARTICLE DOES NOT ANNUL,
 ALTER, OR AFFECT THE LAWS, ORDINANCES, REGULATIONS,  OR  THE  EQUIVALENT
 ADOPTED BY ANY LOCAL ENTITY REGARDING THE PROCESSING, COLLECTION, TRANS-
 FER, DISCLOSURE, AND SALE OF CONSUMERS' PERSONAL DATA BY A CONTROLLER OR
 PROCESSOR  SUBJECT  TO  THIS  ARTICLE,  EXCEPT TO THE EXTENT THOSE LAWS,
 ORDINANCES, REGULATIONS, OR THE EQUIVALENT CREATE REQUIREMENTS OR  OBLI-
 A. 680--B                          21
 
 GATIONS THAT CONFLICT WITH OR REDUCE THE PROTECTIONS AFFORDED TO CONSUM-
 ERS UNDER THIS ARTICLE.
   2. IMPACT REPORT: THE ATTORNEY GENERAL SHALL ISSUE A REPORT EVALUATING
 THIS  ARTICLE,  ITS SCOPE, ANY COMPLAINTS FROM CONSUMERS OR PERSONS, THE
 LIABILITY AND ENFORCEMENT PROVISIONS OF THIS ARTICLE INCLUDING, BUT  NOT
 LIMITED  TO,  THE  EFFECTIVENESS OF ITS EFFORTS TO ENFORCE THIS ARTICLE,
 AND ANY RECOMMENDATIONS FOR CHANGES TO  SUCH  PROVISIONS.  THE  ATTORNEY
 GENERAL SHALL SUBMIT THE REPORT TO THE GOVERNOR, THE TEMPORARY PRESIDENT
 OF  THE SENATE, THE SPEAKER OF THE ASSEMBLY, AND THE APPROPRIATE COMMIT-
 TEES OF THE LEGISLATURE WITHIN TWO YEARS OF THE EFFECTIVE DATE  OF  THIS
 SECTION.
   3. REGULATORY AUTHORITY: (A) THE ATTORNEY GENERAL IS HEREBY AUTHORIZED
 AND EMPOWERED TO ADOPT, PROMULGATE, AMEND AND RESCIND SUITABLE RULES AND
 REGULATIONS TO CARRY OUT THE PROVISIONS OF THIS ARTICLE, INCLUDING RULES
 GOVERNING  THE  FORM  AND  CONTENT  OF ANY DISCLOSURES OR COMMUNICATIONS
 REQUIRED BY THIS ARTICLE.
   (B) THE  ATTORNEY  GENERAL  MAY  REQUEST  DATA  AND  INFORMATION  FROM
 CONTROLLERS  CONDUCTING BUSINESS IN NEW YORK STATE, OTHER NEW YORK STATE
 GOVERNMENT ENTITIES ADMINISTERING NOTICE AND CONSENT  REGIMES,  CONSUMER
 PROTECTION  AND  PRIVACY  ADVOCATES  AND RESEARCHERS, INTERNET STANDARDS
 SETTING BODIES, SUCH AS  THE  INTERNET  ENGINEERING  TASKFORCE  AND  THE
 INSTITUTE  OF  ELECTRICAL  AND ELECTRONICS ENGINEERS, AND OTHER RELEVANT
 SOURCES, TO CONDUCT STUDIES TO INFORM SUITABLE  RULES  AND  REGULATIONS.
 THE  ATTORNEY  GENERAL  SHALL RECEIVE, UPON REQUEST, DATA FROM OTHER NEW
 YORK STATE GOVERNMENTAL ENTITIES.
   4. EXERCISE OF RIGHTS: ANY CONSUMER RIGHT SET FORTH  IN  THIS  ARTICLE
 MAY  BE  EXERCISED AT ANY TIME BY THE CONSUMER WHO IS THE SUBJECT OF THE
 DATA OR BY A PARENT OR GUARDIAN AUTHORIZED BY LAW  TO  TAKE  ACTIONS  OF
 LEGAL  CONSEQUENCE  ON  BEHALF OF THE CONSUMER WHO IS THE SUBJECT OF THE
 DATA. AN AGENT AUTHORIZED BY A CONSUMER MAY EXERCISE THE CONSUMER RIGHTS
 SET FORTH IN SUBDIVISIONS THREE THROUGH SIX OF  SECTION  ELEVEN  HUNDRED
 TWO OF THIS ACT ON THE CONSUMERS BEHALF.
   §  4.  This act shall take effect immediately; provided, however, that
 sections 1101, 1102, 1103, 1105, 1106 and 1107 of the  general  business
 law,  as added by section three of this act, shall take effect two years
 after it shall have become a law but the private right of action author-
 ized by subdivision 6 of section 1106 of the general business law  shall
 take effect three years after such section shall have become a law.