Senate File 2208 - Introduced
SENATE FILE 2208
BY NUNN
A BILL FOR
An Act relating to consumer data protection, making penalties1
applicable, and including effective date provisions.2
BE IT ENACTED BY THE GENERAL ASSEMBLY OF THE STATE OF IOWA:3
TLSB 5349XS (1) 89
es/rn
S.F. 2208
Section 1. NEW SECTION. 715D.1 Definitions.1
As used in this chapter, unless the context otherwise2
requires:3
1. “Aggregate data” means information that relates to a4
group or category of consumers, from which individual consumer5
identities have been removed, that is not linked or reasonably6
linkable to any consumer.7
2. “Authenticate” means verifying through reasonable means8
that a consumer, entitled to exercise their consumer rights in9
section 715D.3, is the same consumer exercising such consumer10
rights with respect to the personal data at issue.11
3. “Biometric data” means data generated by automatic12
measurements of an individual’s biological characteristics,13
such as a fingerprint, voiceprint, eye retinas, irises, or14
other unique biological patterns or characteristics that is15
used to identify a specific individual. “Biometric data”16
does not include a physical or digital photograph, a video or17
audio recording or data generated therefrom, or information18
collected, used, or stored for health care treatment, payment,19
or operations under HIPAA.20
4. “Child” means any natural person younger than thirteen21
years of age.22
5. “Consent” means a clear affirmative act signifying a23
consumer’s freely given, specific, informed, and unambiguous24
agreement to process personal data relating to the consumer.25
“Consent” may include a written statement, including a26
statement written by electronic means, or any other unambiguous27
affirmative action.28
6. “Controller” means the person that, alone or jointly with29
others, determines the purpose and means of processing personal30
data.31
7. “De-identified data” means data that cannot reasonably be32
linked to an identified or identifiable natural person.33
8. “Health Insurance Portability and Accountability34
Act” or “HIPAA” means the Health Insurance Portability and35
-1-
LSB 5349XS (1) 89
es/rn 1/16
S.F. 2208
Accountability Act of 1996, Pub. L. No. 104-191, including1
amendments thereto and regulations promulgated thereunder.2
9. “Precise geolocation data” means information derived from3
technology, including but not limited to global positioning4
system level latitude and longitude coordinates or other5
mechanisms, that identifies the specific location of a natural6
person with precision and accuracy within a radius of one7
thousand seven hundred fifty feet. “Precise geolocation8
data” does not include the content of communications or any9
data generated by or connected to advanced utility metering10
infrastructure systems or equipment for use by a utility.11
10. “Process” or “processing” means any operation or set12
of operations performed, whether by manual or automated means,13
on personal data or on sets of personal data, such as the14
collection, use, storage, disclosure, analysis, deletion, or15
modification of personal data.16
11. “Processor” means a person that processes personal data17
on behalf of a controller.18
12. “Profiling” means any form of solely automated19
processing performed on personal data to evaluate, analyze,20
or predict personal aspects related to an identified or21
identifiable natural person’s economic situation, health,22
personal preferences, interests, reliability, behavior,23
location, or movements.24
13. “Pseudonymous data” means personal data that cannot25
be attributed to a specific natural person without the use26
of additional information, provided that such additional27
information is kept separately and is subject to appropriate28
technical and organizational measures to ensure that29
the personal data is not attributed to an identified or30
identifiable natural person.31
14. “Sale of personal data” means the exchange of personal32
data for monetary consideration by the controller to a third33
party. “Sale of personal data” does not include:34
a. The disclosure of personal data to a processor that35
-2-
LSB 5349XS (1) 89
es/rn 2/16
S.F. 2208
processes the personal data on behalf of the controller.1
b. The disclosure of personal data to a third party for2
purposes of providing a product or service requested by the3
consumer or a parent of a child.4
c. The disclosure or transfer of personal data to an5
affiliate of the controller.6
d. The disclosure of information that the consumer7
intentionally made available to the general public via a8
channel of mass media and did not restrict to a specific9
audience.10
e. The disclosure or transfer of personal data to a third11
party as an asset that is part of a proposed or actual merger,12
acquisition, bankruptcy, or other transaction in which the13
third party assumes control of all or part of the controller's14
assets.15
15. “Sensitive data” means a category of personal data that16
includes:17
a. Personal data revealing racial or ethnic origin,18
religious beliefs, mental or physical health diagnosis, sexual19
orientation, or citizenship or immigration status.20
b. Genetic or biometric data that is processed for the21
purpose of uniquely identifying a natural person.22
c. The personal data collected from a child.23
d. Precise geolocation data.24
16. “Targeted advertising” means displaying advertisements25
to a consumer where the advertisement is selected based on26
personal data obtained from that consumer’s activities over27
time and across nonaffiliated websites or online applications28
to predict such consumer’s preferences or interests. “Targeted29
advertising” does not include:30
a. Advertisements based on activities within a controller’s31
own or affiliated websites or online applications.32
b. Advertisements based on the context of a consumer’s33
current search query, visit to a website, or online34
application.35
-3-
LSB 5349XS (1) 89
es/rn 3/16
S.F. 2208
c. Advertisements directed to a consumer in response to the1
consumer’s request for information or feedback.2
d. Processing personal data solely for measuring or3
reporting advertising performance, reach, or frequency.4
17. “Trade secret” means information, including but not5
limited to a formula, pattern, compilation, program, device,6
method, technique, or process, that:7
a. Derives independent economic value, actual or potential,8
from not being generally known to, and not being readily9
ascertainable by proper means by, other persons who can obtain10
economic value from its disclosure or use.11
b. Is the subject of efforts that are reasonable under the12
circumstances to maintain its secrecy.13
Sec. 2. NEW SECTION. 715D.2 Scope and exemptions.14
1. This chapter applies to persons conducting business in15
the state or producing products or services that are targeted16
to residents of the state and that during a calendar year17
either:18
a. Control or process personal data of at least one hundred19
thousand consumers.20
b. Control or process personal data of at least twenty-five21
thousand consumers and derive over fifty percent of gross22
revenue from the sale of personal data.23
2. This chapter shall not apply to the state or any24
political subdivision of the state, financial institutions25
or data subject to Tit. V of the federal Gramm-Leach-Bliley26
Act of 1999, 15 U.S.C. §6801 et seq., covered entities or27
business associates governed by the privacy, security, and28
breach notification rules issued by the department of human29
services, the department of health, 45 C.F.R. pts. 160 and 16430
established pursuant to HIPAA, nonprofit organizations, or31
institutions of higher education.32
3. Protected information and personal data collected33
under state or federal law, including but not limited to data34
protected under HIPAA; the federal Fair Credit Reporting Act,35
-4-
LSB 5349XS (1) 89
es/rn 4/16
S.F. 2208
15 U.S.C. §1681 et seq.; confidential records protected under1
42 U.S.C. §290dd-2; in the course of employment or application2
for employment; emergency contact information for employees;3
and for purposes of the protection of natural persons under 454
C.F.R. pt. 46; are exempt from requirements in this chapter.5
Sec. 3. NEW SECTION. 715D.3 Consumer data rights.6
1. A consumer may invoke the consumer rights authorized7
pursuant to this section at any time by submitting a request to8
a controller specifying the consumer rights the consumer wishes9
to invoke. A child’s parent or legal guardian may invoke such10
consumer rights on behalf of the child regarding processing11
personal data belonging to the child. A controller shall12
comply with an authenticated consumer request to exercise all13
of the following:14
a. To confirm whether a controller is processing the15
consumer’s personal data and to access such personal data.16
b. To correct inaccuracies in the consumer’s personal data,17
taking into account the nature of the personal data and the18
purposes of the processing of the consumer’s personal data.19
c. To delete personal data provided by or obtained about20
the consumer.21
d. To obtain a copy of the consumer’s personal data that the22
consumer previously provided to the controller in a portable23
and, to the extent technically practicable, readily usable24
format that allows the consumer to transmit the data to another25
controller without hindrance, where the processing is carried26
out by automated means.27
e. To opt out of the processing of the personal data for28
purposes of targeted advertising, the sale of personal data,29
or profiling in furtherance of decisions that produce legal or30
similarly significant effects concerning the consumer.31
2. Except as otherwise provided in this chapter, a32
controller shall comply with a request by a consumer to33
exercise the consumer rights authorized pursuant to this34
section as follows:35
-5-
LSB 5349XS (1) 89
es/rn 5/16
S.F. 2208
a. A controller shall respond to the consumer without undue1
delay, but in all cases within forty-five days of receipt2
of a request submitted pursuant to the methods described in3
this section. The response period may be extended once by4
forty-five additional days when reasonably necessary upon5
considering the complexity and number of the consumer’s6
requests by informing the consumer of any such extension within7
the initial forty-five-day response period, together with the8
reason for the extension.9
b. If a controller declines to take action regarding the10
consumer’s request, the controller shall inform the consumer11
without undue delay of the justification for declining to take12
action and instructions for how to appeal the decision pursuant13
to this section.14
c. Information provided in response to a consumer request15
shall be provided by a controller free of charge, up to16
twice annually per consumer. If requests from a consumer17
are manifestly unfounded, excessive, or repetitive, the18
controller may charge the consumer a reasonable fee to cover19
the administrative costs of complying with the request or20
decline to act on the request. The controller bears the burden21
of demonstrating the manifestly unfounded, excessive, or22
repetitive nature of the request.23
d. If a controller is unable to authenticate the request24
using commercially reasonable efforts, the controller shall25
not be required to comply with a request to initiate an action26
under this section and may request that the consumer provide27
additional information reasonably necessary to authenticate the28
consumer and the consumer’s request.29
3. A controller shall establish a process for a consumer30
to appeal the controller’s refusal to take action on a request31
within a reasonable period of time after the consumer’s32
receipt of the decision pursuant to this section. The appeal33
process shall be conspicuously available and similar to the34
process for submitting requests to initiate action pursuant to35
-6-
LSB 5349XS (1) 89
es/rn 6/16
S.F. 2208
this section. Within sixty days of receipt of an appeal, a1
controller shall inform the consumer in writing of any action2
taken or not taken in response to the appeal, including a3
written explanation of the reasons for the decisions. If4
the appeal is denied, the controller shall also provide the5
consumer with an online mechanism through which the consumer6
may contact the attorney general to submit a complaint.7
Sec. 4. NEW SECTION. 715D.4 Data controller duties.8
1. A controller shall limit the collection of personal data9
to what is reasonably necessary in relation to the purposes for10
which such data is processed and disclose the collection of the11
data to the consumer and obtain consent from the consumer for12
the data collection. A controller shall adopt and implement13
reasonable administrative, technical, and physical data14
security practices to protect the confidentiality, integrity,15
and accessibility of personal data. A controller shall not16
process sensitive data without the consumer’s consent.17
2. A controller shall not discriminate against a consumer18
for exercising any of the consumer rights contained in this19
chapter, including denying goods or services, charging20
different prices or rates for goods or services, or providing21
a different level of quality of goods and services to the22
consumer.23
3. Any provision of a contract or agreement that purports to24
waive or limit in any way consumer rights pursuant to section25
715E.3 shall be deemed contrary to public policy and shall be26
void and unenforceable.27
4. A controller shall provide consumers with a reasonably28
accessible, clear, and meaningful privacy notice that includes:29
a. The categories of personal data processed by the30
controller.31
b. The purpose for processing personal data.32
c. How consumers may exercise their consumer rights pursuant33
to section 715D.3, including how a consumer may appeal a34
controller’s decision with regard to the consumer’s request.35
-7-
LSB 5349XS (1) 89
es/rn 7/16
S.F. 2208
d. The categories of personal data that the controller1
shares with third parties, if any.2
e. The categories of third parties, if any, with whom the3
controller shares personal data.4
5. If a controller sells a consumer’s personal data to third5
parties or uses such personal data for targeted advertising,6
the controller shall clearly and conspicuously disclose such7
activity, as well as the manner in which a consumer may8
exercise the right to opt out of such sales or use.9
6. A controller shall establish, and shall describe in10
a privacy notice, secure and reliable means for consumers to11
submit a request to exercise their consumer rights under this12
chapter. Such means shall consider the need for secure and13
reliable communication of such requests and the ability of14
the controller to authenticate the identity of the consumer15
making the request. A controller shall not require a consumer16
to create a new account in order to exercise consumer rights17
pursuant to section 715D.3.18
Sec. 5. NEW SECTION. 715D.5 Processor duties.19
1. A processor shall assist a controller in duties required20
under this chapter.21
2. A contract between a controller and a processor shall22
govern the processor’s data processing procedures with respect23
to processing performed on behalf of the controller. The24
contract shall clearly set forth instructions for processing25
personal data, the nature and purpose of processing, the type26
of data subject to processing, the duration of processing, and27
the rights and duties of both parties. The contract shall also28
include requirements that the processor shall do all of the29
following:30
a. Ensure that each person processing personal data is31
subject to a duty of confidentiality with respect to the data.32
b. At the controller’s direction, delete or return all33
personal data to the controller as requested at the end of the34
provision of services, unless retention of the personal data35
-8-
LSB 5349XS (1) 89
es/rn 8/16
S.F. 2208
is required by law.1
c. Upon the reasonable request of the controller, make2
available to the controller all information in the processor’s3
possession necessary to demonstrate the processor’s compliance4
with the duties in this chapter.5
d. Cooperate with reasonable assessments by the controller,6
the controller’s designated assessor, or qualified and7
independent third-party assessor as chosen by the processor8
that will provide a report of such assessment to the controller9
upon request.10
e. Engage any subcontractor or agent pursuant to a written11
contract in accordance with this section that requires the12
subcontractor to meet the duties of the processor with respect13
to the personal data.14
Sec. 6. NEW SECTION. 715D.6 Data protection assessments.15
1. A controller shall conduct and document a data protection16
assessment regarding processing activities involving personal17
data, including but not limited to the sale of personal18
data, the use of personal data for targeted advertising, and19
processing that results in a reasonably foreseeable risk of20
unfair discrimination, injury, or intrusions to a consumer’s21
expectation of privacy.22
2. Data protection assessments conducted pursuant to23
subsection 1 shall identify and evaluate benefits and risks24
regarding data processing, the controller, the consumer,25
other stakeholders, and the public. Safeguards used by26
the controller and processor may be considered. The use27
of de-identified data and the reasonable expectations of28
consumers, as well as the context of the processing and the29
relationship between the controller and the consumer whose30
personal data will be processed, shall be factored into this31
assessment by the controller.32
3. The attorney general may request, pursuant to a consumer33
complaint, that a controller disclose relevant data protection34
assessment information during an investigation conducted by the35
-9-
LSB 5349XS (1) 89
es/rn 9/16
