PRINTER'S NO. 2617
THE GENERAL ASSEMBLY OF PENNSYLVANIA
HOUSE BILL
No. 2257 Session of
2022
INTRODUCED BY KENYATTA, SHUSTERMAN, GUENST, GALLOWAY, HILLEVANS, BENHAM, SCHLOSSBERG, KINSEY, SAMUELSON, FREEMAN,
SANCHEZ, HOWARD, ISAACSON, PARKER, MADDEN, O'MARA, NEILSON,
GUZMAN, CIRESI, ZABEL, McNEILL, D. WILLIAMS, FITZGERALD, LEE
AND DRISCOLL, JANUARY 20, 2022
REFERRED TO COMMITTEE ON CONSUMER AFFAIRS, JANUARY 20, 2022
AN ACT
Providing for protection of certain personal data of consumers;
imposing duties on controllers and processors of personal
data of consumers; providing for enforcement; prescribing
penalties; and establishing the Consumer Privacy Fund.
TABLE OF CONTENTS
Chapter 1. Preliminary Provisions
Section 101. Short title.
Section 102. Definitions.
Section 103. Applicability.
Chapter 3. Enumeration of Rights and Responsibilities
Section 301. Rights of consumers and controllers.
Section 302. Controller responsibilities.
Section 303. Responsibility of processors.
Section 304. Data protection assessments.
Section 305. Processing de-identified data and exemptions.
Section 306. Limitations.
Chapter 5. Administration and Enforcement
1
2
3
4
5
6
7
8
9
10
11
12
13
14
15
16
17
Section 501. Powers and duties of Attorney General.
Section 502. Enforcement procedure.
Section 503. Consumer Privacy Fund.
Chapter 7. Miscellaneous Provisions
Section 701. (Reserved).
Section 702. Effective date.
The General Assembly of the Commonwealth of Pennsylvania
hereby enacts as follows:
CHAPTER 1
PRELIMINARY PROVISIONS
Section 101. Short title.
This act shall be known and may be cited as the Consumer Data
Protection Act.
Section 102. Definitions.
The following words and phrases when used in this act shall
have the meanings given to them in this section unless the
context clearly indicates otherwise:
"Affiliate," "affiliate of" or "person affiliated with." A
person that directly or indirectly, through one or more
intermediaries, controls, is controlled by or is under common
control with a specified person. For the purposes of this
definition, "control" or "controlled" means:
(1) ownership of, or the power to vote, more than 50% of
the outstanding shares of any class of voting security of a
company;
(2) control in any manner over the election of a
majority of the directors or of individuals exercising
similar functions; or
(3) the power to exercise controlling influence over the
management of a company.
20220HB2257PN2617 - 2 -
1
2
3
4
5
6
7
8
9
10
11
12
13
14
15
16
17
18
19
20
21
22
23
24
25
26
27
28
29
30
"Authenticate." Verifying through reasonable means that a
consumer, entitled to exercise the consumer rights under this
act, is the same consumer exercising the consumer rights with
respect to the personal data at issue.
"Automated means." A computer program or an electronic or
other automated means used independently to initiate an action
or respond to electronic records or performances, in whole or in
part, without review or action by an individual.
"Biometric data." Data generated by automatic measurements
of an individual's biological characteristics, such as a
fingerprint, voiceprint, eye retinas, irises or other unique
biological patterns or characteristics that are used to identify
a specific individual. The term does not include a physical or
digital photograph, a video or audio recording or data generated
therefrom or information collected, used or stored for health
care treatment, payment or operations under HIPAA.
"Breach of the security of the system" or "breach." The
unauthorized access and acquisition of unencrypted data, or
encrypted data with the confidential process or key required to
decrypt the data, that is likely to compromise the security or
confidentiality of personal information maintained by the entity
as part of a database of personal information regarding multiple
individuals that causes or the entity reasonably believes has
caused or will cause loss or injury to any resident of this
Commonwealth. Good faith acquisition of personal information by
an employee or agent of the entity for the purposes of the
entity is not a breach of the security of the system if the
personal information is not used for a purpose other than the
lawful purpose of the entity and is not subject to further
authorized disclosure.
20220HB2257PN2617 - 3 -
1
2
3
4
5
6
7
8
9
10
11
12
13
14
15
16
17
18
19
20
21
22
23
24
25
26
27
28
29
30
"Business associate."
(1) Except as provided in paragraph (4), business
associate means, with respect to a covered entity, a person
who:
(i) on behalf of such covered entity or of an
organized health care arrangement in which the covered
entity participates, but other than in the capacity of a
member of the workforce of the covered entity or
arrangement, creates, receives, maintains or transmits
protected health information for a function or activity
regulated by this chapter, including claims processing or
administration, data analysis, processing or
administration, utilization review, quality assurance,
patient safety activities as defined in 42 CFR 3.20
(relating to definitions), billing, benefit management,
practice management and repricing; or
(ii) provides, other than in the capacity of a
member of the workforce of the covered entity, legal,
actuarial, accounting, consulting, data aggregation,
management, administrative, accreditation, or financial
services to or for such covered entity, or to or for an
organized health care arrangement in which the covered
entity participates, where the provision of the service
involves the disclosure of protected health information
from such covered entity or arrangement, or from another
business associate of such covered entity or arrangement,
to the person.
(2) A covered entity may be a business associate of
another covered entity.
(3) A person who is or does any of the following:
20220HB2257PN2617 - 4 -
1
2
3
4
5
6
7
8
9
10
11
12
13
14
15
16
17
18
19
20
21
22
23
24
25
26
27
28
29
30
(i) A Health Information Organization, E-prescribing
Gateway or other person that provides data transmission
services with respect to protected health information to
a covered entity and that requires access on a routine
basis to such protected health information.
(ii) Offers a personal health record to one or more
individuals on behalf of a covered entity.
(iii) A subcontractor that creates, receives,
maintains or transmits protected health information on
behalf of the business associate.
(4) The term does not include:
(i) A health care provider, with respect to
disclosures by a covered entity to the health care
provider concerning the treatment of the individual.
(ii) A plan sponsor, with respect to disclosures by
a group health plan (or by a health insurance issuer or
HMO with respect to a group health plan) to the plan
sponsor.
(iii) A government agency, with respect to
determining eligibility for, or enrollment in, a
government health plan that provides public benefits and
is administered by another government agency, or
collecting protected health information for such
purposes, to the extent the activities are authorized by
law.
(iv) A covered entity participating in an organized
health care arrangement that performs a function or
activity as described by paragraph (1)(i) for or on
behalf of such organized health care arrangement, or that
provides a service as described in paragraph (1)(ii) to
20220HB2257PN2617 - 5 -
1
2
3
4
5
6
7
8
9
10
11
12
13
14
15
16
17
18
19
20
21
22
23
24
25
26
27
28
29
30
or for the organized health care arrangement by virtue of
the activities or services.
"Child." An individual who is younger than 13 years of age.
"Consent." A clear affirmative act signifying a consumer's
freely given, specific, informed and unambiguous agreement to
process personal data relating to the consumer. The act may
include a written statement, including a statement written by
electronic means, or any other unambiguous affirmative action.
"Consumer." A natural person who is a resident of this
Commonwealth acting only in a personal or household context. The
term does not include a natural person who acts in a commercial
or employment context.
"Controller." An entity that, alone or jointly with others,
collects, uses, processes or stores personal information or
directs others to collect, use, process or store personal
information on its behalf.
"Covered entity." A covered entity means:
(1) A health plan.
(2) A health care clearinghouse.
(3) A health care provider that transmits health
information in electronic form in connection with a
transaction covered by this chapter.
"Data protection assessment." A process to identify and
minimize the data protection risks of a project by:
(1) Describing the nature, scope, context and purpose of
processing.
(2) Assessing necessity, proportionality and compliance
measures.
(3) Identifying and assessing risk to individuals.
(4) Identifying additional measures to mitigate those
20220HB2257PN2617 - 6 -
1
2
3
4
5
6
7
8
9
10
11
12
13
14
15
16
17
18
19
20
21
22
23
24
25
26
27
28
29
30
risks.
"Decision of the controller." A decision made by a
controller to provide or deny a consumer's request for
financial or lending services, housing, insurance, education
enrollment, criminal justice, an employment opportunity, health
care services or access to a basic necessity, such as food and
water.
"De-identified data." Data that cannot reasonably be linked
to an identified or identifiable individual or data on a device
linked to the individual.
"Entity." An individual or business conducting business or
other activities involving residents of this Commonwealth
whether or not physically located in this Commonwealth or a
Commonwealth agency or political subdivision of the
Commonwealth.
"Financial institution." Any regulated financial institution
insured by the Federal Deposit Insurance Corporation or its
successor or an affiliate of the financial institution.
"Fund." The Consumer Privacy Fund established under section
503.
"Health care practitioner." An individual who is authorized
to practice some component of the healing arts by a license,
permit, certificate or registration issued by a Commonwealth
licensing agency or board.
"Health care provider" or "provider." An individual, trust
or estate, partnership, corporation (including associations,
joint stock companies and insurance companies) or the
Commonwealth or a political subdivision or instrumentality,
including a municipal corporation or authority, thereof that
operates a health care facility.
20220HB2257PN2617 - 7 -
1
2
3
4
5
6
7
8
9
10
11
12
13
14
15
16
17
18
19
20
21
22
23
24
25
26
27
28
29
30
"Health record." A written, printed or electronically
recorded material maintained by a health care entity in the
course of providing health services to an individual concerning
the individual and the services provided. The term includes the
substance of a communication made by an individual to a health
care entity in confidence during or in connection with the
provision of health services or information otherwise acquired
by the health care entity about an individual in confidence and
in connection with the provision of health services to the
individual.
"HIPAA." The Health Insurance Portability and Accountability
Act of 1996 (Public Law 104-191, 110 Stat. 1936).
"Identifiable private information." Any of the following:
(1) An individual's first name or first initial and last
name in combination with and linked to one or more of the
following data elements when the elements are not encrypted
or redacted:
(i) Social Security number;
(ii) driver's license number;
(iii) State identification card number issued in
lieu of a driver's license;
(iv) passport number;
(v) taxpayer identification number;
(vi) medical information;
(vii) health insurance information;
(viii) biometric data; or
(ix) a financial account number or a credit or debit
card number in combination with other information that
allows a financial, credit or debit account to be used or
accessed.
20220HB2257PN2617 - 8 -
1
2
3
4
5
6
7
8
9
10
11
12
13
14
15
16
17
18
19
20
21
22
23
24
25
26
27
28
29
30
(2) A data element enumerated in paragraph (1) if the
information would reasonably permit the fraudulent assumption
of the identity of an individual.
(3) An individual's username or e-mail address in
combination with a password or security question and answer,
biometric information or other information that would permit
access to an online account.
(4) The term does not include information that an
individual has made public himself or herself, information
that the individual has consented in writing to be made
public or information that was lawfully made public under
Federal or State law or court order.
"Identified or identifiable natural person." An individual
who can be readily identified, directly or indirectly.
"Institution of higher education." The term includes the
following:
(1) A community college operating under Article XIX-A of
the act of March 10, 1949 (P.L.30, No.14), known as the
Public School Code of 1949.
(2) A university within the State System of Higher
Education.
(3) The Pennsylvania State University.
(4) The University of Pittsburgh.
(5) Temple University.
(6) Lincoln University.
(7) Another institution that is designated as "Staterelated" by the Commonwealth.
(8) An accredited private or independent college or
university.
(9) A private licensed school as defined in the act of
20220HB2257PN2617 - 9 -
1
2
3
4
5
6
7
8
9
10
11
12
13
14
15
16
17
18
19
20
21
22
23
24
25
26
27
28
29
30
December 15, 1986 (P.L.1585, No.174), known as the Private
Licensed Schools Act.
"International Council for Harmonisation of Technical
Requirements for Pharmaceuticals for Human Use" or "(ICH)." An
initiative that brings together regulatory authorities and the
pharmaceutical industry to discuss scientific and technical
aspects of pharmaceutical product development and registration.
"Minor." An individual who is under 18 years of age.
"Nonprofit organization." An organization exempt from
taxation under section 501(c)(3), (6) or (12) of the Internal
Revenue Code of 1986 (Public Law 99-514, 26 U.S.C. § 501(c)(3),
(6) or (12)).
"Person." An individual.
"Personal data" or "consumer personal data." Information
that is linked or reasonably linkable to an identified or
identifiable natural person. The term does not include deidentified data or publicly available information.
"Precise geolocation data." Information derived from
technology, including global positioning system level latitude
and longitude coordinates or other mechanisms, that directly
identifies the specific location of an individual with precision
and accuracy within a radius of 1,750 feet. The term does not
include the content of communications or data generated by or
connected to advanced utility metering infrastructure systems or
equipment for use by a public utility.
"Process" or "processing." An operation or set of operations
performed, whether by manual or automated means, on personal
data or on sets of personal data, such as the collection, use,
storage, disclosure, analysis, deletion or modification of
personal data.
20220HB2257PN2617 - 10 -
1
2
3
4
5
6
7
8
9
10
11
12
13
14
15
16
17
18
19
20
21
22
23
24
25
26
27
28
29
30
"Processor." A person that processes personal data on behalf
of a controller.
"Profiling." A form of automated processing performed on
personal data to evaluate, analyze or predict personal aspects
related to an identified or identifiable natural person's
economic situation, health, personal preferences, interests,
reliability, behavior, location or movements.
"Protected health information." As defined in 45 CFR 160.103
(relating to definitions).
"Pseudonymous data." Personal data that cannot be attributed
to a specific natural person without the use of additional
information, provided that the additional information is kept
separately and is subject to appropriate technical and
organizational measures to ensure that the personal data is not
attributed to an identified or identifiable natural person.
"Publicly available information." Information that:
(1) an individual has made public himself or herself;
(2) an individual has consented in writing to be made
public;
(3) was lawfully made public under Federal or State law
or court order; or
(4) is from another publicly available source, including
news reports, periodicals, public social media posts or other
widely distributed media.
"Qualified service organization." An entity that provides
services such as data processing, bill collecting, dosage
preparation, laboratory analysis or legal, accounting,
population health management, medical staffing or other
professional services or services to prevent or treat child
abuse or neglect, including training on nutrition and child care
20220HB2257PN2617 - 11 -
1
2
3
4
5
6
7
8
9
10
11
12
13
14
15
16
17
18
19
20
21
22
23
24
25
26
27
28
29
30
and individual and group therapy.
"Sale of personal data." The exchange of personal data for
monetary consideration by a controller to a third party. The
term does not include any of the following:
(1) The disclosure of personal data to a processor that
processes the personal data on behalf of a controller.
(2) The disclosure of personal data to a third party for
purposes of providing a product or service requested by a
consumer.
(3) The disclosure or transfer of personal data to an
affiliate of a controller.
(4) The disclosure of information that a consumer:
(i) intentionally made available to the general
public through publicly available sources, including news
reports, periodicals, public social media posts or other
widely distributed media; and
(ii) did not restrict disclosure to a specific
audience.
(5) The disclosure or transfer of personal data to a
third party as an asset that is part of a merger,
acquisition, bankruptcy or other transaction in which the
third party assumes control of all or part of the
controller's assets.
"Sensitive data." A category of personal data that includes
any of the following:
(1) personal data revealing racial or ethnic origin,
religious beliefs, mental behavioral or physical health
diagnosis, sexual orientation, gender or gender identity,
citizenship or immigration status;
(2) the processing of genetic or biometric data for the
20220HB2257PN2617 - 12 -
1
2
3
4
5
6
7
8
9
10
11
12
13
14
15
16
17
18
19
20
21
22
23
24
25
26
27
28
29
30
purpose of uniquely identifying a natural person;
(3) the personal data collected from a minor; or
(4) precise geolocation data.
"Targeted advertising." Displaying advertisements to a
consumer where the advertisement is selected based on personal
data obtained from the consumer's online activities over time
and across nonaffiliated websites or online applications to
predict the consumer's preferences or interests. The term does
not include any of the following:
(1) Advertisements based on activities within a
controller's own websites or online applications.
(2) Advertisements based on the context of a consumer's
current search query, visit to a website or online
application.
(3) Advertisements directed to a consumer in response to
the consumer's request for information or feedback.
(4) Processing personal data processed solely for
measuring or reporting advertising performance, reach or
frequency.
"Third party." A person, other than a consumer, controller
or processor or an affiliate of a processor or controller. The
term shall include an agency of the Federal Government, a
Commonwealth agency or a local agency.
"Third party controller or processor." A person or entity
acting on behalf of a controller or processor.
Section 103. Applicability.
(a) General rule.--This act applies to persons that conduct
business in this Commonwealth or produce goods, products or
services that are sold or offered for sale to residents of this
Commonwealth and that:
20220HB2257PN2617 - 13 -
1
2
3
4
5
6
7
8
9
10
11
12
13
14
15
16
17
18
19
20
21
22
23
24
25
26
27
28
29
30
(1) during a calendar year, control or process personal
data of at least 100,000 consumers; or
(2) control or process personal data of at least 25,000
consumers and derive over 50% of gross revenue from the sale
of personal data.
(b) Nonapplicability.--This act shall not apply to any of
the following:
(1) The Commonwealth or a political subdivision of the
Commonwealth or an agency, office, authority, board, bureau
or commission of the Commonwealth or a political subdivision.
(2) A financial institution or data subject to Title V
of the Gramm-Leach-Bliley Act (Public Law 106-102, 113 Stat.
1338).
(3) A covered entity or business associate of a covered
entity governed by the privacy, security and breach
notification rules issued by the Department of Health and
Human Services under 45 CFR Pts. 160 (relating to general
administrative requirements) and 164 (relating to security
and privacy) established under HIPAA, and Title XIII of the
American Recovery and Reinvestment Act of 2009 (Public Law
111-5, 123 Stat. 115).
(4) A nonprofit organization.
(5) An institution of higher education.
(c) Exempt information and data.--The following information
and data is exempt from this act:
(1) Protected health information under HIPAA.
(2) Health records as defined by and for lawful purposes
under State law.
(3) Patient identifying information for purposes of
section 522 of the Public Health Service Act (58 Stat. 682,
20220HB2257PN2617 - 14 -
1
2
3
4
5
6
7
8
9
10
11
12
13
14
15
16
17
18
19
20
21
22
23
24
25
26
27
28
29
30
42 U.S.C. § 290dd-2).
(4) Identifiable private information for purposes of the
Federal policy for the protection of human subjects under 45
CFR Pt. 46 (relating to protection of human subjects),
identifiable private information that is otherwise
information collected as part of human subjects research
pursuant to the good clinical practice guidelines issued by
The International Council for Harmonisation of Technical
Requirements for Pharmaceuticals for Human Use or the
protection of human subjects under 21 CFR Pts. 50 (relating
to protection of human subjects) and 56 (relating to
institutional review boards) or personal data used or shared
in research conducted in accordance with the requirements
specified in this act or other research conducted in
accordance with applicable law.
(5) Information and documents created for purposes of
the Health Care Quality Improvement Act of 1986 (Public Law
99-660, 42 U.S.C. § 11101 et seq.).
(6) Patient safety work product for purposes of the
Patient Safety and Quality Improvement Act of 2005 (Public
Law 109-41, 42 U.S.C. § 299 et seq.).
(7) Information derived from any of the health carerelated information that is de-identified in accordance with
the requirements for de-identification under HIPAA.
(8) Information originating from, and intermingled to be
indistinguishable with, or information treated in the same
manner as information exempt under this subsection that is
maintained by a covered entity or business associate of a
covered entity as defined by HIPAA or a program or a
qualified service organization as defined by section 522 of
20220HB2257PN2617 - 15 -
1
2
3
4
5
6
7
8
9
10
11
12
13
14
15
16
17
18
19
20
21
22
23
24
25
26
27
28
29
30
the Public Health Services Act.
(9) Information used only for public health activities
and purposes as authorized by HIPAA.
(10) The collection, maintenance, disclosure, sale,
communication or use of personal information bearing on a
consumer's credit worthiness, credit standing, credit
capacity, character, general reputation, personal
characteristics or mode of living by a consumer reporting
agency, business or public utility that provides information
for use in a consumer report, and by a user of a consumer
report, but only to the extent that the activity is regulated
by and authorized under the Fair Credit Reporting Act (Public
Law 91-508, 15 U.S.C. § 1681 et seq.).
(11) Data collected, processed, sold or disclosed in
compliance with 18 U.S.C. § 2721 (relating to prohibition on
release and use of certain personal information from State
motor vehicle records).
(12) Personal data regulated by the Family Educational
Rights and Privacy Act of 1974 (Public Law 90-247, 20 U.S.C.
§ 1232g).
(13) Personal data collected, processed, sold or
disclosed in compliance with the Farm Credit Act of 1971
(Public Law 92-181, 12 U.S.C. § 2001 et seq.).
(14) Data processed or maintained:
(i) to the extent that data is collected and used in
the course of employment with, or the performance of, a
contract for a controller, processor or third party;
(ii) as the emergency contact information of an
individual under this act used for emergency contact
purposes; or
20220HB2257PN2617 - 16 -
1
2
3
4
5
6
7
8
9
10
11
12
13
14
15
16
17
18
19
20
21
22
23
24
25
26
27
28
29
30
(iii) as necessary to retain or administer benefits
for another individual relating to the individual under
subparagraph (i) and used for the purposes of
administering those benefits.
(d) Compliance under other Federal law.--A controller or
processor that complies with the verifiable parental consent
requirements of the Children's Online Privacy Protection Act
(Public Law 105-277, 15 U.S.C. § 6501 et seq.) shall be deemed
compliant with any obligation to obtain parental consent under
this act.
CHAPTER 3
ENUMERATION OF RIGHTS AND RESPONSIBILITIES
Section 301. Rights of consumers and controllers.
(a) Consumer rights.--A consumer may invoke the consumer
rights authorized under this subsection at any time by
submitting a request to a controller specifying the consumer
rights the consumer wishes to invoke. A known child's parent or
legal guardian may invoke the consumer rights on behalf of the
child regarding processing personal data belonging to the known
child. The consumer may invoke the right:
(1) To confirm whether or not the controller is
processing the consumer's personal data and to access the
personal data.
(2) To correct inaccuracies in the consumer's personal
data.
(3) To delete personal data provided by the consumer or
obtained by the controller about the consumer.
(4) To obtain a copy of the consumer's personal data
that the consumer previously provided to the controller in a
portable and, to the extent technically feasible, readily
20220HB2257PN2617 - 17 -
1
2
3
4
5
6
7
8
9
10
11
12
13
14
15
16
17
18
19
20
21
22
23
24
25
26
27
28
29
30
usable format that allows the consumer to transmit the data
to another controller without hindrance, where the processing
is carried out by automated means.
(5) To opt out of the processing of the personal data
for purposes of:
(i) targeted advertising;
(ii) the sale of personal data; or
(iii) profiling in furtherance of decisions that
produce legal or similarly significant effects concerning
the consumer.
(b) Controller duties.--Except as otherwise provided in this
act, a controller shall comply with a request by a consumer to
exercise the consumer rights authorized under subsection (a) as
follows:
(1) The controller shall respond to the consumer within
45 days of receipt of a request submitted under subsection
(a). The response period may be extended once by 45
additional days when reasonably necessary, taking into
account the complexity and number of the consumer's requests,
so long as the controller informs the consumer of the
extension within the initial 45-day response period, together
with the reason for the extension.
(2) If the controller declines to take action regarding
a consumer's request, the controller shall:
(i) inform the consumer within 45 days of receipt of
the request of the justification for declining to take
action; and
(ii) provide the consumer with instructions on how
to appeal the decision under subsection (c).
(3) (i) Information provided in response to a
20220HB2257PN2617 - 18 -
1
2
3
4
5
6
7
8
9
10
11
12
13
14
15
16
17
18
19
20
21
22
23
24
25
26
27
28
29
30
consumer's request to invoke consumer rights shall,
except as provided in subparagraph (ii), be provided by
the controller free of charge and up to twice annually
per consumer.
(ii) If a request from a consumer is determined by
the comptroller to be unfounded, excessive or repetitive,
the controller may charge the consumer a reasonable fee
to cover the administrative costs of complying with the
request or decline to act on the request. The controller
shall bear the burden of demonstrating that a consumer's
request under subsection (a) is unfounded, excessive or
repetitive.
(4) If the controller is unable to authenticate the
request using reasonable efforts, the controller may not be
required to comply with a request to initiate an action under
subsection (a). The controller may request that the consumer
provide additional information reasonably necessary to
authenticate the consumer and the consumer's request.
(c) Appeal process.--
(1) A controller shall establish a process for a
consumer to appeal the controller's refusal to take action on
a request within a reasonable period of time after the
consumer's receipt of the decision.
(2) The appeal process shall be stated in plain
language. The controller shall respond to the consumer within
45 days of receipt of the appeal, notifying the consumer that
the appeal has been received.
(3) Within 60 days of receipt of an appeal, the
controller shall inform the consumer in writing of any action
taken or not taken in response to the appeal, including a
20220HB2257PN2617 - 19 -
1
2
3
4
5
6
7
8
9
10
11
12
13
14
15
16
17
18
19
20
21
22
23
24
25
26
27
28
29
30
written explanation of the reasons for the decisions.
(4) If the appeal is denied, the controller shall
provide the consumer with an online form that the consumer
may use to contact the Attorney General to submit a
complaint. Information about the form shall be published on a
publicly accessible Internet website.
Section 302. Controller responsibilities.
(a) General rule.--A controller shall:
(1) Limit the collection of personal data to what is
necessary in relation to the purposes for which the data is
collected, processed and maintained by the controller, as
disclosed to the consumer.
(2) Except as otherwise provided in this act, not
collect and process personal data for purposes that are
neither reasonably necessary to nor compatible with the
disclosed purposes for which the personal data is collected,
processed and maintained, as disclosed to the consumer,
unless the controller obtains the consumer's prior consent.
(3) Establish, implement and maintain reasonable
administrative, technical data security practices to protect
the confidentiality, integrity and accessibility of personal
data. The data security practices shall be appropriate to the
volume and nature of all consumer personal data collected,
processed and maintained by the controller.
(4) (i) Not process personal data in violation of
Federal and State laws that prohibit unlawful
discrimination against consumers, including the act of
December 17, 1968 (P.L.1224, No.387), known as the Unfair
Trade Practices and Consumer Protection Law. A controller
shall not discriminate against a consumer by:
20220HB2257PN2617 - 20 -
1
2
3
4
5
6
7
8
9
10
11
12
13
14
15
16
17
18
19
20
21
22
23
24
25
26
27
28
29
30
(A) exercising a consumer right under section
301(a);
(B) denying goods, products or services;
(C) charging different prices or rates for
goods, products or services; or
(D) providing a different level of quality of
goods and services to the consumer.
(ii) Nothing in this paragraph shall be construed to
require a controller to:
(A) provide a good, product or service that
requires the personal data of a consumer that the
controller does not collect or maintain in the normal
course of business or otherwise; or
(B) prohibit a controller from offering a
different price, rate, level, quality or selection of
goods, products or services to a consumer, including
offering goods, products or services for no fee, if
the consumer has exercised the right to opt out under
this act or the offer is related to a consumer's
voluntary participation in a bona fide loyalty,
promotional, rewards, premium features, discounts or
club card program or any other similar program.
(5) Not process sensitive data concerning a consumer
without obtaining the consumer's written consent or, in the
case of the processing of sensitive data concerning a known
child, without processing the data in accordance with the
Children's Online Privacy Protection Act (Public Law 105-277,
15 U.S.C. § 6501 et seq.).
(b) Void contract provisions.--A provision of a contract or
agreement that purports to waive or limit a consumer right under
20220HB2257PN2617 - 21 -
1
2
3
4
5
6
7
8
9
10
11
12
13
14
15
16
17
18
19
20
21
22
23
24
25
26
27
28
29
30
this act shall be deemed contrary to the intent and policy
purposes of this act and shall be void and unenforceable.
(c) Consumer notice from controller.--
(1) A controller shall provide a consumer with an
accessible, clear and meaningful privacy notice that
includes:
(i) The categories of personal data collected,
processed and maintained by the controller.
(ii) The purpose for processing the consumer's
personal data.
(iii) How the consumer may exercise the consumer's
rights under section 301, including how the consumer may
appeal the controller's decision with regard to a
consumer's request under section 301(a).
(iv) The categories of personal data that the
controller shares with third parties, if any.
(v) The categories of third parties, if any, with
whom the controller shares personal data.
(2) The privacy notice shall be provided to the consumer
by United States Postal Service mail, annually, and shall be
accessible, electronically on the controller's publicly
accessible Internet website.
(d) Disclosure of sale and advertising processes.--If a
controller sells consumer personal data to third parties or
processes consumer personal data for targeted advertising, the
controller shall clearly and conspicuously disclose the sale or
processing, to the affected consumers, as well as the manner in
which a consumer may opt out of the sale and processing of the
consumer's personal data under this subsection.
(e) Privacy notice.--
20220HB2257PN2617 - 22 -
1
2
3
4
5
6
7
8
9
10
11
12
13
14
15
16
17
18
19
20
21
22
23
24
25
26
27
28
29
30
(1) A controller shall establish and describe in a
privacy notice the reliable procedures a consumer may use to
submit a request to exercise the consumer rights under this
act. The procedures shall take into account:
(i) the ways in which a consumer normally
communicates or interacts with the controller;
(ii) the need for secure and reliable communication
of the request; and
(iii) the method the controller will use to
authenticate the identity of the consumer making a
request.
(2) The controller shall not require a consumer with an
existing account to create a new account in order to exercise
a consumer right under this act.
Section 303. Responsibility of processors.
(a) Processors.--A processor shall adhere to the
instructions of a controller and shall assist the controller in
meeting its obligations under this act. The assistance shall
include:
(1) Technical and organizational measures that take into
account the nature of processing consumer personal data and
the information available to the processor, as reasonably
practicable, to fulfill the controller's obligation to
respond to consumer rights requests under section 301.
(2) Security measures that take into account the nature
of processing consumer personal data and the information
available to the processor, in order to assist the controller
in meeting the controller's obligations in relation to the
security of processing consumer personal data and in relation
to the notification of a breach of the security of the system
20220HB2257PN2617 - 23 -
1
2
3
4
5
6
7
8
9
10
11
12
13
14
15
16
17
18
19
20
21
22
23
24
25
26
27
28
29
30
of the processor.
(3) Providing necessary information to enable the
controller to conduct and document data protection
assessments.
(b) Contract between controllers and processors.--
(1) A contract between a controller and a processor
shall include provisions to govern the processor's data
processing procedures with respect to the processing of
consumer personal data performed by a processor on behalf of
a controller.
(2) A contract under this subsection shall:
(i) be binding;
(ii) clearly state instructions for processing data,
including the nature and purpose of processing, and the
type of data subject to processing;
(iii) indicate the duration of processing; and
(iv) specify the rights and obligations of both the
controller and the processor.
(3) The contract shall also include requirements that
the processor shall:
(i) Ensure that a person processing consumer
personal data is informed of and subject to
confidentiality requirements under Federal laws and
regulations and the laws and regulations of this
Commonwealth with respect to the data.
(ii) At the controller's direction, delete and
return all consumer personal data to the controller as
requested at the end of the contract, unless retention of
the personal data is required by law.
(iii) Upon the request of the controller, make
20220HB2257PN2617 - 24 -
1
2
3
4
5
6
7
8
9
10
11
12
13
14
15
16
17
18
19
20
21
22
23
24
25
26
27
28
29
30
available to the controller all information in the
processor's possession necessary to demonstrate the
processor's compliance with the processor's obligations
under this act.
(iv) Allow, and cooperate with, audits by the
controller or the controller's designated assessor or,
alternatively, allow the processor to arrange for a
qualified and independent assessor to conduct an
assessment of the processor's policies and technical and
organizational measures in support of the obligations
under this act using an appropriate and accepted control
standard or framework and assessment procedure for the
assessment. The processor shall provide a report of the
assessment to the controller upon request.
(4) In order to meet a processor's obligations to a
controller, a processor may contract with a subcontractor to
process consumer personal data in accordance with the
requirements of this act. A contract entered into under this
paragraph shall include provisions informing the
subcontractor of the confidentiality requirements under
Federal laws and regulations and State laws and regulations
and making the subcontractor subject to the confidentiality
requirements.
(5) A subcontractor under paragraph (4) shall be subject
to all the requirements that relate to the obligations of a
processor under this act.
(c) Construction.--Nothing in this section shall be
construed to relieve a controller or a processor and a
contractor or subcontractor under subsection (b) from the
liabilities imposed on such controller, processor, contractor or
20220HB2257PN2617 - 25 -
1
2
3
4
5
6
7
8
9
10
11
12
13
14
15
16
17
18
19
20
21
22
23
24
25
26
27
28
29
30
subcontractor by virtue of their roles in the processing of
consumer personal data under this act.
Section 304. Data protection assessments.
(a) Duty of controller.--A controller shall conduct and
document a data protection assessment of each of the following
processing activities involving personal data:
(1) The processing of personal data for purposes of
targeted advertising.
(2) The sale of personal data.
(3) The processing of personal data for purposes of
profiling, where the profiling presents a reasonably
foreseeable risk of:
(i) discriminatory, unfair or deceptive treatment
of, or unlawful disparate impact on, consumers;
(ii) financial, physical or reputational injury to
consumers;
(iii) a physical or other intrusion upon the
solitude or seclusion, or the private affairs or
concerns, of consumers, where the intrusion would be
offensive to a reasonable person; or
(iv) other substantial injury to consumers.
(4) The processing of sensitive data.
(5) Any processing activity involving personal data that
presents a heightened risk of harm to consumers.
(b) Identification and weighing of benefits.--
(1) Data protection assessments conducted under
subsection (a) shall identify and weigh the benefits that may
flow, directly and indirectly, from the processing to the
controller, the consumer, other persons and the public
against the potential risks to the rights of the consumer
20220HB2257PN2617 - 26 -
1
2
3
4
5
6
7
8
9
10
11
12
13
14
15
16
17
18
19
20
21
22
23
24
25
26
27
28
29
30
associated with the processing, as mitigated by safeguards
that can be employed by the controller to reduce the risks.
(2) The use of de-identified data and the reasonable
expectations of consumers, as well as the context of the
processing and the relationship between the controller and
the consumer whose personal data will be processed, shall be
factored into the assessment by the controller.
(c) Authority of Attorney General.--
(1) The Attorney General may request by subpoena that a
controller disclose any data protection assessment that is
relevant to an investigation conducted by the Attorney
General, and the controller shall make the data protection
assessment available to the Attorney General.
(2) The Attorney General may evaluate the data
protection assessment for compliance with the
responsibilities specified in this act.
(3) Data protection assessments shall be confidential
and exempt from public inspection and copying.
(4) The disclosure of a data protection assessment as a
result of a request from the Attorney General shall not
constitute a waiver of attorney-client privilege or work
product protection with respect to the assessment and any
information contained in the assessment.
(d) Comparable set of processing operations permitted.--A
single data protection assessment may address a comparable set
of processing operations that include similar activities.
(e) Compliance with other laws.--A data protection
assessment conducted by a controller for the purpose of
compliance with Federal or State laws or regulations may comply
under this section if the assessment has a reasonably comparable
20220HB2257PN2617 - 27 -
1
2
3
4
5
6
7
8
9
10
11
12
13
14
15
16
17
18
19
20
21
22
23
24
25
26
27
28
29
30
scope and effect.
(f) Applicability.--Data protection assessment requirements
shall apply to processing activities created or generated after
January 1, 2023, and are not retroactive.
Section 305. Processing de-identified data and exemptions.
(a) Duties of controller.--The controller in possession of
de-identified data shall:
(1) Take reasonable measures to ensure that the data
cannot be associated with a natural person.
(2) Publicly commit to maintaining and using deidentified data without attempting to re-identify the data.
(3) Contractually obligate a recipient of the deidentified data to comply with all provisions of this act.
(b) Construction.--Nothing in this act shall be construed to
require a controller or processor to:
(1) Re-identify de-identified data or pseudonymous data;
or maintain data in identifiable form, or collect, obtain,
retain or access any data or technology in order to be
capable of associating an authenticated consumer request with
personal data.
(2) Require a controller or processor to comply with an
authenticated consumer rights request under this act if all
of the following are true:
(i) The controller is not reasonably capable of
associating the request with the personal data or it
would be unreasonably burdensome for the controller to
associate the request with the personal data.
(ii) The controller does not use the personal data
to recognize or respond to the specific consumer who is
the subject of the personal data, or associate the
20220HB2257PN2617 - 28 -
1
2
3
4
5
6
7
8
9
10
11
12
13
14
15
16
17
18
19
20
21
22
23
24
25
26
27
28
29
30
personal data with other personal data about the same
consumer.
(iii) The controller does not sell the personal data
to a third party or otherwise voluntarily disclose the
personal data to a third party other than a processor,
except as otherwise permitted in this act.
(c) Pseudonymous data.--The consumer rights contained in
this act shall not apply to pseudonymous data in a case where
the controller is able to demonstrate that information necessary
to identify the consumer is maintained separately from the
original data and is secured in such a way that prevents the
controller from accessing the information.
(d) Duty to exercise reasonable oversight.--A controller
that discloses pseudonymous data or de-identified data shall
exercise reasonable oversight to monitor compliance with safety
standards, contracts with consumer, and Federal and State laws
to which the pseudonymous data or de-identified data is subject
and shall take appropriate steps to address a breach of the
contractual commitment.
Section 306. Limitations.
(a) General rule.--Nothing in this act shall be construed to
restrict a controller's or processor's ability to:
(1) Comply with Federal, State or local law, rule or
regulation.
(2) Comply with a civil, criminal or regulatory inquiry,
investigation, subpoena or summons by a Federal, State, local
or other governmental authority.
(3) Cooperate with a law enforcement agency concerning
conduct or activity that the controller or processor
reasonably and in good faith believes may violate Federal,
20220HB2257PN2617 - 29 -
1
2
3
4
5
6
7
8
9
10
11
12
13
14
15
16
17
18
19
20
21
22
23
24
25
26
27
28
29
30
State or local law, rule or regulation.
(4) Investigate, establish, exercise, prepare for or
defend a legal claim.
(5) Provide a good, product or service specifically
requested by a consumer, perform a contract to which the
consumer is a party, including fulfilling the terms of a
written warranty, or take steps at the request of the
consumer prior to entering into a contract.
(6) Take immediate steps to protect an interest that is
essential for the life or physical safety of the consumer or
of another individual, and where the processing cannot be
manifestly based on another legal basis.
(7) Prevent, detect, protect against or respond to
security incidents, identity theft, fraud, harassment,
malicious or deceptive activities, or any illegal activity,
preserve the integrity or security of data systems or
investigate, report or prosecute a person responsible for
that action.
(8) Engage in public or peer-reviewed scientific or
statistical research in the public interest that adheres to
all other Federal, State or local ethics and privacy laws and
is approved, monitored and governed by an independent
oversight entity that determines:
(i) if the deletion of the information is likely to
provide substantial benefits to the consumer that do not
exclusively accrue to the controller;
(ii) the expected benefits of the research outweigh
the privacy risks; and
(iii) the controller has implemented reasonable
safeguards to mitigate privacy risks associated with
20220HB2257PN2617 - 30 -
1
2
3
4
5
6
7
8
9
10
11
12
13
14
15
16
17
18
19
20
21
22
23
24
25
26
27
28
29
30
research, including risks associated with reidentification.
(9) Assist another controller, processor or third party
with an obligation under this subsection.
(b) Other abilities preserved.--The obligations imposed on
controllers or processors under this act shall not be construed
to restrict a controller's or processor's ability to collect,
use or retain data to:
(1) Conduct internal research to develop, improve or
repair products, services or technology.
(2) Effectuate a product recall.
(3) Identify and repair technical errors that impair
existing or intended functionality of the data.
(4) Perform internal operations that are reasonably
aligned with the expectations of a consumer or reasonably
anticipated by a consumer based on a consumer's existing
relationship with the controller or are otherwise compatible
with processing data in furtherance of the provision of a
good, product or service specifically requested by a consumer
or the performance of a contract to which a consumer is a
party.
(c) Evidentiary privileges.--
(1) The obligations imposed on controllers or processors
under this act shall not apply where compliance by the
controller or processor with this act would violate an
evidentiary privilege under the laws of this Commonwealth.
(2) Nothing in this act shall be construed to prevent a
controller or processor from providing personal data
concerning a consumer to a person covered by an evidentiary
privilege under the laws of this Commonwealth as part of a
20220HB2257PN2617 - 31 -
1
2
3
4
5
6
7
8
9
10
11
12
13
14
15
16
17
18
19
20
21
22
23
24
25
26
27
28
29
30
privileged communication.
(d) Defenses.--
(1) A controller or processor that discloses personal
data to a third-party controller or processor in compliance
with the requirements of this act is not in violation of this
act if the third-party controller or processor that receives
and processes the personal data is in violation of this act,
provided that, at the time of disclosing the personal data,
the disclosing controller or processor did not have actual
knowledge that the recipient intended to commit a violation.
(2) A third-party controller or processor receiving
personal data from a controller or processor in compliance
with the requirements of this act is not in violation of this
act for the transgressions of the controller or processor
from which it receives the personal data.
(e) Construction.--Nothing in this act shall be construed as
imposing an obligation on a controller or processor that
adversely affects the right or freedom of a person, such as
exercising the right of free speech pursuant to the First
Amendment to the Constitution of the United States, or applies
to the processing of personal data by a person in the course of
a purely personal or household activity.
(f) Permissible processing.--
(1) Personal data processed by a controller or processor
under contract with a controller under this section shall not
be processed for any purpose other than those expressly
listed in this section unless otherwise allowed by this act.
Personal data processed by a controller or processor under
contract with a controller under this section may be
processed to the extent that such processing is:
20220HB2257PN2617 - 32 -
1
2
3
4
5
6
7
8
9
10
11
12
13
14
15
16
17
18
19
20
21
22
23
24
25
26
27
28
29
30
(i) Reasonably necessary and proportionate to the
purposes listed in this section.
(ii) Limited to what is necessary in relation to the
specific purposes listed in this section.
(2) Personal data collected, used or retained under
subsection (b) shall, where applicable, take into account the
nature and purpose or purposes of the collection, use or
retention. The data shall be subject to reasonable
administrative, technical and physical measures to protect
the confidentiality, integrity and accessibility of the
personal data and to reduce reasonably foreseeable risks of
harm to consumers relating to such collection, use or
retention of personal data.
(g) Controller burden to demonstrate exemption.--If a
controller processes personal data by virtue of an exemption
under this section, the controller bears the burden of
demonstrating that the processing qualifies for the exemption
and complies with the requirements of subsection (f).
(h) Status as controller.--Processing personal data for the
purposes expressly identified in subsection (a) shall not solely
make an entity a controller with respect to the processing.
CHAPTER 5
ADMINISTRATION AND ENFORCEMENT
Section 501. Powers and duties of Attorney General.
(a) Administration.--The Attorney General shall administer
and enforce the provisions of this act and may adopt regulations
to carry out the requirements of this act.
(b) Investigative authority.--Whenever the Attorney General
has reasonable cause to believe that a person has engaged in, is
engaging in or is about to engage in a violation of this act,
20220HB2257PN2617 - 33 -
1
2
3
4
5
6
7
8
9
10
11
12
13
14
15
16
17
18
19
20
21
22
23
24
25
26
27
28
29
30
the Attorney General may issue a civil investigative demand.
Section 502. Enforcement procedure.
(a) Notice of violation.--Prior to initiating an action
under this act, the Attorney General shall provide a controller
or processor 30 days' written notice identifying the specific
provisions of this act that the Attorney General alleges have
been or are being violated.
(b) Cure of violation.--If within the 30-day period
specified under subsection (b), the controller or processor
cures the noticed violation and provides the Attorney General an
express written statement that the alleged violations have been
cured and that no further violations shall occur, no action
shall be initiated against the controller or processor.
(c) Failure to cure.--If a controller or processor continues
to violate this act following the cure period in subsection (b)
or breaches an express written statement provided to the
Attorney General under this section, the Attorney General may
initiate an action in the name of the Commonwealth and may seek
an injunction to restrain the violation of this act and civil
penalties of up to $7,500 for each violation under this act.
(d) Recovery of reasonable expenses.--The Attorney General
may recover reasonable expenses incurred in investigating and
preparing the case, including attorney fees, in an action
initiated under this act.
(e) Construction.--Nothing in this act shall be construed as
providing the basis for, or be subject to, a private right of
action for violations of this act or under any other law.
Section 503. Consumer Privacy Fund.
(a) Establishment.--The Consumer Privacy Fund is established
in the State Treasury.
20220HB2257PN2617 - 34 -
1
2
3
4
5
6
7
8
9
10
11
12
13
14
15
16
17
18
19
20
21
22
23
24
25
26
27
28
29
30
(b) Contents of fund.--All civil penalties, expenses and
attorney fees collected under this act shall be paid into the
State Treasury and credited to the fund. Interest earned on
money in the fund shall remain in the fund and shall be credited
to the fund. Any money remaining in the fund, including
interest, at the end of each fiscal year shall not revert to the
General Fund but shall remain in the fund.
(c) Use of fund.--The money in the fund shall be used by the
Office of the Attorney General to enforce the provisions of this
act.
CHAPTER 7
MISCELLANEOUS PROVISIONS
Section 701. (Reserved).
Section 702. Effective date.
This act shall take effect January 1, 2023, or in 18 months,
whichever is later.
20220HB2257PN2617 - 35 -
1
2
3
4
5
6
7
8
9
10
11
12
13
14
15
16