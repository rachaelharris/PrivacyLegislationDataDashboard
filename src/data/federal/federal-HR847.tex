To support research on privacy enhancing technologies and promote responsible data use, and for other purposes.

Be it enacted by the Senate and House of Representatives of the United States of America in Congress assembled,

SECTION 1. SHORT TITLE.

This Act may be cited as the “Promoting Digital Privacy Technologies Act”.

SEC. 2. DEFINITIONS.

In this Act:

(1) PERSONAL DATA.—The term “personal data” means information that identifies, is linked to, or is reasonably linkable to, an individual or a consumer device, including derived data.

(2) PRIVACY ENHANCING TECHNOLOGY.—The term “privacy enhancing technology”—

(A) means any software solution, technical processes, or other technological means of enhancing the privacy and confidentiality of an individual’s personal data in data or sets of data; and

(B) includes anonymization and pseudonymization techniques, filtering tools, anti-tracking technology, differential privacy tools, synthetic data, and secure multi-party computation.

SEC. 3. NATIONAL SCIENCE FOUNDATION SUPPORT OF RESEARCH ON PRIVACY ENHANCING TECHNOLOGY.

The Director of the National Science Foundation, in consultation with other relevant Federal agencies (as determined by the Director), shall support merit-reviewed and competitively awarded research on privacy enhancing technologies, which may include—

(1) fundamental research on technologies for de-identification, pseudonymization, anonymization, or obfuscation of personal data in data sets while maintaining fairness, accuracy, and efficiency;

(2) fundamental research on algorithms and other similar mathematical tools used to protect individual privacy when collecting, storing, sharing, or aggregating data;

(3) fundamental research on technologies that promote data minimization principles in data collection, sharing, and analytics; and

(4) research awards on privacy enhancing technologies coordinated with other relevant Federal agencies and programs.

SEC. 4. INTEGRATION INTO THE COMPUTER AND NETWORK SECURITY PROGRAM.

Subparagraph (D) of section 4(a)(1) of the Cyber Security Research and Development Act (15 U.S.C. 7403(a)(1)(D)) is amended to read as follows:


“(D) privacy enhancing technologies and confidentiality;”.

SEC. 5. COORDINATION WITH THE NATIONAL INSTITUTE OF STANDARDS AND TECHNOLOGY AND OTHER STAKEHOLDERS.

(a) In General.—The Director of the Office of Science and Technology Policy, acting through the Networking and Information Technology Research and Development Program, shall coordinate with the Director of the National Science Foundation, the Director of the National Institute of Standards and Technology, and the Federal Trade Commission to accelerate the development and use of privacy enhancing technologies.

(b) Outreach.—The Director of the National Institute of Standards and Technology shall conduct outreach to—

(1) receive input from private, public, and academic stakeholders, including the National Institutes of Health and the Centers for Disease Control and Prevention, for the purpose of facilitating public health research, on the development of privacy enhancing technologies; and

(2) develop ongoing public and private sector engagement to create and disseminate voluntary, consensus-based resources to increase the integration of privacy enhancing technologies in data collection, sharing, and analytics by the public and private sectors.

SEC. 6. REPORT ON RESEARCH AND STANDARDS DEVELOPMENT.

Not later than 2 years after the date of enactment of this Act, the Director of the Office of Science and Technology Policy, acting through the Networking and Information Technology Research and Development Program, shall, in coordination with the Director of the National Science Foundation and the Director of the National Institute of Standards and Technology, submit to the Committee on Commerce, Science, and Transportation of the Senate, the Subcommittee on Commerce, Justice, Science, and Related Agencies of the Committee on Appropriations of the Senate, the Committee on Science, Space, and Technology of the House of Representatives, and the Subcommittee on Commerce, Justice, Science, and Related Agencies of the Committee on Appropriations of the House of Representatives, a report containing—

(1) the progress of research on privacy enhancing technologies;

(2) the progress of the development of voluntary resources described under section 5(b)(2); and

(3) any policy recommendations of the Directors that could facilitate and improve communication and coordination between the private sector, the National Science Foundation, and relevant Federal agencies through the implementation of privacy enhancing technologies.

