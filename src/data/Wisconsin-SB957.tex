LRB-4835/1
KP:amn&wlj
2021 - 2022 LEGISLATURE
2021 ASSEMBLY BILL 957
February 3, 2022 - Introduced by Representatives ZIMMERMAN, ARMSTRONG,
DITTRICH, DUCHOW, HORLACHER, KNODL, KUGLITSCH, KURTZ, MACCO,
MAGNAFICI, MOSES, MURPHY, PENTERMAN, PLUMER, PRONSCHINSKE, SORTWELL,
STEFFEN, SUBECK, TAUCHEN, THIESFELDT, TITTL and WITTKE, cosponsored by
Senators KOOYENGA, BALLWEG, FEYEN and NASS. Referred to Committee on
Consumer Protection.
***AUTHORS SUBJECT TO CHANGE***
AN ACT to create 134.985 of the statutes; relating to: consumer data protection
and providing a penalty.
Analysis by the Legislative Reference Bureau
This bill establishes requirements for controllers and processors of the personal
data of consumers. The bill defines a “controller” as a person that, alone or jointly
with others, determines the purpose and means of processing personal data, and the
bill applies to controllers that control or process the personal data of at least 100,000
consumers or that control or process the personal data of at least 25,000 consumers
and derive over 50 percent of their gross revenue from the sale of personal data.
Under the bill, “personal data” means any information that is linked or reasonably
linkable to an individual except for publicly available information.
The bill provides consumers with the following rights regarding their personal
data: 1) to confirm whether a controller is processing the consumer's personal data
and to access the personal data; 2) to correct inaccuracies in the consumer's personal
data; 3) to require a controller to delete personal data provided by or about the
consumer; 4) to obtain a copy of the personal data that the consumer previously
provided to the controller; and 5) to opt out of the processing of the consumer's
personal data for targeted advertising; the sale of the consumer's personal data; and
certain forms of automated processing of the consumer's personal data. These rights
are subject to certain exceptions specified in the bill. Controllers may not
discriminate against a consumer for exercising rights under the bill, including by
charging different prices for goods or providing a different level of quality of goods
or services.
1
2
2021 - 2022 Legislature - 2 - LRB-4835/1
KP:amn&wlj
 ASSEMBLY BILL 957
The bill requires controllers to respond to consumers' requests to invoke rights
under the bill without undue delay. If a controller declines to take action regarding
a consumer's request, the controller must inform the consumer of its justification
without undue delay. The bill also requires that information provided in response
to a consumer's request be provided free of charge up to twice annually per consumer.
Controllers must also establish processes for consumers to appeal a refusal to take
action on a consumer's request. Within 60 days of receiving an appeal, a controller
must inform the consumer in writing of any action taken or not taken in response to
the appeal, including a written explanation of the reasons for its decisions. If the
appeal is denied, the controller must provide the consumer with a method through
which the consumer can contact the attorney general to submit a complaint.
Under the bill, a controller must provide consumers with a privacy notice that
discloses the categories of personal data processed by the controller; the purpose of
processing the personal data; the categories of third parties, if any, with whom the
controller shares personal data; the categories of personal data that the controller
shares with third parties; and information about how consumers may exercise their
rights under the bill. Controllers may not collect or process personal data for
purposes that are not relevant to or reasonably necessary for the purposes disclosed
in the privacy notice. The bill's requirements do not restrict a controller's ability to
collect, use, or retain data for conducting internal research, effectuating a product
recall, identifying and repairing technical errors, or performing internal operations
that are reasonably aligned with consumer expectations or reasonably anticipated
on the basis of a consumer's relationship with the controller.
Persons that process personal data on behalf of a controller must adhere to a
contract between the controller and the processor, and such contracts must satisfy
certain requirements specified in the bill. The bill also requires controllers to
conduct data protection assessments related to certain activities, including
processing personal data for targeted advertising, selling personal data, processing
personal data for profiling purposes, and processing sensitive data, as defined in the
bill. The attorney general may request that a controller disclose a data protection
assessment that is relevant to an investigation being conducted by the attorney
general.
The attorney general has exclusive authority to enforce violations of the bill's
requirements. A controller or processor that violates the bill's requirements is
subject to a forfeiture of up to $7,500 per violation, and the attorney general may
recover reasonable investigation and litigation expenses incurred. Before bringing
an action to enforce the bill's requirements, the attorney general must first provide
a controller or processor with a written notice identifying the violations. If within
30 days of receiving the notice the controller or processor cures the violation and
provides the attorney general with an express written statement that the violation
is cured and that no further violations will occur, then the attorney general may not
bring an action against the controller or processor. The bill also prohibits cities,
villages, towns, and counties from enacting or enforcing ordinances that regulate the
collection, processing, or sale of personal data.
2021 - 2022 Legislature - 3 - LRB-4835/1
KP:amn&wlj
 ASSEMBLY BILL 957
For further information see the state fiscal estimate, which will be printed as
an appendix to this bill.
The people of the state of Wisconsin, represented in senate and assembly, do
enact as follows:
SECTION 1. 134.985 of the statutes is created to read:
134.985 Consumer data protection. (1) DEFINITIONS. In this section:
(a) “Affiliate” means a legal entity that controls, is controlled by, or is under
common control with another legal entity or shares common branding with another
legal entity. For the purposes of this definition, “control" or “controlled" means
ownership of, or the power to vote, more than 50 percent of the outstanding shares
of any class of voting security of a company; control in any manner over the election
of a majority of the directors or of individuals exercising similar functions; or the
power to exercise controlling influence over the management of a company.
(b) “Authenticate" means verifying through reasonable means that the
consumer, entitled to exercise his or her consumer rights under sub. (2), is the same
consumer exercising such consumer rights with respect to the personal data at issue.
(c) “Biometric data" means data generated by automatic measurements of an
individual's biological characteristics, such as a fingerprint, voiceprint, eye retinas,
irises, or other unique biological patterns or characteristics that are used to identify
a specific individual. “Biometric data" does not include a physical or digital
photograph, a video or audio recording or data generated therefrom, or information
collected, used, or stored for health care treatment, payment, or operations under the
federal Health Insurance Portability and Accountability Act of 1996.
(d) “Business associate” has the meaning given in 45 CFR 160.103.
(e) “Child” means an individual younger than 13 years of age.
1
2
3
4
5
6
7
8
9
10
11
12
13
14
15
16
17
18
19
20
21
2021 - 2022 Legislature - 4 - LRB-4835/1
KP:amn&wlj
 ASSEMBLY BILL 957 SECTION 1
(f) “Consent" means a clear affirmative act signifying a consumer's freely given,
specific, informed, and unambiguous agreement to process personal data relating to
the consumer. “Consent” may include a written statement, including a statement
written by electronic means, or any other unambiguous affirmative action.
(g) “Consumer" means an individual who is a resident of this state acting only
in an individual or household context. “Consumer" does not include an individual
acting in a commercial or employment context.
(h) “Controller" means a person that, alone or jointly with others, determines
the purpose and means of processing personal data.
(i) “Covered entity” has the meaning given in 45 CFR 160.103.
(j) “Decisions that produce legal or similarly significant effects concerning a
consumer" means a decision made by a controller that results in the provision or
denial by the controller of financial and lending services, housing, insurance,
education enrollment, criminal justice, employment opportunities, health care
services, or access to basic necessities, such as food and water.
(k) “Deidentified data" means data that cannot reasonably be linked to an
identified or identifiable individual, or a device linked to such person.
(L) “Identified or identifiable individual" means a person who can be readily
identified, directly or indirectly.
(m) “Institution of higher education” has the meaning given in s. 39.32 (1) (a).
(n) “Nonprofit organization" means any corporation organized under ch. 181
or any organization exempt from taxation under section 501 (c) (3), (6), or (12) of the
Internal Revenue Code.
1
2
3
4
5
6
7
8
9
10
11
12
13
14
15
16
17
18
19
20
21
22
23
2021 - 2022 Legislature - 5 - LRB-4835/1
KP:amn&wlj
SECTION 1 ASSEMBLY BILL 957
(o) “Personal data" means any information that is linked or reasonably linkable
to an identified or identifiable individual. “Personal data" does not include
deidentified data or publicly available information.
(p) “Precise geolocation data" means information derived from technology,
including global positioning system level latitude and longitude coordinates or other
mechanisms, that directly identifies the specific location of an individual with
precision and accuracy within a radius of 1,750 feet. “Precise geolocation data" does
not include the content of communications or any data generated by or connected to
advanced utility metering infrastructure systems or equipment for use by a utility.
(q) “Process" or “processing" means any operation or set of operations
performed, whether by manual or automated means, on personal data or on sets of
personal data, such as the collection, use, storage, disclosure, analysis, deletion, or
modification of personal data.
(r) “Processor” means an individual or person that processes personal data on
behalf of a controller.
(s) “Profiling" means any form of automated processing performed on personal
data to evaluate, analyze, or predict personal aspects related to an identified or
identifiable individual's economic situation, health, personal preferences, interests,
reliability, behavior, location, or movements.
(t) “Pseudonymous data" means personal data that cannot be attributed to a
specific individual without the use of additional information, provided that such
additional information is kept separately and is subject to appropriate technical and
organizational measures to ensure that the personal data is not attributed to an
identified or identifiable individual.
1
2
3
4
5
6
7
8
9
10
11
12
13
14
15
16
17
18
19
20
21
22
23
24
2021 - 2022 Legislature - 6 - LRB-4835/1
KP:amn&wlj
 ASSEMBLY BILL 957 SECTION 1
(u) “Publicly available information" means information that is lawfully made
available through federal, state, or local government records, or information that a
business has a reasonable basis to believe is lawfully made available to the general
public through widely distributed media, by the consumer, or by a person to whom
the consumer has disclosed the information, unless the consumer has restricted the
information to a specific audience.
(v) “Sale of personal data" means the exchange of personal data for monetary
consideration by the controller to a 3rd party. “Sale of personal data" does not include
any of the following:
1. The disclosure of personal data to a processor that processes the personal
data on behalf of the controller.
2. The disclosure of personal data to a 3rd party for purposes of providing a
product or service requested by the consumer.
3. The disclosure or transfer of personal data to an affiliate of the controller.
4. The disclosure of information that a consumer intentionally made available
to the general public via a channel of mass media and did not restrict to a specific
audience.
5. The disclosure or transfer of personal data to a 3rd party as an asset that is
part of a merger, acquisition, bankruptcy, or other transaction in which the 3rd party
assumes control of all or part of the controller's assets.
(w) “Sensitive data” includes the following:
1. Personal data revealing racial or ethnic origin, religious beliefs, mental or
physical health diagnosis, sexual orientation, or citizenship or immigration status.
2. The processing of genetic or biometric data for the purpose of uniquely
identifying an individual.
1
2
3
4
5
6
7
8
9
10
11
12
13
14
15
16
17
18
19
20
21
22
23
24
25
2021 - 2022 Legislature - 7 - LRB-4835/1
KP:amn&wlj
SECTION 1 ASSEMBLY BILL 957
3. The personal data collected from a known child.
4. Precise geolocation data.
(x) “Targeted advertising" means displaying advertisements to a consumer
where the advertisement is selected based on personal data obtained from that
consumer's activities over time and across nonaffiliated websites or online
applications to predict such consumer's preferences or interests. “Targeted
advertising" does not include any of the following:
1. Advertisements based on activities within a controller's own websites or
online applications.
2. Advertisements based on the context of a consumer's current search query,
visit to a website, or online application.
3. Advertisements directed to a consumer in response to the consumer's request
for information or feedback.
4. Processing personal data processed solely for measuring or reporting
advertising performance, reach, or frequency.
(y) “Third party” means a person or association, authority, board, department,
commission, independent agency, institution, office, society, or other body in state or
local government created or authorized to be created by the constitution or any law,
other than a consumer, controller, processor, or an affiliate of the processor or the
controller.
(2) PERSONAL DATA RIGHTS; CONSUMERS. (a) A consumer may invoke the
consumer rights authorized under this subsection at any time by submitting a
request to a controller specifying the consumer rights the consumer wishes to invoke.
A known child's parent or legal guardian may invoke such consumer rights on behalf
of the child regarding processing personal data belonging to the known child. A
1
2
3
4
5
6
7
8
9
10
11
12
13
14
15
16
17
18
19
20
21
22
23
24
25
2021 - 2022 Legislature - 8 - LRB-4835/1
KP:amn&wlj
 ASSEMBLY BILL 957 SECTION 1
controller shall comply with an authenticated consumer request to exercise any of
the following rights:
1. To confirm whether or not a controller is processing the consumer's personal
data and to access such personal data.
2. To correct inaccuracies in the consumer's personal data, taking into account
the nature of the personal data and the purposes of the processing of the consumer's
personal data.
3. To delete personal data provided by or obtained about the consumer.
4. To obtain a copy of the consumer's personal data that the consumer
previously provided to the controller in a portable and, to the extent technically
feasible, readily usable format that allows the consumer to transmit the data to
another controller without hindrance, where the processing is carried out by
automated means.
5. To opt out of the processing of the personal data for purposes of targeted
advertising, the sale of personal data, or profiling in furtherance of decisions that
produce legal or similarly significant effects concerning the consumer.
(b) 1. Except as otherwise provided in this section, a controller shall comply
with a request by a consumer to exercise the consumer rights authorized under par.
(a).
2. A controller shall respond to a consumer without undue delay, but in all cases
within 45 days of receipt of a request submitted under par. (a). The response period
may be extended once by 45 additional days when reasonably necessary, taking into
account the complexity and number of the consumer's requests, so long as the
controller informs the consumer of any such extension within the initial 45-day
response period, together with the reason for the extension.
1
2
3
4
5
6
7
8
9
10
11
12
13
14
15
16
17
18
19
20
21
22
23
24
25
2021 - 2022 Legislature - 9 - LRB-4835/1
KP:amn&wlj
SECTION 1 ASSEMBLY BILL 957
3. If a controller declines to take action regarding a consumer's request, the
controller shall inform the consumer without undue delay, but in all cases and at the
latest within 45 days of receipt of the request, of the justification for declining to take
action and instructions for how to appeal the decision under par. (c).
4. Information provided in response to a consumer request shall be provided
by a controller free of charge, up to twice annually per consumer. If requests from
a consumer are manifestly unfounded, excessive, or repetitive, the controller may
charge the consumer a reasonable fee to cover the administrative costs of complying
with the request or decline to act on the request. The controller bears the burden of
demonstrating the manifestly unfounded, excessive, or repetitive nature of the
request.
5. If a controller is unable to authenticate the request using commercially
reasonable efforts, the controller may not be required to comply with a request to
initiate an action under par. (a) and may request that the consumer provide
additional information reasonably necessary to authenticate the consumer and the
consumer's request.
(c) A controller shall establish a process for a consumer to appeal the
controller's refusal to take action on a request within a reasonable period of time
after the consumer's receipt of the decision pursuant to par. (b) 3. The appeal process
shall be conspicuously available and similar to the process for submitting requests
to initiate action under par. (a). Within 60 days of receipt of an appeal, a controller
shall inform the consumer in writing of any action taken or not taken in response to
the appeal, including a written explanation of the reasons for the decisions. If the
appeal is denied, the controller shall also provide the consumer with an online
1
2
3
4
5
6
7
8
9
10
11
12
13
14
15
16
17
18
19
20
21
22
23
24
2021 - 2022 Legislature - 10 - LRB-4835/1
KP:amn&wlj
 ASSEMBLY BILL 957 SECTION 1
mechanism, if available, or other method through which the consumer may contact
the attorney general to submit a complaint.
(3) DATA CONTROLLER RESPONSIBILITIES; TRANSPARENCY. (a) 1. A controller shall
limit the collection of personal data to what is adequate, relevant, and reasonably
necessary in relation to the purposes for which such data is processed, as disclosed
to the consumer.
2. Except as otherwise provided in this section, a controller may not process
personal data for purposes that are not reasonably necessary to and not compatible
with the disclosed purposes for which such personal data is processed, as disclosed
to the consumer, unless the controller obtains the consumer's consent.
3. A controller shall establish, implement, and maintain reasonable
administrative, technical, and physical data security practices to protect the
confidentiality, integrity, and accessibility of personal data. Such data security
practices shall be appropriate to the volume and nature of the personal data at issue.
4. A controller may not process personal data in violation of state and federal
laws that prohibit unlawful discrimination against consumers. A controller may not
discriminate against a consumer for exercising any of the consumer rights contained
in this section, including denying goods or services, charging different prices or rates
for goods or services, or providing a different level of quality of goods and services to
the consumer. Nothing in this subdivision shall be construed to require a controller
to provide a product or service that requires the personal data of a consumer that the
controller does not collect or maintain, or to prohibit a controller from offering a
different price, rate, level, quality, or selection of goods or services to a consumer,
including offering goods or services for no fee, if the consumer has exercised his or
her right to opt out under sub. (2) (a) 5. or the offer is related to a consumer's
1
2
3
4
5
6
7
8
9
10
11
12
13
14
15
16
17
18
19
20
21
22
23
24
25
2021 - 2022 Legislature - 11 - LRB-4835/1
KP:amn&wlj
SECTION 1 ASSEMBLY BILL 957
voluntary participation in a bona fide loyalty, rewards, premium features, discounts,
or club card program.
5. A controller may not process sensitive data concerning a consumer without
obtaining the consumer's consent, or, in the case of the processing of sensitive data
concerning a known child, without processing such data in accordance with the
federal Children's Online Privacy Protection Act, 15 USC 6501 et seq.
(b) Any provision of a contract or agreement that purports to waive or limit
consumer rights under sub. (2) is void and unenforceable.
(c) A controller shall provide consumers with a reasonably accessible, clear, and
meaningful privacy notice that includes all of the following:
1. The categories of personal data processed by the controller.
2. The purpose of processing personal data.
3. How consumers may exercise their consumer rights under sub. (2), including
how a consumer may appeal a controller's decision with regard to the consumer's
request.
4. The categories of personal data that the controller shares with 3rd parties,
if any.
5. The categories of 3rd parties, if any, with whom the controller shares
personal data.
(d) If a controller sells personal data to 3rd parties or processes personal data
for targeted advertising, the controller shall clearly and conspicuously disclose such
processing, as well as the manner in which a consumer may exercise the right to opt
out of such processing.
(e) A controller shall establish, and shall describe in a privacy notice, one or
more secure and reliable means for consumers to submit a request to exercise their
1
2
3
4
5
6
7
8
9
10
11
12
13
14
15
16
17
18
19
20
21
22
23
24
25
2021 - 2022 Legislature - 12 - LRB-4835/1
KP:amn&wlj
 ASSEMBLY BILL 957 SECTION 1
consumer rights under this section. Such means shall take into account the ways in
which consumers normally interact with the controller, the need for secure and
reliable communication of such requests, and the ability of the controller to
authenticate the identity of the consumer making the request. Controllers may not
require a consumer to create a new account in order to exercise consumer rights
under sub. (2) but may require a consumer to use an existing account.
(4) RESPONSIBILITY ACCORDING TO ROLE; CONTROLLER AND PROCESSOR. (a) A
processor shall adhere to the instructions of a controller and shall assist the
controller in meeting its obligations under this section. Such assistance shall include
the following:
1. Taking into account the nature of processing and the information available
to the processor, by appropriate technical and organizational measures, insofar as
this is reasonably practicable, to fulfill the controller's obligation to respond to
consumer rights requests under sub. (2).
2. Taking into account the nature of processing and the information available
to the processor, by assisting the controller in meeting the controller's obligations in
relation to the security of processing the personal data and in relation to giving notice
of unauthorized acquisition of personal information under s. 134.98.
3. Providing necessary information to enable the controller to conduct and
document data protection assessments under sub. (5).
(b) A contract between a controller and a processor shall govern the processor's
data processing procedures with respect to processing performed on behalf of the
controller. The contract shall be binding and clearly set forth instructions for
processing data, the nature and purpose of processing, the type of data subject to
processing, the duration of processing, and the rights and obligations of both parties.
1
2
3
4
5
6
7
8
9
10
11
12
13
14
15
16
17
18
19
20
21
22
23
24
25
2021 - 2022 Legislature - 13 - LRB-4835/1
KP:amn&wlj
SECTION 1 ASSEMBLY BILL 957
The contract shall also include requirements that the processor shall do all of the
following:
1. Ensure that each person processing personal data is subject to a duty of
confidentiality with respect to the data.
2. At the controller's direction, delete or return all personal data to the
controller as requested at the end of the provision of services, unless retention of the
personal data is required by law.
3. Upon the reasonable request of the controller, make available to the
controller all information in its possession necessary to demonstrate the processor's
compliance with the obligations in this section.
4. At least one of the following:
a. Allow, and cooperate with, reasonable assessments by the controller or the
controller's designated assessor.
b. Arrange for a qualified and independent assessor to conduct an assessment
of the processor's policies and technical and organizational measures in support of
the obligations under this section using an appropriate and accepted control
standard or framework and assessment procedure for such assessments. The
processor shall provide a report of such assessment to the controller upon request.
5. Engage any subcontractor pursuant to a written contract in accordance with
par. (c) that requires the subcontractor to meet the obligations of the processor with
respect to the personal data.
(c) Nothing in this section shall be construed to relieve a controller or a
processor from the liabilities imposed on it by virtue of its role in the processing
relationship as defined by this section.
1
2
3
4
5
6
7
8
9
10
11
12
13
14
15
16
17
18
19
20
21
22
23
24
2021 - 2022 Legislature - 14 - LRB-4835/1
KP:amn&wlj
 ASSEMBLY BILL 957 SECTION 1
(d) Determining whether a person is acting as a controller or processor with
respect to a specific processing of data is a fact-based determination that depends
upon the context in which personal data is to be processed. A processor that
continues to adhere to a controller's instructions with respect to a specific processing
of personal data remains a processor.
(5) DATA PROTECTION ASSESSMENTS. (a) A controller shall conduct and document
a data protection assessment of each of the following processing activities involving
personal data:
1. The processing of personal data for purposes of targeted advertising.
2. The sale of personal data.
3. The processing of personal data for purposes of profiling, where such
profiling presents a reasonably foreseeable risk of any of the following:
a. Unfair or deceptive treatment of, or unlawful disparate impact on,
consumers.
b. Financial, physical, or reputational injury to consumers.
c. Physical or other intrusion upon the solitude or seclusion, or the private
affairs or concerns, of consumers, where such intrusion would be offensive to a
reasonable person.
d. Other substantial injury to consumers.
4. The processing of sensitive data.
5. Any processing activities involving personal data that present a heightened
risk of harm to consumers.
(b) Data protection assessments conducted under par. (a) shall identify and
weigh the benefits that may flow, directly and indirectly, from the processing to the
controller, the consumer, other stakeholders, and the public against the potential
1
2
3
4
5
6
7
8
9
10
11
12
13
14
15
16
17
18
19
20
21
22
23
24
25
2021 - 2022 Legislature - 15 - LRB-4835/1
KP:amn&wlj
SECTION 1 ASSEMBLY BILL 957
risks to the rights of the consumer associated with such processing, as mitigated by
safeguards that can be employed by the controller to reduce such risks. The use of
deidentified data and the reasonable expectations of consumers, as well as the
context of the processing and the relationship between the controller and the
consumer whose personal data will be processed, shall be factored into this
assessment by the controller.
(c) The attorney general may request, pursuant to a civil investigative demand
issued under sub. (10) (a), that a controller disclose any data protection assessment
that is relevant to an investigation conducted by the attorney general, and the
controller shall make the data protection assessment available to the attorney
general. The attorney general may evaluate the data protection assessment for
compliance with the responsibilities set forth in sub. (3). Data protection
assessments shall be confidential and not subject to the right of inspection and
copying under s. 19.35 (1). The disclosure of a data protection assessment pursuant
to a request from the attorney general shall not constitute a waiver of attorney-client
privilege or work product protection with respect to the assessment and any
information contained in the assessment.
(d) A single data protection assessment may address a comparable set of
processing operations that include similar activities.
(e) Data protection assessments conducted by a controller for the purpose of
compliance with other laws or regulations may comply under this section if the
assessments have a reasonably comparable scope and effect.
(f) Data protection assessment requirements shall apply to processing
activities created or generated after January 1, 2024, and are not retroactive.
1
2
3
4
5
6
7
8
9
10
11
12
13
14
15
16
17
18
19
20
21
22
23
24
2021 - 2022 Legislature - 16 - LRB-4835/1
KP:amn&wlj
 ASSEMBLY BILL 957 SECTION 1
(6) PROCESSING DEIDENTIFIED DATA; EXEMPTIONS. (a) A controller in possession
of deidentified data shall do all of the following:
1. Take reasonable measures to ensure that the data cannot be associated with
an individual.
2. Publicly commit to maintaining and using deidentified data without
attempting to reidentify the data.
3. Contractually obligate any recipients of the deidentified data to comply with
all provisions of this section.
(b) Nothing in this section shall be construed to require a controller or processor
to reidentify deidentified data or pseudonymous data; maintain data in identifiable
form; or collect, obtain, retain, or access any data or technology, in order to be capable
of associating an authenticated consumer request with personal data.
(c) Nothing in this section shall be construed to require a controller or processor
to comply with an authenticated consumer rights request under sub. (2) if all of the
following apply:
1. The controller is not reasonably capable of associating the request with the
personal data or it would be unreasonably burdensome for the controller to associate
the request with the personal data.
2. The controller does not use the personal data to recognize or respond to the
specific consumer who is the subject of the personal data, or associate the personal
data with other personal data about the same specific consumer.
3. The controller does not sell the personal data to any 3rd party or otherwise
voluntarily disclose the personal data to any 3rd party other than a processor, except
as otherwise permitted in this subsection.
1
2
3
4
5
6
7
8
9
10
11
12
13
14
15
16
17
18
19
20
21
22
23
24
2021 - 2022 Legislature - 17 - LRB-4835/1
KP:amn&wlj
SECTION 1 ASSEMBLY BILL 957
(d) The consumer rights contained in subs. (2) (a) 1. to 4. and (3) shall not apply
to pseudonymous data in cases where the controller is able to demonstrate any
information necessary to identify the consumer is kept separately and is subject to
effective technical and organizational controls that prevent the controller from
accessing such information.
(e) A controller that discloses pseudonymous data or deidentified data shall
exercise reasonable oversight to monitor compliance with any contractual
commitments to which the pseudonymous data or deidentified data is subject and
shall take appropriate steps to address any breaches of those contractual
commitments.
(7) LIMITATIONS. (a) Nothing in this section shall be construed to restrict a
controller's or processor's ability to do any of the following:
1. Comply with federal, state, or local laws, rules, or regulations.
2. Comply with a civil, criminal, or regulatory inquiry, investigation, subpoena,
or summons by federal, state, local, or other governmental authorities.
3. Cooperate with law enforcement agencies concerning conduct or activity that
the controller or processor reasonably and in good faith believes may violate federal,
state, or local laws, rules, or regulations.
4. Investigate, establish, exercise, prepare for, or defend legal claims.
5. Provide a product or service specifically requested by a consumer, perform
a contract to which the consumer is a party, including fulfilling the terms of a written
warranty, or take steps at the request of the consumer prior to entering into a
contract.
1
2
3
4
5
6
7
8
9
10
11
12
13
14
15
16
17
18
19
20
21
22
23
2021 - 2022 Legislature - 18 - LRB-4835/1
KP:amn&wlj
 ASSEMBLY BILL 957 SECTION 1
6. Take immediate steps to protect an interest that is essential for the life or
physical safety of the consumer or of another individual, and where the processing
cannot be manifestly based on another legal basis.
7. Prevent, detect, protect against, or respond to security incidents, identity
theft, fraud, harassment, malicious or deceptive activities, or any illegal activity;
preserve the integrity or security of systems; or investigate, report, or prosecute
those responsible for any such action.
8. Engage in public or peer-reviewed scientific or statistical research in the
public interest that adheres to all other applicable ethics and privacy laws and is
approved, monitored, and governed by an institutional review board, or similar
independent oversight entities that determine all of the following:
a. If the deletion of the information is likely to provide substantial benefits that
do not exclusively accrue to the controller.
b. The expected benefits of the research outweigh the privacy risks.
c. If the controller has implemented reasonable safeguards to mitigate privacy
risks associated with research, including any risks associated with reidentification.
9. Assist another controller, processor, or 3rd party with any of the obligations
under this section.
(b) The obligations imposed on controllers or processors under this section shall
not restrict a controller's or processor's ability to collect, use, or retain data to do any
of the following:
1. Conduct internal research to develop, improve, or repair products, services,
or technology.
2. Effectuate a product recall.
1
2
3
4
5
6
7
8
9
10
11
12
13
14
15
16
17
18
19
20
21
22
23
24
2021 - 2022 Legislature - 19 - LRB-4835/1
KP:amn&wlj
SECTION 1 ASSEMBLY BILL 957
3. Identify and repair technical errors that impair existing or intended
functionality.
4. Perform internal operations that are reasonably aligned with the
expectations of the consumer or reasonably anticipated on the basis of the
consumer's existing relationship with the controller or are otherwise compatible
with processing data in furtherance of the provision of a product or service
specifically requested by a consumer or the performance of a contract to which the
consumer is a party.
(c) The obligations imposed on controllers or processors under this section shall
not apply where compliance by the controller or processor with this section would
violate an evidentiary privilege under ch. 905. Nothing in this section shall be
construed to prevent a controller or processor from providing personal data
concerning a consumer to a person covered by an evidentiary privilege under ch. 905
as part of a privileged communication.
(d) A controller or processor that discloses personal data to a 3rd-party
controller or processor, in compliance with the requirements of this section, is not in
violation of this section if the 3rd-party controller or processor that receives and
processes such personal data is in violation of this section, provided that, at the time
of disclosing the personal data, the disclosing controller or processor did not have
actual knowledge that the recipient intended to commit a violation. A 3rd-party
controller or processor receiving personal data from a controller or processor in
compliance with the requirements of this section is likewise not in violation of this
section for the transgressions of the controller or processor from which it receives
such personal data.
1
2
3
4
5
6
7
8
9
10
11
12
13
14
15
16
17
18
19
20
21
22
23
24
2021 - 2022 Legislature - 20 - LRB-4835/1
KP:amn&wlj
 ASSEMBLY BILL 957 SECTION 1
(e) Nothing in this section shall be construed as an obligation imposed on
controllers and processors that adversely affects the rights or freedoms of any
persons, such as exercising the right of free speech pursuant to the First Amendment
to the U.S. Constitution, or applies to the processing of personal data by a person in
the course of a purely personal or household activity.
(f) Personal data processed by a controller pursuant to this subsection may not
be processed for any purpose other than those expressly listed in this subsection
unless otherwise allowed by this section. Personal data processed by a controller
pursuant to this subsection may be processed to the extent that such processing is
both of the following:
1. Reasonably necessary and proportionate to the purposes listed in this
subsection.
2. Adequate, relevant, and limited to what is necessary in relation to the
specific purposes listed in this subsection. Personal data collected, used, or retained
pursuant to par. (b) shall, where applicable, take into account the nature and purpose
or purposes of such collection, use, or retention. Such data shall be subject to
reasonable administrative, technical, and physical measures to protect the
confidentiality, integrity, and accessibility of the personal data and to reduce
reasonably foreseeable risks of harm to consumers relating to such collection, use,
or retention of personal data.
(g) If a controller processes personal data pursuant to an exemption in this
section, the controller bears the burden of demonstrating that such processing
qualifies for the exemption and complies with the requirements in par. (f).
(h) Processing personal data for the purposes expressly identified in par. (a)
shall not solely make an entity a controller with respect to such processing.
1
2
3
4
5
6
7
8
9
10
11
12
13
14
15
16
17
18
19
20
21
22
23
24
25
2021 - 2022 Legislature - 21 - LRB-4835/1
KP:amn&wlj
SECTION 1 ASSEMBLY BILL 957
(8) SCOPE; EXEMPTIONS. (a) This section applies to persons that conduct
business in this state or produce products or services that are targeted to residents
of this state and who satisfy either of the following:
1. During a calendar year, the person controls or processes personal data of at
least 100,000 consumers.
2. The person controls or processes personal data of at least 25,000 consumers
and derives over 50 percent of gross revenue from the sale of personal data.
(b) This section shall not apply to any of the following:
1. An association, authority, board, department, commission, independent
agency, institution, office, society, or other body in state or local government created
or authorized to be created by the constitution or any law.
2. Financial institutions or data subject to Title V of the federal
Gramm-Leach-Bliley Act, 15 USC 6801 et seq.
3. A covered entity or business associate governed by the privacy, security, and
breach notification rules issued by the federal department of health and human
services, 45 CFR Parts 160 and 164 established pursuant to the federal Health
Insurance Portability and Accountability Act of 1996, and the federal Health
Information Technology for Economic and Clinical Health Act.
4. A nonprofit organization.
5. An institution of higher education.
(c) The following information and data are exempt from this section:
1. Protected health information under the federal Health Insurance Portability
and Accountability Act of 1996.
2. Patient health care records, as defined in s. 146.81 (4).
3. Treatment records, as defined in s. 51.30 (1) (b).
1
2
3
4
5
6
7
8
9
10
11
12
13
14
15
16
17
18
19
20
21
22
23
24
25
2021 - 2022 Legislature - 22 - LRB-4835/1
KP:amn&wlj
 ASSEMBLY BILL 957 SECTION 1
4. Patient identifying information for purposes of 42 USC 290dd-2.
5. Any of the following:
a. Identifiable private information for purposes of the federal policy for the
protection of human subjects under 45 CFR Part 46.
b. Identifiable private information that is otherwise information collected as
part of human subjects research pursuant to the good clinical practice guidelines
issued by the International Council for Harmonisation of Technical Requirements
for Pharmaceuticals for Human Use or under 21 CFR Parts 50 and 56.
c. Personal data used or shared in research conducted in accordance with the
requirements set forth in this section, or other research conducted in accordance with
applicable law.
6. Information and documents created for purposes of the federal Health Care
Quality Improvement Act of 1986, 42 USC 11101 et seq.
7. Patient safety work product for purposes of the federal Patient Safety and
Quality Improvement Act, 42 USC 299b-21 et seq.
8. Information derived from any of the health care-related information listed
in this paragraph that is deidentified in accordance with the requirements for
deidentification pursuant to the federal Health Insurance Portability and
Accountability Act of 1996.
9. Information originating from, and intermingled to be indistinguishable
with, or information treated in the same manner as information exempt under this
paragraph that is maintained by a covered entity or business associate as defined by
the federal Health Insurance Portability and Accountability Act of 1996 or a program
or a qualified service organization as defined by 42 USC 290dd-2.
1
2
3
4
5
6
7
8
9
10
11
12
13
14
15
16
17
18
19
20
21
22
23
24
2021 - 2022 Legislature - 23 - LRB-4835/1
KP:amn&wlj
SECTION 1 ASSEMBLY BILL 957
10. Information used only for public health activities and purposes as
authorized by the federal Health Insurance Portability and Accountability Act of
1996.
11. The collection, maintenance, disclosure, sale, communication, or use of any
personal information bearing on a consumer's credit worthiness, credit standing,
credit capacity, character, general reputation, personal characteristics, or mode of
living by a consumer reporting agency, furnisher, or user that provides information
for use in a consumer report, and by a user of a consumer report, but only to the extent
that such activity is regulated by and authorized under the federal Fair Credit
Reporting Act, 15 USC 1681 et seq.
12. Personal data collected, processed, sold, or disclosed in compliance with the
federal Driver's Privacy Protection Act of 1994, 18 USC 2721 et seq.
13. Personal data regulated by the federal Family Educational Rights and
Privacy Act, 20 USC 1232g et seq.
14. Personal data collected, processed, sold, or disclosed in compliance with the
federal Farm Credit Act, 12 USC 2001 et seq.
15. Data processed or maintained for any of the following purposes:
a. In the course of an individual applying to, employed by, or acting as an agent
or independent contractor of a controller, processor, or 3rd party, to the extent that
the data is collected and used within the context of that role.
b. As the emergency contact information of an individual under this section
used for emergency contact purposes.
c. That is necessary to retain to administer benefits for another individual
relating to an individual described in subd. 15. a. and used for the purposes of
administering those benefits.
1
2
3
4
5
6
7
8
9
10
11
12
13
14
15
16
17
18
19
20
21
22
23
24
25
2021 - 2022 Legislature - 24 - LRB-4835/1
KP:amn&wlj
 ASSEMBLY BILL 957 SECTION 1
(d) Controllers and processors that comply with the verifiable parental consent
requirements of the Children's Online Privacy Protection Act, 15 USC 6501 et seq.,
shall be deemed compliant with any obligation to obtain parental consent under this
section.
(9) VIOLATIONS. (a) The attorney general shall have exclusive authority to
enforce violations of this section.
(b) 1. Prior to initiating any action under this section, the attorney general shall
provide a controller or processor 30 days' written notice identifying the specific
provisions of this section the attorney general, on behalf of a consumer, alleges have
been or are being violated. If within the 30 days the controller or processor cures the
noticed violation and provides the attorney general an express written statement
that the alleged violations have been cured and that no further violations shall occur,
no action for statutory damages shall be initiated against the controller or processor.
2. If a controller or processor continues to violate this section in breach of an
express written statement provided to the consumer under this section, the attorney
general may initiate an action and seek damages for up to $7,500 for each violation
under this section.
(c) Nothing in this section shall be construed as providing the basis for, or be
subject to, a private right of action to violations of this section or under any other law.
(10) ENFORCEMENT. (a) The attorney general retains exclusive authority to
enforce this section by bringing an action in the name of the state, or on behalf of
persons residing in the state. The attorney general may issue a civil investigative
demand to any controller or processor believed to be engaged in, or about to engage
in, any violation of this section, and by the civil investigative demand the attorney
general may compel the attendance of any officers or agents of the controller or
1
2
3
4
5
6
7
8
9
10
11
12
13
14
15
16
17
18
19
20
21
22
23
24
25
2021 - 2022 Legislature - 25 - LRB-4835/1
KP:amn&wlj
SECTION 1 ASSEMBLY BILL 957
processor, examine the officers or agents of the controller or processor under oath,
require the production of any books or papers that the attorney general deems
relevant or material to the inquiry, and issue written interrogatories to be answered
by the officers or agents of the controller or processor.
(b) Any controller or processor that violates this section is subject to an
injunction and liable for a forfeiture of not more than $7,500 for each violation.
(c) Notwithstanding s. 814.04 (1), the attorney general may recover reasonable
expenses incurred in investigating and preparing the case, including attorney fees,
of any action initiated under this section.
(11) LOCAL PREEMPTION. No city, village, town, or county may enact or enforce
an ordinance that regulates the collection, processing, or sale of personal data.
SECTION 2.0Effective date.
(1) This act takes effect on January 1, 2024.
(END)
1
2
3
4
5
6
7
8
9
10
11
12
13
14