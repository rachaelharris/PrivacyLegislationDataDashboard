Req. No. 9493 Page 1
1
2
3
4
5
6
7
8
9
10
11
12
13
14
15
16
17
18
19
20
21
22
23
24
STATE OF OKLAHOMA
2nd Session of the 58th Legislature (2022)
HOUSE BILL 3447 By: O'Donnell
AS INTRODUCED
An Act relating to technology; enacting a new title
of law; enacting the Personal Data Protection Act;
defining terms; clarifying applicability of this act;
directing that a controller provide, correct, or
amend data subject's personal data on request;
directing that a controller provide notice of
practices and obtain consent; prohibiting controllers
use of certain data practice; requiring controllers
provide redress for performed prohibited practices;
directing processors to make available data subject's
personal data and correct inaccuracy upon request;
prohibiting processors processing of personal data
when not requested by controller; requiring processor
provide redress for prohibited data practices;
defining prohibited data practices; directing
procedures for unpseudonymized data; prohibiting
waiver of rights; requiring the creation and
disclosure of a privacy policy; permitting compatible
data practice; defining compatible data practice;
authorizing certain disclosures of personal data to
third parties for certain limited purposes;
clarifying when controller or processor engages in an
incompatible data practice; providing certain
exemptions for use of incompatible data practice;
requiring controller or processor to conduct and
maintain record of data-privacy and security-risk
assessment; authorizing Attorney General to determine
compliance; clarifying compliance; authorizing
stakeholder to initiate development of voluntary
consensus standard for compliance; clarifying when
Attorney General may or may not recognize a voluntary
consensus standard; authorizing Attorney General to
adopt rules; providing remedies for violation of act;
providing for noncodification; providing for
codification; and providing an effective date.
Req. No. 9493 Page 2
1
2
3
4
5
6
7
8
9
10
11
12
13
14
15
16
17
18
19
20
21
22
23
24
BE IT ENACTED BY THE PEOPLE OF THE STATE OF OKLAHOMA:
SECTION 1. NEW LAW A new section of law not to be
codified in the Oklahoma Statutes reads as follows:
In publishing the decennial Oklahoma Statutes, and the
cumulative supplements after July 1, 2022, West Publishing Company
shall include in such decennial statutes and supplements a new Title
75A, to be designated "Technology".
SECTION 2. NEW LAW A new section of law to be codified
in the Oklahoma Statutes as Section 200 of Title 75A, unless there
is created a duplication in numbering, reads as follows:
This act shall be known and may be cited as the "Uniform
Personal Data Protection Act".
SECTION 3. NEW LAW A new section of law to be codified
in the Oklahoma Statutes as Section 200.1 of Title 75A, unless there
is created a duplication in numbering, reads as follows:
In this act:
1. "Collecting controller" means a controller that collects
personal data directly from a data subject;
2. "Compatible data practice" means processing consistent with
Section 8 of this act;
3. "Controller" means a person that, alone or with others,
determines the purpose and means of processing;
4. "Data subject" means an individual who is identified or
described by personal data;
Req. No. 9493 Page 3
1
2
3
4
5
6
7
8
9
10
11
12
13
14
15
16
17
18
19
20
21
22
23
24
5. "De-identified data" means data that is modified to remove
all direct identifiers and to reasonably ensure that the record
cannot be linked to an identified data subject by a person who does
not have personal knowledge of or special access to the data
subject's information;
6. "Direct identifier" means information that is commonly used
to identify a data subject, including name, physical address, email
address, recognizable photograph, and telephone number;
7. "Incompatible data practice" means processing that may be
performed consistent with Section 9 of this act;
8. "Maintains", with respect to personal data, means to retain,
hold, store, or preserve personal data as a system of records used
to retrieve records about individual data subjects for the purpose
of individualized communication or treatment;
9. "Person" means an individual, estate, business or nonprofit
entity, or other legal entity. The term does not include a public
corporation or government or governmental subdivision, agency, or
instrumentality;
10. "Personal data" means a record that identifies or describes
a data subject by a direct identifier or is pseudonymized data. The
term does not include de-identified data;
11. "Processing" means performing or directing performance of
an operation on personal data, including collection, transmission,
use, disclosure, analysis, prediction, and modification of the 
Req. No. 9493 Page 4
1
2
3
4
5
6
7
8
9
10
11
12
13
14
15
16
17
18
19
20
21
22
23
24
personal data, whether or not by automated means. "Process" has a
corresponding meaning;
12. "Processor" means a person who processes personal data on
behalf of a controller;
13. "Prohibited data practice" means processing which is
prohibited by Section 10 of this act;
14. "Pseudonymized data" means personal data without a direct
identifier that can be reasonably linked to a data subject's
identity or is maintained to allow individualized communication
with, or treatment of, the data subject. The term includes a record
without a direct identifier if the record contains an Internet
protocol address, browser, software, or hardware identification
code, or other data uniquely linked to a particular device. The
term does not include de-identified data;
15. "Publicly available information" means information:
a. lawfully made available from a federal, state, or
local government record,
b. available to the general public in widely distributed
media, including:
(1) a publicly accessible website,
(2) a website or other forum with restricted access
if the information is available to a broad
audience,
(3) a telephone book or online directory,
Req. No. 9493 Page 5
1
2
3
4
5
6
7
8
9
10
11
12
13
14
15
16
17
18
19
20
21
22
23
24
(4) a television, Internet, or radio program, and
(5) news media,
c. observable from a publicly accessible location, or
d. that a person reasonably believes is made available
lawfully to the general public if:
(1) the information is of a type generally available
to the public, and
(2) the person has no reason to believe that a data
subject with authority to remove the information
from public availability has directed the
information to be removed;
16. "Record" means information:
a. inscribed on a tangible medium, or
b. stored in an electronic or other medium and
retrievable in perceivable form;
17. "Sensitive data" means personal data that reveals:
a. racial or ethnic origin, religious belief, gender,
sexual orientation, citizenship, or immigration
status,
b. credentials sufficient to access an account remotely,
c. a credit or debit card number or financial account
number,
d. a Social Security number, tax-identification number,
driver license number, military identification number, 
Req. No. 9493 Page 6
1
2
3
4
5
6
7
8
9
10
11
12
13
14
15
16
17
18
19
20
21
22
23
24
or identifying number on a government-issued
identification,
e. geolocation in real time,
f. a criminal record,
g. income,
h. diagnosis or treatment for a disease or health
condition,
i. genetic sequencing information, or
j. information about a data subject the controller knows
or has reason to know is under thirteen (13) years of
age;
18. "Sign" means, with present intent to authenticate or adopt
a record:
a. execute or adopt a tangible symbol, or
b. attach to or logically associate with the record an
electronic symbol, sound, or procedure;
19. "Stakeholder" means a person who, or represents a person
who has, a direct interest in the development of a voluntary
consensus standard;
20. "State" means a state of the United States, the District of
Columbia, Puerto Rico, the United States Virgin Islands, or any
other territory or possession subject to the jurisdiction of the
United States. The term includes federally recognized Indian tribes.
Req. No. 9493 Page 7
1
2
3
4
5
6
7
8
9
10
11
12
13
14
15
16
17
18
19
20
21
22
23
24
21. "Third-party controller" means a controller that receives
from another controller authorized access to personal data or
pseudonymized data and determines the purpose and means of
additional processing.
SECTION 4. NEW LAW A new section of law to be codified
in the Oklahoma Statutes as Section 200.2 of Title 75A, unless there
is created a duplication in numbering, reads as follows:
A. Except as provided in subsections B and C of this section,
this act applies to the activities of a controller or processor that
conducts business in this state or produces products or provides
services purposefully directed to residents of this state and:
1. At any time during a calendar year maintains personal data
about more than fifty thousand (50,000) data subjects who are
residents of this state, excluding data subjects whose data is
collected or maintained solely to complete a payment transaction;
2. Earns more than fifty percent (50%) of its gross annual
revenue during a calendar year from maintaining personal data as a
controller or processor;
3. Is a processor acting on behalf of a controller the
processor knows or has reason to know satisfies paragraph 1 or 2 of
this subsection; or
4. Maintains personal data, unless it processes the personal
data solely using compatible data practices.
Req. No. 9493 Page 8
1
2
3
4
5
6
7
8
9
10
11
12
13
14
15
16
17
18
19
20
21
22
23
24
B. This act does not apply to an agency or instrumentality of
this state or a political subdivision of this state.
C. This act does not apply to personal data that is:
1. Publicly available information;
2. Processed or maintained solely as part of human-subjects
research conducted in compliance with legal requirements for the
protection of human subjects;
3. Processed or disclosed as required or permitted by a warrant,
subpoena, or court order or rule, or otherwise as specifically required
by law;
4. Subject to a public-disclosure requirement under the
Oklahoma Open Records Act, Sections 24A.1 through 24A.33 of Title 51
of the Oklahoma Statutes; or
5. Processed or maintained in the course of a data subject's
employment or application for employment.
SECTION 5. NEW LAW A new section of law to be codified
in the Oklahoma Statutes as Section 200.3 of Title 75A, unless there
is created a duplication in numbering, reads as follows:
A. A controller shall:
1. If a collecting controller, provide under Section 6 of this
act a copy of a data subject's personal data to the data subject on
request;
2. Correct or amend under Section 6 of this act a data
subject's personal data upon the data subject's request;
Req. No. 9493 Page 9
1
2
3
4
5
6
7
8
9
10
11
12
13
14
15
16
17
18
19
20
21
22
23
24
3. Provide notice under Section 7 of this act about the
personal data it maintains and its processing practices;
4. Obtain consent under Section 9 of this act for processing
that is an incompatible data practice;
5. Not use a prohibited data practice;
6. Conduct and maintain under Section 11 of this act dataprivacy and security-risk assessments; and
7. Provide redress for a prohibited data practice the
controller performs or is responsible for performing while
processing a data subject's personal data.
B. A processor shall:
1. On request of the controller, provide the controller with a
data subject's personal data or enable the controller to access the
personal data at no cost to the controller;
2. On request of the controller, correct an inaccuracy in a
data subject's personal data;
3. Not process personal data for a purpose other than one
requested by the controller;
4. Conduct and maintain data-privacy and security-risk
assessments in accordance with Section 11 of this act; and
5. Provide redress for a prohibited data practice that the
processor knowingly performs in the course of processing a data
subject's personal data at the direction of the controller.
C. A controller is responsible under this act for a prohibited 
Req. No. 9493 Page 10
1
2
3
4
5
6
7
8
9
10
11
12
13
14
15
16
17
18
19
20
21
22
23
24
data practice conducted by another if:
1. The practice is conducted with respect to personal data
collected by the controller; and
2. The controller knew the personal data would be used for the
practice and was in a position to prevent it.
D. A processor is responsible under this act for a prohibited data
practice conducted by another if:
1. The practice is conducted with respect to personal data
processed by the processor; and
2. The processor knew the personal data would be used for the
practice and was in a position to prevent it.
SECTION 6. NEW LAW A new section of law to be codified
in the Oklahoma Statutes as Section 200.4 of Title 75A, unless there
is created a duplication in numbering, reads as follows:
A. Unless personal data is pseudonymized and not maintained
with sensitive data, a collecting controller, with respect to
personal data initially collected by the controller and maintained
by the controller or a third-party controller or processor, shall:
1. Establish a reasonable procedure for a data subject to
request, receive a copy of, and propose an amendment or correction
to personal data about the data subject;
2. Establish a procedure to authenticate the identity of a data
subject who requests a copy of the data subject's personal data;
Req. No. 9493 Page 11
1
2
3
4
5
6
7
8
9
10
11
12
13
14
15
16
17
18
19
20
21
22
23
24
3. Not later than forty-five (45) days after receiving a
request from a data subject authenticated under paragraph 2 of this
subsection for a copy of personal data about the data subject,
comply with the request or provide an explanation of action being
taken to comply with the request;
4. On request, provide the data subject one copy of the data
subject's personal data free of charge once every twelve (12) months
and additional copies on payment of a fee reasonably based on the
collecting controller's administrative costs;
5. Make an amendment or correction as requested by a data
subject if the collecting controller has no reason to believe the
request is inaccurate, unreasonable, or excessive; and
6. Confirm to the data subject that an amendment or correction
has been made or explain why the amendment or correction has not
been made.
B. A collecting controller shall make a reasonable effort to
ensure that a correction of personal data performed by the
controller also is performed on personal data maintained by a thirdparty controller or processor that directly or indirectly received
the personal data from the collecting controller. A third-party
controller or processor shall make a reasonable effort to assist the
collecting controller, if necessary to satisfy a request of a data
subject under this section.
Req. No. 9493 Page 12
1
2
3
4
5
6
7
8
9
10
11
12
13
14
15
16
17
18
19
20
21
22
23
24
C. A controller may not deny a data subject a good or service,
charge a different rate, or provide a different level of quality to
a data subject in retaliation for exercising a right under this
section. It is not retaliation under this subsection for a
controller to make a data subject ineligible to participate in a
program if:
1. Corrected information requested by the data subject makes
the data subject ineligible for the program; and
2. The program's terms of service specify the eligibility
requirements for all participants.
D. An agreement that waives or limits a right or duty under
this section is unenforceable.
SECTION 7. NEW LAW A new section of law to be codified
in the Oklahoma Statutes as Section 200.5 of Title 75A, unless there
is created a duplication in numbering, reads as follows:
A. A controller shall adopt and comply with a reasonably clear
and accessible privacy policy that discloses:
1. Categories of personal data maintained by or on behalf of
the controller;
2. Categories of personal data the controller provides to a
processor or another controller and the purpose of providing the
personal data;
3. Compatible data practices applied routinely to personal data
by the controller or by an authorized processor;
Req. No. 9493 Page 13
1
2
3
4
5
6
7
8
9
10
11
12
13
14
15
16
17
18
19
20
21
22
23
24
4. Incompatible data practices that, if the data subject
consents under Section 9 of this act, will be applied by the
controller or an authorized processor;
5. The procedure for a data subject to request a copy of, or
propose an amendment or correction to, personal data under Section 6
of this act;
6. Federal, state, or international privacy laws or frameworks
with which the controller complies; and
7. Any voluntary consensus standard adopted by the controller.
B. The privacy policy under subsection A of this section must
be reasonably available to a data subject at the time personal data
is collected about the data subject.
C. If a controller maintains a public website, the controller
shall publish the privacy policy on the website.
SECTION 8. NEW LAW A new section of law to be codified
in the Oklahoma Statutes as Section 200.6 of Title 75A, unless there
is created a duplication in numbering, reads as follows:
A. A controller or processor may engage in a compatible data
practice without the data subject's consent. A controller or
processor engages in a compatible data practice if the processing is
consistent with the ordinary expectations of data subjects or is
likely to benefit data subjects substantially. The following factors
apply to determine whether processing is a compatible data practice:
1. The data subject's relationship with the controller;
Req. No. 9493 Page 14
1
2
3
4
5
6
7
8
9
10
11
12
13
14
15
16
17
18
19
20
21
22
23
24
2. The type of transaction in which the personal data was
collected;
3. The type and nature of the personal data processed;
4. The risk of a negative consequence on the data subject by use
or disclosure of the personal data;
5. The effectiveness of safeguards against unauthorized use or
disclosure of the personal data; and
6. The extent to which the practice advances the economic,
health, or other interests of the data subject.
B. A compatible data practice includes processing that:
1. Initiates or effectuates a transaction with a data subject with
the data subject's knowledge or participation;
2. Is reasonably necessary to comply with a legal obligation or
regulatory oversight of the controller;
3. Meets a particular and explainable managerial, personnel,
administrative, or operational need of the controller or processor;
4. Permits appropriate internal oversight of the controller by the
controller's or processor's agent or external oversight by a government
unit;
5. Is reasonably necessary to create pseudonymized or deidentified data;
6. Permits analysis:
a. to discover insights related to public health, public
policy, or other matters of general public interest and 
Req. No. 9493 Page 15
1
2
3
4
5
6
7
8
9
10
11
12
13
14
15
16
17
18
19
20
21
22
23
24
does not include use of personal data to make a
prediction or determination about a particular data
subject, or
b. for research and development of a product or service;
7. Is reasonably necessary to prevent, detect, investigate,
report on, prosecute, or remediate an actual or potential:
a. fraud,
b. unauthorized transaction or claim,
c. security incident,
d. malicious, deceptive, or illegal activity,
e. legal liability of the controller or processor, or
f. threat to national security;
8. Assists a person or government entity acting under paragraph
7 of this subsection;
9. Is reasonably necessary to comply with or defend a legal
claim; or
10. Accomplishes any other purpose determined to be a
compatible data practice under subsection A of this section.
C. A controller may use personal data, or disclose
pseudonymized data to a third-party controller, to deliver to a data
subject targeted advertising and other purely expressive content. A
controller may not use personal data, or disclose pseudonymized
data, to offer terms to a data subject that are different from terms
offered to data subjects generally, including terms relating to 
Req. No. 9493 Page 16
1
2
3
4
5
6
7
8
9
10
11
12
13
14
15
16
17
18
19
20
21
22
23
24
price or quality. Processing personal data or pseudonymized data
for differential treatment is an incompatible data practice unless
the processing is otherwise compatible under this section. This
subsection does not prevent providing different treatment to members
of a program if the program's terms of service specify the
eligibility requirements for all participants.
D. A controller or processor may process personal data in
accordance with the rules of a voluntary consensus standard under
Sections 13 through 16 of this act unless a court has prohibited the
processing or found it to be an incompatible data practice.
Processing under a voluntary consensus standard is permitted only if
a controller adopts and commits to the standard in its privacy
policy.
SECTION 9. NEW LAW A new section of law to be codified
in the Oklahoma Statutes as Section 200.7 of Title 75A, unless there
is created a duplication in numbering, reads as follows:
A. A controller or processor engages in an incompatible data
practice if the processing:
1. Is not a compatible data practice under Section 8 of this act
or a prohibited data practice under Section 10 of this act; or
2. Even if a compatible data practice under Section 8 of this
act, is inconsistent with a privacy policy adopted under Section 7 of
this act.
Req. No. 9493 Page 17
1
2
3
4
5
6
7
8
9
10
11
12
13
14
15
16
17
18
19
20
21
22
23
24
B. A controller may use an incompatible data practice to process
personal data that does not include sensitive data if, at the time the
personal data is collected about a data subject, the controller
provides the data subject:
1. Notice and information sufficient to allow the data subject to
understand the nature of the incompatible data processing; and
2. A reasonable opportunity to withhold consent to the practice.
C. A controller may not process a data subject's sensitive data
using an incompatible data practice without the data subject's express
consent in a signed record for each practice.
D. Unless processing is a prohibited data practice, a
controller may require a data subject to consent to an incompatible
data practice as a condition for access to the controller's goods or
services. The controller may offer a reward or discount in exchange
for the data subject's consent to process the data subject's
personal data.
SECTION 10. NEW LAW A new section of law to be codified
in the Oklahoma Statutes as Section 200.8 of Title 75A, unless there
is created a duplication in numbering, reads as follows:
A. A controller may not engage in a prohibited data practice.
Processing personal data is a prohibited data practice if the
processing is likely to:
1. Subject a data subject to specific and significant:
a. financial, physical, or reputational harm,
Req. No. 9493 Page 18
1
2
3
4
5
6
7
8
9
10
11
12
13
14
15
16
17
18
19
20
21
22
23
24
b. embarrassment, ridicule, intimidation, or harassment, or
c. physical or other intrusion on solitude or seclusion if
the intrusion would be highly offensive to a reasonable
person;
2. Result in misappropriation of personal data to assume
another's identity;
3. Constitute a violation of other law, including federal or state
law against discrimination;
4. Fail to provide reasonable data-security measures, including
appropriate administrative, technical, and physical safeguards to
prevent unauthorized access; or
5. Process without consent under Section 9 of this act personal
data in a manner that is an incompatible data practice.
B. Reidentifying or causing the reidentification of pseudonymized
or de-identified data is a prohibited data practice unless:
1. The reidentification is performed by a controller or processor
that previously had pseudonymized or de-identified the personal data;
2. The data subject expects the personal data to be maintained in
identified form by the controller performing the reidentification; or
3. The purpose of the reidentification is to assess the privacy
risk of de-identified data and the person performing the
reidentification does not use or disclose reidentified personal data
except to demonstrate a privacy vulnerability to the controller or
processor that created the de-identified data.
Req. No. 9493 Page 19
1
2
3
4
5
6
7
8
9
10
11
12
13
14
15
16
17
18
19
20
21
22
23
24
SECTION 11. NEW LAW A new section of law to be codified
in the Oklahoma Statutes as Section 200.9 of Title 75A, unless there
is created a duplication in numbering, reads as follows:
A. A controller or processor shall conduct and maintain in a
record a data-privacy and security-risk assessment. The assessment
may take into account the size, scope, and type of business of the
controller or processor and the resources available to it. The
assessment must evaluate:
1. Privacy and security risks to the confidentiality and
integrity of the personal data being processed or maintained, the
likelihood of the risks, and the impact that the risks would have on
the privacy and security of the personal data;
2. Efforts taken to mitigate the risks; and
3. The extent to which the data practices comply with this act.
B. A controller or processor shall update the data-privacy and
security-risk assessment if there is a change in the risk
environment or in a data practice that may materially affect the
privacy or security of the personal data.
C. A data-privacy and security-risk assessment is confidential
and is not subject to the Oklahoma Open Records Act, Sections 24A.1
through 24A.33 of the Oklahoma Statutes, and the Oklahoma Discovery
Code, Sections 3224 through 3237 of Title 12 of the Oklahoma
Statutes. The fact that a controller or processor conducted an 
Req. No. 9493 Page 20
1
2
3
4
5
6
7
8
9
10
11
12
13
14
15
16
17
18
19
20
21
22
23
24
assessment, the records analyzed in the assessment, and the date of
the assessment are not confidential under this section.
SECTION 12. NEW LAW A new section of law to be codified
in the Oklahoma Statutes as Section 200.10 of Title 75A, unless
there is created a duplication in numbering, reads as follows:
A. A controller or processor complies with this act if it
complies with a comparable law protecting personal data in another
jurisdiction and the Attorney General determines the law in the
other jurisdiction is at least as protective of personal data as
this act. The Attorney General may charge a fee to a controller or
processor that requests a determination of compliance with a comparable
law under this subsection. The fee must reflect the cost reasonably
expected to be incurred by the Attorney General to determine whether
the comparable law is at least as protective as this act.
B. A controller or processor complies with this act with
respect to processing that is subject to the following acts as
amended:
1. The Health Insurance Portability and Accountability Act,
P.L. No. 104-191, if the controller or processor is regulated by
that act;
2. The Fair Credit Reporting Act, 15 U.S.C. Section 1681 et
seq., or otherwise is used to generate a consumer report by a
consumer reporting agency as defined in Section 603(f) of the Fair 
Req. No. 9493 Page 21
1
2
3
4
5
6
7
8
9
10
11
12
13
14
15
16
17
18
19
20
21
22
23
24
Credit Reporting Act, 15 U.S.C. Section 1681a(f), a furnisher of the
information, or a person procuring or using a consumer report;
3. The Gramm-Leach-Bliley Act of 1999, 15 U.S.C. Section 6801
et seq.;
4. The Driver's Privacy Protection Act of 1994, 18 U.S.C.
Section 2721 et seq.;
5. The Family Education Rights and Privacy Act of 1974, 20
U.S.C. Section 1232g; or
6. The Children's Online Privacy Protection Act of 1998, 15
U.S.C. Section 6501 et seq.
SECTION 13. NEW LAW A new section of law to be codified
in the Oklahoma Statutes as Section 200.11 of Title 75A, unless
there is created a duplication in numbering, reads as follows:
A controller or processor complies with a requirement of this
act if it adopts and complies with a voluntary consensus standard
that addresses that requirement and is recognized by the Attorney
General under Section 16 of this act.
SECTION 14. NEW LAW A new section of law to be codified
in the Oklahoma Statutes as Section 200.12 of Title 75A, unless
there is created a duplication in numbering, reads as follows:
A stakeholder may initiate the development of a voluntary
consensus standard for compliance with this act. A voluntary
consensus standard may address any requirement including:
1. Identification of compatible data practices for an industry;
Req. No. 9493 Page 22
1
2
3
4
5
6
7
8
9
10
11
12
13
14
15
16
17
18
19
20
21
22
23
24
2. The procedure and method for securing consent of a data
subject for an incompatible data practice;
3. A common method for responding to a request by a data
subject for a copy or correction of personal data, including a
mechanism for authenticating the identity of the data subject;
4. A format for a privacy policy that provides consistent and
fair communication of the policy to data subjects;
5. Practices that provide reasonable security for personal data
maintained by a controller or processor; and
6. Any other policy or practice that relates to compliance with
this act.
SECTION 15. NEW LAW A new section of law to be codified
in the Oklahoma Statutes as Section 200.13 of Title 75A, unless
there is created a duplication in numbering, reads as follows:
The Attorney General shall not recognize a voluntary consensus
standard unless it is developed through a consensus procedure that:
1. Achieves general agreement, but not necessarily unanimity,
and:
a. includes stakeholders representing a diverse range of
industry, consumer, and public interests,
b. gives fair consideration to each comment by a
stakeholder,
c. responds to each good-faith objection by a stakeholder,
Req. No. 9493 Page 23
1
2
3
4
5
6
7
8
9
10
11
12
13
14
15
16
17
18
19
20
21
22
23
24
d. attempts to resolve each good-faith objection by a
stakeholder,
e. provides each stakeholder an opportunity to change the
stakeholder's position after reviewing comments, and
f. informs each stakeholder of the disposition of each
objection and the reason for the disposition;
2. Provides stakeholders a reasonable opportunity to contribute
their knowledge, talents, and efforts to the development of the
standard;
3. Is responsive to the concerns of all stakeholders;
4. Consistently complies with documented and publicly available
policies and procedures that provide adequate notice of meetings and
standards development; and
5. Permits a stakeholder to file a statement of dissent.
SECTION 16. NEW LAW A new section of law to be codified
in the Oklahoma Statutes as Section 200.14 of Title 75A, unless
there is created a duplication in numbering, reads as follows:
A. On filing of a request by any person, the Attorney General may
recognize a voluntary consensus standard if the Attorney General finds
the standard:
1. Does not conflict with any requirement of Sections 6 through
11 of this act;
2. Is developed through a procedure that substantially complies
with Section 15 of this act; and
Req. No. 9493 Page 24
1
2
3
4
5
6
7
8
9
10
11
12
13
14
15
16
17
18
19
20
21
22
23
24
3. If necessary, reasonably reconciles a requirement of this act
with the requirements of other law.
B. The Attorney General shall adopt rules under the Administrative
Procedures Act, Sections 250 through 323 of Title 75 of the Oklahoma
Statutes, or otherwise establish a procedure for filing a request under
subsection A of this section. The rules may require:
1. That the request be in a record demonstrating the standard and
procedure through which it was adopted complies with this act;
2. The person filing the request to indicate whether the standard
has been recognized as appropriate in another jurisdiction and, if so,
identify the authority that recognized it; and
3. The person filing the request to pay a fee, which must reflect
the cost reasonably expected to be incurred by the Attorney General in
acting on a request.
C. The Attorney General shall determine whether to grant or deny
the request and provide the reason for a grant or denial. In making
the determination, the Attorney General shall consider the need to
promote predictability and uniformity among the states and give
appropriate deference to a voluntary consensus standard developed
consistent with this act and recognized by a privacy-enforcement agency
in another state.
D. After notice and hearing, the Attorney General may withdraw
recognition of a voluntary consensus standard if the Attorney General
Req. No. 9493 Page 25
1
2
3
4
5
6
7
8
9
10
11
12
13
14
15
16
17
18
19
20
21
22
23
24
finds that the standard or its implementation is not consistent with
this act.
E. A voluntary consensus standard recognized by the Attorney
General is a public record under the Oklahoma Open Records Act,
Sections 24A.1 through 24A.33 of Title 51 of the Oklahoma Statutes.
SECTION 17. NEW LAW A new section of law to be codified
in the Oklahoma Statutes as Section 200.15 of Title 75A, unless
there is created a duplication in numbering, reads as follows:
A. Subject to subsection D of this section, the enforcement
authority, remedies, and penalties provided by the Oklahoma Consumer
Protection Act, beginning in Section 751 of Title 15 of the Oklahoma
Statutes, apply to a violation of this act.
B. The Attorney General may adopt rules under Administrative
Procedures Act, Sections 250 through 323 of Title 75 of the Oklahoma
Statutes, to implement this act.
C. In adopting rules under this section, the Attorney General
shall consider the need to promote predictability for data subjects,
controllers, and processors and uniformity among the states. The
Attorney General may:
1. Consult with Attorneys General and other agencies with
authority to enforce personal-data privacy in other jurisdictions
that have laws substantially similar to this act;
Req. No. 9493 Page 26
1
2
3
4
5
6
7
8
9
10
11
12
13
14
15
16
17
18
19
20
21
22
23
24
2. Consider suggested or model rules or enforcement guidelines
promulgated by the National Association of Attorneys General or a
successor organization;
3. Consider the rules and practices of Attorneys General and
other agencies with authority to enforce personal-data privacy in
other jurisdictions; and
4. Consider voluntary consensus standards developed consistent
with this act, that have been recognized by other Attorneys General
or other agencies with authority to enforce personal-data privacy.
D. A private cause of action for a violation of this act is not
authorized by this act or the Oklahoma Consumer Protection Act,
beginning in Section 751 of Title 15 of the Oklahoma Statutes.
SECTION 18. NEW LAW A new section of law to be codified
in the Oklahoma Statutes as Section 200.16 of Title 75A, unless
there is created a duplication in numbering, reads as follows:
This act does not create or affect a cause of action under other
law of this state.
SECTION 19. NEW LAW A new section of law to be codified
in the Oklahoma Statutes as Section 200.17 of Title 75A, unless
there is created a duplication in numbering, reads as follows:
In applying and construing this uniform act, a court shall
consider the promotion of uniformity of the law among jurisdictions
that enact it.
Req. No. 9493 Page 27
1
2
3
4
5
6
7
8
9
10
11
12
13
14
15
16
17
18
19
20
21
22
23
24
SECTION 20. NEW LAW A new section of law to be codified
in the Oklahoma Statutes as Section 200.18 of Title 75A, unless
there is created a duplication in numbering, reads as follows:
This act modifies, limits, or supersedes the Electronic
Signatures in Global and National Commerce Act, 15 U.S.C. Section
7001 et seq., as amended, but does not modify, limit, or supersede
15 U.S.C. Section 7001(c), or authorize electronic delivery of any
of the notices described in 15 U.S.C. Section 7003(b).
SECTION 21. This act shall become effective November 1, 2022.
58-2-9493 MJ 12/29/21