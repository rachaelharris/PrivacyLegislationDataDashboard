BILL NUMBER: S567

SPONSOR: HOYLMAN
 
TITLE OF BILL:

An act to amend the general business law and the state finance law, in
relation to allowing consumers the right to request from businesses the
categories of personal information the business has sold or disclosed to
third parties

 
PURPOSE OR GENERAL IDEA OF BILL:

To enact data privacy protections for consumers.

 
SUMMARY OF SPECIFIC PROVISIONS:

Section 1 amends the article heading of section 39-F of the General
Business Law

Section 2 adds a new section to the General Business Law. It provides
for various definitions. It also creates numerous consumer rights,
including:

a right for the consumer to request from a business the category of
personal information it shares, such as name, social security number, IP
address, etc...

a right for the consumer to request the identities of any third parties
to whom the business has sold or shared personal information

a right for the consumer to opt-out of having their information sold to
third parties

a right to protection from discrimination for consumers who choose to
opt-out

Section 3 amends the state finance law to create a '"consumer privacy
fund" which shall consist of monies received by the state pursuant to
any violation of this bill.

Section 4 is the effective date.

 
JUSTIFICATION:

The Internet and the sharing of information play such a tremendous role
in the lives of New Yorkers. However, despite the Internet's increased
role in society, privacy protections have not kept the same pace.
According to the Pew Research Center, "68% of internet users believe
current laws are not good enough in protecting people's privacy online."

We trust that the information we give to businesses online is safe.
Everything from passport details when booking a vacation to data about
our children when helping them with homework is given away, often with-
out a second thought. However, countless investigations show that our
data is often sold or traded in order to enhance companies' bottom
lines. According to the New York Times, "personal data has become the
most prized commodity of the digital age, traded on a vast scale by some
of the most powerful companies in Silicon Valley and beyond."

With data privacy protections being adopted in Europe and California,
New York State should ensure that our privacy is also protected. This
legislation enacts similar protections to those enacted in the State of
California.

 
PRIOR LEGISLATIVE HISTORY:

S.4411 of 2019-2020 (Hoylman): Died in Consumer Protection
A.6351 of 2019-2020 (Gunther): Died in Consumer Affairs and Protection

 
FISCAL IMPLICATIONS:

None to state

 
EFFECTIVE DATE:
This act shall take effect on the one hundred eightieth day after it
shall have become a law. Effective immediately, the addition, amendment
and/or repeal of any rule or regulation necessary for the implementation
of this act on its effective date are authorized and directed to be made
and completed on or before such effective date.
VIEW LESS
S567 (ACTIVE) - BILL TEXT
DOWNLOAD PDF
 
                     S T A T E   O F   N E W   Y O R K
 ________________________________________________________________________
 
                                    567
 
                        2021-2022 Regular Sessions
 
                             I N  S E N A T E
 
                                (PREFILED)
 
                              January 6, 2021
                                ___________
 
 Introduced  by  Sen. HOYLMAN -- read twice and ordered printed, and when
   printed to be committed to the Committee on Consumer Protection
 
 AN ACT to amend the general business law and the state finance  law,  in
   relation  to  allowing  consumers the right to request from businesses
   the categories of  personal  information  the  business  has  sold  or
   disclosed to third parties
 
   THE  PEOPLE OF THE STATE OF NEW YORK, REPRESENTED IN SENATE AND ASSEM-
 BLY, DO ENACT AS FOLLOWS:
 
   Section 1. The article heading of article 39-F of the general business
 law, as amended by chapter 117 of the laws of 2019, is amended  to  read
 as follows:
 
          [NOTIFICATION OF UNAUTHORIZED] ACQUISITION AND CONTROL
            OF PRIVATE AND PERSONAL INFORMATION; DATA SECURITY
                                PROTECTIONS
 
   §  2. The general business law is amended by adding a new section 899-
 cc to read as follows:
   § 899-CC. CONSUMER CONTROL OF PERSONAL INFORMATION. 1. FOR PURPOSES OF
 THIS SECTION, THE FOLLOWING DEFINITIONS SHALL APPLY:
   (A) "BIOMETRIC DATA" MEANS AN INDIVIDUAL'S  PHYSIOLOGICAL,  BIOLOGICAL
 OR  BEHAVIORAL  CHARACTERISTICS,  INCLUDING AN INDIVIDUAL'S DEOXYRIBONU-
 CLEIC ACID THAT CAN BE USED, SINGLY OR IN COMBINATION WITH EACH OTHER OR
 WITH OTHER IDENTIFYING DATA TO ESTABLISH INDIVIDUAL IDENTITY.  BIOMETRIC
 DATA INCLUDES BUT IS NOT LIMITED TO IMAGERY OF THE IRIS, RETINA, FINGER-
 PRINT, FACE, HAND, PALM, VEIN PATTERNS, AND VOICE RECORDINGS, FROM WHICH
 AN  IDENTIFIER  TEMPLATE, SUCH AS A FACEPRINT, A MINUTIAE TEMPLATE, OR A
 VOICEPRINT, CAN BE EXTRACTED, AND KEYSTROKE PATTERNS  OR  RHYTHMS,  GAIT
 PATTERNS  OR  RHYTHMS,  AND SLEEP, HEALTH, OR EXERCISE DATA THAT CONTAIN
 IDENTIFYING INFORMATION.
 
  EXPLANATION--Matter in ITALICS (underscored) is new; matter in brackets